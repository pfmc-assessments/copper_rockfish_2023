\input{input_accessability.tex}
\documentclass[11pt,
  english,
  letterpaper,
]{article}
\usepackage{sa4ss}
\usepackage{amsmath,amssymb,array}
\usepackage{booktabs}

% From tagged-template.latex
\usepackage{lmodern}
\usepackage{ifxetex,ifluatex}
\ifnum 0\ifxetex 1\fi\ifluatex 1\fi=0 % if pdftex
  \usepackage[T1]{fontenc}
  \usepackage[utf8]{inputenc}
  \usepackage{textcomp} % provide euro and other symbols
\else % if luatex or xetex
  \usepackage{unicode-math}
  \defaultfontfeatures{Scale=MatchLowercase}
  \defaultfontfeatures[\rmfamily]{Ligatures=TeX,Scale=1}
\fi

% Use upquote if available, for straight quotes in verbatim environments
\IfFileExists{upquote.sty}{\usepackage{upquote}}{}
\IfFileExists{microtype.sty}{% use microtype if available
  \usepackage[]{microtype}
  \UseMicrotypeSet[protrusion]{basicmath} % disable protrusion for tt fonts
}{}
\makeatletter
\@ifundefined{KOMAClassName}{% if non-KOMA class
  \IfFileExists{parskip.sty}{%
    \usepackage{parskip}
  }{% else
    \setlength{\parindent}{0pt}
    \setlength{\parskip}{6pt plus 2pt minus 1pt}}
}{% if KOMA class
  \KOMAoptions{parskip=half}}
\makeatother
\usepackage{xcolor}
\IfFileExists{xurl.sty}{\usepackage{xurl}}{} % add URL line breaks if available
\hypersetup{
  pdflang={en},
  hidelinks,
  pdfcreator={LaTeX via pandoc}}
\urlstyle{same} % disable monospaced font for URLs
\usepackage{longtable}
% Correct order of tables after \paragraph or \subparagraph
\usepackage{etoolbox}
\makeatletter
\patchcmd\longtable{\par}{\if@noskipsec\mbox{}\fi\par}{}{}
\makeatother
% Allow footnotes in longtable head/foot
\IfFileExists{footnotehyper.sty}{\usepackage{footnotehyper}}{\usepackage{footnote}}
\makesavenoteenv{longtable}
\usepackage{graphicx}
\makeatletter
\def\maxwidth{\ifdim\Gin@nat@width>\linewidth\linewidth\else\Gin@nat@width\fi}
\def\maxheight{\ifdim\Gin@nat@height>\textheight\textheight\else\Gin@nat@height\fi}
\makeatother
% Scale images if necessary, so that they will not overflow the page
% margins by default, and it is still possible to overwrite the defaults
% using explicit options in \includegraphics[width, height, ...]{}
\setkeys{Gin}{width=\maxwidth,height=\maxheight,keepaspectratio}
% Set default figure placement to htbp
\makeatletter
\def\fps@figure{htbp}
\makeatother
\setlength{\emergencystretch}{3em} % prevent overfull lines
\providecommand{\tightlist}{%
  \setlength{\itemsep}{0pt}\setlength{\parskip}{0pt}}
\setcounter{secnumdepth}{5}
\ifxetex
  % Load polyglossia as late as possible: uses bidi with RTL langages (e.g. Hebrew, Arabic)
  \usepackage{polyglossia}
  \setmainlanguage[]{}
\else
  \usepackage[shorthands=off,main=english]{babel}
\fi

%Define cslreferences environment, required by pandoc 2.8
%https://github.com/rstudio/rmarkdown/issues/1649
\newlength{\csllabelwidth}
\setlength{\csllabelwidth}{3em}
\newlength{\cslhangindent}
\setlength{\cslhangindent}{1.5em}
% for Pandoc 2.8 to 2.10.1
\newenvironment{cslreferences}%
  {}%
  {\par}
% For Pandoc 2.11+
\newenvironment{CSLReferences}[2] % #1 hanging-ident, #2 entry spacing
 {% don't indent paragraphs
  \setlength{\parindent}{0pt}
  % turn on hanging indent if param 1 is 1
  \ifodd #1 \everypar{\setlength{\hangindent}{\cslhangindent}}\ignorespaces\fi
  % set entry spacing
  \ifnum #2 > 0
  \setlength{\parskip}{#2\baselineskip}
  \fi
 }%
 {}
\usepackage{calc}  % for \widthof, \maxof in minipage
\newcommand{\CSLBlock}[1]{#1\hfill\break}
\newcommand{\CSLLeftMargin}[1]{\parbox[t]{\csllabelwidth}{#1}}
\newcommand{\CSLRightInline}[1]{\parbox[t]{\linewidth - \csllabelwidth}{#1}\break}
\newcommand{\CSLIndent}[1]{\hspace{\cslhangindent}#1}


\providecommand{\tightlist}{%
  \setlength{\itemsep}{0pt}\setlength{\parskip}{0pt}}


\date{}
\newcommand{\trTitle}{}
\newcommand{\trYear}{2023}
\newcommand{\trMonth}{April}
\newcommand{\trAuthsLong}{truetruetrue}
\newcommand{\trAuthsBack}{Monk, M.H., C.R. Wetzel, J. Coates}
\newcommand{\trCitation}{
\begin{hangparas}{1em}{1}
\trAuthsBack{}. \trYear{}. \trTitle{}. \glsentrylong{pfmc}, Portland, Oregon. \pageref{LastPage}{}\,p.
\end{hangparas}}

\newcommand\includegraphicsifexists[2][width=\linewidth]{\IfFileExists{#2}{\includegraphics[#1]{#2}}{}}

\begin{document}

%%%%% Frontmatter %%%%%

% Footnote symbols in front matter
\renewcommand*{\thefootnote}{\fnsymbol{footnote}}

\small
\thispagestyle{empty}
\pagenumbering{roman}
\noindent
\begin{center}
\title{}
% \textnormal{\MakeTextUppercase{\trTitle{}}}
\vspace{1.5cm}
{\Large\textbf\newline{}}

\includegraphicsifexists[width=4in]{figure_title.png}
\vfill
by\\
Melissa H. Monk\textsuperscript{1}\\
Chantel R. Wetzel\textsuperscript{2}\\
Julia Coates\textsuperscript{3}\vfill
\textsuperscript{1}Southwest Fisheries Science Center, U.S. Department of Commerce, National Oceanic and Atmospheric Administration, National Marine Fisheries Service, 110 McAllister Way, Santa Cruz, California 95060\\
\textsuperscript{2}Northwest Fisheries Science Center, U.S. Department of Commerce, National Oceanic and Atmospheric Administration, National Marine Fisheries Service, 2725 Montlake Boulevard East, Seattle, Washington 98112\\
\textsuperscript{3}.na.character\vfill
\trMonth{} \trYear{}
\end{center}
\clearpage

% Fourth page: Colophon
\thispagestyle{empty}
\vspace*{\fill}
\begin{center}
\copyright{} \glsentrylong{pfmc}, \trYear{}\\
\end{center}
\par
\bigskip
\noindent
Correct citation for this publication:
\bigskip
\par
\trCitation{}
\clearpage

% Add TOC to pdf bookmarks (clickable pdf)
\pdfbookmark[1]{\contentsname}{toc}

% Table of contents page, lists of figures and tables
\tableofcontents\clearpage
\label{TRlastRoman}
\clearpage

% Table of contents
\newpage
\thispagestyle{empty} % to remove page number

% Settings for the main document
\pagenumbering{arabic}  % Regular page numbers
\pagestyle{plain}  % No page number on first page of main document, use 'empty'
\renewcommand*{\thefootnote}{\arabic{footnote}}  % Back to numeric footnotes
\setcounter{footnote}{0}  % And start at 1
\renewcommand{\headrulewidth}{0.5pt}
\renewcommand{\footrulewidth}{0.5pt}
%\pagestyle{fancy}\fancyhead[c]{Draft: Do not cite or circulate}

\newcommand{\lt}{\ensuremath <}
\newcommand{\gt}{\ensuremath >}

\hypertarget{executive-summary}{%
\section*{Executive summary}\label{executive-summary}}
\addcontentsline{toc}{section}{Executive summary}

\hypertarget{stock}{%
\subsection*{Stock}\label{stock}}
\addcontentsline{toc}{subsection}{Stock}

This assessment reports the status of copper rockfish (\emph{Sebastes caurinus}) off the California coast in U.S. waters, using data through 2022. The copper rockfish stock was assessed using two sub-area models that captured distinct inter-stock dynamics split north and south of Point Conception. The estimated dynamics for each assessed sub-area is described here along with the combined stock status for the California stock. This assessment does not account for populations located in Mexico waters or other areas off the U.S. coast and assumes that these southern and northern populations do not contribute to the population being assessed here.

\hypertarget{catches}{%
\subsection*{Catches}\label{catches}}
\addcontentsline{toc}{subsection}{Catches}

Replace text with trends and current levels. Include Table for last 10 years. Include Figure with long-term estimates.

\begingroup\fontsize{10}{12}\selectfont
\begingroup\fontsize{10}{12}\selectfont

\begin{longtable}[t]{r>{\centering\arraybackslash}p{1.83cm}>{\centering\arraybackslash}p{1.83cm}>{\centering\arraybackslash}p{1.83cm}>{\centering\arraybackslash}p{1.83cm}>{\centering\arraybackslash}p{1.83cm}}
\caption{\label{tab:south-removalsES}Recent catch by fleet and total catch summed across fleets for the sub-area model south of Point Conception.}\\
\toprule
Year & Commercial dead & Commercial live & Rec CPFV & Rec Private & Total Catch\\
\midrule
\endfirsthead
\caption[]{Recent catch by fleet and total catch summed across fleets for the sub-area model south of Point Conception. \textit{(continued)}}\\
\toprule
Year & Commercial dead & Commercial live & Rec CPFV & Rec Private & Total Catch\\
\midrule
\endhead

\endfoot
\bottomrule
\endlastfoot
2013 & 1.26 & 2.67 & 61.65 & 13.96 & 79.54\\
2014 & 1.79 & 2.29 & 47.58 & 10.04 & 61.71\\
2015 & 2.11 & 4.09 & 67.00 & 8.97 & 82.18\\
2016 & 2.11 & 3.57 & 82.20 & 11.07 & 98.95\\
2017 & 1.74 & 2.82 & 70.58 & 11.72 & 86.86\\
2018 & 2.93 & 2.20 & 81.97 & 14.21 & 101.31\\
2019 & 2.71 & 3.08 & 60.25 & 14.66 & 80.70\\
2020 & 3.54 & 3.58 & 56.39 & 23.01 & 86.52\\
2021 & 2.74 & 1.94 & 44.25 & 8.28 & 57.20\\
2022 & 0.69 & 0.21 & 14.12 & 4.50 & 19.52\\*
\end{longtable}
\endgroup{}
\endgroup{}


\begingroup\fontsize{10}{12}\selectfont
\begingroup\fontsize{10}{12}\selectfont

\begin{table}[t]{r>{\centering\arraybackslash}p{1.83cm}>{\centering\arraybackslash}p{1.83cm}>{\centering\arraybackslash}p{1.83cm}>{\centering\arraybackslash}p{1.83cm}>{\centering\arraybackslash}p{1.83cm}}
\caption{\label{tab:north-removalsES}Recent catches (mt) by fleet and total catch (mt) summed across fleets for the sub-area model north of Point Conception.}\\
\toprule
Year & Commercial Dead & Commercial Live & Rec CPFV & Rec PR & Total Catch\\
\midrule
\endfirsthead
\caption[]{Recent catches (mt) by fleet and total catch (mt) summed across fleets for the sub-area model north of Point Conception. \textit{(continued)}}\\
\toprule
Year & Commercial Dead & Commercial Live & Rec CPFV & Rec PR & Total Catch\\
\midrule
\endhead

\endfoot
\bottomrule
\endlastfoot
2013 & 0.70 & 2.11 & 8.83 & 14.00 & 25.64\\
2014 & 0.74 & 2.47 & 16.10 & 17.63 & 36.94\\
2015 & 0.78 & 2.69 & 24.22 & 37.77 & 65.46\\
2016 & 0.83 & 2.57 & 28.69 & 34.23 & 66.32\\
2017 & 1.41 & 4.60 & 56.48 & 76.13 & 138.62\\
2018 & 3.04 & 6.36 & 43.97 & 49.01 & 102.38\\
2019 & 2.49 & 6.85 & 39.16 & 53.39 & 101.89\\
2020 & 3.90 & 7.55 & 36.55 & 55.17 & 103.17\\
2021 & 3.10 & 7.55 & 24.98 & 41.42 & 77.05\\
2022 & 1.19 & 1.92 & 11.50 & 32.53 & 47.15\\
\end{table}
\endgroup{}
\endgroup{}


\begin{figure}
\centering
\includegraphics[width=1\textwidth,height=1\textheight]{S:/copper_rockfish_2023/models/sca/5.5_est_m/plots/catch2 landings stacked.png}
\caption{Landings by fleet used in the base model for the area south of Point Conception where catches in metric tons by fleet are stacked.\label{fig:es-south-catch}}
\end{figure}

\begin{figure}
\centering
\includegraphics[width=1\textwidth,height=1\textheight]{S:/copper_rockfish_2023/models/nca/8.5_update_deb_index/plots/catch2 landings stacked.png}
\caption{Landings by fleet used in the base model for the area north of Point Conception where catches in metric tons by fleet are stacked.\label{fig:es-north-catch}}
\end{figure}

\hypertarget{data-and-assessment}{%
\subsection*{Data and assessment}\label{data-and-assessment}}
\addcontentsline{toc}{subsection}{Data and assessment}

This assessment uses the stock assessment framework Stock Synthesis version 3.30.21 (SS3).

Replace text with date of last assessment, type of assessment model, data available, new information, and information lacking.

\hypertarget{stock-biomass-and-dynamics}{%
\subsection*{Stock biomass and dynamics}\label{stock-biomass-and-dynamics}}
\addcontentsline{toc}{subsection}{Stock biomass and dynamics}

Replace text with trends and current levels relative to virgin or historic levels and description of uncertainty. Include Table for last 10 years. Include Figure with long-term estimates.

\begingroup\fontsize{10}{12}\selectfont
\begingroup\fontsize{10}{12}\selectfont

\begin{longtable}[t]{r>{\centering\arraybackslash}p{1.57cm}>{\centering\arraybackslash}p{1.57cm}>{\centering\arraybackslash}p{1.57cm}>{\centering\arraybackslash}p{1.57cm}>{\centering\arraybackslash}p{1.57cm}>{\centering\arraybackslash}p{1.57cm}}
\caption{\label{tab:south-ssbES}Estimated recent trend in spawning output and the fraction unfished and the 95 percent intervals for the sub-area model south of Point Conception.}\\
\toprule
Year & Spawning Output & Lower Interval & Upper Interval & Fraction Unfished & Lower Interval & Upper Interval\\
\midrule
\endfirsthead
\caption[]{Estimated recent trend in spawning output and the fraction unfished and the 95 percent intervals for the sub-area model south of Point Conception. \textit{(continued)}}\\
\toprule
Year & Spawning Output & Lower Interval & Upper Interval & Fraction Unfished & Lower Interval & Upper Interval\\
\midrule
\endhead

\endfoot
\bottomrule
\endlastfoot
2013 & 54.30 & 19.87 & 88.72 & 0.27 & 0.10 & 0.44\\
2014 & 55.93 & 19.85 & 92.01 & 0.28 & 0.10 & 0.46\\
2015 & 60.61 & 21.87 & 99.35 & 0.30 & 0.11 & 0.50\\
2016 & 64.75 & 22.90 & 106.59 & 0.32 & 0.11 & 0.53\\
2017 & 67.82 & 22.69 & 112.96 & 0.34 & 0.11 & 0.56\\
2018 & 71.25 & 22.73 & 119.77 & 0.36 & 0.11 & 0.60\\
2019 & 71.19 & 19.91 & 122.48 & 0.36 & 0.10 & 0.61\\
2020 & 69.46 & 16.44 & 122.48 & 0.35 & 0.08 & 0.61\\
2021 & 64.22 & 10.68 & 117.77 & 0.32 & 0.06 & 0.58\\
2022 & 60.01 & 6.78 & 113.23 & 0.30 & 0.04 & 0.56\\
2023 & 58.78 & 6.39 & 111.17 & 0.29 & 0.04 & 0.55\\*
\end{longtable}
\endgroup{}
\endgroup{}


\begingroup\fontsize{10}{12}\selectfont
\begingroup\fontsize{10}{12}\selectfont

\begin{longtable}[t]{r>{\centering\arraybackslash}p{1.57cm}>{\centering\arraybackslash}p{1.57cm}>{\centering\arraybackslash}p{1.57cm}>{\centering\arraybackslash}p{1.57cm}>{\centering\arraybackslash}p{1.57cm}>{\centering\arraybackslash}p{1.57cm}}
\caption{\label{tab:north-removalsES}Estimated recent trend in spawning output and the fraction unfished and the 95 percent intervals for the sub-area model north of Point Conception.}\\
\toprule
Year & Spawning Output & Lower Interval & Upper Interval & Fraction Unfished & Lower Interval & Upper Interval\\
\midrule
\endfirsthead
\caption[]{Estimated recent trend in spawning output and the fraction unfished and the 95 percent intervals for the sub-area model north of Point Conception. \textit{(continued)}}\\
\toprule
Year & Spawning Output & Lower Interval & Upper Interval & Fraction Unfished & Lower Interval & Upper Interval\\
\midrule
\endhead

\endfoot
\bottomrule
\endlastfoot
2013 & 173.43 & 108.00 & 238.85 & 0.36 & 0.25 & 0.46\\
2014 & 187.64 & 117.88 & 257.40 & 0.38 & 0.27 & 0.50\\
2015 & 204.06 & 128.84 & 279.29 & 0.42 & 0.30 & 0.54\\
2016 & 218.46 & 137.45 & 299.48 & 0.45 & 0.32 & 0.58\\
2017 & 231.77 & 144.97 & 318.56 & 0.47 & 0.34 & 0.61\\
2018 & 237.31 & 144.76 & 329.85 & 0.49 & 0.34 & 0.63\\
2019 & 245.64 & 147.26 & 344.02 & 0.50 & 0.34 & 0.66\\
2020 & 253.37 & 149.21 & 357.53 & 0.52 & 0.35 & 0.69\\
2021 & 254.85 & 145.11 & 364.59 & 0.52 & 0.34 & 0.70\\
2022 & 260.41 & 145.04 & 375.79 & 0.53 & 0.35 & 0.72\\
2023 & 272.27 & 150.92 & 393.61 & 0.56 & 0.36 & 0.76\\*
\end{longtable}
\endgroup{}
\endgroup{}


\begingroup\fontsize{10}{12}\selectfont
\begingroup\fontsize{10}{12}\selectfont

\begin{longtable}[t]{c>{\centering\arraybackslash}p{2cm}>{\centering\arraybackslash}p{2cm}}
\caption{\label{tab:ca-status}The estimated spawning ouput in number of million eggs across California and fraction unfished by year.}\\
\toprule
Year & Spawning Output & Fraction Unfished\\
\midrule
\endfirsthead
\caption[]{\label{tab:ca-status}The estimated spawning ouput in number of million eggs across California and fraction unfished by year. \textit{(continued)}}\\
\toprule
Year & Spawning Output & Fraction Unfished\\
\midrule
\endhead

\endfoot
\bottomrule
\endlastfoot
2013 & 227.72 & 0.331\\
2014 & 243.57 & 0.354\\
2015 & 264.67 & 0.384\\
2016 & 283.21 & 0.411\\
2017 & 299.59 & 0.435\\
2018 & 308.56 & 0.448\\
2019 & 316.83 & 0.460\\
2020 & 322.83 & 0.469\\
2021 & 319.07 & 0.463\\
2022 & 320.42 & 0.465\\
2023 & 331.05 & 0.481\\*
\end{longtable}
\endgroup{}
\endgroup{}

\begin{figure}
\centering
\includegraphics[width=1\textwidth,height=1\textheight]{C:/Users/melissa.monk/Documents/GitHub/copper_rockfish_2023/documents/shared_figures/compare2_spawnbio_uncertainty.png}
\caption{Estimated time series of spawning output (circles and line: median; light broken lines: 95 percent intervals) for the model areas south and north of Point Conception.\label{fig:es-sb}}
\end{figure}

\begin{figure}
\centering
\includegraphics[width=1\textwidth,height=1\textheight]{C:/Users/melissa.monk/Documents/GitHub/copper_rockfish_2023/documents/shared_figures/compare4_Bratio_uncertainty.png}
\caption{Estimated time series of fraction of unfished spawning output (circles and line: median; light broken lines: 95 percent intervals) for the model areas south and north of Point Conception.\label{fig:es-depl}}
\end{figure}

\clearpage

\hypertarget{recruitment}{%
\subsection*{Recruitment}\label{recruitment}}
\addcontentsline{toc}{subsection}{Recruitment}

Replace text with trends and current levels relative to virgin or historic levels and description of uncertainty. Include Table for last 10 years. Include Figure with long-term estimates.

\begingroup\fontsize{10}{12}\selectfont
\begingroup\fontsize{10}{12}\selectfont

\begin{longtable}[t]{r>{\centering\arraybackslash}p{1.57cm}>{\centering\arraybackslash}p{1.57cm}>{\centering\arraybackslash}p{1.57cm}>{\centering\arraybackslash}p{1.57cm}>{\centering\arraybackslash}p{1.57cm}>{\centering\arraybackslash}p{1.57cm}}
\caption{\label{tab:south-recrES}Estimated recent trend in recruitment and recruitment deviations and the 95 percent intervals for the sub-area model south of Point Conception.}\\
\toprule
Year & Recruitment & Lower Interval & Upper Interval & Recruitment Deviations & Lower Interval & Upper Interval\\
\midrule
\endfirsthead
\caption[]{Estimated recent trend in recruitment and recruitment deviations and the 95 percent intervals for the sub-area model south of Point Conception. \textit{(continued)}}\\
\toprule
Year & Recruitment & Lower Interval & Upper Interval & Recruitment Deviations & Lower Interval & Upper Interval\\
\midrule
\endhead

\endfoot
\bottomrule
\endlastfoot
2013 & 460.73 & 361.13 & 587.80 & 1.24 & 1.01 & 1.48\\
2014 & 122.60 & 76.54 & 196.38 & -0.09 & -0.56 & 0.37\\
2015 & 50.87 & 29.98 & 86.31 & -1.01 & -1.53 & -0.49\\
2016 & 124.48 & 81.84 & 189.32 & -0.15 & -0.53 & 0.24\\
2017 & 62.66 & 34.31 & 114.43 & -0.88 & -1.45 & -0.30\\
2018 & 45.02 & 22.72 & 89.22 & -1.25 & -1.91 & -0.59\\
2019 & 51.27 & 23.69 & 110.97 & -1.14 & -1.89 & -0.39\\
2020 & 86.40 & 34.59 & 215.82 & -0.63 & -1.55 & 0.30\\
2021 & 165.61 & 130.37 & 210.38 & 0.00 & 0.00 & 0.00\\
2022 & 160.68 & 122.78 & 210.27 & 0.00 & 0.00 & 0.00\\
2023 & 159.66 & 120.60 & 211.38 & 0.00 & 0.00 & 0.00\\*
\end{longtable}
\endgroup{}
\endgroup{}


\begingroup\fontsize{10}{12}\selectfont
\begingroup\fontsize{10}{12}\selectfont

\begin{longtable}[t]{r>{\centering\arraybackslash}p{1.57cm}>{\centering\arraybackslash}p{1.57cm}>{\centering\arraybackslash}p{1.57cm}>{\centering\arraybackslash}p{1.57cm}>{\centering\arraybackslash}p{1.57cm}>{\centering\arraybackslash}p{1.57cm}}
\caption{\label{tab:north-recrES}Estimated recent trend in recruitment and recruitment deviations and the 95 percent intervals for the sub-area model north of Point Conception.}\\
\toprule
Year & Recruitment & Lower Interval & Upper Interval & Recruitment Deviations & Lower Interval & Upper Interval\\
\midrule
\endfirsthead
\caption[]{Estimated recent trend in recruitment and recruitment deviations and the 95 percent intervals for the sub-area model north of Point Conception. \textit{(continued)}}\\
\toprule
Year & Recruitment & Lower Interval & Upper Interval & Recruitment Deviations & Lower Interval & Upper Interval\\
\midrule
\endhead

\endfoot
\bottomrule
\endlastfoot
2013 & 623.28 & 325.04 & 1195.19 & 0.33 & -0.28 & 0.94\\
2014 & 529.14 & 271.64 & 1030.75 & 0.15 & -0.48 & 0.78\\
2015 & 665.19 & 373.31 & 1185.28 & 0.36 & -0.16 & 0.88\\
2016 & 315.54 & 151.19 & 658.56 & -0.40 & -1.11 & 0.31\\
2017 & 966.99 & 559.44 & 1671.44 & 0.71 & 0.24 & 1.18\\
2018 & 665.26 & 354.10 & 1249.87 & 0.30 & -0.29 & 0.88\\
2019 & 367.35 & 171.95 & 784.79 & -0.34 & -1.09 & 0.41\\
2020 & 518.59 & 420.32 & 639.84 & 0.00 & 0.00 & 0.00\\
2021 & 519.11 & 419.65 & 642.15 & 0.00 & 0.00 & 0.00\\
2022 & 521.01 & 420.68 & 645.28 & 0.00 & 0.00 & 0.00\\
2023 & 524.85 & 424.08 & 649.55 & 0.00 & 0.00 & 0.00\\*
\end{longtable}
\endgroup{}
\endgroup{}


\begin{figure}
\centering
\includegraphics[width=1\textwidth,height=1\textheight]{C:/Users/melissa.monk/Documents/GitHub/copper_rockfish_2023/documents/shared_figures/compare10_recruits_uncertainty.png}
\caption{Estimated time series of age-0 recruits (1000s) for the model areas south and north of Point Conception with 95 percent intervals.\label{fig:es-recruits}}
\end{figure}

\begin{figure}
\centering
\includegraphics[width=1\textwidth,height=1\textheight]{C:/Users/melissa.monk/Documents/GitHub/copper_rockfish_2023/documents/shared_figures/compare12_recdevs_uncertainty.png}
\caption{Estimated time series of recruitment deviations for the model areas south and north of Point Conception.\label{fig:es-rec-devs}}
\end{figure}

\clearpage

\hypertarget{exploitation-status}{%
\subsection*{Exploitation status}\label{exploitation-status}}
\addcontentsline{toc}{subsection}{Exploitation status}

Replace text with total catch divided by exploitable biomass or SPR harvest rate. Include Table for last 10 years. Include Figure with trend in f relative to target vs.~trend in biomass relative to the target.

\begingroup\fontsize{10}{12}\selectfont
\begingroup\fontsize{10}{12}\selectfont

\begin{longtable}[t]{r>{\centering\arraybackslash}p{1.57cm}>{\centering\arraybackslash}p{1.57cm}>{\centering\arraybackslash}p{1.57cm}>{\centering\arraybackslash}p{1.57cm}>{\centering\arraybackslash}p{1.57cm}>{\centering\arraybackslash}p{1.57cm}}
\caption{\label{tab:south-exploitES}Estimated recent trend in the 1-SPR where SPR is the spawning potential ratio the exploitation rate, and the 95 percent intervals for the sub-area model south of Point Conception.}\\
\toprule
Year & 1-SPR & Lower Interval & Upper Interval & Exploitation Rate & Lower Interval & Upper Interval\\
\midrule
\endfirsthead
\caption[]{Estimated recent trend in the 1-SPR where SPR is the spawning potential ratio the exploitation rate, and the 95 percent intervals for the sub-area model south of Point Conception. \textit{(continued)}}\\
\toprule
Year & 1-SPR & Lower Interval & Upper Interval & Exploitation Rate & Lower Interval & Upper Interval\\
\midrule
\endhead

\endfoot
\bottomrule
\endlastfoot
2013 & 0.84 & 0.79 & 0.89 & 0.18 & 0.14 & 0.21\\
2014 & 0.74 & 0.67 & 0.80 & 0.13 & 0.10 & 0.16\\
2015 & 0.80 & 0.74 & 0.86 & 0.16 & 0.13 & 0.20\\
2016 & 0.84 & 0.78 & 0.90 & 0.18 & 0.14 & 0.22\\
2017 & 0.81 & 0.74 & 0.88 & 0.16 & 0.12 & 0.20\\
2018 & 0.87 & 0.81 & 0.94 & 0.20 & 0.14 & 0.26\\
2019 & 0.86 & 0.78 & 0.94 & 0.18 & 0.11 & 0.24\\
2020 & 0.87 & 0.78 & 0.96 & 0.17 & 0.10 & 0.25\\
2021 & 0.80 & 0.67 & 0.94 & 0.13 & 0.07 & 0.19\\
2022 & 0.58 & 0.39 & 0.76 & 0.06 & 0.03 & 0.10\\*
\end{longtable}
\endgroup{}
\endgroup{}


\begingroup\fontsize{10}{12}\selectfont
\begingroup\fontsize{10}{12}\selectfont

\begin{longtable}[t]{r>{\centering\arraybackslash}p{1.57cm}>{\centering\arraybackslash}p{1.57cm}>{\centering\arraybackslash}p{1.57cm}>{\centering\arraybackslash}p{1.57cm}>{\centering\arraybackslash}p{1.57cm}>{\centering\arraybackslash}p{1.57cm}}
\caption{\label{tab:north-removalsES}Estimated recent trend in the 1-SPR where SPR is the spawning potential ratio the exploitation rate, and the 95 percent intervals for the sub-area model north of Point Conception.}\\
\toprule
Year & 1-SPR & Lower Interval & Upper Interval & Exploitation Rate & Lower Interval & Upper Interval\\
\midrule
\endfirsthead
\caption[]{Estimated recent trend in the 1-SPR where SPR is the spawning potential ratio the exploitation rate, and the 95 percent intervals for the sub-area model north of Point Conception. \textit{(continued)}}\\
\toprule
Year & 1-SPR & Lower Interval & Upper Interval & Exploitation Rate & Lower Interval & Upper Interval\\
\midrule
\endhead

\endfoot
\bottomrule
\endlastfoot
2013 & 0.17 & 0.12 & 0.23 & 0.01 & 0.01 & 0.02\\
2014 & 0.22 & 0.16 & 0.28 & 0.02 & 0.01 & 0.02\\
2015 & 0.33 & 0.24 & 0.41 & 0.03 & 0.02 & 0.04\\
2016 & 0.32 & 0.24 & 0.40 & 0.03 & 0.02 & 0.04\\
2017 & 0.50 & 0.40 & 0.61 & 0.06 & 0.04 & 0.08\\
2018 & 0.41 & 0.31 & 0.51 & 0.04 & 0.03 & 0.05\\
2019 & 0.40 & 0.30 & 0.51 & 0.04 & 0.02 & 0.05\\
2020 & 0.52 & 0.40 & 0.64 & 0.06 & 0.04 & 0.08\\
2021 & 0.41 & 0.29 & 0.52 & 0.04 & 0.02 & 0.06\\
2022 & 0.22 & 0.14 & 0.30 & 0.02 & 0.01 & 0.03\\*
\end{longtable}
\endgroup{}
\endgroup{}


\begin{figure}
\centering
\includegraphics[width=1\textwidth,height=1\textheight]{C:/Users/melissa.monk/Documents/GitHub/copper_rockfish_2023/documents/shared_figures/compare6_SPRratio_uncertainty.png}
\caption{Estimated 1 - relative spawning ratio (SPR) by year for the model areas south and north of Point Conception. The management target is plotted as a red horizontal line and values above this reflect harvest in excess of the proxy harvest rate.\label{fig:es-1-spr}}
\end{figure}

\hypertarget{ecosystem-considerations}{%
\subsection*{Ecosystem considerations}\label{ecosystem-considerations}}
\addcontentsline{toc}{subsection}{Ecosystem considerations}

shared text

\hypertarget{reference-points}\), i.e., the \(B_{MSY}\) proxy and the equilibrium stock size that results from fishing at the default harvest rate, i.e., the \(F_{MSY}\) proxy. Include Table of estimated reference points for ssb, SPR, exploitation rate, and yield based on SSB proxy for MSY, SPR proxy for MSY, and estimated MSY values.

\begingroup\fontsize{10}{12}\selectfont
\begingroup\fontsize{10}{12}\selectfont

\begin{longtable}[t]{r>{\centering\arraybackslash}p{2cm}>{\centering\arraybackslash}p{2cm}>{\centering\arraybackslash}p{2cm}}
\caption{\label{tab:south-referenceES}Summary of reference points and management quantities, including estimates of the 95 percent intervals for the sub-area model south of Point Conception.}\\
\toprule
 & Estimate & Lower Interval & Upper Interval\\
\midrule
\endfirsthead
\caption[]{Summary of reference points and management quantities, including estimates of the 95 percent intervals for the sub-area model south of Point Conception. \textit{(continued)}}\\
\toprule
 & Estimate & Lower Interval & Upper Interval\\
\midrule
\endhead

\endfoot
\bottomrule
\endlastfoot
Unfished Spawning Output & 201.35 & 163.79 & 238.91\\
Unfished Age 3+ Biomass (mt) & 1991.18 & 1619.51 & 2362.85\\
Unfished Recruitment (R0) & 240.94 & 195.99 & 285.89\\
Spawning Output (2023) & 27.58 & 11.63 & 43.54\\
Fraction Unfished (2023) & 0.14 & 0.06 & 0.22\\
Reference Points Based SB40\textbackslash{}\% & NA & NA & NA\\
Proxy Spawning Output SB40\textbackslash{}\% & 80.54 & 65.51 & 95.56\\
SPR Resulting in SB40\textbackslash{}\% & 0.46 & 0.46 & 0.46\\
Exploitation Rate Resulting in SB40\textbackslash{}\% & 0.06 & 0.06 & 0.06\\
Yield with SPR Based On SB40\textbackslash{}\% (mt) & 50.12 & 40.86 & 59.38\\
Reference Points Based on SPR Proxy for MSY & NA & NA & NA\\
Proxy Spawning Output (SPR50) & 89.83 & 73.07 & 106.59\\
SPR50 & 0.50 & NA & NA\\
Exploitation Rate Corresponding to SPR50 & 0.05 & 0.05 & 0.05\\
Yield with SPR50 at SB SPR (mt) & 47.90 & 39.05 & 56.75\\
Reference Points Based on Estimated MSY Values & NA & NA & NA\\
Spawning Output at MSY (SB MSY) & 55.67 & 45.32 & 66.03\\
SPR MSY & 0.35 & 0.34 & 0.35\\
Exploitation Rate Corresponding to SPR MSY & 0.08 & 0.08 & 0.09\\
MSY (mt) & 53.05 & 43.25 & 62.86\\*
\end{longtable}
\endgroup{}
\endgroup{}


\begingroup\fontsize{10}{12}\selectfont
\begingroup\fontsize{10}{12}\selectfont

\begin{longtable}[t]{r>{\centering\arraybackslash}p{2cm}>{\centering\arraybackslash}p{2cm}>{\centering\arraybackslash}p{2cm}}
\caption{\label{tab:north-removalsES}Summary of reference points and management quantities, including estimates of the 95 percent intervals for the sub-area model north of Point Conception.}\\
\toprule
 & Estimate & Lower Interval & Upper Interval\\
\midrule
\endfirsthead
\caption[]{Summary of reference points and management quantities, including estimates of the 95 percent intervals for the sub-area model north of Point Conception. \textit{(continued)}}\\
\toprule
 & Estimate & Lower Interval & Upper Interval\\
\midrule
\endhead

\endfoot
\bottomrule
\endlastfoot
Unfished Spawning Output & 488.13 & 395.18 & 581.08\\
Unfished Age 3+ Biomass (mt) & 4732.53 & 3849.56 & 5615.50\\
Unfished Recruitment (R0) & 565.27 & 457.63 & 672.91\\
Spawning Output (2023) & 272.27 & 150.92 & 393.61\\
Fraction Unfished (2023) & 0.56 & 0.36 & 0.76\\
Reference Points Based SB40\textbackslash{}\% & NA & NA & NA\\
Proxy Spawning Output SB40\textbackslash{}\% & 195.25 & 158.07 & 232.43\\
SPR Resulting in SB40\textbackslash{}\% & 0.46 & 0.46 & 0.46\\
Exploitation Rate Resulting in SB40\textbackslash{}\% & 0.06 & 0.06 & 0.06\\
Yield with SPR Based On SB40\textbackslash{}\% (mt) & 129.88 & 105.62 & 154.15\\
Reference Points Based on SPR Proxy for MSY & NA & NA & NA\\
Proxy Spawning Output (SPR50) & 217.78 & 176.31 & 259.25\\
SPR50 & 0.50 & NA & NA\\
Exploitation Rate Corresponding to SPR50 & 0.05 & 0.05 & 0.05\\
Yield with SPR50 at SB SPR (mt) & 124.08 & 100.90 & 147.26\\
Reference Points Based on Estimated MSY Values & NA & NA & NA\\
Spawning Output at MSY (SB MSY) & 134.86 & 109.14 & 160.59\\
SPR MSY & 0.35 & 0.34 & 0.35\\
Exploitation Rate Corresponding to SPR MSY & 0.09 & 0.08 & 0.09\\
MSY (mt) & 137.57 & 111.87 & 163.28\\*
\end{longtable}
\endgroup{}
\endgroup{}


\begin{figure}
\centering
\includegraphics[width=1\textwidth,height=1\textheight]{C:/Users/melissa.monk/Documents/GitHub/copper_rockfish_2023/documents/shared_figures/compare15_phase_plot.png}
\caption{Phase plot of estimated 1-SPR versus fraction unfished for the model areas south and north of Point Conception.\label{fig:es-phase}}
\end{figure}

\begin{figure}
\centering
\includegraphics[width=1\textwidth,height=1\textheight]{S:/copper_rockfish_2023/models/sca/5.5_est_m/plots/yield2_yield_curve_with_refpoints.png}
\caption{Equilibrium yield curve for the base case model for model south of Point Conception. Values are based on the 2022 fishery selectivities and with steepness fixed at 0.72.\label{fig:south-es-yield}}
\end{figure}

\begin{figure}
\centering
\includegraphics[width=1\textwidth,height=1\textheight]{S:/copper_rockfish_2023/models/nca/8.5_update_deb_index/plots/yield2_yield_curve_with_refpoints.png}
\caption{Equilibrium yield curve for the base case model for model north of Point Conception. Values are based on the 2022 fishery selectivities and with steepness fixed at 0.72.\label{fig:north-es-yield}}
\end{figure}

\hypertarget{management-performance}{%
\subsection*{Management performance}\label{management-performance}}
\addcontentsline{toc}{subsection}{Management performance}

Include Table of most recent 10 years of catches in comparison with OFL, ABC, HG, and OY/ACL values, overfishing levels, actual catch and discard. Include OFL (encountered), OFL (retained), and OFL (dead) if different due to discard and discard mortality.

\begingroup\fontsize{10}{12}\selectfont
\begingroup\fontsize{10}{12}\selectfont

\begin{longtable}[t]{c>{\centering\arraybackslash}p{2cm}>{\centering\arraybackslash}p{2cm}>{\centering\arraybackslash}p{2cm}}
\caption{\label{tab:es-ca-management}The portion of the Overfishing Limit (OFL) and Annual Catch Limit (ACL) and estimated catch in California waters.}\\
\toprule
Year & OFL (mt) & ACL (mt) & Catch (mt)\\
\midrule
\endfirsthead
\caption[]{\label{tab:es-ca-management}The portion of the Overfishing Limit (OFL) and Annual Catch Limit (ACL) and estimated catch in California waters. \textit{(continued)}}\\
\toprule
Year & OFL (mt) & ACL (mt) & Catch (mt)\\
\midrule
\endhead

\endfoot
\bottomrule
\endlastfoot
2012 & 163.15 & 136.17 & 85.95\\
2013 & 148.00 & 123.42 & 105.18\\
2014 & 148.00 & 123.42 & 98.65\\
2015 & 303.75 & 277.32 & 147.64\\
2016 & 286.88 & 261.95 & 165.27\\
2017 & 313.70 & 286.38 & 225.48\\
2018 & 319.60 & 291.85 & 203.69\\
2019 & 325.08 & 296.83 & 182.59\\
2020 & 330.35 & 301.60 & 242.73\\
2021 & 249.85 & 206.43 & 164.20\\
2022 & 249.48 & 204.02 & 66.67\\*
\end{longtable}
\endgroup{}
\endgroup{}

\hypertarget{unresolved-problems-and-major-uncertainties}{%
\subsection*{Unresolved problems and major uncertainties}\label{unresolved-problems-and-major-uncertainties}}
\addcontentsline{toc}{subsection}{Unresolved problems and major uncertainties}

shared text

\hypertarget{decision-table-and-projections}{%
\subsection*{Decision table and projections}\label{decision-table-and-projections}}
\addcontentsline{toc}{subsection}{Decision table and projections}

Replace text with projected yields (OFL, ABC, and ACL), spawning biomass, and stock depletion levels for each year. OFL calculations should be based on the assumption that future catches equal ABCs and not OFLs.

\begingroup\fontsize{10}{12}\selectfont

\begin{landscape}\begingroup\fontsize{10}{12}\selectfont

\begin{longtable}[t]{c>{\centering\arraybackslash}p{1.38cm}>{\centering\arraybackslash}p{1.38cm}>{\centering\arraybackslash}p{1.38cm}>{\centering\arraybackslash}p{1.38cm}>{\centering\arraybackslash}p{1.38cm}>{\centering\arraybackslash}p{1.38cm}>{\centering\arraybackslash}p{1.38cm}}
\caption{\label{tab:es-ca-proj}The estimated spawning output in number of million eggs across California and fraction unfished by year.}\\
\toprule
Year & Adopted OFL (mt) & Adopted ABC (mt) & Assumed Catch (mt) & OFL (mt) & ABC (mt) & Spawning Biomass & Fraction Unfished\\
\midrule
\endfirsthead
\caption[]{\label{tab:es-ca-proj}The estimated spawning output in number of million eggs across California and fraction unfished by year. \textit{(continued)}}\\
\toprule
Year & Adopted OFL (mt) & Adopted ABC (mt) & Assumed Catch (mt) & OFL (mt) & ABC (mt) & Spawning Biomass & Fraction Unfished\\
\midrule
\endhead

\endfoot
\bottomrule
\endlastfoot
2023 & 116.4 & 91.53 & 91.5 & - & - & 331.05 & 0.481\\
2024 & 121.32 & 94.69 & 94.7 & - & - & 331.91 & 0.482\\
2025 & - & - & - & 182.58 & 169.79 & 331.76 & 0.482\\
2026 & - & - & - & 182.54 & 169.45 & 330.85 & 0.481\\
2027 & - & - & - & 182.16 & 169.11 & 329.74 & 0.479\\
2028 & - & - & - & 181.54 & 168.81 & 328.80 & 0.478\\
2029 & - & - & - & 180.82 & 168.58 & 328.09 & 0.477\\
2030 & - & - & - & 180.13 & 168.42 & 327.58 & 0.476\\
2031 & - & - & - & 179.53 & 168.35 & 327.20 & 0.475\\
2032 & - & - & - & 179.02 & 168.33 & 326.90 & 0.475\\
2033 & - & - & - & 178.61 & 168.37 & 326.66 & 0.474\\
2034 & - & - & - & 178.28 & 168.44 & 326.45 & 0.474\\*
\end{longtable}
\endgroup{}
\end{landscape}
\endgroup{}

\hypertarget{scientific-uncertainty}{%
\subsection*{Scientific uncertainty}\label{scientific-uncertainty}}
\addcontentsline{toc}{subsection}{Scientific uncertainty}

The model estimated uncertainty around the 2023 spawning output was \(\sigma\) = 0.22 and the uncertainty around the OFL was \(\sigma\) = 0.22. This is likely an underestimate of overall uncertainty because of the necessity to fix several population dynamic parameters (e.g., steepness, recruitment variance, female natural mortality) and no explicit incorporation of model structural uncertainty (although see the decision table for alternative states of nature).

\hypertarget{research-and-data-needs}{%
\subsection*{Research and data needs}\label{research-and-data-needs}}
\addcontentsline{toc}{subsection}{Research and data needs}

shared text

\hypertarget{introduction}{%
\section{Introduction}\label{introduction}}

\hypertarget{basic-information-and-life-history}{%
\subsection{Basic Information and Life History}\label{basic-information-and-life-history}}

This assessment report describes the sub-area population of copper rockfish (\emph{Sebastes caurinus}) off the California coast south of Point Conception in U.S. waters, using data through 2022. The sub-area population north of Point Conception in California waters was also evaluated and is described in a separate assessment report. The copper rockfish status for the California stock of is determined by the combined estimates of spawning output from both sub-areas and is detailed in the \protect\hyperlink{management}{management} section. This assessment does not account for populations located in Mexico waters or other areas off the U.S. coast and assumes that these southern and northern populations do not contribute to the population being assessed here.

Copper rockfish have historically been a part of both commercial and recreational fisheries throughout its range. Copper rockfish are a demersal, relatively nearshore species within the subgenus \emph{Pteropodus.} Copper rockfish range from xxx to xx at depth of xxx (M. S. Love, Yoklavich, and Thorsteinson 2002). The core range is comparatively large, from northern Baja Mexico to the Gulf of Alaska, with copper rockfish also found in Puget Sound. Copper rockfish are commonly found in waters less than 100 meters in depth inhabiting nearshore kelp forests and complex low-relief rocky habitat (M. Love 1996). Adult copper rockfish have high site fidelity and do not make long-range movements. An acoustic telemetry study displaced copper rockfish 4km from their capture location to an artificial reef and within 10 days, half of the copper rockfish returned to the original capture location (Reynolds, Powers, and Bishop 2010).

long the posterior two-thirds of the lateral line. The copper rockfish has high variation in coloration throughout its range, taking on coloration from dark brown, olive, orange-red and pink, with patches of yellow and pink (D. J. Miller and Lea 1972). In general the copper rockfish towards the northern part of the range are often darker in color than fish encountered in southern California. The distinct change in coloration resulted in copper rockfish described as two separate species, copper rockfish (\emph{S. caurinus}) and whitebelly rockfish (\emph{S. vexillaris}).

The \emph{Sebastes} genus are viviparous with internal fertilization, many exhibit dimorphic growth with females larger at size-at-age than males, and a number of species have reproductive strategies that vary with latitude. There are very few fecundity samples from copper rockfish available from available from California, although copper rockfish are assumed to produce a single brood annually. In southern California, peak parturition occurs xxxx central California peak parturition occurs in February and March.

The pelagic larvae are encountered in the CalCOFI surveys, but neither larval nor young-of-the-year (YOY) can be identified copper rockfish visually (Thompson et al. 2017). The size at birth ranges from 5-6 mm and the larvae remain pelagic until approximately 22-23 mm standard length at which time they recruit to the kelp forest canopy (Anderson 1983).

Juvenile Copper rockfish are indistinguishable from kelp (\emph{S. atrovirens}), black-and-yellow (\emph{S. chrysomelas}), and gopher rockfish (\emph{S. carnatus}), all of which recruit to the kelp forest canopy in the spring months. Copper rockfish is the first of the species group to recruit to the kelp forest from April to May and can be distinguished from the other species once it reaches a size around 50 mm standard length (Anderson 1983). Baetscher genetically identified YOY rockfish from surveys in Carmel and Monterey Bays in California and provided the authors with the length and genotyped species idenifications from her study (Baetscher et al. 2019). The average length of copper rockfish in July was 3-4 cm total length \ref{fig:copper-smurf-length}. Anderson observed benthic copper rockfish nocturnally active over sandy bottom outside the kelp forest (Anderson 1983).

Copper rockfish are a relatively long-lived rockfish, estimated to live at least 50 years (M. Love 1996). Copper rockfish was determined to have the highest vulnerability (V = 2.27) of any West Coast groundfish stock evaluated in a productivity susceptibility analysis (Cope et al. 2011). This analysis calculated species-specific vulnerability scores based on two dimensions: productivity characterized by the life history and susceptibility that characterized how the stock could be impacted by fisheries and other activities.

As adults, there is little evidence of movement, with Hanan and CCFRP citations

Copper rockfish are opportunistic carnivores and commonly consume crustaceans, mollusks, and fish whole (Lea, McAllister, and VenTresca 1999; Bizzarro, Yoklavich, and Wakefield 2017). -Prince (1972) observed a shift in a diet dominated by arthropods in age 0 and 1 fish, and a shift to a more diverse diet including molluscs and fish as they aged. the study also noted that juvenile copper rockfish were predated on by harbor seals and lingcod.

There is currently no evidence of significant stock structure from genetic studies of copper rockfish across the west coast. -Buonaccorsi et al. (2002) looked at genetic variation across six micosatellite DNA loci from samples ranging from British Columbia to southern California. Significant population subdivision was detected between th Puget Sound and coastal samples and support the model of isolation-by-distance for copper rockfish. Sivasundar and Palumbi (2010) conducted a genetic study to determine the potential for biogeographic boundaries to prohibit gene flow for 15 \emph{Sebastes} species. The study's sample sizes of copper rockfish with samples form Oregon, Monterey Bay and Santa Barbara. Sivasundar and Palumbi (2010) used mtDNA and could differentiate samples from Santa Barbara from those collected in Oregon and Monterey Bay, but the Monterey Bay and Oregon samples could not be distinguished. Micosatellite data did not reveal any genetic differentiation among the sampels from the three locations for copper rockfish and suggests low genetic differentiation coastwide.

The most recent genetic analysis of copper rockfish to date was conducted by Johansson et al. (2008). The study included 749 samples from along the west coast ranging from Neah Bay, Washington to San Diego, California with the majority of sampling locations clustered north of Cape Mendocino in northern California. The study included 185 samples collected within California. Eleven microsatellite DNA loci were analyzed. The study found significant evidence to support isolation by distance at the coast wide scale. Weak, but significant, genetic structure was identified from samples collected along the Oregon coast suggesting that habitat barriers may limit larval dispersal.

\hypertarget{ecosystem-considerations-1}{%
\subsection{Ecosystem Considerations}\label{ecosystem-considerations-1}}

This stock assessment does not explicitly incorporate trophic interactions, habitat factors (other than as they inform relative abundance indices) or environmental factors into the assessment model, but a brief description of likely or potential ecosystem considerations are provided below.

As with most other rockfish and groundfish in the California Current, recruitment, or cohort (year-class) strength appears to be highly variable for the copper rockfish complex, with only a modest apparent relationship to estimated levels of spawning output. Oceanographic and ecosystem factors are widely recognized to be key drivers of recruitment variability for most species of groundfish, as well as most elements of California Current food webs. Empirical estimates of recruitment from pelagic juvenile rockfish surveys have been used to inform incoming year class strength for some of these stocks, however copper rockfish are infrequently encountered in these surveys. Between 1998 and 2013 the California Cooperative Oceanic Fisheries Investigation (CalCOFI) survey observed had 34 positive observations copper rockfish out of nearly 300,000 total juvenile \emph{Sebastes} encountered in juvenile surveys.

\hypertarget{historical-and-current-fishery-information}{%
\subsection{Historical and Current Fishery Information}\label{historical-and-current-fishery-information}}

Off the coast of California south of Point Conception copper rockfish is caught in both commercial and recreational fisheries. Recreational removals have been the largest source of fishing mortality of copper rockfish across all years (Table \ref{tab:allcatches} and Figure \ref{fig:catch}). The recreational fishery is comprised of individual recreational fishers (Private/Rental, PR) and charter recreational private vessels (CPFV) which take groups of individuals out for day fishing trips. Across both types of recreational fishing the majority of effort occurs around rocky reefs that can be accessed via a day-trips.

The recreational fishery in the early part of the 20th century was focused on nearshore waters near ports, with expanded activity further from port and into deeper depths over time (R. R. Miller et al. 2014). Prior to the groundfish fishery being declared a federal disaster in 2000, and the subsequent rebuilding period, there were no time or area closures for groundfish. Access to deeper depths during this period spread effort over a larger area and filled bag limits with a greater diversity of species from both the shelf and nearshore. This resulted in lower catch of nearshore rockfish relative to the period after 2000 when 20 to 60 fm depth restrictions ranging from 20 fm in the Northern Management Area to 60 fm in the Southern Management Area were put in place in various management area delineations along the state. This shifting effort onto the nearshore, concomitantly increased catch rates for nearshore rockfish including copper rockfish in the remaining open depths, though season lengths were greatly curtailed.

Following all previously overfished groundfish species, other than yelloweye rockfish, being declared rebuilt by 2019, deeper depth restrictions were offered in the Southern Management area allowing resumed access to shelf rockfish in less than 75 fm and are currently 100 fm as of 2021. The increased access to deeper depths south of Point Conception with the rebuilding of cowcod is expected to reduce the effort in nearshore waters where copper rockfish is most prevalent. To the north of Point Conception where yelloweye rockfish are prevalent, depth constraints persist and effort remains focused on the nearshore in 30 to 50 fm depending on the management area. As yelloweye rockfish continues to rebuild, incremental increases in access to deeper depths are expected, which will likely further reduce the effort in nearshore waters where copper rockfish is most prevalent.

Prior to development of the live fish market in the 1980s, there was very little commercial catch of copper rockfish, with dead copper rockfish fetching a low ex-vessel price per pound. Copper rockfish were targeted along with other rockfish to some degree in the nearshore or caught as incidental catch by vessels targeting other more valuable stocks such as lingcod. Most fish were caught using hook and line gear, though some were caught using traps, gill nets and, rarely, trawl gear. Trawling was prohibited within three miles of shore in 1953 and gill netting within three miles of shore was prohibited in 1994, preventing access to a high proportion of the species habitat with these gear types. Copper rockfish were caught along with other rockfish to some degree in the nearshore or caught as bycatch by vessels targeting other more valuable stocks such as lingcod.

In the late 1980s and early 1990s a market for fish landed live arose out of Los Angeles and the Bay area, driven by demand from Asian restaurants and markets. The growth of the live fish market was driven by consumers willing to pay a higher price for live fish, ideally plate-sized (12 - 14 inches or 30.5 - 35.6 cm). Live fish landed for the restaurant market are lumped into two categories, small (1 - 3 lbs.) or large (3 - 6 lbs.), with small, plate-sized, fish fetching higher prices at market ranging between \$5 -7 per fish (Bill James, personal communication). Copper rockfish is one of the many rockfish species that is included in the commercial live fish fishery. The proportion of copper rockfish being landed live vs.~dead since 2000 by California commercial fleets ranges between 50 to greater than 70 percent in the southern and northern areas, respectively.

With the development and expansion of the nearshore live fish fishery during the 1980s and 1990s, new entrants in this open access fishery were drawn by premium ex-vessel price per pound for live fish, resulting in over-capitalization of the fishery. Since 2002, the California Department of Fish and Wildlife (CDFW) has managed 19 nearshore species in accordance with Nearshore Fisheries Management Plan (Wilson-Vandenberg, Larinto, and Key 2014). In 2003, the CDFW implemented a Nearshore Restricted Access Permit system, including the requirement of a Deeper Nearshore Fishery Species Permit to retain copper rockfish, with the overall goal of reducing the number of participants to a more sustainable level, with permit issuance based on historical landings history by the retrospective qualifying date. The result was a reduction in permits issued from 1,127 in 1999 to 505 in 2003, greatly reducing catch levels. In addition, reduced trip limits, season closures in March and April and depth restrictions were implemented to address bycatch of overfished species and associated constraints from their low catch limits.

Copper rockfish residing between Point Conception and the California/Oregon border are assessed here as a single, separate stock (Figure \ref{fig:ca-map}). This designation was made based on oceanographic, geographic, and fishery conditions. The copper rockfish population in California waters was split at Point Conception due to water circulation patterns that create a natural barrier between nearshore rockfish populations to the north and south. The northern border for this assessment was defined as the California/Oregon border due to substantial differences in historical and current exploitation levels. Additionally, the fairly sedentary nature of adult copper rockfish, likely limits flow of fish between northern California and areas to the north.

\hypertarget{summary-of-management-history-and-performance}{%
\subsection{Summary of Management History and Performance}\label{summary-of-management-history-and-performance}}

Replace text.

\hypertarget{foreign-fisheries}{%
\subsection{Foreign Fisheries}\label{foreign-fisheries}}

Replace text.

\hypertarget{ccfrp-index}{%
\section{Appendix F. CCFRP Index of Abundance}\label{ccfrp-index}}

\textbf{California Collaborative Fisheries Research Program Index}

The California Collaborative Fisheries Research Program, \href{https://www.mlml.calstate.edu/ccfrp/}{CCFRP}, is a fishery-independent hook-and-line survey designed to monitor nearshore fish populations at a series of sampling locations both inside and adjacent to MPAs (Starr et al. 2015; Wendt and Starr 2009). The CCFRP survey began in 2007 along the central coast of California and was designed in collaboration with academics, NMFS scientists and fishermen. From 2007-2016 the CCFRP project was focused on the central California coast, and has monitored four MPAs consistently. In 2017, the CCFRP expanded coastwide within California.\\
The index of abundance was developed from the four MPAs sampled consistently (Año Nuevo and Point Lobos by Moss Landing Marine Labs; Point Buchon and Piedras Blancas by Cal Poly).

The survey design for CCFRP consists 500 x 500 m cells both within and adjacent to each MPA. On any given survey day site cells are randomly selected within a stratum (MPA and/or reference cells). CPFVs are chartered for the survey and the fishing captain is allowed to search within the cell for a fishing location. During a sampling event, each cell is fished for a total of 30-45 minutes by volunteer anglers. Each fish encountered is recorded, measured, and can be linked back to a particular angler, and released (or descended to depth). CCFRP samples shallower depths to avoid barotrauma-induced mortality.\\
Starting in 2017, a subset of fish have been retained to collect otoliths and fin clips that provide needed biological information for nearshore species. For the index of abundance, CPUE was modeled at the level of the drift, similar to the fishery-dependent onboard observer survey described above.

\emph{CCFRP Index: Data Preparation, Filtering, and Sample Sizes}

The CCFRP data are quality controlled at the time they are key punched and little filtering was needed for the index. Cells not consistently sampled over time were excluded as well as cells that never encountered copper rockfish. The full dataset for northern California contained 8,770 drifts, 23\% of which encountered copper rockfish. After applying filters to remove drfits from sites that were not consistently sampled, marked for exclusion in the data, or did not fish a minimum of xxx, 7,078 drifts remained for for index standardization, with 1,757 drifts encountering copper rockfish.

\emph{CCFRP Index: Model Selection, Fits, and Diagnostics}

The CCFRP index includes all of the MPAs currently sampled from 2017-2020 and the core central California sampling sites from 2007-2016. Trends among all of the MPAs sampled increased along the entire coast from 2017-2020. The final index (Table \ref{tab:tab-index-ccfrp}) represents a similar trend to the arithmetic mean of the annual CPUE (Figure \ref{fig:fig-cpue-ccfrp}).

We modeled retained catch per angler hour (CPUE; number of fish per angler hour) using MLE fr. Indices with a year and area (location along the coast) interaction were not considered in model selection; trends in the average CPUE by region were similar in the filtered data set (Figure \ref{fig:fig-areacpue-ccfrp}). Plots of the arithmetic mean by inside (MPA) and outside (REF) MPAs over time is in Figure \ref{fig:fig-sitecpue-ccfrp} and shows the distinct trends of increasing average CPUE over time.

A negative binomial model was fit to the drift-level data (catch with a log offset for angler hours). Because the average observed CPUE inside MPAs and in the reference sites exhibited differing trends, we explored a YEAR:SITE interaction, which was selected as the best fit model by AIC Table \ref{tab:tab-model-select-ccfrp}), The final model included yrea, mpa/reference categorization, depth, depth squared, and a year:mpa/reference interaction. The model was fit using the sdmTMB R package (version xxx1).

Based on work completed at the SWFSC, we estimate that the percent of rocky reef habitat from Point Conception to the California border within California state waters is 892 \(km^2\), of which approximately 23\% is in MPAs that prohibit the harvest of groundfish (pers comm. Rebecca Miller, UCSC). There is recreational fishing outside of state waters, but habitat maps are not available at the same 2-m resolution and do not allow for direct comparisons. To estimate the area of rocky substrate south of Point conception, we separted the southern California Bight into four areas, 1) CRFS District 1 along the mainland coast, 2) CRFS District 2 along the mainland coast, 3) state waters encompassing the southern Channel Islands, and 4) state waters encompassing the northern Channel Islands. We calculated the total area in each of the four regions, as well as the total area with available interpretted substrate. By also calculating the total area open and closed to fishing, i.e., MPAs and CCAs, we expanded the known fraction of rocky substrate to the areas within state waters where no substrated interpretted maps exist. This resulted in an estimate of 27\% of the available rocky substrate within closed areas to fishing in southern California state waters.

The final index was weighted, giving 20\% of the model weight to MPAs and 80\% of model weight to the ``open'' areas within the state.

\begingroup\fontsize{7}{9}\selectfont

\begin{landscape}\begingroup\fontsize{7}{9}\selectfont

\begin{longtable}[t]{l>{\raggedright\arraybackslash}p{2cm}>{\raggedright\arraybackslash}p{2cm}>{\raggedright\arraybackslash}p{2cm}}
\caption{\label{tab:ccfrp-data-filter}Data filtering for the CCFRP survey.}\\
\toprule
Filter & Description & Samples & Positive\_Samples\\
\midrule
\endfirsthead
\caption[]{\label{tab:ccfrp-data-filter}Data filtering for the CCFRP survey. \textit{(continued)}}\\
\toprule
Filter & Description & Samples & Positive\_Samples\\
\midrule
\endhead

\endfoot
\bottomrule
\endlastfoot
All data &  & 8770 & 1979\\
Sampling frequency & Remove locations and cells not well 
                                          sampled and drifts marked for exclusion & 7850 & 1773\\
Location & Remove grid cells that never observed
                                           the target species & 7205 & 1773\\
Time fished & Remove drifts less than two minutes 
                                          and cells fished less than 15 minutes
                                          during a sampling event & 7078 & 1757\\*
\end{longtable}
\endgroup{}
\end{landscape}
\endgroup{}

\newpage

\begingroup\fontsize{7}{9}\selectfont

\begin{landscape}\begingroup\fontsize{7}{9}\selectfont

\begin{longtable}[t]{l>{\raggedright\arraybackslash}p{1cm}>{\raggedright\arraybackslash}p{1cm}>{\raggedright\arraybackslash}p{1cm}>{\raggedright\arraybackslash}p{1cm}>{\raggedright\arraybackslash}p{1cm}>{\raggedright\arraybackslash}p{1cm}>{\raggedright\arraybackslash}p{1cm}>{\raggedright\arraybackslash}p{1cm}>{\raggedright\arraybackslash}p{1cm}>{\raggedright\arraybackslash}p{1cm}}
\caption{\label{tab:ccfrp-model-selection}Model selection for the CCFRP survey.}\\
\toprule
Depth & Depth.Squared & Mpaorref & Region & Year & Interaction & Effort.Offset & Df & Log.Likelihood & AICc & Delta\\
\midrule
\endfirsthead
\caption[]{\label{tab:ccfrp-model-selection}Model selection for the CCFRP survey. \textit{(continued)}}\\
\toprule
Depth & Depth.Squared & Mpaorref & Region & Year & Interaction & Effort.Offset & Df & Log.Likelihood & AICc & Delta\\
\midrule
\endhead

\endfoot
\bottomrule
\endlastfoot
0.402 & -0.008 & Included & Included & Included & Included & Included & 36 & -5319.3 & 10710.9 & 0.0\\
0.393 & -0.008 & Included & Excluded & Included & Included & Included & 35 & -5321.0 & 10712.3 & 1.4\\
0.406 & -0.008 & Included & Included & Included & Excluded & Included & 21 & -5351.1 & 10744.4 & 33.5\\
0.397 & -0.008 & Included & Excluded & Included & Excluded & Included & 20 & -5353.0 & 10746.1 & 35.2\\
0.145 & Excluded & Included & Excluded & Included & Included & Included & 34 & -5350.2 & 10768.8 & 57.9\\
0.144 & Excluded & Included & Included & Included & Included & Included & 35 & -5350.1 & 10770.5 & 59.6\\
0.143 & Excluded & Included & Excluded & Included & Excluded & Included & 19 & -5383.4 & 10804.9 & 94.0\\
0.143 & Excluded & Included & Included & Included & Excluded & Included & 20 & -5383.2 & 10806.5 & 95.6\\
0.464 & -0.010 & Excluded & Included & Included & Excluded & Included & 20 & -5508.1 & 11056.3 & 345.4\\
0.454 & -0.010 & Excluded & Excluded & Included & Excluded & Included & 19 & -5510.5 & 11059.2 & 348.3\\
0.144 & Excluded & Excluded & Excluded & Included & Excluded & Included & 18 & -5554.0 & 11144.1 & 433.2\\
0.144 & Excluded & Excluded & Included & Included & Excluded & Included & 19 & -5553.8 & 11145.6 & 434.7\\
Excluded & Excluded & Included & Excluded & Included & Included & Included & 33 & -5632.6 & 11331.5 & 620.6\\
Excluded & Excluded & Included & Included & Included & Included & Included & 34 & -5632.2 & 11332.7 & 621.8\\
Excluded & Excluded & Included & Excluded & Included & Excluded & Included & 18 & -5661.2 & 11358.4 & 647.5\\
Excluded & Excluded & Included & Included & Included & Excluded & Included & 19 & -5660.7 & 11359.5 & 648.6\\
Excluded & Excluded & Excluded & Excluded & Included & Excluded & Included & 17 & -5815.9 & 11665.8 & 954.9\\
Excluded & Excluded & Excluded & Included & Included & Excluded & Included & 18 & -5815.3 & 11666.8 & 955.9\\*
\end{longtable}
\endgroup{}
\end{landscape}
\endgroup{}

\newpage

\begingroup\fontsize{10}{12}\selectfont
\begingroup\fontsize{10}{12}\selectfont

\begin{longtable}[t]{c>{\centering\arraybackslash}p{2cm}>{\centering\arraybackslash}p{2cm}}
\caption{\label{tab:ccfrp-index}Estimated relative index of abundance for the CCFRP survey.}\\
\toprule
Year & Estimate & logSE\\
\midrule
\endfirsthead
\caption[]{\label{tab:ccfrp-index}Estimated relative index of abundance for the CCFRP survey. \textit{(continued)}}\\
\toprule
Year & Estimate & logSE\\
\midrule
\endhead

\endfoot
\bottomrule
\endlastfoot
2007 & 0.0582160 & 0.1394863\\
2008 & 0.0275242 & 0.1493542\\
2009 & 0.0599728 & 0.1562757\\
2010 & 0.0329613 & 0.1665564\\
2011 & 0.0302584 & 0.1638784\\
2012 & 0.0359084 & 0.1446754\\
2013 & 0.0237656 & 0.1726645\\
2014 & 0.0495890 & 0.1397864\\
2015 & 0.0371527 & 0.2124289\\
2016 & 0.0962345 & 0.1096466\\
2017 & 0.0920281 & 0.1075274\\
2018 & 0.1107285 & 0.0950086\\
2019 & 0.1284849 & 0.0884973\\
2020 & 0.1693210 & 0.0947559\\
2021 & 0.1546231 & 0.0894429\\
2022 & 0.1363272 & 0.0914945\\*
\end{longtable}
\endgroup{}
\endgroup{}

\newpage

\begin{figure}
\centering
\includegraphics[width=1\textwidth,height=1\textheight]{S:/copper_rockfish_2023/data/survey_indices/ccfrp/north/area_weighted/qq.png}
\caption{QQ-plot for the CCFRP survey.\label{fig:ccfrp-qq}}
\end{figure}

\newpage

\begin{figure}
\centering
\includegraphics[width=1\textwidth,height=1\textheight]{S:/copper_rockfish_2023/data/survey_indices/ccfrp/north/mpa_site_cpue.png}
\caption{Average CPUE by site with trends prior to standardization in the MPA and REF areas.\label{fig:ccfrp-avg-cpue}}
\end{figure}

\newpage

\begin{figure}
\centering
\includegraphics[width=1\textwidth,height=1\textheight]{S:/copper_rockfish_2023/data/survey_indices/ccfrp/north/area_weighted/Index.png}
\caption{The weighted relative index of abundance.\label{fig:ccfrp-index}}
\end{figure}

\hypertarget{refs}{}
\begin{CSLReferences}{1}{0}
\leavevmode\vadjust pre{\hypertarget{ref-anderson_identification_1983}{}}%
Anderson, Todd Wilson. 1983. {``Identification and Development of Nearshore Juvenile Rockfishes (Genus Genus{\textbackslash{}}emph\{{Sebastes}\}) in Central {California} Kelp Forests.''} PhD thesis, California State University, Fresno.

\leavevmode\vadjust pre{\hypertarget{ref-baetscher_dispersal_2019}{}}%
Baetscher, Diana S., Eric C. Anderson, Elizabeth A. Gilbert Horvath, Daniel P. Malone, Emily T. Saarman, Mark H. Carr, and John Carlos Garza. 2019. {``Dispersal of a Nearshore Marine Fish Connects Marine Reserves and Adjacent Fished Areas Along an Open Coast.''} \emph{Molecular Ecology} 28: 1611--23. \url{https://doi.org/10.1111/mec.15044}.

\leavevmode\vadjust pre{\hypertarget{ref-bizzarro_diet_2017}{}}%
Bizzarro, Joseph J., Mary M. Yoklavich, and W. Waldo Wakefield. 2017. {``Diet Composition and Foraging Ecology of {U}.{S}. {Pacific} {Coast} Groundfishes with Applications for Fisheries Management.''} \emph{Environmental Biology of Fishes} 100 (4): 375--93. \url{https://doi.org/10.1007/s10641-016-0529-2}.

\leavevmode\vadjust pre{\hypertarget{ref-buonaccorsi_population_2002}{}}%
Buonaccorsi, Vincent P, Carol A Kimbrell, Eric A Lynn, and Russell D Vetter. 2002. {``Population Structure of Copper Rockfish (\emph{{Sebastes} Caurinus}) Reflects Postglacial Colonization and Contemporary Patterns of Larval Dispersal.''} \emph{Canadian Journal of Fisheries and Aquatic Sciences} 59 (8): 1374--84. \url{https://doi.org/10.1139/f02-101}.

\leavevmode\vadjust pre{\hypertarget{ref-cope_approach_2011}{}}%
Cope, Jason M., John DeVore, E. J. Dick, Kelly Ames, John Budrick, Daniel L. Erickson, Joanna Grebel, et al. 2011. {``An {Approach} to {Defining} {Stock} {Complexes} for {U}.{S}. {West} {Coast} {Groundfishes} {Using} {Vulnerabilities} and {Ecological} {Distributions}.''} \emph{North American Journal of Fisheries Management} 31 (4): 589--604. \url{https://doi.org/10.1080/02755947.2011.591264}.

\leavevmode\vadjust pre{\hypertarget{ref-johansson_influence_2008}{}}%
Johansson, M. L., M. A. Banks, K. D. Glunt, H. M. Hassel-Finnegan, and V. P. Buonaccorsi. 2008. {``Influence of Habitat Discontinuity, Geographical Distance, and Oceanography on Fine-Scale Population Genetic Structure of Copper Rockfish ( \emph{{Sebastes} Caurinus} ).''} \emph{Molecular Ecology} 17 (13): 3051--61. \url{https://doi.org/10.1111/j.1365-294X.2008.03814.x}.

\leavevmode\vadjust pre{\hypertarget{ref-lea_biological_1999}{}}%
Lea, Robert N, Robert D McAllister, and David A VenTresca. 1999. {``Biological Sspects of Nearshore Rockfishes of the Genus Sebastes from {Central} {California} with Notes on Ecologically Related Sport Fishes.''} Fish Bulletin 177. State of California The Resources Agency Department of Fish; Game.

\leavevmode\vadjust pre{\hypertarget{ref-love_probably_1996}{}}%
Love, Milton. 1996. \emph{Probably More Than You Want to Know about the Fishes of the {Pacific} {Coast}}. Santa Barbara, California: Really Big Press.

\leavevmode\vadjust pre{\hypertarget{ref-love_rockfishes_2002}{}}%
Love, Milton S., Mary M. Yoklavich, and L. Thorsteinson. 2002. \emph{Rockfishes of the {Northeast} {Pacific}}. Berkeley, CA: University of California Press.

\leavevmode\vadjust pre{\hypertarget{ref-miller_guide_1972}{}}%
Miller, Daniel J, and Robert N Lea. 1972. {``Guide to Coastal {Marine} {Fishes} of {California}.''} Fish Bulletin 157. State of California Department of Fish; Game Bureau of Marine Fisheries.

\leavevmode\vadjust pre{\hypertarget{ref-miller_spatially_2014}{}}%
Miller, Rebecca R., John C. Field, Jarrod A. Santora, Isaac D. Schroeder, David D. Huff, Meisha Key, Don E. Pearson, and Alec D. MacCall. 2014. {``A {Spatially} {Distinct} {History} of the {Development} of {California} {Groundfish} {Fisheries}.''} Edited by David Hyrenbach. \emph{PLoS ONE} 9 (6): e99758. \url{https://doi.org/10.1371/journal.pone.0099758}.

\leavevmode\vadjust pre{\hypertarget{ref-prince_food_1972}{}}%
Prince, Eric D. 1972. {``The Food and Behavior of the Copper Rockfish, {Sebastes} Caurinus {Richardson}, Associated with an Artificial Reef in {South} {Humboldt} {Bay}, {California}.''} \{PhD\} \{Thesis\}, California State University.

\leavevmode\vadjust pre{\hypertarget{ref-reynolds_application_2010}{}}%
Reynolds, Brad F., Sean P. Powers, and Mary Anne Bishop. 2010. {``Application of {Acoustic} {Telemetry} to {Assess} {Residency} and {Movements} of {Rockfish} and {Lingcod} at {Created} and {Natural} {Habitats} in {Prince} {William} {Sound}.''} Edited by Daniel J. Rankin. \emph{PLoS ONE} 5 (8): e12130. \url{https://doi.org/10.1371/journal.pone.0012130}.

\leavevmode\vadjust pre{\hypertarget{ref-sivasundar_life_2010}{}}%
Sivasundar, Arjun, and Stephen R. Palumbi. 2010. {``Life History, Ecology and the Biogeography of Strong Genetic Breaks Among 15 Species of {Pacific} Rockfish, {Sebastes}.''} \emph{Marine Biology} 157 (7): 1433--52. \url{https://doi.org/10.1007/s00227-010-1419-3}.

\leavevmode\vadjust pre{\hypertarget{ref-starr_variation_2015a}{}}%
Starr, Richard M., Dean E. Wendt, Cheryl L. Barnes, Corina I. Marks, Dan Malone, Grant Waltz, Katherine T. Schmidt, Jennifer Chiu, Andrea L. Launer, and Nathan C. Hall. 2015. {``Variation in Responses of Fishes Across Multiple Reserves Within a Network of Marine Protected Areas in Temperate Waters.''} \emph{PLoS ONE} 10 (3): 1--24. \url{https://doi.org/10.5061/dryad.6hk4h.Funding}.

\leavevmode\vadjust pre{\hypertarget{ref-thompson_larval_2017}{}}%
Thompson, Andrew R., Dustin C. Chen, Lian W. Guo, John R. Hyde, and William Watson. 2017. {``Larval Abundances of Rockfishes That Were Historically Targeted by Fishing Increased over 16 Years in Association with a Large Marine Protected Area.''} \emph{Royal Society Open Science} 4 (9). \url{https://doi.org/10.1098/rsos.170639}.

\leavevmode\vadjust pre{\hypertarget{ref-wendt_collaborative_2009}{}}%
Wendt, Dean E., and Richard M. Starr. 2009. {``Collaborative Research: An Effective Way to Collect Data for Stock Assessments and Evaluate Marine Protected Areas in {California}.''} \emph{Marine and Coastal Fisheries} 1 (1): 315--24. \url{https://doi.org/10.1577/c08-054.1}.

\leavevmode\vadjust pre{\hypertarget{ref-wilson-vandenberg_implementing_2014}{}}%
Wilson-Vandenberg, Deb, Traci Larinto, and Meisha Key. 2014. {``Implementing {California}'s {Nearshore} {Fishery} {Management} {Plan} --- Twelve Years Later.''} \emph{California Department of Fish and Game} 100 (2): 32.

\end{CSLReferences}
\end{document}
