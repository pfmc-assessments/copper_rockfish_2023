\input{input_accessability.tex}
\documentclass[11pt,
  english,
  letterpaper,
]{article}
\usepackage{sa4ss}
\usepackage{amsmath,amssymb,array}
\usepackage{booktabs}

% From tagged-template.latex
\usepackage{lmodern}
\usepackage{ifxetex,ifluatex}
\ifnum 0\ifxetex 1\fi\ifluatex 1\fi=0 % if pdftex
  \usepackage[T1]{fontenc}
  \usepackage[utf8]{inputenc}
  \usepackage{textcomp} % provide euro and other symbols
\else % if luatex or xetex
  \usepackage{unicode-math}
  \defaultfontfeatures{Scale=MatchLowercase}
  \defaultfontfeatures[\rmfamily]{Ligatures=TeX,Scale=1}
\fi

% Use upquote if available, for straight quotes in verbatim environments
\IfFileExists{upquote.sty}{\usepackage{upquote}}{}
\IfFileExists{microtype.sty}{% use microtype if available
  \usepackage[]{microtype}
  \UseMicrotypeSet[protrusion]{basicmath} % disable protrusion for tt fonts
}{}
\makeatletter
\@ifundefined{KOMAClassName}{% if non-KOMA class
  \IfFileExists{parskip.sty}{%
    \usepackage{parskip}
  }{% else
    \setlength{\parindent}{0pt}
    \setlength{\parskip}{6pt plus 2pt minus 1pt}}
}{% if KOMA class
  \KOMAoptions{parskip=half}}
\makeatother
\usepackage{xcolor}
\IfFileExists{xurl.sty}{\usepackage{xurl}}{} % add URL line breaks if available
\hypersetup{
  pdflang={en},
  hidelinks,
  pdfcreator={LaTeX via pandoc}}
\urlstyle{same} % disable monospaced font for URLs
\usepackage{longtable}
% Correct order of tables after \paragraph or \subparagraph
\usepackage{etoolbox}
\makeatletter
\patchcmd\longtable{\par}{\if@noskipsec\mbox{}\fi\par}{}{}
\makeatother
% Allow footnotes in longtable head/foot
\IfFileExists{footnotehyper.sty}{\usepackage{footnotehyper}}{\usepackage{footnote}}
\makesavenoteenv{longtable}
\usepackage{graphicx}
\makeatletter
\def\maxwidth{\ifdim\Gin@nat@width>\linewidth\linewidth\else\Gin@nat@width\fi}
\def\maxheight{\ifdim\Gin@nat@height>\textheight\textheight\else\Gin@nat@height\fi}
\makeatother
% Scale images if necessary, so that they will not overflow the page
% margins by default, and it is still possible to overwrite the defaults
% using explicit options in \includegraphics[width, height, ...]{}
\setkeys{Gin}{width=\maxwidth,height=\maxheight,keepaspectratio}
% Set default figure placement to htbp
\makeatletter
\def\fps@figure{htbp}
\makeatother
\setlength{\emergencystretch}{3em} % prevent overfull lines
\providecommand{\tightlist}{%
  \setlength{\itemsep}{0pt}\setlength{\parskip}{0pt}}
\setcounter{secnumdepth}{5}
\usepackage{lineno}
\usepackage[inline]{showlabels}
\ifxetex
  % Load polyglossia as late as possible: uses bidi with RTL langages (e.g. Hebrew, Arabic)
  \usepackage{polyglossia}
  \setmainlanguage[]{}
\else
  \usepackage[shorthands=off,main=english]{babel}
\fi

%Define cslreferences environment, required by pandoc 2.8
%https://github.com/rstudio/rmarkdown/issues/1649


\providecommand{\tightlist}{%
  \setlength{\itemsep}{0pt}\setlength{\parskip}{0pt}}

\usepackage{lineno}
\usepackage[inline]{showlabels}
\date{}
\newcommand{\trTitle}{}
\newcommand{\trYear}{2023}
\newcommand{\trMonth}{May}
\newcommand{\trAuthsLong}{truetruetrue}
\newcommand{\trAuthsBack}{Monk, M.H., C.R. Wetzel, J. Coates}
\newcommand{\trCitation}{
\begin{hangparas}{1em}{1}
\trAuthsBack{}. \trYear{}. \trTitle{}. \glsentrylong{pfmc}, Portland, Oregon. \pageref{LastPage}{}\,p.
\end{hangparas}}

\newcommand\includegraphicsifexists[2][width=\linewidth]{\IfFileExists{#2}{\includegraphics[#1]{#2}}{}}

\begin{document}

%%%%% Frontmatter %%%%%

% Footnote symbols in front matter
\renewcommand*{\thefootnote}{\fnsymbol{footnote}}

\small
\thispagestyle{empty}
\pagenumbering{roman}
\noindent
\begin{center}
\title{}
% \textnormal{\MakeTextUppercase{\trTitle{}}}
\vspace{1.5cm}
{\Large\textbf\newline{}}

\includegraphicsifexists[width=4in]{figure_title.png}
\vfill
by\\
Melissa H. Monk\textsuperscript{1}\\
Chantel R. Wetzel\textsuperscript{2}\\
Julia Coates\textsuperscript{3}\vfill
\textsuperscript{1}Southwest Fisheries Science Center, U.S. Department of Commerce, National Oceanic and Atmospheric Administration, National Marine Fisheries Service, 110 McAllister Way, Santa Cruz, California 95060\\
\textsuperscript{2}Northwest Fisheries Science Center, U.S. Department of Commerce, National Oceanic and Atmospheric Administration, National Marine Fisheries Service, 2725 Montlake Boulevard East, Seattle, Washington 98112\\
\textsuperscript{3}California Department of Fish and Wildlife, Marine Region 1933 Cliff Drive, Suite 9, Santa Barbara, California 93109\vfill
\trMonth{} \trYear{}
\end{center}
\clearpage

% Fourth page: Colophon
\thispagestyle{empty}
\vspace*{\fill}
\begin{center}
\copyright{} \glsentrylong{pfmc}, \trYear{}\\
\end{center}
\par
\bigskip
\noindent
Correct citation for this publication:
\bigskip
\par
\trCitation{}
\clearpage

% Add TOC to pdf bookmarks (clickable pdf)
\pdfbookmark[1]{\contentsname}{toc}

% Table of contents page, lists of figures and tables
\tableofcontents\clearpage
\label{TRlastRoman}
\clearpage

% Table of contents
\newpage
\thispagestyle{empty} % to remove page number

% Settings for the main document
\pagenumbering{arabic}  % Regular page numbers
\pagestyle{plain}  % No page number on first page of main document, use 'empty'
\renewcommand*{\thefootnote}{\arabic{footnote}}  % Back to numeric footnotes
\setcounter{footnote}{0}  % And start at 1
\renewcommand{\headrulewidth}{0.5pt}
\renewcommand{\footrulewidth}{0.5pt}
%\pagestyle{fancy}\fancyhead[c]{Draft: Do not cite or circulate}

\newcommand{\lt}{\ensuremath <}
\newcommand{\gt}{\ensuremath >}

\linenumbers

\newcommand\CapeM{$40^\circ 10^\prime$ N. lat.}
\newcommand\PtC{$34^\circ 27^\prime$ N. lat.}
\newcommand\CAOR{$42^\circ 00^\prime$ N. lat.}

\hypertarget{base-model-results}{%
\subsection{Base Model Results}\label{base-model-results}}

The base model described here is only for the portion of the stock copper rockfish in California from Point Conception, $34^\circ 27^\prime$ N. lat. to the California/Oregon border, $42^\circ 00^\prime$ N. lat. Descriptions of the summed biomass and stock status for the California stock of copper rockfish are described in later sections.

The base model parameter estimates along with approximate asymptotic standard errors are shown in Table \ref{tab:north-params} and the likelihood components are shown in Table \ref{tab:likes}. Estimates of derived reference points and approximate 95 percent asymptotic confidence intervals are shown in Table \ref{tab:referenceES}. Estimates of stock size and status over time are shown in Table \ref{tab:timeseries}.

The full r4ss plotting output is available in the supplementary material on the Council's website.

\hypertarget{parameter-estimates}{%
\subsubsection{Parameter Estimates}\label{parameter-estimates}}

Estimated parameter values are provided in Table \ref{tab:params}. The log(\(R_0\)) was estimated at 6.34.

The northern California base model estimated reasonable growth parameters for \(k\) and lengths at age 2 and age 20 for males and females. The estimates differed from those estimated externally, which was not unexpected given the lack of consistent age data across fleets and years. The direct estimation of male \(L_{age=2}=12.6\) cm was reasonable compared to female \$L\_\{age=20\}=14.6. While \(k\) was estimated larger for males (0.20) than females (0.15), female \(L_{age=30}\) of 48.3 cm was larger than males at 46.4 cm. These results are consistent with other studies that have looked at sex-specific growth in copper rockfish and similar to estimates from the southern California pre-STAR base model.

Length-based selectivity curves were estimated for the fishery and survey fleets, and age-based selectivity of 1.0 starting at age 1 for the growth fleet. Model explorations included parameterizing the fleets with double normal selectivity. Selectivity of the commercial dead fleet and the CDFW ROV survey were continually estimated as asypmtotic through base model development and were simplified to two parameter logistic selectivity in the base model. Peak selectivity for the commercial dead fleet was estimated at 34 cm and 32 cm for the ROV survey. Plots of the estimated selectivities are shown in Figure \ref{fig:est-selex}.

The commercial live fishery selectivity was estimated in two blocks of time; 1916 - 2010 and 2011 - 2022. The block in selectivity was included to capture a shift from asymptotic selectivity prior to 2011 to the selection of plate-sized (approx. 2 pounds) fish preferred in the live-fish fishery (Figure \ref{fig:com-len-data}). Both recreational fleets were fit to the same three time blocks. From 1916-2001, peak selectivity was estimated around 36 cm with selectivity decreasing for larger fish; dome-shaped selectivity was estimated from 2002-2016 representing the years the fishery was restricted to shallower depths, and asymptotic selectivity starting in 2017 when the fishery gained access to deeper depths. The two estimated PR fleet selecitivities were both dome-shape with the wider peak selecitivity estimted in 2017-2022 representing the change in depth regulations.

The CCFRP survey estimated peak selectivity at 33 cm in both time blocks with the first time blocks estimating decreased selectivity of larger fish. The survey expanded to northern California in 2017 where larger copper rockfish were observed and estimated asymptotic selecitivity for fish larger than 33 cm.

The catchability for each of the surveys was analytically solved comparing observed to expected vulnerable biomass across all years. The analytical values for catchability were very small given the survey methodologies and are reported in Table \ref{tab:params} in log-space. Additional fishery and survey index variability, process error added directly to each year's input standard deviation for the were estimated within the model. The model estimated the largest added variance of for the recreational PR fishery index. In contrast the model estimated only limited additional variability in order to fit the recreational CRFS CPFV fishery index (0.095). The model fit the trend in the CCFRP survey with added variance estimated to fit the time series of 0.22, while the model added and still did not fit the trend in the index. The model fit the CDFW ROV survey index well and estimated a small added variance of 0.089.

The estimated annual recruitment and recruitment deviations are shown in Figures \ref{fig:recruits} and \ref{fig:rec-devs}. The bias adjustment applied to the annual recruimtent deviations across time is shown in Figure \ref{fig:bias-adj}. Strong recruitments are estimated to have occurred in 1966-1967, 2007 and 2017 and the years of lowest estimated recuitment being 1979 and 1980. The uncertainy in recruitment deviations is highest for the first two years 1970 and 1971 and relatively consistent for the remainder of the time series. There is limited information in the data on recruitment variability from the available data. During model explorations the recruitment deviations were most sensitive to the removal of the available age and fishery index data.

Recruitment is estimated based on the spawner-recruit curve in 2021 and 2022 (Figure \ref{fig:bh-curve}). The recruitment bias adjustment applied within the model across years is shown in Figure \ref{fig:bias-adjust}.

\hypertarget{fits-to-the-data}{%
\subsubsection{Fits to the Data}\label{fits-to-the-data}}

\hypertarget{fits-to-length-and-age-composition}{%
\paragraph{Fits to length and age composition}\label{fits-to-length-and-age-composition}}

Fits to the length data are shown based on the Pearson residuals-at-length, the annual mean lengths, and aggregated length composition data for the commercial and recreational fleets. Annual length composition fits are shown in the Appendix, Section \ref{append-fit}. Aggregate fits by fleet are shown in Figure \ref{fig:agg-len-fit}.

The aggregated lengths for the commercial dead fleet reflected a wide selection across sizes, with the model under-predicting the selection for both small males and females. The majority of the length data for the commercial dead fleet consisted of unsexed fish with sex-specific lengths available from 1980, 1984, 1999, and 2019-2022. The aggregate length compositionfit well with the asymptotic selectivity curve for the commercial dead fleet. Multiple sensitivities were conducted to explore alternative parameterization of commercial dead fleet selectivity. The Pearson residuals for the commercial dead fishery length data area shown in Figure \ref{fig:com-dead-pearson}. The mean length observed in the commercial lengths of unsexed fish were generally stable between 1990 - 2019 and decreased to smaller sizes from 2019 - 2022, with high undertainty in the mean lengths of unsexed fish in 2022 (Figure \ref{fig:com-dead-mean-len-fit}). The observations of larger fish, greater than 40 cm, are minimally greater than the model expectations after 2010. A limited number of ages from the commercial dead fleet were available from 2019-2022. The model estimated mean age was within the bounds of uncertainty, but not well fit (Figure \ref{fig:agg-marg-age-fit})

Starting in 2010, the commercial live fleet length data shifts to smaller fish with observations greater than model expectations for fish between 25 - 30 cm. All available lengths for the commercial live fleet were from unsexed fish and the aggregated length data were fit relativley well given the change in selectivity in 2011. There were no ages available from the commercial live fleet and the The Pearson residuals for the commercial dead fishery length data area shown in Figure \ref{fig:com-live-pearson}. The mean length observed in the commercial lengths of unsexed fish were not stable prior to 2011 (Figure \ref{fig:com-dead-mean-len-fit}). From 2011-2022 the mean length of fish in the live fishery are relatively stable, with a notable decreased in 2016.

The lenth composition fits to the recreational CPFV fleet were relatively well fit throughout the time series, except for a efw years where the a number of fish in a single size class were observed that the model did not expect given the selectivity. The Pearson residuals do not show an indication of any strong year classes from the avaialble lengths (Figure \ref{fig:rec-cpfv-pearson}). The mean length observed, unsexed fish from the CPFV fleet was fit relatively well, indicating a slight increase in mean size around 2000, a decrease from 2007-2011 and a slight increase again from 2013-2018 (Figure \textbackslash ref\{fig:rec-cpfv-mean-len-fit). The number of sexed fish available from the CPFV fleet is small andthe last year of data was not well fit and estimated with high uncertainty. Only one year of age data were available, 2022, from a combination of NMFS Coopeartive Reserach collections and the CDFW groundfish group. A small fraction of these fish were unsexed, and the Pearson residuls indicate these data were generally well-fit (Figure \ref{fig:rec-cpfv-age-pearson}).

The Pearson residuals for the recreational PR length data are variable by year (Figure \ref{fig:rec-pr-pearson}). Pearson residuals were positive, observations greater than expected, for small fish prior to 1997 and are generally variable showing no clear misfit in the model in recent years. The aggregate length composition data from the PR fleet had a slightly higher peak around 29 cm with fewer observations. The length composition across yeras is fit well from 2004-2022 when CDFW implemented the CRFS sampling program. A wide range of sizes were observed from 1959-1987 with poorer fits in years with less data such as 1989 and 1996-2002. The mean length by year for the recreational PR fleet was highly variable across years (Figure \ref{fig:rec-pr-mean-len-fit}). The implementation of the MPA network may have impacted the shift to smaller mean size in those years. The CDFW collected ages from the recreational PR fleet in 2022. The peak of the age distribution was underestimated by the model (Figure \ref{fig:agg-marg-age-fit}).

The aggregated length composition to both fishery-independent surveys CCFRP and the CDFW ROV, were fit reasonably well with an underestimation of fish at around 40-45 cm in the ROV survey. Both of these surveys were conducted in California state waters and represent samples from inside and outside the MPAs. The annual fits to the CCFRP length data were not as well fit as other data sources in any given year, but the observation of larger fish when the survey expanded north in 2017 is pronounced. The Pearson residuals are presented in Figure \ref{fig:ccfrp-len-pearson} and exhibit no clear pattern. The model estiamted mean length was increaseing from 2014-2016 priot to the survey's expansion (Figure \ref{fig:ccfrp-mean-len-fit}). The model did not fit the decreased observed mean length in 2019. However, the model estimated mean length given the observed data from this survey. Age data were available from 2018, 2019 and 2022 from the CCFRP survey and input as condition age-at-length data. The data had a slightly higher proportion of older fish given estimated growth (Figure \ref{fig:ccfrp-age-pearson}). Of note is that all of these ages represent the time period after the survey expanded and selectivity was estimated to be asymptotic.

The length composition data for the ROV survey were available at a finer scale than the super years available for the index of abundance. Not surprising, the year with the most available length observations, 2021, had the best fit to the length data, although was an underestimate for fish in the 35-45 cm range. No trend was observed from the Pearson residuals (Figure \ref{fig:ccfrp-len-pearson}). The survey covered a wide range of depths and the same increasing trend in mean size as the CCFRP data show, was observed in the ROV survey from 2019-2021 (Figure \ref{fig:rov-mean-len-fit}).

\hypertarget{fits-to-indices-of-abundance}{%
\paragraph{Fits to Indices of Abundance}\label{fits-to-indices-of-abundance}}

<<<<<<< Updated upstream
Fits to the indices vary in quality. The Deb Wilson-Vandenberg onboard survey from 1988-1998 indicated a decline from 1992-1998 that was not fit well by the model. However, this is the highest quality data source for the time period and with the added variance, the model fit was fairly flat and uninformative (Figure @ref(fig:dwv\_cpfv-index-fit)). The index spans the years where the stock biomass begins to increase, creating a conflict between the index and the population trend. The Deb Wilson-Vandenberg survey effort was concentrated in central California, similar to the area surveyed from 2007-2016 in the CCFRP survey.
=======
\endfoot
\bottomrule
\endlastfoot
2013 & 2627.91 & 2588.61 & 928.75 & 210.14 & 0.306\\
2014 & 2795.04 & 2739.48 & 244.85 & 223.28 & 0.325\\
2015 & 2973.06 & 2916.90 & 102.22 & 240.91 & 0.350\\
2016 & 3097.56 & 3064.44 & 245.60 & 255.75 & 0.372\\
2017 & 3182.40 & 3150.11 & 122.12 & 268.46 & 0.390\\
2018 & 3182.40 & 3153.24 & 96.39 & 273.80 & 0.398\\
2019 & 3176.81 & 3131.98 & 112.02 & 279.23 & 0.406\\
2020 & 3179.52 & 3148.18 & 163.48 & 283.32 & 0.412\\
2021 & 3110.47 & 3087.05 & 317.57 & 278.75 & 0.405\\
2022 & 3108.97 & 3077.18 & 302.69 & 279.64 & 0.407\\
2023 & 3199.84 & 3163.70 & 299.87 & 289.74 & 0.421\\*
\end{longtable}
\endgroup{}
\endgroup{}
>>>>>>> Stashed changes

The CDFW and Cal Poly onboard index was flat from 2004-2015 and the increase in relative CPUE in the ending years 2017-2019 represent time periods when the fishery had access to deeper water, but the increase in relative CPUE in 2016 was not due to changes in regulations (Figure @ref(fig:rec\_cpfv-index-fit)). The model fit the ending years or data to the upper bound of the added variability.

The recreaitonal dockside PR index showed a similar trends to the CPFV onboard index. Tndex was well fit during the first part of the time series when it was relatively flat (2004-2015), but the increase in relative CPUE in the ending years 2017-2019 was not well captured by the model. Even with selectivity time blocks for these periods, the index was not fit in 2017.

The CDFW ROW survey contained data grouped into two super years and the model estimated a relatively flat line with the added variance (Figure \ref{fig:crfs-pr-index-fit}). The CCFRP index reflects the same increase in relative cpue in 2016 as the CPFV and PR indicies, prior to the survey expansion and releae of recreational depth restrictions. This index was weighted by the area within the MPAs, which exhibits an increasing trends compared to sites outside the survey at the end of the time series (Figure \ref{fig:ccfrp-index-fit}). The fit to the early part of the time series was reasonable given the available data. Similar to the 2019 gopher/black-and-yellow rockfish complex CCFRP survey, the lowest estimate year in the CCFRP year was 2013, which was also not fit in the 2019 gopher stock assessment. No explanation for the decrease in relative CPUE was identified.

\hypertarget{population-trajectory-in-the-modeled-area}{%
\subsubsection{Population Trajectory in the Modeled Area}\label{population-trajectory-in-the-modeled-area}}

The predicted spawning output (in millions of eggs) is given in Table \ref{tab:timeseries} and shown in Figure \ref{fig:north-ssb}. The estimated spawning output decreases sharply in the late-1970s and continues to decline until reaching low levels in the late-1990s. The spawning output slowly increases between 2000 - 2010 with the rate of population growth increasing after 2011 as fish from recent years of above average recruitment begin to mature. The estimate of total biomass follows the same trend over time is shown in Figure \ref{fig:tot-bio}. The estimated spawning output relative to unfished equilibrium spawning output for the sub-area north of Point Conception reached a minimum of 0.17 in 1994 and then increased over the recent time period, with an ending year estimate of 0.52 in 2022 (Figure \ref{fig:depl}).

\hypertarget{population-trajectory-for-the-stock}{%
\subsubsection{Population Trajectory for the Stock}\label{population-trajectory-for-the-stock}}

The predicted spawning output for the California stock or copper rockfish is given in Table \ref{tab:ca-status} and shown in Figure \ref{fig:sb-all}. The predicted trajectory of spawning output for the stock is generally similar to the trend observed for each area north and south of Point Conception with spawning output declining starting late 1970s when catches across California peaked. The spawning output of the stock declined the the lowest level in the mid-1990s and then began to steadily increase through the end of the time series. The spawning output relative unfished spawning output declined to the stock's lowest point in 1994 of 0.15 of spawning output (Figure \ref{fig:depl-all}). After hitting a low in 1994, the relative spawning output of the stock has steadily increased with an estimated final stock status of 40 percent in 2023.

\hypertarget{reference-points}\) reference harvest rate. The spawning output equivalent to 40\% of the unfished level (\(SB_{40\%}\)) was 194 million eggs.

The 2022 spawning biomass relative to unfished equilibrium spawning biomass is just below the target of 40\% of unfished levels (Figure \ref{fig:depl}). The relative fishing intensity, \((1-SPR)/(1-SPR_{50\%})\), was near the management target in 2020, and has fluctuated around the target level for the past decade (Figure \ref{fig:1-spr} and \ref{fig:phase}).

Table \ref{tab:north-referenceES} shows the full suite of estimated reference points for the base model and Figures \ref{fig:north-yield2} and \ref{fig:north-yield3} show the equilibrium yield curve and net production based on a steepness value fixed at 0.72.

\hypertarget{cpfv-samples}{%
\section{Appendix H. CPFV Fleet Description, Sampling, and Trip Types}\label{cpfv-samples}}

During the 2021 copper rockfish stock assessment meeting and during the pre-assessment workshop for the 2023 groundfish assessments, concern was raised regarding possible biases in the available data by trip types. This appendix describes the sampling and available data from two recent years as an example of the relationship between the available data and catches for copper rockfish.

The CDFW conducts several surveys are part of the CRFS program, including an onboard observer survey, angler interviews, and a required CPFV logbook for each trip. The \href{https://nrm.dfg.ca.gov/FileHandler.ashx?DocumentID=62348\&inline}{CRFS sampling manual} outlines the sampling methodologies for each survey component. The state of California allows the filleting of fish at sea; to accurately identify rockfish to species and obtain measurements of whole fish the majority of angler interviews occur onboard the vessel for the CPFV fleet. The CPFV fleet is not required to speciate the catch in the and only one block number is recorded per entry. For this reasons the CPFV logbook data are rarely used to develop an index of abundance for groundfish species. However, the CPFV logbook data provide insight into the spatial distribution of fishing effort.

The \href{https://nrm.dfg.ca.gov/FileHandler.ashx?DocumentID=36136\&inline}{California Fisheries Recreational FIsheries Survey Methods} indicates the CPFV ``sampling goal is to sample onboard at the historical sampling frequency of two to five percent of estimated CPFV trips of interest (e.g., trips targeting groundfish, inshore and coastal pelagic species) at each CPFV landing and to sample other CPFV trip types dockside.'' California is divided into six regions for the CRFS sampling, and these represent the finest stratification for catch and effort estimates (Figure \ref{fig:crfs-districts}).

The CPFV fleet north and south of Point Conception are fundamentally different in terms of the vessels fishing, available target species, and also fishery regulations over the past twenty years. The CPFVs north of Point Conception targeting groundfish to not typically target other species on the same trip, with the exception being a half day bottomfish trip mixed with a half day of Dungeness crab pot fishing. The fleet north of Bodega Bay California is dominated by smaller 6-pack vessels with limited capacity. It may not be possible for a sampler to ride onboard a 6-pack vessel if it's at capacity. Overnight trips are extremely rare in northern California. In addition, it's important to note that even though Santa Barbara spans Point Conception, all of the fish are landed in San Luis Obispo county in District 3.

In southern California, the diveristy of targt species is higher an includes several bass species, state managed gamefish, coastal pelagics, and highly migratory species, as well as boats that fish in Mexican waters. The trip types in southern California include 1/2 trips where a CPFV will run two trips a day, 3/4 day trips to fishing grounds further offshore, such as the Channel Islands, and overnight trips that can access areas like San Nicholas in the southern Channel Island or may be combination trips fishing for tunas, but also catching rockfish limits. The diversity of trip types and mixed target trips in southern California pose a challenge for both sampling and managing the fisheries.

We filtered the CPFV logbook data to trips that recorded at least one rockfish and the CPFV onboard observer data to trips that encountered at least 1\% groundfish species. We aggregated within each dataset across 2018 and 2019 and areas where necessary to ensure confidentiality requirements were met. The tables below represent an aggregation of data from 2018 and 2019 combined as they are the two years pre-COVID that best represent the CPFV fleet's activity. We extracted data from the CPFV logbook data and the onboard observer sampled trips to describe the differences in portion of trips sampled and the distribution of copper rockfish catches. In the CPFV logbooks, 60 trips were recorded as multi-day trips in northern California. This represented less then 1\% of the trips in northern California and we did not investigate the accuracy of these entries. Approximately 6\% of the logbooks from District 1 recorded a block number in Mexican waters as the primary location fished. Fewer than 10 trips that recorded block numbers along the mainland of southern California were recorded a multi-day trips. These were removed from the tables below.

The combined onboard observer sampling rate for 2018-2019 was 3\% percent of all CPFV trips in the data used. There were no trips observed onboard in CRFS District 5 during 2018-2019, but 1,052 CPFV logbook records were submitted that recorded at least one rockfish. Grouping the districts by north and south of Point Conception, 2\% of all trips in northern California were sampled onboard (Table @ref(\url{tag:percent-observed})). In southern California, 4\% of single day trips were observed and less than 1\% of multi-day trips were observed.

We broke the data in southern California down further to explore sampling effort by trip type, and assigned a CPFV logbook entry to a region based on the recorded block number (Table \ref{tab:logbook-triptype}). The offshore region was a catch-all that included blocks outside the range of a 1/2 trip, but not within the vicinity of an island. In CRFS District 1, 79\% of the observd trips were 1/2 day trips, which can access the District 1 mainland. In District 2, 29\% of the observed trips were 1/2 day trips. The vessels in District 2 typically offer fewer 1/2 day trips and fish the nearshore when weather precludes crossing the channel to the northern Channel Islands.

To put these sampling rates in context for copper rockfish, we explored the total estimated moratlity related to the distribution of sampling. The total mortality of copper rockfish in metric tons from 2018-2019 from the CPFV fleet by CRFS District is in Table \ref{tab:catch-example}. Fifty-two percent of the total copper rockfish mortality was from northern California and 48\% from southern California. Within northern California, 50\% of the total mortality originated from District 3, and in southern California, 81\% of the total mortality originated from District 2.

From the onboard observer trips in District 1, 37\% of the observed coppers were from 1/2 day trips, 42\% from 3/4-1 day trips, and 21\% of the observed copper rockfish were from the five observed multi-day trips (Table \ref{tab:onboard-coppers}).

In District 2, 17\% of the observed copper were from 1/2 day trips, 75\% from the 3/4-1 day trips and 8\% from overnight trips. When weather allows, the northern Channel Islands can be accessed from CRFS District 2 during a 3/4-1 day trip or a multi-day trip, depending on the port. For instance, the CPFVs from Oxnard, California access the northern Channel Island during multi-day trips, and the same areas are accessed by the 3/4-1 day boats out of Santa Barbara Landing.

\newpage

\begin{figure}
\centering
\includegraphics[width=1\textwidth,height=1\textheight]{S:/copper_rockfish_2023/data/rec_indices/cpfv_logbook_summaries/crfs_districts.png}
\caption{Map of the CDFW CRFS sampling Districts.\label{fig:crfs-districts}}
\end{figure}

\begingroup\fontsize{10}{12}\selectfont
\begingroup\fontsize{10}{12}\selectfont

\begin{longtable}[t]{c>{\centering\arraybackslash}p{2cm}>{\centering\arraybackslash}p{2cm}}
\caption{\label{tab:percent-observed}The percent of trips sampled onboard by trip type and area for 2018 and 2019, using the number of trips observed and CPFV logbook trips described in the text.}\\
\toprule
Area & Single day & Multi day\\
\midrule
\endfirsthead
\caption[]{\label{tab:percent-observed}The percent of trips sampled onboard by trip type and area for 2018 and 2019, using the number of trips observed and CPFV logbook trips described in the text. \textit{(continued)}}\\
\toprule
Area & Single day & Multi day\\
\midrule
\endhead

\endfoot
\bottomrule
\endlastfoot
<<<<<<< Updated upstream
Northern CA & 6\% & NA\\
Southern CA & 10\% & 1\%\\*
=======
2012 & 163.15 & 136.17 & 85.95\\
2013 & 148.00 & 123.42 & 105.18\\
2014 & 148.00 & 123.42 & 98.65\\
2015 & 303.75 & 277.32 & 147.64\\
2016 & 286.88 & 261.95 & 165.27\\
2017 & 313.70 & 286.38 & 225.48\\
2018 & 319.60 & 291.85 & 203.69\\
2019 & 325.08 & 296.83 & 182.59\\
2020 & 330.35 & 301.60 & 242.73\\
2021 & 249.85 & 206.43 & 164.20\\
2022 & 249.48 & 204.02 & 66.67\\*
>>>>>>> Stashed changes
\end{longtable}
\endgroup{}
\endgroup{}

\newpage

\begingroup\fontsize{10}{12}\selectfont
\begingroup\fontsize{10}{12}\selectfont

\begin{longtable}[t]{r>{\raggedleft\arraybackslash}p{2cm}>{\raggedleft\arraybackslash}p{2cm}>{\raggedleft\arraybackslash}p{2cm}}
\caption{\label{tab:catch-example}Total mortality in metric tons of copper rockfish from 2018 and 2019 from the CPFV fleet by CRFS District.}\\
\toprule
District & 2018 & 2019 & Total\\
\midrule
\endfirsthead
\caption[]{\label{tab:catch-example}Total mortality in metric tons of copper rockfish from 2018 and 2019 from the CPFV fleet by CRFS District. \textit{(continued)}}\\
\toprule
District & 2018 & 2019 & Total\\
\midrule
\endhead

\endfoot
\bottomrule
\endlastfoot
<<<<<<< Updated upstream
1 & 9.2 & 23.1 & 32.3\\
2 & 87.0 & 51.8 & 138.8\\
3 & 49.3 & 44.3 & 93.6\\
4 & 30.0 & 27.9 & 57.9\\
5 & 7.7 & 13.1 & 20.7\\
6 & 6.0 & 7.3 & 13.2\\*
=======
2023 & 116.4 & 91.53 & 91.5 & - & - & 289.74 & 0.421\\
2024 & 121.32 & 94.69 & 94.7 & - & - & 295.18 & 0.429\\
2025 & - & - & - & 171.29 & 148.79 & 302.30 & 0.440\\
2026 & - & - & - & 171.74 & 147.12 & 303.82 & 0.442\\
2027 & - & - & - & 172.05 & 145.66 & 304.90 & 0.443\\
2028 & - & - & - & 172.2 & 144 & 305.69 & 0.444\\
2029 & - & - & - & 172.26 & 141.82 & 306.33 & 0.445\\
2030 & - & - & - & 172.31 & 140.09 & 306.97 & 0.446\\
2031 & - & - & - & 172.39 & 138.4 & 307.60 & 0.447\\
2032 & - & - & - & 172.54 & 136.34 & 308.29 & 0.448\\
2033 & - & - & - & 172.8 & 134.85 & 309.09 & 0.449\\
2034 & - & - & - & 173.15 & 143.44 & 309.99 & 0.451\\*
>>>>>>> Stashed changes
\end{longtable}
\endgroup{}
\endgroup{}

<<<<<<< Updated upstream
\newpage
=======
\hypertarget{scientific-uncertainty}{%
\subsection*{Scientific uncertainty}\label{scientific-uncertainty}}
\addcontentsline{toc}{subsection}{Scientific uncertainty}

The model estimated uncertainty around the 2023 spawning output was for the sub-area south of Point Conception is \(\sigma\) = 0.29 and north of Point Conception is \(\sigma\) = 0.26. The uncertainty around the OFL south and north of Point Conception was \(\sigma\) = 0.27 and 0.25, respectively. These is likely an underestimates of overall uncertainty because of the necessity to fix several population dynamic parameters (e.g., steepness, recruitment variance, natural mortality) and no explicit incorporation of model structural uncertainty (although see the decision table for alternative states of nature).

\hypertarget{research-and-data-needs}{%
\subsection*{Research and data needs}\label{research-and-data-needs}}
\addcontentsline{toc}{subsection}{Research and data needs}

There were some major sources of uncertainty within the assessments for copper rockfish. To improve our understanding of the copper rockfish stock in California waters the following research and data collection should be prioritized:

\begin{enumerate}

  \item Continue to investigate historical catch reconstructions and all other historical data sources.

  \item Continue to investigate the most appropriate model structure for the NWFSC Hook and Line survey index. The NWFSC Hook and Line survey is the only long-term fishery-independent survey in rocky (untrawlable) habitat in the Southern California Bight. We also recommend evaluating how to structure the NWFSC Hook and Line survey index, given its expansion into the CCAs and increase of site now within designated MPAs, and independent analysis of information content in NWFSC Hook and Line survey across observed species. Finally, increased spatiotemporal sampling around Point Conception would aid in identifying stock boundaries.


    \item The assessment area appears to have a mixture of observations from areas experiencing variable fishing mortality. In the region there are likely a mixture of areas: open access rocky reefs that are close to port that are heavily fished, open access rocky reefs that are inaccessible via day-trips that are fished but likely lower levels, and rocky reefs that fall within marine protect areas.  A spatially-explicit assessment model may be able to capture this complexity but will require data (indices of abundance and composition data) from each of the regions. 
    
    \item Future nearshore assessments would greatly benefit from additional CDFW ROV surveys which could increase the power of these data to inform assessments.

    \item There are very limited age data for copper rockfish across California arising from fishery-dependent sources.  Collecting otoliths from the recreational fishery, a large source of mortality, would support future assessments  and would improve the understanding of the population structure and life history of copper rockfish. 


\end{enumerate}

\pagebreak
\setlength{\parskip}{5mm plus1mm minus1mm}
\pagenumbering{arabic}
\setcounter{page}{1}
\renewcommand{\thefigure}{\arabic{figure}}
\renewcommand{\thetable}{\arabic{table}}
\setcounter{table}{0}
\setcounter{figure}{0}

\hypertarget{introduction}{%
\section{Introduction}\label{introduction}}

This assessment report describes the sub-area population of copper rockfish (\emph{Sebastes caurinus}) off the California coast north of Point Conception in U.S. waters, using data through 2022. The sub-area population south of Point Conception in California waters was also evaluated and is described in a separate assessment report. The copper rockfish status for the California stock is determined by the combined estimates of spawning output from both sub-areas and is detailed in the \protect\hyperlink{management}{management} Section. This assessment does not account for populations located in Mexico waters or other areas off the U.S. coast and assumes that these southern and northern populations do not contribute to the population being assessed here.

\hypertarget{basic-information-and-life-history}{%
\subsection{Basic Information and Life History}\label{basic-information-and-life-history}}

Copper rockfish have historically been a part of both commercial and recreational fisheries throughout its range. Copper rockfish are a demersal, relatively nearshore species within the subgenus \emph{Pteropodus.} The core range of copper rockfish is comparatively large, ranging from northern Baja Mexico to the Gulf of Alaska, with copper rockfish also found in Puget Sound. Copper rockfish range from the sub-tidal (as juveniles) to depths of 183 m (Love et al. 2002). Copper rockfish are commonly found in waters less than 100 meters in depth inhabiting nearshore kelp forests and complex low-relief rocky habitat (Love 1996). Adult copper rockfish have high site fidelity and are thought to not make long-range movements. An acoustic telemetry study displaced copper rockfish 4km from their capture location to an artificial reef and within 10 days, half of the copper rockfish returned to the original capture location (Reynolds et al. 2010).

Copper rockfish have a clearly defined long white band the posterior two-thirds of the lateral line. Copper rockfish has high variation in coloration throughout its range, taking on coloration from dark brown, olive, orange-red and pink, with patches of yellow and pink (Miller and Lea 1972). In general, the copper rockfish towards the northern part of the range are often darker in color than fish encountered in southern California. The distinct change in coloration resulted in copper rockfish initially being described as two separate species, copper rockfish (\emph{S. caurinus}) and whitebelly rockfish (\emph{S. vexillaris}).

The \emph{Sebastes} genus are viviparous with internal fertilization, many exhibit dimorphic growth with females larger at size-at-age than males, and a number of species have reproductive strategies that vary with latitude. There are very few fecundity samples from copper rockfish available from available from California, although copper rockfish are assumed to produce a single brood annually during the winter months.

The pelagic larvae are encountered in the California Cooperative Oceanic Fisheries Investigations (CalCOFI) surveys, but neither larval nor young-of-the-year can be identified copper rockfish visually (Thompson et al. 2017). The size at birth ranges from 5-6 mm and the larvae remain pelagic until approximately 22-23 mm standard length at which time they recruit to the kelp forest canopy (Anderson 1983).

Juvenile Copper rockfish are indistinguishable from kelp (\emph{S. atrovirens}), black-and-yellow (\emph{S. chrysomelas}), and gopher rockfishes (\emph{S. carnatus}), all of which recruit to the kelp forest canopy in the spring months. Copper rockfish is the first of the species group to recruit to the kelp forest from April to May and can be distinguished from the other species once it reaches a size around 50 mm standard length (Anderson 1983). (2019) genetically identified young-of-the-year rockfish from surveys in Carmel and Monterey Bays in California and provided the authors with the length and genotyped species identifications from her study. The average length of copper rockfish in July was 3-4 cm total length (Figure \ref{fig:copper-smurf-length}). Anderson observed benthic copper rockfish nocturnally active over sandy bottom outside the kelp forest (Anderson 1983).

Copper rockfish are a relatively long-lived rockfish, estimated to live at least 50 years (Love 1996). Copper rockfish was determined to have the highest vulnerability (V = 2.27) of any West Coast groundfish stock evaluated in a productivity susceptibility analysis (Cope et al. 2011). This analysis calculated species-specific vulnerability scores based on two dimensions: productivity characterized by the life history and susceptibility that characterized how the stock could be impacted by fisheries and other activities.

Copper rockfish are opportunistic carnivores and commonly consume crustaceans, mollusks, and fish whole (Lea et al. 1999; Bizzarro et al. 2017). (1972) observed a shift in a diet dominated by arthropods in age 0 and 1 fish to a more diverse diet including molluscs and fish as they aged. The study also noted that juvenile copper rockfish were preyed upon by harbor seals and lingcod.

There is currently no evidence of significant stock structure from genetic studies of copper rockfish across the west coast. -Buonaccorsi et al. (2002) looked at genetic variation across six micosatellite DNA loci from samples ranging from British Columbia to southern California. Significant population subdivision was detected between the Puget Sound and coastal samples which supports the model of isolation-by-distance for copper rockfish. Sivasundar and Palumbi (2010) conducted a genetic study to determine the potential for biogeographic boundaries to prohibit gene flow for 15 \emph{Sebastes} species. The study's sample sizes of copper rockfish with samples form Oregon, Monterey Bay and Santa Barbara. -Sivasundar and Palumbi (2010) used mtDNA and could differentiate samples from Santa Barbara from those collected in Oregon and Monterey Bay, but the Monterey Bay and Oregon samples could not be distinguished. Micosatellite data did not reveal any genetic differentiation among the samples from the three locations for copper rockfish and suggests low genetic differentiation coastwide.

The most recent genetic analysis of copper rockfish to date was conducted by Johansson et al. (2008). The study included 749 samples from along the west coast ranging from Neah Bay, Washington to San Diego, California with the majority of sampling locations clustered north of Cape Mendocino in northern California. The study included 185 samples collected within California. Eleven microsatellite DNA loci were analyzed. The study found significant evidence to support isolation by distance at the coast wide scale. Weak, but significant, genetic structure was identified from samples collected along the Oregon coast suggesting that habitat barriers may limit larval dispersal.

\hypertarget{ecosystem-considerations-1}{%
\subsection{Ecosystem Considerations}\label{ecosystem-considerations-1}}

This stock assessment does not explicitly incorporate trophic interactions, habitat factors (other than as they inform relative abundance indices) or environmental factors into the assessment model, but a brief description of likely or potential ecosystem considerations are provided below.

As with most other rockfish and groundfish in the California Current, recruitment, or cohort (year-class) strength appears to be highly variable for the copper rockfish complex, with only a modest apparent relationship to estimated levels of spawning output. Oceanographic and ecosystem factors are widely recognized to be key drivers of recruitment variability for most species of groundfish, as well as most elements of California Current food webs. Empirical estimates of recruitment from pelagic juvenile rockfish surveys have been used to inform incoming year class strength for some of these stocks, however copper rockfish are infrequently encountered in these surveys. Between 1998 and 2013 the California Cooperative Oceanic Fisheries Investigation (CalCOFI) survey observed had 34 positive observations copper rockfish out of nearly 505 bongo net tows.

\hypertarget{historical-and-current-fishery-information}{%
\subsection{Historical and Current Fishery Information}\label{historical-and-current-fishery-information}}

Off the coast of California south of Point Conception copper rockfish is caught in both commercial and recreational fisheries. Recreational removals have been the largest source of fishing mortality of copper rockfish across all years (Table \ref{tab:allcatches} and Figure \ref{fig:catch}). The recreational fishery is comprised of individual recreational fishers (Private/Rental, PR) and commercial passenger fishing vessels (CPFV) also known as party/charter (PC) which take groups of individuals out for day fishing trips. Across both types of recreational fishing the majority of effort occurs around rocky reefs that can be accessed via a day-trips.

The recreational fishery in the early part of the 20th century was focused on nearshore waters near ports, with expanded activity further from port and into deeper depths over time (Miller et al. 2014). Prior to the groundfish fishery being declared a federal disaster in 2000, and the subsequent rebuilding period, there were no time or area closures for groundfish. Access to deeper depths during this period spread effort over a larger area and filled bag limits with a greater diversity of species from both the shelf and nearshore. This resulted in lower catch of nearshore rockfish relative to the period after 2000 when 20 to 60 fathom (fm) depth restrictions ranging from 20 fm in the Northern Management Area to 60 fm in the Southern Management Area were put in place in various management area delineations along the state. This shifting effort onto the nearshore, concomitantly increased catch rates for nearshore rockfish including copper rockfish in the remaining open depths, though season lengths were greatly curtailed.

Following all previously overfished groundfish species, other than yelloweye rockfish, being declared rebuilt by 2019, deeper depth restrictions were offered in the California Southern Management area allowing resumed access to shelf rockfish in less than 75 fm and are currently 100 fm as of 2021. The increased access to deeper depths south of Point Conception with the rebuilding of cowcod is expected to reduce the effort in nearshore waters where copper rockfish is most prevalent. To the north of Point Conception where yelloweye rockfish are prevalent, depth constraints persist and effort remains focused on the nearshore in 30 to 50 fm depending on the management area. As yelloweye rockfish continues to rebuild, incremental increases in access to deeper depths are expected, which will likely further reduce the effort in nearshore waters where copper rockfish is most prevalent.

Prior to development of the live fish market in the 1980s, there was very little commercial catch of copper rockfish, with dead copper rockfish fetching a low ex-vessel price per pound. Copper rockfish were targeted along with other rockfish to some degree in the nearshore or caught as incidental catch by vessels targeting other more valuable stocks such as lingcod. Most fish were caught using hook and line gear, though some were caught using traps, gill nets and, rarely, trawl gear. Trawling was prohibited within three miles of shore in 1953 and gill netting within three miles of shore was prohibited in 1994, preventing access to a high proportion of the species habitat with these gear types. Copper rockfish were caught in the nearshore, along with other rockfish, or caught as bycatch by vessels targeting other more valuable stocks such as lingcod.

In the late 1980s and early 1990s a market for fish landed live arose out of Los Angeles and the Bay area, driven by demand from Asian restaurants and markets. The growth of the live fish market was driven by consumers willing to pay a higher price for live fish, ideally plate-sized (12 - 14 inches or 30.5 - 35.6 cm). Live fish landed for the restaurant market are lumped into two categories, small (1 - 3 lbs.) or large (3 - 6 lbs.), with small, plate-sized, fish fetching higher prices at market ranging between \$5 -7 per fish (Bill James, personal communication). Copper rockfish is one of the many rockfish species that is included in the commercial live fish fishery. The proportion of copper rockfish being landed live vs.~dead since 2000 by California commercial fleets ranges between 50 to greater than 70 percent in the southern and northern areas, respectively.

With the development and expansion of the nearshore live fish fishery during the 1980s and 1990s, new entrants in this open access fishery were drawn by premium ex-vessel price per pound for live fish, resulting in over-capitalization of the fishery. Since 2002, the California Department of Fish and Wildlife (CDFW) has managed 19 nearshore species in accordance with the Nearshore Fisheries Management Plan (Wilson-Vandenberg et al. 2014). In 2003, the CDFW implemented a Nearshore Restricted Access Permit system, including the requirement of a Deeper Nearshore Fishery Species Permit to retain copper rockfish, with the overall goal of reducing the number of participants to a more sustainable level, with permit issuance based on historical landings history by the retrospective qualifying date. The result was a reduction in permits issued from 1,127 in 1999 to 505 in 2003, greatly reducing catch levels. In addition, reduced trip limits, season closures in March and April and depth restrictions were implemented to address bycatch of overfished species and associated constraints from their low catch limits.

Copper rockfish residing between Point Conception and the California/Oregon border are assessed here as a separate sub-area (Figure \ref{fig:ca-map}). This designation was made based on oceanographic, geographic, and fishery conditions. The copper rockfish population in California waters was split at Point Conception due to water circulation patterns that create a natural barrier between nearshore rockfish populations to the north and south. The northern border for this assessment was defined as the California/Oregon border due to substantial differences in historical and current exploitation levels. Additionally, the fairly sedentary nature of adult copper rockfish likely limits flow of fish between south and north of Point Conception.

\hypertarget{summary-of-management-history-and-performance}{%
\subsection{Summary of Management History and Performance}\label{summary-of-management-history-and-performance}}

Prior to the adoption of the Pacific Coast Groundfish Fishery Management Plan (FMP) in 1982, copper rockfish were managed through a regulatory process that included the California Department of Fish and Wildlife (CDFW), the California State Legislature, and the Fish and Game Commission (FGC). With implementation of the Pacific Coast Groundfish FMP, copper rockfish came under the management authority of the Pacific Fishery Management Council (PFMC) and were managed as part of the Sebastes complex. Because copper rockfish had not undergone rigorous stock assessment and did not compose a large fraction of the landings it was classified and managed as part of the ``Minor Nearshore Rockfish'' group (\emph{Status of the pacific coast groundfish fishery} 2008).

Since the early 1980s, a number of federal regulatory measures have been used to manage the commercial rockfish fishery including cumulative trip limits (generally for two- month periods) and seasons. Starting in 1994 the commercial groundfish fishery sector was divided into two components: limited entry and open access with specific regulations designed for each component. Limited entry programs were designed in part to limit bottom contact gears and the open access sector includes gears not making bottom contact, e.g., hook and line. Other regulatory actions for the general rockfish categories included area closures and gear restrictions set for the four different commercial sectors - limited entry fixed gear, limited entry trawl, open access trawl, and open access non-trawl (which includes the nearshore fishery) .

During the late 1990s and early 2000s, major changes also occurred in the way that California managed its nearshore fishery. The Marine Life Management Act (MLMA), which was passed in 1998 by the California Legislature and enacted in 1999, required that the FGC adopt an FMP for nearshore finfish (Wilson-Vandenberg et al. 2014). It also gave authority to the FGC to regulate commercial and recreational nearshore fisheries through FMPs and provided broad authority to adopt regulations for the nearshore fishery during the time prior to adoption of the nearshore finfish FMP. Within this legislation, the Legislature also included a requirement that commercial fishermen landing nearshore species possess a nearshore fishery permit.

In 2000, the PFMC's rockfish management structure changed significantly with the replacement of the Sebastes complex -north and -south areas with Minor Rockfish North (Vancouver, Columbia, and Eureka, International North Pacific Fisheries Commission (INPFC) areas) and Minor Rockfish South (Monterey and Conception INPFC areas only). The optimum yield for these two groups was further divided (between north and south of 40\(^\circ\) 10' N. lat., Cape Mendocino, California) into nearshore, shelf, and slope rockfish categories with allocations set for Limited Entry and Open Access fisheries within each of these three categories. Species were parceled into these new categories depending on primary catch depths and geographical distribution. copper rockfish was included in the nearshore rockfish category.

Following adoption of the Nearshore FMP and accompanying regulations by the FGC in fall of 2002, the FGC adopted regulations in November 2002 which established a set of marine protected areas (MPAs) around the Channel Islands in southern California (which became effective April 2003). The FGC also adopted a restricted access program in December 2002 which established the Deeper Nearshore Species Fishery Permit, to be effective starting in the 2003 fishing year. Also, since the enactment of the MLMA, the PFMC and State coordinated to develop and adopt various management specifications to keep harvest within the harvest targets, including seasonal and area closures, depth restrictions, and bag limits to regulate the recreational fishery and license and permit regulations, finfish trap permits, gear restrictions, seasonal and area closures, depth restrictions, trip limits, and minimum size limits to regulate the commercial fishery. The MPAs were later expanded under authority of the Marine Life Protection Act (MLPA) enacted in 1999, creating a network of MPAs which went into place in phases beginning with the central coast in 2007, north central coast in 2010, and the south and north coasts in 2012. The implementation of the cowcod conservation area (CCA) in 2001 closed a large area of the Southern California Bight west of Santa Catalina and San Clemente Islands and offshore of San Diego. The CCA prohibited retention of groundfish, except for some take of nearshore species in depths less than 20 fm around islands and banks, and later, less than 40 fm. The rockfish conservation areas (RCAs) are seasonally adjusted depth limits impacting trawl and non-trawl gears that were initially established in 2002 to protect overfished species. The RCAs also restricted catch of nearshore species to depths less than 30 fm, and in some areas along California to less than 20 fm. Thus, the MPAs, CCAs and RCAs represent three types of spatial and/or depth closures impacting rockfish.

The state of California has adopted regulatory measures to manage the nearshore fishery based on the harvest guidelines set by the PFMC for the minor nearshore rockfish complexes north and south of 40\(^\circ\) 10' N. lat. The complexes are managed based on overfishing limits (OFL) and annual catch limits (ACL) that are determined by summing the species-specific OFLs and ACLs (ACLs set equal to the Acceptable Biological Catches) contributions for all stocks managed in the complexes). Limits are shared among all commercial and recreational fleets with the various management procedures intended to maintain removals below the total OFL and ACL for the nearshore rockfish north and south complexes as a whole, rather than on a species by species basis. The nearshore commercial fishery is managed based on bimonthly allowable catches per vessel, that have ranged from 200 pounds to 2,000 pounds per two months since 2000. The limited entry trawl fleet is managed on monthly limits on an annual basis. Since 2011, the limit has been 300 pounds per month for non-IFQ species, such as nearshore rockfish.

The species-specific OFL and ACL contribution for copper rockfish that is allocated to California waters, Nearshore Rockfish South and 25 percent of the Nearshore Rockfish North for copper rockfish, is shown in Table \ref{tab:ca-management} as well as the total catch, south and north of Point Conception, of copper rockfish in California for the last ten years. Over the last ten years the catches of copper rockfish have been below the species-specific ACLs. In 2021 all U.S. West Coast stocks of copper rockfish were assessed that informed the 2023-24 harvest specifications species-specific OFLs and ACLs for copper rockfish. In California waters the new OFLs and ACLs for the 2023-24 management cycle were significantly lower than early years, resulting in in-season management action by CDFW for 2022 to reduce removals based on the latest stock assessment. January 1, 2022, a statewide commercial sub-trip limit of 75 lbs. per 2-month and statewide recreational sub-bag limit of 1 fish within the overall 10 fish allowed for the RCG complex went into effect. No change in recreational seasons or depth limits occurred in 2022 but changes were implemented in 2023. In 2022, the Northern and Mendocino management areas were closed January through April and allowed fishing to 30 fathoms May through October and at all depths November through December. The San Francisco and Central management areas were closed January through March and allowed fishing to 50 fathoms the remainder of the year. The Southern management area was closed January and February and allowed fishing to 100 fathoms the remainder of the year. Beginning in 2023, closed seasons are extended in all management areas. Depth restrictions are eased during some months and tightened in others.
>>>>>>> Stashed changes

\begingroup\fontsize{10}{12}\selectfont
\begingroup\fontsize{10}{12}\selectfont

\begin{longtable}[t]{r>{\raggedleft\arraybackslash}p{2cm}>{\raggedleft\arraybackslash}p{2cm}>{\raggedleft\arraybackslash}p{2cm}}
\caption{\label{tab:onboard-coppers}Number of copper rockfish observed during the CPFV trips sampled onboard by district and trip type from 2018-2019.}\\
\toprule
District & 1/2 day trips & 3/4-1 day trips & Overnight trips\\
\midrule
\endfirsthead
\caption[]{\label{tab:onboard-coppers}Number of copper rockfish observed during the CPFV trips sampled onboard by district and trip type from 2018-2019. \textit{(continued)}}\\
\toprule
District & 1/2 day trips & 3/4-1 day trips & Overnight trips\\
\midrule
\endhead

\endfoot
\bottomrule
\endlastfoot
1 & 111 & 123 & 62\\
2 & 136 & 588 & 59\\
3 & 140 & 351 & NA\\
4 and 6 & 12 & 138 & NA\\*
\end{longtable}
\endgroup{}
\endgroup{}

\newpage

\begingroup\fontsize{10}{12}\selectfont
\begingroup\fontsize{10}{12}\selectfont

<<<<<<< Updated upstream
\begin{longtable}[t]{c>{\centering\arraybackslash}p{1.83cm}>{\centering\arraybackslash}p{1.83cm}>{\centering\arraybackslash}p{1.83cm}>{\centering\arraybackslash}p{1.83cm}>{\centering\arraybackslash}p{1.83cm}}
\caption{\label{tab:onboard-trips}Number of CPFV trips sampled as part of the onboard observer sruvey during 2018-2019 by trip type and District. District 4 and 6 were combined to retain confidentiality.  No trips from District 5 were sampled.}\\
\toprule
District & 1/2 day trips & 3/4-1 day trips & Overnight trips & Percent of trips with copper & Total number of copper observed\\
\midrule
\endfirsthead
\caption[]{\label{tab:onboard-trips}Number of CPFV trips sampled as part of the onboard observer sruvey during 2018-2019 by trip type and District. District 4 and 6 were combined to retain confidentiality.  No trips from District 5 were sampled. \textit{(continued)}}\\
\toprule
District & 1/2 day trips & 3/4-1 day trips & Overnight trips & Percent of trips with copper & Total number of copper observed\\
\midrule
\endhead
=======
Data from a wide range of programs were available for possible inclusion in the current assessment model. Descriptions of each data source included in the model (Figure \ref{fig:data-plot}) and sources that were explored but not included in the base model are provided below. Data that were excluded from the base model were excluded only after being explicitly explored during the development of this stock assessment and found to be inappropriate for use or had not changed since their past exploration in a previous copper rockfish stock assessment when they were not used. In some cases, the inclusion of excluded data sources were explored through sensitivity analyses (see Section \ref{assessment-model}).
>>>>>>> Stashed changes

\endfoot
\bottomrule
\endlastfoot
1 & 435 & 119 & 5 & 21\% & 296\\
2 & 36 & 93 & 4 & 72\% & 783\\
3 & 86 & 55 & NA & 67\% & 864\\
4 and 6 & 10 & 69 & NA & 61\% & 150\\*
\end{longtable}
\endgroup{}
\endgroup{}

\newpage

\begingroup\fontsize{10}{12}\selectfont
\begingroup\fontsize{10}{12}\selectfont

\begin{longtable}[t]{r>{\raggedleft\arraybackslash}p{2cm}>{\raggedleft\arraybackslash}p{2cm}}
\caption{\label{tab:logbook-triptype}Number of CPFV logbook entries with at least one rockfish, grouped by region fished and trip type from 2018-2019.}\\
\toprule
Region & Multi-day & Single day\\
\midrule
\endfirsthead
\caption[]{\label{tab:logbook-triptype}Number of CPFV logbook entries with at least one rockfish, grouped by region fished and trip type from 2018-2019. \textit{(continued)}}\\
\toprule
Region & Multi-day & Single day\\
\midrule
\endhead

\endfoot
\bottomrule
\endlastfoot
Mexico & 223 & 636\\
District 1 mainland & NA & 8324\\
Southern Channel Islands & 1170 & 1572\\
District 2 mainland & NA & 663\\
Northern Channel Islands & 1135 & 2600\\
Southern CA Offshore & 119 & 2243\\
District 3 & 58 & 5195\\
District 4 & NA & 3156\\
District 5 & NA & 1051\\
District 6 & NA & 1189\\*
\end{longtable}
\endgroup{}
\endgroup{}

\clearpage

\hypertarget{tables}{%
\section{Tables}\label{tables}}

\begingroup\fontsize{10}{12}\selectfont
\begingroup\fontsize{10}{12}\selectfont

\begin{longtable}[t]{l>{\raggedright\arraybackslash}p{1.83cm}>{\raggedright\arraybackslash}p{1.83cm}>{\raggedright\arraybackslash}p{1.83cm}>{\raggedright\arraybackslash}p{1.83cm}>{\raggedright\arraybackslash}p{1.83cm}}
\caption{\label{tab:allcatches}Removals (mt) by fleet and the summed total landings (mt).}\\
\toprule
Year & Commercial (Dead) & Commercial (Live) & Rec. CPFV & Rec. PR & Total Landings\\
\midrule
\endfirsthead
\caption[]{\label{tab:allcatches}Removals (mt) by fleet and the summed total landings (mt). \textit{(continued)}}\\
\toprule
Year & Commercial (Dead) & Commercial (Live) & Rec. CPFV & Rec. PR & Total Landings\\
\midrule
\endhead

<<<<<<< Updated upstream
\endfoot
\bottomrule
\endlastfoot
1916 & 4.0 & 0.0 & 0.0 & 0.0 & 4.0\\
1917 & 6.2 & 0.0 & 0.0 & 0.0 & 6.2\\
1918 & 7.5 & 0.0 & 0.0 & 0.0 & 7.5\\
1919 & 4.9 & 0.0 & 0.0 & 0.0 & 4.9\\
1920 & 5.1 & 0.0 & 0.0 & 0.0 & 5.1\\
1921 & 4.3 & 0.0 & 0.0 & 0.0 & 4.3\\
1922 & 3.7 & 0.0 & 0.0 & 0.0 & 3.7\\
1923 & 3.9 & 0.0 & 0.0 & 0.0 & 3.9\\
1924 & 2.6 & 0.0 & 0.0 & 0.0 & 2.6\\
1925 & 3.8 & 0.0 & 0.0 & 0.0 & 3.8\\
1926 & 4.9 & 0.0 & 0.0 & 0.0 & 4.9\\
1927 & 3.6 & 0.0 & 0.0 & 0.0 & 3.6\\
1928 & 3.6 & 0.0 & 1.0 & 0.6 & 5.2\\
1929 & 3.0 & 0.0 & 1.9 & 1.2 & 6.2\\
1930 & 5.3 & 0.0 & 2.2 & 1.4 & 9.0\\
1931 & 6.3 & 0.0 & 3.0 & 1.9 & 11.1\\
1932 & 5.7 & 0.0 & 3.7 & 2.4 & 11.7\\
1933 & 4.9 & 0.0 & 4.4 & 2.8 & 12.1\\
1934 & 3.6 & 0.0 & 5.2 & 3.3 & 12.0\\
1935 & 5.7 & 0.0 & 5.9 & 3.8 & 15.3\\
1936 & 5.2 & 0.0 & 6.6 & 4.2 & 16.1\\
1937 & 5.9 & 0.0 & 7.9 & 5.0 & 18.8\\
1938 & 5.2 & 0.0 & 7.7 & 5.0 & 17.9\\
1939 & 5.0 & 0.0 & 6.8 & 4.3 & 16.1\\
1940 & 4.8 & 0.0 & 9.7 & 6.2 & 20.8\\
1941 & 5.2 & 0.0 & 9.0 & 5.8 & 20.0\\
1942 & 1.8 & 0.0 & 4.8 & 3.1 & 9.6\\
1943 & 2.9 & 0.0 & 4.6 & 2.9 & 10.4\\
1944 & 8.7 & 0.0 & 3.8 & 2.4 & 14.8\\
1945 & 21.4 & 0.0 & 5.0 & 3.2 & 29.6\\
1946 & 23.9 & 0.0 & 8.6 & 5.5 & 38.0\\
1947 & 7.2 & 0.0 & 6.8 & 4.4 & 18.3\\
1948 & 9.6 & 0.0 & 13.6 & 8.7 & 31.9\\
1949 & 5.2 & 0.0 & 17.6 & 11.3 & 34.1\\
1950 & 4.1 & 0.0 & 21.5 & 13.8 & 39.3\\
1951 & 8.9 & 0.0 & 24.5 & 20.5 & 53.9\\
1952 & 5.9 & 0.0 & 21.3 & 17.8 & 45.1\\
1953 & 2.9 & 0.0 & 18.2 & 15.2 & 36.3\\
1954 & 5.5 & 0.0 & 22.6 & 18.9 & 46.9\\
1955 & 2.9 & 0.0 & 26.9 & 22.5 & 52.4\\
1956 & 4.9 & 0.0 & 30.1 & 25.1 & 60.1\\
1957 & 5.6 & 0.0 & 28.1 & 24.5 & 58.3\\
1958 & 6.5 & 0.0 & 52.4 & 40.3 & 99.2\\
1959 & 7.4 & 0.0 & 39.2 & 33.7 & 80.3\\
1960 & 10.0 & 0.0 & 32.3 & 26.1 & 68.3\\
1961 & 7.3 & 0.0 & 24.1 & 19.7 & 51.1\\
1962 & 5.2 & 0.0 & 27.1 & 31.3 & 63.6\\
1963 & 6.2 & 0.0 & 32.3 & 40.8 & 79.3\\
1964 & 4.2 & 0.0 & 22.5 & 44.0 & 70.7\\
1965 & 4.5 & 0.0 & 37.1 & 63.3 & 104.9\\
1966 & 5.5 & 0.0 & 40.8 & 74.8 & 121.0\\
1967 & 6.2 & 0.0 & 38.3 & 83.8 & 128.4\\
1968 & 3.3 & 0.0 & 37.6 & 95.1 & 136.0\\
1969 & 2.4 & 0.0 & 36.8 & 106.6 & 145.8\\
1970 & 2.5 & 0.0 & 53.7 & 125.0 & 181.2\\
1971 & 4.4 & 0.0 & 39.8 & 125.0 & 169.2\\
1972 & 6.9 & 0.0 & 60.9 & 147.5 & 215.2\\
1973 & 6.7 & 0.0 & 69.3 & 170.4 & 246.3\\
1974 & 15.7 & 0.0 & 70.4 & 184.3 & 270.4\\
1975 & 8.4 & 0.0 & 67.3 & 192.2 & 268.0\\
1976 & 15.9 & 0.0 & 69.5 & 211.1 & 296.5\\
1977 & 13.9 & 0.0 & 78.6 & 213.7 & 306.1\\
1978 & 2.5 & 0.0 & 62.3 & 216.7 & 281.5\\
1979 & 2.8 & 0.0 & 56.4 & 233.6 & 292.8\\
1980 & 39.6 & 0.0 & 55.1 & 210.4 & 305.2\\
1981 & 9.6 & 0.0 & 106.9 & 171.2 & 287.8\\
1982 & 12.9 & 0.0 & 106.7 & 164.4 & 284.0\\
1983 & 69.0 & 0.0 & 64.4 & 76.3 & 209.8\\
1984 & 43.2 & 0.0 & 49.0 & 92.9 & 185.1\\
1985 & 25.4 & 0.0 & 42.6 & 138.4 & 206.5\\
1986 & 10.4 & 0.0 & 47.6 & 106.9 & 165.0\\
1987 & 13.8 & 0.0 & 17.6 & 68.8 & 100.2\\
1988 & 17.9 & 0.0 & 25.5 & 69.2 & 112.7\\
1989 & 33.8 & 0.0 & 42.3 & 46.3 & 122.4\\
1990 & 43.3 & 0.0 & 28.5 & 61.4 & 133.2\\
1991 & 52.4 & 0.0 & 25.7 & 53.7 & 131.8\\
1992 & 71.3 & 0.0 & 24.7 & 46.0 & 142.0\\
1993 & 68.6 & 0.2 & 22.8 & 71.2 & 162.7\\
1994 & 25.4 & 6.0 & 17.1 & 44.9 & 93.5\\
1995 & 34.3 & 8.5 & 11.3 & 21.9 & 76.1\\
1996 & 36.5 & 17.3 & 10.3 & 19.9 & 84.0\\
1997 & 38.6 & 7.1 & 18.5 & 15.8 & 80.0\\
1998 & 23.2 & 5.3 & 5.2 & 11.1 & 44.9\\
1999 & 8.0 & 7.8 & 11.8 & 9.4 & 37.0\\
2000 & 2.9 & 4.8 & 19.8 & 4.2 & 31.6\\
2001 & 4.3 & 7.4 & 12.3 & 4.9 & 28.9\\
2002 & 3.2 & 6.2 & 10.3 & 2.1 & 21.8\\
2003 & 1.0 & 1.6 & 3.8 & 17.4 & 23.8\\
2004 & 1.3 & 2.0 & 6.5 & 9.1 & 18.9\\
2005 & 0.9 & 2.8 & 18.2 & 13.0 & 34.9\\
2006 & 0.8 & 2.2 & 16.8 & 16.5 & 36.2\\
2007 & 1.1 & 4.7 & 17.4 & 18.8 & 42.0\\
2008 & 1.0 & 4.0 & 9.8 & 17.0 & 31.8\\
2009 & 0.8 & 1.7 & 14.7 & 22.0 & 39.2\\
2010 & 0.6 & 1.1 & 14.3 & 11.5 & 27.5\\
2011 & 0.6 & 1.9 & 8.8 & 14.6 & 25.9\\
2012 & 0.9 & 2.3 & 12.2 & 19.5 & 34.9\\
2013 & 0.7 & 2.1 & 8.8 & 14.0 & 25.6\\
2014 & 0.7 & 2.5 & 16.1 & 17.6 & 36.9\\
2015 & 0.8 & 2.7 & 24.2 & 37.8 & 65.5\\
2016 & 0.8 & 2.6 & 28.7 & 34.2 & 66.3\\
2017 & 1.4 & 4.6 & 56.5 & 76.1 & 138.6\\
2018 & 3.0 & 6.4 & 44.0 & 49.0 & 102.4\\
2019 & 2.5 & 6.9 & 39.2 & 53.4 & 101.9\\
2020 & 3.9 & 7.5 & 59.6 & 85.1 & 156.2\\
2021 & 3.1 & 7.5 & 54.9 & 41.4 & 107.0\\
2022 & 1.2 & 1.9 & 11.5 & 32.5 & 47.1\\*
\end{longtable}
\endgroup{}
\endgroup{}
=======
\hfill\break

Biological data were extracted from the PacFIN Biological Data System on March 20, 2023. Length data for the commercial fleet were extracted from the PacFIN Biological Data System (BDS) with samples from north↨ of Point Conception beginning in 1978 (Tables \ref{tab:dead-com-len} and \ref{tab:live-com-len}). The commercial data was split by landed condition, live or dead, with the first data for the live fish fishery beginning in 1994. The number of length samples by fleet were highly variable with the largest number of samples by year being recorded in the 1990s for the dead fish fishery. In recent years, the number of length samples by year are limited for both fleets with annual sample sizes less than 100 per year. The number of samples prior to the 1990s and in the 2000s for the dead fish fishery were sparse and variable across sizes. During model explorations any years with less than 20 sampled fish were considered too sparse to accurately reflect the fleet selectivity for that year (see \protect\hyperlink{excluded-data}{Appendix A} for implied fits to these lengths).

The majority of lengths observed by the commercial fleet landing dead copper rockfish ranged between approximately 25 - 50 cm (Figure \ref{fig:com-dead-len-data}, detailed length compositions by year can be found in the Appendix, Section \ref{length-data}). Notably, fewer small fish were observed in the early years of data prior to 1990 compared to later years. The mean length observed by year ranged between approximately 30 - 45 cm (Figure \ref{fig:mean-com-dead-len-data}). The mean observed length since 2010 slowly increased through 2018 with a drop in the mean observed age in the most recent years data.

The observed distribution of sizes sampled from the commercial live fish fleet were generally variable prior to 2011 with the length distributions thereafter indicating a smaller range of sizes being landed (Figure \ref{fig:com-live-len-data}). The observed mean length of fish landed live also clearly shows a drop starting in 2011 (Figure \ref{fig:mean-com-live-len-data}).

The input sample sizes for all commercial data were calculated based on a combination of trips and fish sampled:

\begin{centering}

Input effN = $N_{\text{trips}} + 0.138 * N_{\text{fish}}$ if $N_{\text{fish}}/N_{\text{trips}}$ is $<$ 44

Input effN = $7.06 * N_{\text{trips}}$ if $N_{\text{fish}}/N_{\text{trips}}$ is $\geq$ 44

\end{centering}

\hypertarget{recreational-fishery}{%
\subsubsection{Recreational Fishery}\label{recreational-fishery}}

\hypertarget{landings-and-discards-1}{%
\paragraph{Landings and Discards}\label{landings-and-discards-1}}

\hfill\break

The recreational fishery is the main source of exploitation of copper rockfish across California. The recreational catches of copper rockfish south of Point Conception in California waters peaked in the late 1970s and early 1980s. Catches declined in the 1990s and early 2000s (Table \ref{tab:allcatches} and Figure \ref{fig:catch}). The removals remained relatively low until 2015. Catches began to increase in 2015, likely due to changes in harvest specifications (Cope et al. 2013). The catches decreased in 2020 due to COVID-19 impacts and remained relatively low in 2021 and 2022 due to reductions in the sub-bag limits in California for copper rockfish. The recreational fishery was split into two fleets based on fishing type (termed `modes'), a commercial passenger fishing vessel (CPFV, party/charter mode) fleet and a combined private or rental boats (PR mode). Shoreside fishing (man-made and beach/bank modes) was combined with the PR mode. The catches associated with the shoreside mode for copper rockfish are limited and did not justify a separate fishing fleet within the model.

Recreational landing estimates from 1928 to 1980 were obtained from the historical reconstruction (Ralston et al. 2010). The historical landings reconstruction split removals north and south of Point Conception and by recreational modes. CPFV landings of all rockfish were based on logbook data (which do not report rockfish to the species level), scaled by compliance estimates, while total recreational landings from PR vessels were based on a combination of the relative catch rates observed in the CPFV fleet and a linear ramp between catch estimates in the early 1960s and those in the early 1980s (as described in Ralston et al. (2010)). The species composition of rockfish landings was estimated using a combination of the 1980s Marine Recreational Fisheries Statistics Survey (MRFSS) data as well as limited CPFV mode species composition data from onboard observer programs in the late 1970s (south of Point Conception) and dockside recreational creel surveys in the late 1950s and early 1960s (north of Point Conception).

Recreational removals from 1981-1989 and 1993-2003 were obtained from MRFSS downloaded from the Recreational Fisheries Information Network (RecFIN). Historically, copper rockfish were occasionally referred to as whitebelly rockfish in select California areas. MRFSS catches were pulled for both species names and for all ocean areas. MRFSS includes estimates of removals for 1980. However, due to inconsistencies in the estimates of this year in MRFSS, likely due to it being the first year of the survey with low sample sizes, the value for recreational landings from the historical reconstruction were used (2010).

Some known issues with the MRFSS estimates include 1) a change in the spatial definition of California subregions after 1989, 2) missing or imprecise estimates of catch in weight for some strata that reported catch in numbers, and 3) a hiatus in sampling from 1990-1992 (all modes) and also 1993-1995 in the party/charter mode north of Point Conception. The STAT attempted to address each of these issues, as described below. CRFS estimates from 2004 were also included in the MRFSS analysis, as they were not available on the current RecFIN website but are included with the MRFSS catch estimate tables

The MRFSS definition of ``Southern California'' included San Luis Obispo County between 1981-1989, requiring the catches from this county to be split out and removed from the recreational catch south of Point Conception. The MRFSS catches between southern and northern California were adjusted in a similar fashion as previous assessments split at Point Conception. Albin et al. (1993) used MRFSS data to estimate catch at a finer spatial scale from the California/Oregon border to the southern edge of San Luis Obispo (SLO) County. Over the period 1981-1986, numbers of copper rockfish landed in SLO County were found to be approximately one third (0.317) of the numbers of copper rockfish landed in all California counties north of SLO County (Albin et al. 1993). Therefore, to approximate catches north and south of Point Conception from 1980-1989, the STAT reduced the `southern' subregion annual catch (which included SLO County) from 1980-1989 by 0.317 during the same period, and added this amount to the northern subregion catch. On average, this `moves' the estimated SLO County catch from the southern region to the northern region from 1980-1989, creating a spatially consistent time series of landings over the entire time series.

The STAT chose to use catch in terms of weight (WGT\_AB1 column) within MRFSS. The catch weights were converted from kilograms to metric tons and any records with missing catch weights were examined. The number of records with missing catch weights for copper rockfish in MRFSS were limited (only 18 out of 713). The missing catch weights were imputed based on the number of fish (TOT\_CAT column) and the calculated average fish weight by year and area north and south of Point Conception.

MRFSS sampling was halted from 1990-1992 due to funding issues. The survey resumed in 1993 in all modes, except for the PC boat mode which resumed in 1996 for counties north of Santa Barbara County. To produce catch estimates for the missing subregion, mode, and year combinations linear interpolations were used to fill in the missing data.

Two additional revisions were applied to select years and modes in the MRFSS data based on conversations with California Department of Fish and Wildlife (CDFW). The catches for the PR mode north of Point Conception in MRFSS for 1981 were 50 to 90 percent greater than the catches in 1980 and 1982, respectively. The high catches in this year were assumed to be a result of issues in the catch expansions due to limited sampling. The catches for the PR fleet were revised downward to be equal to the average removals in surrounding years (1979, 1980, 1982, and 1983). The catches in MRFSS south of Point Conception in 1987 were identified as abnormally low by CDFW (John Budrick, pers. communication, 13 to 27 percent of catches in 1986 and 1988) which was due to no catch information for waves 1-3 (January - June) for either mode. Absence of data in 1987 for these waves was not observed across other rockfish species in southern California indicating that the absence of catch data was likely not due to closures in the fishery. The catches for this year and mode were set equal to the average catch by mode 2 years before and after 1987.

Recreational landings from 2004-2022 were obtained from California Recreational Fisheries Survey (CRFS) available on RecFIN for for all ocean areas. This survey improves upon the MRFSS sampling design, employing higher sampling rates and producing estimates with finer spatial and temporal resolution. CRFS also employs onboard CPFV observers, providing spatially referenced, drift-level estimates of catch and discard for a subset of anglers on observed groundfish trips. Any CRFS records of fish caught in Mexican waters were removed and catch estimates were split north and south of Point Conception for each fleet. Due to database issues, catches for 2004 are currently not available on RecFIN. The catches for this year were set equal to data pulled in 2021 for the previous assessment of copper rockfish.

Adjustments to the recreational catches for 2020-2022 were provided directly by CDFW to deal with sampling issues due to COVID-19. During 2020, dockside sampling by observers was halted April through June leading to missing catch data within the CRFS database for this period. CDFW provided proxy catch values for these months directly by CRFS district (personal communication, Melanie Parker). The total proxy catches south of Point Conception (districts 1 and 2) for these months were 18.9 mt and 15.0 mt north of Point Conception in California (districts 3 - 6). These catches were split by mode (CPFV and PR) equally for both areas, noting that effort by mode during this period varied across district based on varying COVID-19 restrictions. When sampling resumed a large number of rockfish catches were not identified to species, and rather were recorded as rockfish genus, for the remainder of 2020 and 2021 due to social distancing for health and safety. The second adjustment to catches was to allocate some of those unidentified rockfish catches to specific species. CDFW provided proxy catch values that allocated a subset of the rockfish genus removals by recreational mode north and south of Point Conception for these years. Finally, the completed catch estimates for 2022 were not available within CRFS on RecFIN by the data deadline for this assessment and estimates were provided directly to the STAT from CDFW.

MRFSS and CRFS both provide estimates of total mortality which combine observed landings plus estimates of discarded fish using depth-dependent mortality rates. While the recreational removals from the historical reconstruction from 1928-1980 account for only landed fish. There is limited information on historical discarding in the recreational fishery. A report by Miller and Gotshall (1965) looked at the number of retained and discard fish in the recreational fishery in California for a select year which showed essentially no discarding of copper rockfish. Based on that no additional discards were applied to the historical data between 1926-1980.

\hypertarget{indices-of-abundance}{%
\paragraph{Indices of Abundance}\label{indices-of-abundance}}

\hfill\break

A number of indices of abundance were explored for the recreational fleet. Discarded catch is available from onboard observer surveys, but was not included in indices. Indices developed for the assessment include:

\begin{itemize}
\tightlist
\item
  MRFSS era dockside survey of the CPFV/PC fleet (1980-1999),
\item
  Deb-Wilson Vandenberg survey of the CPFV/PC fleet (1988-1998),
\item
  CDFW CPFV/PC onboard observer index (1999-2019), and
\item
  CRFS PR1 sites dockside survey (2004-2019).
\end{itemize}

Due to limited sampling during 2020 due to the COVID-19 pandemic and inseason action taken by CDFW for 2022 reducing sub-bag limits for copper rockfish across California, both recreational fishery indices of abundance excluded data collected after 2019.

From 1980 to 2003 the MRFSS program conducted dockside intercept surveys of the recreational CPFV fishing fleet. No MRFSS CPUE data are available for the years 1990-1992, due to a hiatus in sampling related to funding issues. Sampling of California CPFVs north of Point Conception was further delayed, and CPFV samples in 1993 and 1994 are limited to San Luis Obispo County. For the purposes of this assessment, the MRFSS time series was truncated at 1999 due to sampling overlap with the onboard observer program (i.e., the same observer samples the catch while onboard the vessel and also conducts the dockside intercept survey for the same vessel).The onboard observer data provide higher resolution data of retained and discarded catch.

Each entry in the RecFIN Type 3 database corresponds to a single fish examined by a sampler at a particular survey site. Since only a subset of the catch may be sampled, each record also identifies the total number of that species possessed by the group of anglers being interviewed. The number of anglers and the hours fished are also recorded. The data, as they exist in RecFIN, do not indicate which records belong to the same boat trip.

The data were filtered to identify rockfish trips, standardized across the time-series, and modeled to estimate an index of abundance for copper rockfish (see \protect\hyperlink{mrfss-cpfv-index}{Appendix B} for details). The MRFSS CPFV index of abundance between 1980-1999 is generally variable but with an overall flat trend in abundance except for three years that spike in abundance estimates (Figure \ref{fig:mrfss-index-main}). These same patterns in sharp increases in the index for select years were also observed in the raw data. Given the limited information in the data to understand what was behind these unexpected spikes in the time series, the MRFSS index of abundance was not used in the final base model.

The Deb Wilson-Vandenberg data set originated from an onboard observer survey conducted by CDFW survey in California north of Point Conception from 1988-1998 and is referred to as the Deb Wilson-Vandenberg onboard observer survey (\textbf{reilly-onboard-1998?}). During an observer trip the sampler rode along on the CPFV and recorded location-specific catch and discard information to the species level for a subset of anglers onboard the vessel. The subset of observed anglers is usually a maximum of 15 people and the observed anglers change during each fishing stop. The catch cannot be linked to an individual, but rather to a specific fishing location. The sampler also records the starting and ending time, number of anglers observed, starting and ending depth, and measured retained and discarded fish. The fine-scale catch and effort data allow us to better filter the data for indices to fishing stops within suitable habitat for the target species. See \protect\hyperlink{dwv-cpfv-index}{Appendix D} for details on the data filtering, processing, and model selection. The Deb Wilson-Vandenberg CPFV index of abundance increased between 1988-1990 and then generally declined between 1991-1998 (Figure \ref{fig:dwv-index-main}).

The state of California implemented a statewide onboard observer sampling program in 1999 (Monk et al. 2014). California Polytechnic State University (Cal Poly) has conducted an independent onboard sampling program as of 2003 for boats in Port San Luis and Morro Bay, and follows the protocols established in Reilly et al. (1998). During an onboard observer trip the sampler rides along on the CPFV and records location-specific catch and discard information to the species level for a subset of anglers onboard the vessel. The subset of observed anglers is usually a maximum of 15 people and the observed anglers change during each fishing stop.

The catch cannot be linked to an individual, but rather to a specific fishing location. The sampler also records the starting and ending time, number of anglers observed, starting and ending depth, and measures discarded fish. The fine-scale catch and effort data allow us to better filter the data for indices to fishing stops within suitable habitat for copper rockfish. Cal Poly has modified protocols to reflect sampling changes that CDFW has also adopted, e.g., observing fish as they are encountered instead of at the level of a fisher's bag. Therefore, the Cal Poly data are incorporated in the same index as the CDFW data from 1999-2019. The only difference is that Cal Poly measures the length of both retained and discarded fish.

The CRFS CPFV index of abundance was low in the early years of the time-series between 1999-2004 and then began to increase with variability among years until 2013, and declined in the final years (Figure \ref{fig:crfs-index-main}). See \protect\hyperlink{onboard-cpfv-index}{Appendix C} for details on the data filtering, processing, and model selection.

Catch and effort data from CRFS dockside sampling of private boats from 2004-2019 and 2021-2022 were provided by CDFW for use in this assessment. The data include catch (number of fish) by species, number of anglers (i.e., effort units are angler trips), angler-reported distance from shore (Area X: inside/outside of 3 nm), county, port, interview site, year, month, and CRFS district. Only data through 2019 were used to develop an index of abundance due to potential changes in angler behavior in 2021 and management changes in 2022. See \protect\hyperlink{crfs-pr-index}{Appendix D} for details on the data filtering, processing, and model selection. The CRFS PR index of abundance generally increased between 2004-2016, with the final years of the index stabilizing or slightly declining (Figure \ref{fig:crfs-pr-index-main}).

\hypertarget{composition-data-1}{%
\paragraph{Composition Data}\label{composition-data-1}}

\hfill\break

Length compositions were available from the following sources:

\begin{itemize}
\item
  Recreational party/charter mode (CPFV/PC)

  \begin{itemize}
  \tightlist
  \item
    Miller and Goshall dockside survey (1959-1961, 1966)
  \item
    Don Pearson onboard PC survey (1978-1984)
  \item
    MRFSS CPFV/PC dockside survey (1980-1989, 1993-2003)
  \item
    CRFS CPFV/PC onboard dockside survey (2004-2022)
  \item
    Deb Wilson-Vandenberg onboard CPFV survey (1988-1998)
  \end{itemize}
\item
  Recreational private/rental mode (PR)

  \begin{itemize}
  \tightlist
  \item
    Miller and Gotshall dockside PR survey (1959)
  \item
    MRFSS dockside PR survey (1980-1989, 1993-2003)
  \item
    CRFS dockside PR survey (2004-2022)
  \end{itemize}
\end{itemize}

The number of available fish and unique trips by year and fleet are in Tables \ref{tab:rec-len-samps}. MRFSS historical biological data were downloaded from the RecFIN website in December 2022. CRFS biological data were also downloaded from RecFIN on February 18, 2023. The Miller and Goshall, Don Person, and Deb-Wilson Vandenberg recreational survey data were downloaded from the SWFSC databases in February 2023.

Between 1987-1989 and 1993-1998 there were recreational length data for the CPFV fleet from both MRFSS and the Deb Wilson-Vandenberg data sets. During data exploration it was determined that the lengths in MRFSS from 1997 and 1998 were also included in the Deb Wilson-Vandenberg data, indicating that these data sources were duplicative for these years but also potentially other years where they overlapped. In order to avoid duplicate data, the length data from MRFSS, which had far fewer length samples for the overlapping years with Deb Wilson-Vandenberg for the CPFV fleet, were removed from the data used within the model (see \protect\hyperlink{excluded-data}{Appendix A} for implied fits to these lengths).

The majority of length samples for both recreational fleets, CPFV and PR, were unsexed. A wide range of sampled lengths from the recreational CPFV fleet were observed across all years with lengths generally ranging between 25 - 45 cm except for the late 1970s and early 1980s where a higher proportion of larger fish were sampled (Figure \ref{fig:rec-cpfv-len-data}). The mean of lengths observed in the recreational CPFV fleet since approximately 1990s has been relatively stable, varying between 35 - 40 cm, with high variability within the data in the early years (Figure \ref{fig:mean-rec-cpfv-len-data}). The range of lengths sampled from the recreational PR fleet are similar to those from the CPFV fleet with lengths in recent years ranging between 25 - 45 cm with a slightly larger proportion of larger fish observed in the 1980s (Figures \ref{fig:rec-pr-len-data} and \ref{fig:mean-rec-pr-len-data}).

Age composition data were available for select years from both the recreational CPFV and PR fleets. Historical age data collected from the CPFV fishery were available from this fleet from 1978, 1981, and 1984. The majority of these fish were sexed (only 4 total unsexed ages from 1978 and 1984) with an average age ranging from 10 to 14 across these years (Figures \ref{fig:rec-cpfv-caal-data} and \ref{fig:mean-rec-cpfv-age-data}). The historical age data from this fleet were input as marginal ages. There were a total of 250 age samples from the final model year, 2022, collected by a cooperative sampling program with the fleet coordinated by the SWFSC (Figures \ref{fig:rec-cpfv-caal-data} and ADD MARGINAL FIGURE). These data were collected by three CPFV vessels that operate north of Point Conception following random sampling protocols. The cooperative ages were compared to all the CPFV lengths collected by the CRFS sampling program to ensure that the sampling was representative of the fishery (Figure \ref{fig:coop-len-comparison} North). These ages were incorporated as either marginal or conditional age-at-length data depending upon how fish length was measured: carcass or whole fish. The carcass lengths were included as marginals in order to avoid any potential measurement bias in the use of these ages. Finally, a total of 139 ages were collected from the PR fleet in 2022 (Figure \ref{fig:rec-pr-caal-data}). These data were used in the model as conditional age-at-length data as well.

The approach to determine the number of unique trips by data source varied. Some data sources had unique trip numbers within the data (Don Pearson, Deb Wilson Vandenberg). Other data sources that lacked a clear trip identifier applied a similar methodology as developed by Brian Soper that combines multiple fields of information to attempt to estimate trips sampled. The number of trips for MRFSS data was estimated using the year, wave, ID code, sampling site (INSITE), area, and mode. A similar methodology was done for CRFS and Miller and Gotshall data that used data, county, water area, interview site, and mode.

\hypertarget{fishery-independent-data}{%
\subsection{Fishery-Independent Data}\label{fishery-independent-data}}

Two fishery-independent data sources with indices of abundance were included in the base model. These surveys sampled rocky habitat across the area north of Point Conception (Figure \ref{fig:survey-locations}) in both areas opens to fishing (termed reference areas) and Marine Protected Areas (MPAs, Figure \ref{fig:ref-mpa}).

\hypertarget{california-cooperative-fisheries-research-program-survey}{%
\subsubsection{California Cooperative Fisheries Research Program Survey}\label{california-cooperative-fisheries-research-program-survey}}

\hypertarget{index-of-abundance}{%
\paragraph{Index of Abundance}\label{index-of-abundance}}

\hfill\break

Since 2007, the \gls{s-ccfrp} has monitored several areas in California to evaluate the performance of \glspl{mpa} and understand nearshore fish populations (Wendt and Starr 2009; Starr et al. 2015b). In 2017, the survey expanded beyond the four \Gls{mpa}s in central California (Año Nuevo, Point Lobos, Point Buchon, and Piedras Blancas) to include the entire California coast. Fish are collected by volunteer anglers aboard \glspl{cpfv} guided by one of the following academic institutions based on proximity to fishing location: Humboldt State University; Bodega Marine Laboratories; Moss Landing Marine Laboratories; Cal Poly San Luis Obispo; University of California, Santa Barbara; and Scripps Institution of Oceanography.

Surveys consist of fishing with hook-and-line gear for 30-45 minutes within randomly chosen 500 by 500 m grid cells within and outside \glspl{mpa}. Prior to 2017, all fish were measured for length and released or descended to depth; since then, some were sampled for otoliths and fin clips.

The CCRFP Hook and Line survey is one the longest fishery-independent time series available north of Point Conception for nearshore rockfish, having occurred annually between 2007-2022. See \protect\hyperlink{ccfrp-index}{Appendix F} for details on the data filtering, processing, and model selection for these data. The estimated index of abundance was weighted based sample locations outside (reference) and inside MPAs (73 and 80 percent of areas open to fishing in the north and south, respectively, see Appendix Section \ref{cdfw-rov-index} for additional information). The estimated index of abundance was variable but generally flat between 2007-2016, increased in 2017 when sampling locations expanded across the northern coast with an increasing trend up until 2020, and slightly declined in the final two years in the dataset (Figure \ref{fig:ccfrp-index-main}). To account for the range expansion of this survey in areas with potentially larger copper rockfish and/or areas with lower fishing pressure starting in 2017, a time block in selectivity was assumed in the model.

\hypertarget{composition-data-2}{%
\paragraph{Composition Data}\label{composition-data-2}}

~

Length measurements were available for 2007-2022 from the CCFRP survey north of Point Conception and age data were collected between 2017-2022 (Table \ref{tab:ccfrp-samps}). The length data by designation, MPA and Reference, were weighted based on the estimated rocky habitat within each designation north of Point Conception (80 percent of areas open to fishing). The lengths observed by the survey ranged between 25-50 cm across the sample years with the mean length observed ranging between 33-40 cm (Figures \ref{fig:ccfrp-len-data} and \ref{fig:ccfrp-mean-len-data}). The survey collected age data from a subset of fish sampled between 2017-2022 (Figure \ref{fig:ccfrp-age-data}). The read ages from these sampled fish ranged between 2-33 years of age.

\hypertarget{california-department-of-fish-and-wildlife-remotely-operated-vehicle-survey}{%
\subsubsection{California Department of Fish and Wildlife Remotely Operated Vehicle Survey}\label{california-department-of-fish-and-wildlife-remotely-operated-vehicle-survey}}
>>>>>>> Stashed changes

\newpage

\begingroup\fontsize{10}{12}\selectfont
\begingroup\fontsize{10}{12}\selectfont

<<<<<<< Updated upstream
\begin{longtable}[t]{c>{\centering\arraybackslash}p{2cm}>{\centering\arraybackslash}p{2cm}>{\centering\arraybackslash}p{2cm}}
\caption{\label{tab:ca-management}The portion of the Overfishing Limit (OFL) and Annual Catch Limit (ACL) and estimated catch in California waters.}\\
\toprule
Year & OFL (mt) & ACL (mt) & Catch (mt)\\
\midrule
\endfirsthead
\caption[]{\label{tab:ca-management}The portion of the Overfishing Limit (OFL) and Annual Catch Limit (ACL) and estimated catch in California waters. \textit{(continued)}}\\
\toprule
Year & OFL (mt) & ACL (mt) & Catch (mt)\\
\midrule
\endhead

\endfoot
\bottomrule
\endlastfoot
2012 & 163.15 & 136.17 & 85.95\\
2013 & 148.00 & 123.42 & 105.18\\
2014 & 148.00 & 123.42 & 98.65\\
2015 & 303.75 & 277.32 & 147.64\\
2016 & 286.88 & 261.95 & 165.27\\
2017 & 313.70 & 286.38 & 225.48\\
2018 & 319.60 & 291.85 & 203.69\\
2019 & 325.08 & 296.83 & 182.59\\
2020 & 330.35 & 301.60 & 242.73\\
2021 & 249.85 & 206.43 & 164.20\\
2022 & 249.48 & 204.02 & 66.67\\*
\end{longtable}
\endgroup{}
\endgroup{}

\begingroup\fontsize{10}{12}\selectfont
\begingroup\fontsize{10}{12}\selectfont

\begin{longtable}[t]{r>{\centering\arraybackslash}p{2cm}>{\centering\arraybackslash}p{2cm}}
\caption{\label{tab:com-ratio}Ratio estimates of total rockfish landings north and south of Point Conception. "Ratio years" are the range of years over which ratio estimates were calculated. Sources include the NMFS SWFSC ERD Live Access Server and several volumes of the CDFG Fish Bulletin series.}\\
\toprule
Year & Ratio & Ratio Years\\
\midrule
\endfirsthead
\caption[]{Ratio estimates of total rockfish landings north and south of Point Conception. "Ratio years" are the range of years over which ratio estimates were calculated. Sources include the NMFS SWFSC ERD Live Access Server and several volumes of the CDFG Fish Bulletin series. \textit{(continued)}}\\
\toprule
Year & Ratio & Ratio Years\\
\midrule
\endhead

\endfoot
\bottomrule
\endlastfoot
1916 & 0.33 & 1928-33\\
1917 & 0.33 & 1928-33\\
1918 & 0.33 & 1928-33\\
1919 & 0.33 & 1928-33\\
1920 & 0.33 & 1928-33\\
1921 & 0.33 & 1928-33\\
1922 & 0.33 & 1928-33\\
1923 & 0.33 & 1928-33\\
1924 & 0.33 & 1928-33\\
1925 & 0.33 & 1928-33\\
1926 & 0.33 & 1928-33\\
1927 & 0.33 & 1928-33\\
1928 & 0.33 & 1949-51\\
1929 & 0.33 & 1949-51\\
1930 & 0.33 & 1949-51\\
1931 & 0.33 & 1949-51\\
1932 & 0.33 & 1949-51\\
1933 & 0.33 & 1949-51\\
1934 & 0.33 & 1949-51\\
1935 & 0.33 & 1949-51\\
1936 & 0.33 & 1949-51\\
1937 & 0.33 & 1949-51\\
1938 & 0.33 & 1949-51\\
1939 & 0.33 & 1949-51\\
1940 & 0.33 & 1949-51\\
1941 & 0.33 & 1949-51\\
1942 & 0.33 & 1949-51\\
1943 & 0.33 & 1949-51\\
1944 & 0.33 & 1949-51\\
1945 & 0.33 & 1949-51\\
1946 & 0.33 & 1949-51\\
1947 & 0.33 & 1949-51\\
1948 & 0.33 & 1949-51\\
1949 & 0.30 & data\\
1950 & 0.19 & data\\
1951 & 0.44 & data\\
1952 & 0.46 & 1949-51\\
1953 & 0.31 & 1954-57\\
1954 & 0.14 & data\\
1955 & 0.01 & data\\
1956 & 0.06 & data\\
1957 & 0.10 & data\\
1958 & 0.14 & 1954-57\\
1959 & 0.24 & 1954-57\\
1960 & 0.23 & 1954-57\\
1961 & 0.44 & 1954-57\\
1962 & 0.28 & data\\
1963 & 0.25 & data\\
1964 & 0.19 & data\\
1965 & 0.37 & data\\
1966 & 0.27 & data\\
1967 & 0.38 & data\\
1968 & 0.46 & data\\*
\end{longtable}
\endgroup{}
\endgroup{}

=======
The California Department of Fish and Wildlife (CDFW), in collaboration with Marine Applied Research and Exploration (MARE), have been conducting remotely operated vehicle (ROV) surveys along the California coast in Marine Protected Areas (MPAs) and reference sites adjacent to them since 2004 for the purposes of long-term monitoring of changes in size, density (fish/square meter) and length of fish and invertebrate species along the California coast. Surveys of the entire coast have now been undertaken twice, each taking three years to complete, 2014-2016 and again in 2019-2021. The survey conducted multiple 500 meter transects across rocky reef survey sites. Sample sites were selected by first randomly selecting the deepest transect at a given site, then selecting transects on a constant interval into shallower depths. Transects were designed to be oriented parallel to general depth contours, though they were carried out using a fixed bearing that crossed depths in some cases.

The data were explored using a super year approach where the central years, 2015 and 2020, were designated as the super year and the data were split north and south of Point Conception. The effort of the survey was split roughly equally between sites that were within MPAs or areas open to fishing (referred to as ``Reference'') with sampling across most sites (termed ``MPA group'') within each super year. The number of transects and the number of copper rockfish by site and super year are shown in Table \ref{tab:rov-obs} with the number of observations across all years by location shown in Figure \ref{fig:rov-obs-loc}. The trend in the calculated catch-per-unit-effort based on the data alone was highly variable across sampling locations and by MPA or Reference area (Figure \ref{fig:rov-raw-cpue}).

CDFW provided an initial analysis of the CDFW ROV survey data which helped in determination of which modeling approaches would be considered for these data. The final selected model assumed a negative binomial error structure with covariates for super year, site designation (MPA or Reference), and super year site designation interaction. The estimated index of abundance was weighted based on the proportion of hard substrate either in MPA (20 percent) or Reference (80 percent) areas in California north of Point Conception. The weighted index of abundance increases between the two super years 2015 and 2020 (Figure \ref{fig:rov-index-main}). Details regarding the index of abundance, sample sizes, and model selection can be found in the Appendix Section \ref{cdfw-rov-index}.
>>>>>>> Stashed changes

\newpage

\begingroup\fontsize{10}{12}\selectfont
\begingroup\fontsize{10}{12}\selectfont

\begin{table}[t]{r>{\centering\arraybackslash}p{2cm}>{\centering\arraybackslash}p{2cm}}
\caption{\label{tab:dead-com-len}Summary of the number of trips and length samples for fish landed dead by commercial fisheries.}\\
\toprule
Year & Trips & Lengths\\
\midrule
\endfirsthead
\caption[]{Summary of the number of trips and length samples for fish landed dead by commercial fisheries. \textit{(continued)}}\\
\toprule
Year & Trips & Lengths\\
\midrule
\endhead

\endfoot
\bottomrule
\endlastfoot
1978 & 1 & 2\\
1979 & 3 & 26\\
1980 & 4 & 34\\
1981 & 2 & 4\\
1982 & 3 & 6\\
1983 & 5 & 13\\
1984 & 2 & 25\\
1985 & 1 & 1\\
1986 & 1 & 2\\
1987 & 2 & 2\\
1988 & 3 & 4\\
1990 & 2 & 2\\
1991 & 6 & 126\\
1992 & 106 & 662\\
1993 & 169 & 808\\
1994 & 85 & 334\\
1995 & 66 & 255\\
1996 & 87 & 348\\
1997 & 28 & 116\\
1998 & 16 & 32\\
1999 & 58 & 336\\
2000 & 6 & 36\\
2001 & 5 & 10\\
2002 & 2 & 8\\
2003 & 3 & 21\\
2004 & 3 & 14\\
2005 & 1 & 13\\
2007 & 1 & 5\\
2008 & 2 & 5\\
2009 & 3 & 7\\
2010 & 1 & 1\\
2011 & 5 & 7\\
2012 & 7 & 11\\
2013 & 3 & 3\\
2014 & 4 & 4\\
2015 & 3 & 4\\
2016 & 11 & 22\\
2017 & 9 & 14\\
2018 & 7 & 26\\
2019 & 8 & 53\\
2020 & 14 & 56\\
2021 & 19 & 59\\
2022 & 17 & 79\\
\end{table}
\endgroup{}
\endgroup{}


<<<<<<< Updated upstream
\begingroup\fontsize{10}{12}\selectfont
\begingroup\fontsize{10}{12}\selectfont

\begin{longtable}[t]{r>{\centering\arraybackslash}p{2cm}>{\centering\arraybackslash}p{2cm}}
\caption{\label{tab:live-com-len}Summary of the number of trips and length samples for fish landed live by commercial fisheries.}\\
\toprule
Year & Trips & Lengths\\
\midrule
\endfirsthead
\caption[]{Summary of the number of trips and length samples for fish landed live by commercial fisheries. \textit{(continued)}}\\
\toprule
Year & Trips & Lengths\\
\midrule
\endhead

\endfoot
\bottomrule
\endlastfoot
1994 & 3 & 8\\
1995 & 4 & 8\\
1996 & 7 & 15\\
1998 & 5 & 5\\
1999 & 23 & 123\\
2000 & 26 & 34\\
2001 & 39 & 223\\
2002 & 6 & 21\\
2003 & 3 & 6\\
2004 & 5 & 22\\
2005 & 1 & 6\\
2006 & 2 & 4\\
2007 & 18 & 69\\
2008 & 15 & 67\\
2009 & 11 & 20\\
2010 & 19 & 31\\
2011 & 21 & 44\\
2012 & 18 & 51\\
2013 & 11 & 25\\
2014 & 6 & 20\\
2015 & 12 & 18\\
2016 & 18 & 25\\
2017 & 16 & 56\\
2018 & 21 & 76\\
2019 & 18 & 31\\
2020 & 15 & 29\\
2021 & 9 & 62\\
2022 & 9 & 40\\*
\end{longtable}
\endgroup{}
\endgroup{}


\begingroup\fontsize{10}{12}\selectfont
\begingroup\fontsize{10}{12}\selectfont

\begin{longtable}[t]{r>{\centering\arraybackslash}p{1.57cm}>{\centering\arraybackslash}p{1.57cm}>{\centering\arraybackslash}p{1.57cm}>{\centering\arraybackslash}p{1.57cm}>{\centering\arraybackslash}p{1.57cm}>{\centering\arraybackslash}p{1.57cm}}
\caption{\label{tab:rec-len-samps}Summary of the recreational length samples by source for the CPFV and PR fleets.}\\
\toprule
Area & Year & Source & CPFV Trips & CPFV Samples & PR Trips & PR Samples\\
\midrule
\endfirsthead
\caption[]{Summary of the recreational length samples by source for the CPFV and PR fleets. \textit{(continued)}}\\
\toprule
Area & Year & Source & CPFV Trips & CPFV Samples & PR Trips & PR Samples\\
\midrule
\endhead

\endfoot
\bottomrule
\endlastfoot
North & 1959 & MILLER & 1 & 202 & 4 & 337\\
North & 1960 & MILLER & 4 & 715 & - & -\\
North & 1961 & MILLER & 2 & 8 & - & -\\
North & 1966 & MILLER & 2 & 20 & - & -\\
North & 1978 & DON PEARSON & 98 & 343 & - & -\\
North & 1979 & DON PEARSON & 75 & 233 & - & -\\
North & 1980 & DON PEARSON & 115 & 199 & - & -\\
North & 1980 & MRFSS & 53 & 92 & 125 & 286\\
North & 1981 & DON PEARSON & 53 & 92 & - & -\\
North & 1981 & MRFSS & 61 & 172 & 91 & 188\\
North & 1982 & DON PEARSON & 78 & 148 & - & -\\
North & 1982 & MRFSS & 41 & 59 & 118 & 310\\
North & 1983 & DON PEARSON & 55 & 98 & - & -\\
North & 1983 & MRFSS & 50 & 82 & 109 & 209\\
North & 1984 & DON PEARSON & 40 & 102 & - & -\\
North & 1984 & MRFSS & 79 & 193 & 122 & 216\\
North & 1985 & MRFSS & 110 & 175 & 148 & 314\\
North & 1986 & MRFSS & 138 & 248 & 152 & 257\\
North & 1987 & DEB WILSON-VANDENBERG & 15 & 26 & - & -\\
North & 1987 & MRFSS & 23 & 67 & 56 & 134\\
North & 1988 & DEB WILSON-VANDENBERG & 92 & 551 & - & -\\
North & 1988 & MRFSS & 39 & 57 & 41 & 94\\
North & 1989 & DEB WILSON-VANDENBERG & 130 & 824 & - & -\\
North & 1989 & MRFSS & 89 & 187 & 39 & 68\\
North & 1990 & DEB WILSON-VANDENBERG & 44 & 378 & - & -\\
North & 1991 & DEB WILSON-VANDENBERG & 49 & 272 & - & -\\
North & 1992 & DEB WILSON-VANDENBERG & 126 & 735 & - & -\\
North & 1993 & DEB WILSON-VANDENBERG & 136 & 977 & - & -\\
North & 1993 & MRFSS & 27 & 37 & 234 & 428\\
North & 1994 & DEB WILSON-VANDENBERG & 130 & 530 & - & -\\
North & 1994 & MRFSS & 22 & 29 & 140 & 270\\
North & 1995 & DEB WILSON-VANDENBERG & 148 & 725 & - & -\\
North & 1995 & MRFSS & 32 & 59 & 62 & 92\\
North & 1996 & DEB WILSON-VANDENBERG & 120 & 457 & - & -\\
North & 1996 & MRFSS & 134 & 194 & 56 & 76\\
North & 1997 & DEB WILSON-VANDENBERG & 142 & 554 & - & -\\
North & 1998 & DEB WILSON-VANDENBERG & 84 & 252 & - & -\\
North & 1999 & MRFSS & 140 & 191 & 35 & 53\\
North & 2000 & MRFSS & 53 & 85 & 14 & 19\\
North & 2001 & MRFSS & 72 & 94 & 9 & 18\\
North & 2002 & MRFSS & 82 & 107 & 18 & 20\\
North & 2003 & MRFSS & 87 & 107 & 45 & 60\\
North & 2004 & CRFS & 65 & 179 & 130 & 396\\
North & 2005 & CRFS & 60 & 348 & 251 & 880\\
North & 2006 & CRFS & 77 & 414 & 322 & 1354\\
North & 2007 & CRFS & 149 & 677 & 303 & 1284\\
North & 2008 & CRFS & 90 & 411 & 279 & 1125\\
North & 2009 & CRFS & 96 & 487 & 268 & 994\\
North & 2010 & CRFS & 99 & 532 & 232 & 826\\
North & 2011 & CRFS & 126 & 416 & 266 & 912\\
North & 2012 & CRFS & 134 & 558 & 277 & 884\\
North & 2013 & CRFS & 144 & 533 & 319 & 1245\\
North & 2014 & CRFS & 136 & 583 & 350 & 1327\\
North & 2015 & CRFS & 148 & 528 & 453 & 2397\\
North & 2016 & CRFS & 134 & 641 & 415 & 2184\\
North & 2017 & CRFS & 157 & 1088 & 478 & 2904\\
North & 2018 & CRFS & 128 & 806 & 450 & 2226\\
North & 2019 & CRFS & 142 & 716 & 458 & 2099\\
North & 2021 & CRFS & 79 & 246 & 254 & 1014\\
North & 2022 & CRFS & 98 & 247 & 420 & 1278\\*
\end{longtable}
\endgroup{}
\endgroup{}

=======
CDFW provided annual fish length measurements made from images taken with stereo-cameras in 2014, 2015, 2016, 2019, 2020, and 2021 (Figure \ref{fig:rov-len}). The ROV was equipped with two locally recorded stereo cameras mounted in a fixed orientation in parallel with the primary forward-facing camera used for navigation, identification and enumeration. Paired stereo video cameras were calibrated to provide accurate feature measurements by MARE prior to in-field data collection. Fish length was determined based on a photogrammetric intersection to calculate 3D coordinates from measured image coordinates to measure fork length to the nearest millimeter. Processors selected two 3D points, fish head and tail, on both time-synced port and starboard stereo images. Processors made note of an generated precision value that was computed alongside each measurement. The software then computed the Euclidean distance between the two 3D points to give a length measurement. Sizes were rounded to the nearest centimeter. The precision of estimates was determined and represents the repeatability and geometric quality of a measurement, and is dependent upon the location of the individual in the camera view and the clarity of the point measured. Precision values greater than 10 mm were recorded in the database alongside measurements, as precision values less than 10 mm meant measurements were deemed fully repeatable. Fish observations from initial species scoring were marked ``not sizable'' if the nose/tail of the fish was cut off in either stereo view, the fish was missing in one view, or the orientation of the fish obscured measurement or produced an extreme precision measurement. Fish were marked ``not visible'' if the fish was not visible in either the port or starboard image. Fish marked ``not sizable'' and ``not visible'' were removed from the final data set submitted by CDFW.

Explorations around how to use the length data within the base model were conducted (super period, combined externally into two years, by year). These explorations indicated similar fit to these data across the alternative approaches and entering them by year was selected in order to understand the fit to these data relative to the model expectations by year. The length data were weighted by MPA/Reference area and were grouped by the collection year (Figures \ref{fig:rov-len-data} and \ref{fig:mean-rov-len-data}). The input sample size was set equal to the weighted number of unique transects north of Point Conception.
>>>>>>> Stashed changes

\begingroup\fontsize{10}{12}\selectfont
\begingroup\fontsize{10}{12}\selectfont

\begin{longtable}[t]{r>{\centering\arraybackslash}p{2cm}>{\centering\arraybackslash}p{2cm}>{\centering\arraybackslash}p{2cm}}
\caption{\label{tab:ccfrp-samps}The total number of drifts, length, and age samples collected by year from the CCFRP survey north of Point Conception.}\\
\toprule
Year & Drifts & Lengths & Ages\\
\midrule
\endfirsthead
\caption[]{The total number of drifts, length, and age samples collected by year from the CCFRP survey north of Point Conception. \textit{(continued)}}\\
\toprule
Year & Drifts & Lengths & Ages\\
\midrule
\endhead

\endfoot
\bottomrule
\endlastfoot
2007 & 60 & 92 & 0\\
2008 & 70 & 88 & 0\\
2009 & 67 & 92 & 0\\
2010 & 52 & 73 & 0\\
2011 & 60 & 78 & 0\\
2012 & 76 & 108 & 0\\
2013 & 53 & 70 & 0\\
2014 & 109 & 163 & 0\\
2015 & 30 & 43 & 0\\
2016 & 114 & 214 & 0\\
2017 & 117 & 230 & 7\\
2018 & 185 & 335 & 20\\
2019 & 201 & 403 & 27\\
2020 & 182 & 340 & 11\\
2021 & 193 & 355 & 4\\
2022 & 181 & 393 & 45\\*
\end{longtable}
\endgroup{}
\endgroup{}


<<<<<<< Updated upstream
\begingroup\fontsize{10}{12}\selectfont
\begingroup\fontsize{10}{12}\selectfont

\begin{longtable}[t]{r>{\centering\arraybackslash}p{2.2cm}>{\centering\arraybackslash}p{2.2cm}>{\centering\arraybackslash}p{2.2cm}>{\centering\arraybackslash}p{2.2cm}}
\caption{\label{tab:rov-obs}Number of transects and number of observations of copper rockfish for each group and survey year.}\\
\toprule
Super Year & Area & Designation & Transects & Observations\\
\midrule
\endfirsthead
\caption[]{Number of transects and number of observations of copper rockfish for each group and survey year. \textit{(continued)}}\\
\toprule
Super Year & Area & Designation & Transects & Observations\\
\midrule
\endhead

\endfoot
\bottomrule
\endlastfoot
2015 & Ano Nuevo & MPA & 4 & 0\\
2020 & Ano Nuevo & MPA & 10 & 7\\
2015 & Big Creek & MPA & 3 & 3\\
2020 & Big Creek & MPA & 4 & 4\\
2015 & Bodega Bay & MPA & 28 & 11\\
2020 & Bodega Bay & MPA & 45 & 84\\
2015 & Montara & MPA & 11 & 4\\
2020 & Montara & MPA & 19 & 8\\
2015 & Piedras Blancas & MPA & 8 & 6\\
2020 & Piedras Blancas & MPA & 8 & 11\\
2015 & Pillar Point & MPA & 4 & 1\\
2020 & Pillar Point & MPA & 8 & 7\\
2015 & Point Arena & MPA & 7 & 7\\
2020 & Point Arena & MPA & 12 & 41\\
2015 & Point Buchon & MPA & 7 & 4\\
2020 & Point Buchon & MPA & 14 & 17\\
2015 & Point Lobos & MPA & 15 & 11\\
2020 & Point Lobos & MPA & 31 & 110\\
2015 & Point St. George & MPA & 21 & 27\\
2020 & Point St. George & MPA & 17 & 17\\
2015 & Point Sur & MPA & 14 & 20\\
2020 & Point Sur & MPA & 22 & 74\\
2015 & Portuguese Ledge & MPA & 6 & 30\\
2020 & Portuguese Ledge & MPA & 11 & 24\\
2015 & Reading Rock & MPA & 14 & 4\\
2020 & Reading Rock & MPA & 17 & 17\\
2015 & SE Farallon Islands & MPA & 12 & 18\\
2020 & SE Farallon Islands & MPA & 22 & 58\\
2015 & Sea Lion Gulch & MPA & 12 & 0\\
2020 & Sea Lion Gulch & MPA & 21 & 16\\
2015 & Ten Mile & MPA & 20 & 30\\
2020 & Ten Mile & MPA & 17 & 51\\
2015 & Ano Nuevo & Reference & 5 & 0\\
2020 & Ano Nuevo & Reference & 9 & 3\\
2015 & Big Creek & Reference & 20 & 54\\
2020 & Big Creek & Reference & 8 & 35\\
2015 & Bodega Bay & Reference & 16 & 3\\
2020 & Bodega Bay & Reference & 32 & 48\\
2015 & Montara/Pillar Point & Reference & 8 & 0\\
2020 & Montara/Pillar Point & Reference & 20 & 3\\
2015 & Point Arena & Reference & 8 & 8\\
2020 & Point Arena & Reference & 12 & 7\\
2015 & Point Buchon & Reference & 8 & 4\\
2020 & Point Buchon & Reference & 12 & 8\\
2015 & Point Lobos & Reference & 8 & 2\\
2020 & Point Lobos & Reference & 22 & 13\\
2015 & Point St. George & Reference & 14 & 3\\
2020 & Point St. George & Reference & 13 & 3\\
2015 & Point Sur & Reference & 8 & 3\\
2020 & Point Sur & Reference & 17 & 8\\
2015 & Portuguese Ledge & Reference & 6 & 9\\
2020 & Portuguese Ledge & Reference & 8 & 11\\
2015 & Reading Rock & Reference & 19 & 21\\
2020 & Reading Rock & Reference & 17 & 26\\
2015 & SE Farallon Islands & Reference & 13 & 1\\
2020 & SE Farallon Islands & Reference & 16 & 8\\
2015 & Sea Lion Gulch & Reference & 9 & 5\\
2020 & Sea Lion Gulch & Reference & 16 & 18\\
2015 & Ten Mile & Reference & 18 & 28\\
2020 & Ten Mile & Reference & 19 & 16\\*
\end{longtable}
\endgroup{}
\endgroup{}

=======
A significant amount of additional length-at-age data not associated with fishery fleets or surveys incorporated in the model were available for copper rockfish. These independent age data collection efforts were derived from four programs north of Point Conception since 2001: 227 otoliths collected by the NWFSC WCGBT survey, 430 otoliths collected by a research survey conducted by Don Pearson, 45 otoliths from CDFW special collections, and 77 otoliths collected by Jeff Abrams research program, and (Table \ref{tab:growth-age-samps}). The ages collected by these four sources were included in the model as a ``growth'' fleet that was not associated with removals or an index of abundance. These collections had a wide distribution of lengths and ages observed (Figures \ref{fig:growth-len-dist} and \ref{fig:growth-age-dist}).
>>>>>>> Stashed changes

\begingroup\fontsize{10}{12}\selectfont
\begingroup\fontsize{10}{12}\selectfont

\begin{table}[t]{r>{\centering\arraybackslash}p{2cm}>{\centering\arraybackslash}p{2cm}}
\caption{\label{tab:growth-age-samps}Number of ages by year and source used as conditional-age-at-length data to inform estimation of growth.}\\
\toprule
Year & Source & Ages\\
\midrule
\endfirsthead
\caption[]{Number of ages by year and source used as conditional-age-at-length data to inform estimation of growth. \textit{(continued)}}\\
\toprule
Year & Source & Ages\\
\midrule
\endhead

\endfoot
\bottomrule
\endlastfoot
2001 & Pearson Research & 3\\
2002 & Pearson Research & 68\\
2003 & Pearson Research & 260\\
2004 & NWFSC WCGBT & 49\\
2004 & Pearson Research & 82\\
2005 & NWFSC WCGBT & 9\\
2005 & Pearson Research & 13\\
2006 & NWFSC WCGBT & 7\\
2007 & NWFSC WCGBT & 1\\
2008 & NWFSC WCGBT & 25\\
2009 & NWFSC WCGBT & 6\\
2010 & Abrams & 27\\
2010 & NWFSC WCGBT & 10\\
2011 & Abrams & 47\\
2012 & NWFSC WCGBT & 4\\
2013 & NWFSC WCGBT & 8\\
2014 & NWFSC WCGBT & 16\\
2015 & NWFSC WCGBT & 10\\
2016 & NWFSC WCGBT & 2\\
2017 & NWFSC WCGBT & 11\\
2018 & CDFW & 3\\
2018 & NWFSC WCGBT & 12\\
2019 & CDFW & 27\\
2019 & NWFSC WCGBT & 10\\
2021 & CDFW & 15\\
2021 & NWFSC WCGBT & 14\\
2022 & NWFSC WCGBT & 13\\
\end{table}
\endgroup{}
\endgroup{}


\begingroup\fontsize{10}{12}\selectfont
\begingroup\fontsize{10}{12}\selectfont

\begin{longtable}[t]{l>{\raggedright\arraybackslash}p{2.2cm}>{\raggedright\arraybackslash}p{2.2cm}>{\raggedright\arraybackslash}p{2.2cm}>{\raggedright\arraybackslash}p{2.2cm}}
\caption{\label{tab:pisco-data}All and filtered observations by year and sampling institution for PISCO.}\\
\toprule
Year & UCSC Raw Count & UCSC Filtered Count & UCSB Raw Count & UCSB Filtered Count\\
\midrule
\endfirsthead
\caption[]{\label{tab:pisco-data}All and filtered observations by year and sampling institution for PISCO. \textit{(continued)}}\\
\toprule
Year & UCSC Raw Count & UCSC Filtered Count & UCSB Raw Count & UCSB Filtered Count\\
\midrule
\endhead

\endfoot
\bottomrule
\endlastfoot
1999 & 2 & NA & 7 & NA\\
2000 & 1 & NA & 11 & NA\\
2001 & 6 & 4 & 4 & NA\\
2002 & 25 & 21 & 8 & NA\\
2003 & 34 & 25 & 73 & NA\\
2004 & 30 & 9 & 65 & 19\\
2005 & 40 & 6 & 45 & 18\\
2006 & 27 & 12 & 51 & 25\\
2007 & 17 & 4 & 58 & 19\\
2008 & 21 & 5 & 44 & 22\\
2009 & 20 & 7 & 60 & 29\\
2010 & 34 & 10 & 85 & 32\\
2011 & 36 & 1 & 44 & 20\\
2012 & 9 & 4 & 77 & 39\\
2013 & 40 & 17 & 59 & 23\\
2014 & 50 & 28 & 50 & 39\\
2015 & 51 & 16 & 18 & 15\\
2016 & 29 & 17 & 51 & 45\\
2017 & 30 & 11 & 28 & 22\\
2018 & 37 & 15 & 42 & 28\\
2019 & 26 & 15 & 41 & 37\\
2020 & 58 & 26 & 35 & 29\\
2021 & 23 & 12 & 37 & 27\\*
\end{longtable}
\endgroup{}
\endgroup{}

\begingroup\fontsize{10}{12}\selectfont
\begingroup\fontsize{10}{12}\selectfont

\begin{longtable}[t]{r>{\centering\arraybackslash}p{1.83cm}>{\centering\arraybackslash}p{1.83cm}>{\centering\arraybackslash}p{1.83cm}>{\centering\arraybackslash}p{1.83cm}>{\centering\arraybackslash}p{1.83cm}}
\caption{\label{tab:wcgbt-pos-tows}The total number of tows between 55-183 m,  the number of positive tows, the total number of copper rockfish observed, and the number of lengths and agec collected north of Point Conception in California by the NWFSC WCGBT survey.}\\
\toprule
Year & Tows & Positive Tows & Numbers & Lengths & Ages\\
\midrule
\endfirsthead
\caption[]{The total number of tows between 55-183 m,  the number of positive tows, the total number of copper rockfish observed, and the number of lengths and agec collected north of Point Conception in California by the NWFSC WCGBT survey. \textit{(continued)}}\\
\toprule
Year & Tows & Positive Tows & Numbers & Lengths & Ages\\
\midrule
\endhead

\endfoot
\bottomrule
\endlastfoot
2003 & 73 & 4 & 12 & 12 & 0\\
2004 & 75 & 4 & 49 & 49 & 49\\
2005 & 97 & 2 & 9 & 9 & 9\\
2006 & 79 & 2 & 7 & 7 & 7\\
2007 & 80 & 1 & 1 & 1 & 1\\
2008 & 93 & 5 & 25 & 25 & 25\\
2009 & 100 & 5 & 6 & 6 & 6\\
2010 & 103 & 5 & 10 & 10 & 10\\
2011 & 102 & 0 & 0 & 0 & 0\\
2012 & 106 & 3 & 4 & 4 & 4\\
2013 & 74 & 3 & 8 & 8 & 8\\
2014 & 91 & 1 & 23 & 23 & 16\\
2015 & 98 & 4 & 10 & 10 & 10\\
2016 & 91 & 1 & 2 & 2 & 2\\
2017 & 93 & 2 & 11 & 11 & 11\\
2018 & 93 & 5 & 12 & 12 & 12\\
2019 & 48 & 3 & 10 & 10 & 10\\
2021 & 101 & 7 & 14 & 14 & 14\\
2022 & 90 & 5 & 13 & 13 & 13\\*
\end{longtable}
\endgroup{}
\endgroup{}


\begingroup\fontsize{10}{12}\selectfont
\begingroup\fontsize{10}{12}\selectfont

\begin{longtable}[t]{r>{\centering\arraybackslash}p{6cm}}
\caption{\label{tab:model-structure}Specifications and structure of the base model.}\\
\toprule
Model Setup & Base Model\\
\midrule
\endfirsthead
\caption[]{Specifications and structure of the base model \textit{(continued)}}\\
\toprule
\textbackslash{}underline\{Model Setup\} & Base Model\\
\midrule
\endhead

\endfoot
\bottomrule
\endlastfoot
Starting year & 1916\\
 \vphantom{3} \vphantom{2} \vphantom{1} & \\
\underline{Population characteristics} & \\
Maximum age & 50\\
Gender & 2\\
Population lengths & 4-58 cm by 1 cm bins\\
Summary biomass (mt) & Age 3+\\
 & \\
\underline{Data characteristics} & \\
Data lengths & 10-54 cm by 2 cm bins\\
Data ages & 0-50 ages\\
Minimum age for growth calculations & 2\\
Maximum age for growth calculations & 20\\
First mature age & 0\\
Starting year of estimated recruitment in main period & 1970\\
 & \\
\underline{Fishery characteristics} & \\
Fishing mortality method & Hybrid F\\
Maximum F & 3.5\\
Catchability & Analytical estimate\\
Commercial Dead Selectivity & Length-Based Double Normal\\
Commercial Live Selectivity & Length-Based Double Normal\\
Recreational CPFV Selectivity & Length-Based Double Normal\\
Recreational PR Selectivity & Length-Based Double Normal\\
CCFRP Selectivity & Length-Based Double Normal\\
CDFW ROV Selectivity & Length-Based Double Normal\\
Growth Selectivity & Age-Based Double Normal\\
 & \\
\underline{Fishery time blocks} & \\
Commercial Live & 1916-2010, 2011-2022\\
Recreational CPFV & 1916-2001, 2002-2016, 2017-2022\\
Recreational PR & 1916-1999, 2000-2022\\*
\end{longtable}
\endgroup{}
\endgroup{}


\begingroup\fontsize{9}{11}\selectfont

\begin{landscape}\begingroup\fontsize{9}{11}\selectfont

<<<<<<< Updated upstream
=======
The PISCO program conducts larval fish recruitment monitoring by sampling artificial settlement substrates called Standard Monitoring Units for Recruitment of Fishes (SMURFs). Similar to the SCUBA surveys, SMURF surveys are conducted by multiple universities using standardized protocols. We examined data collected by the UCSB and UCSC campuses in southern and central California. Surveys by UCSB were conducted between 2000 and 2018 and by UCSC between 1999 and 2016. Juvenile copper rockfish are difficult to distinguish from juvenile gopher rockfish (\emph{Sebastes carnatus}) and the data from UCSB combines counts of these species into a complex. For this reason, we determined this data to be inappropriate for construction of a copper rockfish recrutiment index to be used within the assessment. While data from UCSC reports distinct counts of copper and gopher rockfish, the concern remains that the copper rockfish counts may not be accurate due to this morphological identification difficulty. Additionally, collections of copper rockfish in this data set were very sparse with many years seeing none. However, an extremely high number were recorded for 2016.

\hypertarget{reef-check}{%
\subsubsection{Reef Check}\label{reef-check}}

Reef Check is an international non-profit organization leading citizen scientists to monitor reef habitats. Data from SCUBA surveys of fish in California are available since 2006. Given the low proportion of copper rockfish detections on PISCO surveys, we did not dedicate time to analysis of Reef Check survey data for the purpose of abundance index d evelopment. However, given the wide distribution of Reef Check survey sites, the data may warrant further exploration for future assessments.\\
\textbf{Reef Check doesn't identify anything less than 10 cm and I don't think speciates - check on this}

\hypertarget{northwest-fisheries-science-center-west-coast-groundfish-bottom-trawl-survey}{%
\subsubsection{Northwest Fisheries Science Center West Coast Groundfish Bottom Trawl Survey}\label{northwest-fisheries-science-center-west-coast-groundfish-bottom-trawl-survey}}

The Northwest Fisheries Science Center (NWFSC) West Coast Groundfish Bottom Trawl (WCGBT) survey is based on a random-grid design; covering the coastal waters from a depth of 55-1,280 m (Bradburn et al. 2011). This design generally uses four industry-chartered vessels per year assigned to a roughly equal number of randomly selected grid cells and divided into two `passes' of the coast. Two vessels fish from north to south during each pass between late May to early October. Therefore, this design incorporates both vessel-to-vessel differences in catchability, as well as variance associated with selecting a relatively small number (approximately 700) of possible cells from a very large set of possible cells spread from the Mexican to the Canadian borders.

The observations of copper rockfish by the NWFSC WCGBT survey are limited. The NWFSC WCGBT survey uses trawl gear to sample sandy bottom areas off the West Coast and \emph{a priori} it would not be expected to be an informative data source for copper rockfish, which are generally more closely associated with rock substrate. The NWFSC WCGBT survey had very limited positive tows by year that observed copper rockfish within this area, preventing the calculation of an index of abundance for copper rockfish (Table \ref{tab:wcgbt-pos-tows}). The catch-per-unit-effort across all years for the NWFSC WCGBT survey is low across all years (Figure \ref{fig:wcgbt-cpue}). The observations of copper rockfish by the NWFSC WCGBT survey commonly occur between 50 - 120 meters (Figure \ref{fig:wcgbt-depth}). The NWFSC WCGBT survey has regularly collected length and age samples from positive tows for copper rockfish north of Point Conception (Figure \ref{fig:wcgbt-len-age}). These data were used as conditional-age-at-length data to inform the estimation of growth within the model. See the \protect\hyperlink{length-at-age}{Length-at-Age} section for data used to inform growth estimation.

\hypertarget{biological-data}{%
\subsection{Biological Data}\label{biological-data}}

\hypertarget{natural-mortality}{%
\subsubsection{Natural Mortality}\label{natural-mortality}}

Natural mortality was not directly measured, so life-history based empirical relationships were used. The Natural Mortality Tool (NMT), a Shiny-based graphical user interface allowing for the application of a variety of natural mortality estimators based on measures such as longevity, size, age and growth, and maturity, was used to obtain estimates of natural mortality. The NMT currently provides 19 options, including the Hamel (2022) method, which is a corrected form of the Then et al. (2015) functional regression model and is a commonly applied method for West Coast groundfish. The NMT also allows for the construction of a natural mortality prior weighted across methods by the user.

The Hamel (2022) method for developing a prior on natural mortality for West Coast groundfish stock assessments combines meta-analytic approaches relating the \(M\) rate to other life-history parameters such as longevity, size, growth rate, and reproductive effort to provide a prior for \(M\). The Hamel (2022) method re-evaluated the data used by Then et al. (2015) by fitting the one-parameter \(A_{\text{max}}\) model under a log-log transformation (such that the slope is forced to be -1 in the transformed space (Hamel 2015), the point estimate and median of the prior for \(M\) is:

\begin{centering}

$M=\frac{5.4}{A_{\text{max}}}$

\end{centering}

\vspace{0.5cm}

where \(A_{\text{max}}\) is the maximum age. The prior is defined as a lognormal distribution with mean \(ln(5.4/A_{\text{max}})\) and standard error = 0.31. Using a maximum age of 50, the point estimate and median of the prior is 0.108 yr\textsuperscript{-1}. The maximum age was selected based on available age data from all West Coast data sources and literature values. The oldest aged copper rockfish observed in California waters was 52 years of age sampled in 2020 in northern California with 15 additional fish aged to be 40 years and older across all data sources.

The maximum age in the model was set at 50 years. This selection was consistent with the literature examining the longevity of copper rockfish within California (Love 1996) and was supported by the observed ages that had multiple observations of fish between 40 and 52 years of age.

\hypertarget{maturation-and-fecundity}{%
\subsubsection{Maturation and Fecundity}\label{maturation-and-fecundity}}

Maturity-at-length was based on maturity reads conducted by Melissa Head at the NWFSC examining a total of 112 samples (18 north of Point Conception and 94 south of Point Conception) collected across California by the NWFSC Hook and Line survey and the NWFSC WCGBT surveys in September and October. Given the limited sample size north of Point Conception, all samples were pooled across California to inform maturity north of Point Conception, while only samples south of Point Conception were used to inform maturity in this region.

The maturity-at-length curve is based on an estimate of functional maturity rather than biological maturity. Biological maturity can include multiple behaviors that functional maturity will exclude (e.g., abortive maturation and skip spawning). Biological maturity indicates that some energy reserves were used to create vitellogenin, but it does not mean that eggs will continue to develop and successfully spawn. This includes juvenile abortive maturation. Female rockfish commonly go through the first stages of spawning the year before they reach actual spawning capability. This is most likely a factor related to their complicated reproductive process of releasing live young. A subset of oocytes will develop early yolk, and then get aborted during the spawning season. Biological maturity also does not account for the proportion of oocytes in atresia (cellular breakdown and reabsorption), which means that fish that were skipping spawning for the season could be listed as biologically mature and functionally immature (Melissa Head, personal communication, NWFSC, NOAA).

The 50 percent size-at-maturity was estimated at 34 cm with a slope of -0.41 (Figure \ref{fig:maturity}). This area-specific maturity-at-length estimate is relatively similar but with fish maturing at a slightly larger size compared to the biological maturity curve assumed for copper rockfish south of Point Conception. Additionally, these values are both slightly smaller compared to estimates by Hannah (2014) for fish observed in Oregon waters (34.8 cm) which estimated the 50 percent size-at-maturity of and slope of -0.60.

The fecundity-at-length was based on research from Dick et al. (2017). The fecundity relationship for copper rockfish was estimated to be equal to 3.362e-07\(L\)\textsuperscript{3.68} in millions of eggs where \(L\) is length in cm. Fecundity-at-length is shown in Figure \ref{fig:fecundity}.

\hypertarget{sex-ratio}{%
\subsubsection{Sex Ratio}\label{sex-ratio}}

There were limited sex-specific observations by length or age of young fish across biological data sources. The NWFSC WCGBT survey had the highest frequency of small fish observed. However, many of the small fish observed by the survey were too small for sex determination (Figure \ref{fig:frac-sex-len}). In the absence of evidence of a differential sex ratio at birth the sex ratio of young fish was assumed to be 1:1.

\hypertarget{length-weight-relationship}{%
\subsubsection{Length-Weight Relationship}\label{length-weight-relationship}}

The length-weight relationship for copper rockfish was estimated outside the model using all coastwide biological data available from fishery-independent data from the NWFSC WCGBT and the NWFSC Hook and Line surveys. The estimated length-weight relationship for female fish was W = 9.6e-06\(L\)\textsuperscript{3.19} and males 1.11e-05\(L\)\textsuperscript{3.15} where \(L\) is length in cm and W is weight in kilograms (Figure \ref{fig:weight-length}).

\hypertarget{length-at-age}{%
\subsubsection{Growth (Length-at-Age)}\label{length-at-age}}

Length-at-age was estimated for male and female copper rockfish informed by age data from the fisheries, the CCFRP survey, and independent age data collection efforts from four programs north of Point Conception since 2002: 207 otoliths collected by the NWFSC WCGBT survey, 426 otoliths collected by a research survey conducted by Don Pearson, 74 from a research survey conducted by Abrams, and 45 from CDFW special collections (Table \ref{tab:growth-age-samps}). The ages collected by these sources were included in the model as a ``growth'' fleet that was not associated with removals or an index of abundance.

Sex-specific growth parameters \texttt{area} were initially estimated external to the model at the following values:

\begin{centering}

Females $L_{\infty}$ = 48.5 cm; $L_0$ = 9.1 cm; $k$ = 0.174 per year

Males $L_{\infty}$ = 46.8 cm; $L_0$ = 5.3 cm; $k$ = 0.207 per year

\end{centering}

\vspace{0.50cm}

These values were used as starting parameter values within the base model prior to estimating each parameter for male and female copper rockfish.

\hypertarget{ageing-precision-and-bias}{%
\subsubsection{Ageing Precision and Bias}\label{ageing-precision-and-bias}}

Uncertainty surrounding the age-reading process for copper rockfish was incorporated by estimating ageing error by age. Age composition data used in the model were from break-and-burn otolith reads. Aged copper rockfish used in the assessment were aged by the Cooperative Ageing Project (CAP) in Newport, Oregon. Within-lab ageing error was estimated by the CAP based on one primary age reader and a second reader producing double reads from 875 otoliths provided by the CAP lab (Figure \ref{fig:age-error-dist}).

An ageing error estimate was made based on these double reads using a computational tool specifically developed for estimating ageing error (Punt et al. 2008) and using release 1.1.0 of the R package \href{https://github.com/nwfsc-assess/nwfscAgeingError}{nwfscAgeingError} (Thorson et al. 2012) for input and output diagnostics. A linear standard error was estimated by age where there is more variability in the age of older fish (Figures \ref{fig:age-error} and \ref{fig:age-error-matrix}). Sensitivities to alternative ageing error estimates (curvilinear relationship with age) were conducted during model development and the model was relatively insensitive to alternative ageing error assumptions.

\hypertarget{environmental-and-ecosystem-data}{%
\subsection{Environmental and Ecosystem Data}\label{environmental-and-ecosystem-data}}

This assessment did not explicitly incorporate environmental data.

\hypertarget{assessment-model}{%
\section{Assessment Model}\label{assessment-model}}

\hypertarget{summary-of-previous-assessments-and-reviews}{%
\subsection{Summary of Previous Assessments and Reviews}\label{summary-of-previous-assessments-and-reviews}}

\hypertarget{history-of-modeling-approaches}{%
\subsubsection{History of Modeling Approaches}\label{history-of-modeling-approaches}}

Copper rockfish was first assessed in 2013 (Cope et al. 2013) using extended depletion-based stock reduction analysis (XDB-SRA), a data-moderate approach, which incorporated catch and index data with priors on select parameters (natural mortality, stock status in a specified year, productivity, and the relative status of maximum productivity). Copper rockfish was assessed as two separated stocks, split north and south of Point Conception where the population north of Point Conception included the population off California, Oregon, and Washington. The 2013 assessment estimated the stock south of Point Conception at 75 percent of unfished spawning output and the stock north of Point Conception at 48 percent of unfished spawning output.

Copper rockfish was last assessed in 2021 using a length-based data moderate assessment approach that included catch, fishery independent index data, and length composition data (Wetzel et al. 2021a, 2021b). The 2021 assessment estimated \(R_0\) and select selectivity parameters with fixed growth and deterministic annual recruitment. The 2021 assessments comprised four regional assessment models for copper rockfish with two model-areas within California split north and south of Point Conception. The estimated stock status in 2021 for the portion of the population south of Point Concept was 18 percent of unfished spawning output, while the California portion of the population north of Point Conception was 39 percent of unfished spawning output.

\hypertarget{most-recent-star-panel-and-ssc-recommendations}{%
\subsubsection{Most Recent STAR Panel and SSC Recommendations}\label{most-recent-star-panel-and-ssc-recommendations}}

This is the first benchmark assessment for Copper rockfish off the coast of California. The previous assessment of this species was a data-moderate assessment conducted in 2021 that were reviewed by the Scientific and Statistical Committee. The following items were identified at that time for future assessments of copper rockfish to consider:

\textbf{Issue}: The model for Northern California estimated a pattern of high recruitment during the 1960s and lower recruitment during the 1970s, which is not consistent with trends in the recruitment for other rockfishes during that time.

\textbf{Response}: The estimated recruitment deviations for the model area north of Point Conception in California for this assessment also estimates a similar pattern despite the additional of additional historical recreational length and ages.

\textbf{Issue}: Concerns were raised regarding the declining trend in the recent time period of the Southern California model, which is inconsistent with population trends from other southern California stocks for which data are available (e.g., bocaccio, cowcod), most of which have seen signs of strong recruitment over the past decade.

\textbf{Response}: The previous data-moderate assessment that incorporated catch, length, and survey indices was unable to estimate annual recruitment deviations in the south of Point Conception model due to lack of information in the data to inform these estimates. This assessment included additional data sources including available age data that supported the estimation of annual recruitment. The south of Point Conception model estimated high recruitment since 2010 similar to trends observed for other rockfish species that have been recently assessed (bocaccio, vermilion/sunset rockfish). Estimates of recruitment were not compared to the most recent cowcod assessment since this model did not estimate annual recruitment deviations.

\textbf{Issue}: Age-length estimates (and hence the growth curve) for northern California may not be representative because they rely on data from Oregon and Washington where water temperatures are different and growth may differ as a result.

\textbf{Response}: Available age data from a range of sources were included within each sub-area model to support area-specific growth for copper rockfish. The majority of the age data that were available to support estimation of growth within the model in the area north of Point Conception (e.g., otoliths collected by the CPFV fleet within a cooperative sampling program coordinated by the SWFSC) were not available for consideration in 2021.

\textbf{Issue}: The fit to the {[}NWFSC{]} hook-and-line survey in the Southern California assessment was poor. This likely reflects differences in the composition from the fishery disproportionately reflecting areas open to fishing closer to port as compared to the more spatially balanced sampling of the survey, more equally representing habitat offshore and in the Cowcod Conservation Areas (CCAs) and in the Rockfish Conservation Areas (RCAs).

\textbf{Response}: It is important to note that the 2021 assessment of copper rockfish south of Point Conception did not estimate annual recruitment deviations which likely limited the ability to fit the variable trends in the index of abundance from the NWFSC Hook and Line survey. However, the NWFSC Hook and Line survey data did appear to see the largest proportion of larger sizes compared to the other surveys and was the only survey with asymptotic selectivity. This is likely due to the sampling locations that would require overnight trips to access from many mainland ports.

\textbf{Issue}: California Department of Fish and Wildlife (CDFW) quantified the percent of habitat in Marine Protected Areas (MPAs), CCAs and RCAs, along with charts for further consideration to make clear the amount of habitat that is not represented in recent years. Data from the recreational fishery only represents areas open to fishing, potentially making the stock appear more depleted than it is as a whole. Two-area models, estimates of biomass from recently reviewed CDFW remotely operated vehicle (ROV) surveys, and inclusion of the California Collaborative Fisheries Research Program that sample in MPAs can be incorporated in future assessments to help reflect differences in composition and fishing mortality in open and closed areas. Additional data to represent the composition in closed areas would be beneficial.

\textbf{Response}: Data from the CDFW ROV survey were not available for consideration in 2021. Additionally, estimates of the percent of habitat within and outside of MPAs and CCAs were provided by CDFW the data of the SSC review in 2021 which precluded their consideration for how to process other available data or model sensitivities for copper rockfish in 2021. This assessment was able to include survey data from multiple sources that do sampling inside and outside of MPAs in California: the CDFW ROV, the CCFRP Hook and Line, and the NWFSC Hook and Line (only available south of Point Conception) surveys In order to properly weight composition data and abundance data collected within and outside MPAs estimates of rocky habitat were developed for the area south of Point Conception from partial seafloor mapping data (see Appendix Section \ref{cdfw-rov-index} for detailed information). The area north of Point Conception has complete seafloor mapping data which has been used to inform data weighting as was done in the 2021 assessment of vermilion/sunset rockfish.

\hypertarget{response-to-groundfish-subcommittee-requests}{%
\subsubsection{Response to Groundfish Subcommittee Requests}\label{response-to-groundfish-subcommittee-requests}}

To be completed post-STAR panel.

\hypertarget{model-structure-and-assumptions}{%
\subsection{Model Structure and Assumptions}\label{model-structure-and-assumptions}}

\hypertarget{modeling-platform-and-structure}{%
\subsubsection{Modeling Platform and Structure}\label{modeling-platform-and-structure}}

The assessment was conducted used Stock Synthesis version 3.30.21 developed by Dr.~Richard Methot at the NOAA, NWFSC (Methot and Wetzel 2013). This most recent version was used because it included improvements and corrections to older model versions. The previous assessment of copper rockfish also used Stock Synthesis but an earlier version, 3.30.16; model bridging was performed between both versions of Stock Synthesis and discussed below. The R package \href{https://github.com/r4ss/r4ss}{r4ss}, version 1.38.0, along with R version 4.0.1 were used to investigate and plot model fits.

\hypertarget{model-selection-and-evaluation}{%
\subsubsection{Model Selection and Evaluation}\label{model-selection-and-evaluation}}

The base assessment model for copper rockfish was developed to balance parsimony and realism, and the goal was to estimate a spawning output trajectory for the population of copper rockfish off the west coast of the U.S. The model contains many assumptions to achieve parsimony and uses many different sources of data to estimate reality. A series of investigative model runs were done to achieve the final base model.

\hypertarget{model-changes-from-the-last-assessment}{%
\subsubsection{Model Changes from the Last Assessment}\label{model-changes-from-the-last-assessment}}

The assessment model structure has substantially changed to the structure used in the 2021 assessment. The 2021 assessment was a length-based data-moderate assessment which, per the Terms of Reference, assumes a more simplified structure and limited data sources. The changes between the 2021 and the assessment are described below.

First, the fishery fleets were broken out into four specific fleets: commercial fishery that lands dead fish, commercial fishery that lands fish for the live market, recreational fishery CPFV vessels, and private/rental recreational anglers. This more disaggregated fleet structure allowed the model structure to account for varying selectivity and historical management actions that may have impacted the fishery and the available data in specific ways.

This assessment also included two additional survey datasets that were not included in the 2021 assessment: the CCFRP and the CDFW ROV surveys. Both of these surveys collect data in areas that are open to fishing and MPAs. These data were weighted according the estimates of the percentage of rocky habitat area within open and MPA areas. Using these data in the assessment allow it to account for dynamics that may not be reflected in the fishery data alone and give a more informed picture of the whole population.

A major change relative to the 2021 assessment is the estimation of biological parameters. Since the 2021 assessment was a length-based data-moderate, growth parameters were externally estimated and fixed within the model. This assessment estimates all growth parameters in the model except for the \(L_1\) parameter which was fixed in the model north of Point Conception at the estimated values due to limited observations of length-at-age of young fish leading to high uncertainty within the model. Additionally, the maturity-at-length parameters were updated from the values use in the 2021 assessment. Maturity estimates conducted by Melissa Head (NWFSC) done in 2021 were re-evaluated limiting the samples to the fall months when copper rockfish are preparing for spawning. The updated length at 50 percent maturity was marginally smaller than the values used in the 2021 assessment.

\hypertarget{bridging-analysis}{%
\subsubsection{Bridging Analysis}\label{bridging-analysis}}

The exploration of models began by bridging from the 2021 data-moderate assessment to Stock Synthesis version 3.30.21, which produced the same estimates for spawning biomass and depletion across the time series (Figures \ref{fig:bridge-ssb} and \ref{fig:bridge-depl}). Additional bridging analysis was conducted examining the impact on a revised model structure and updating and adding new data into the model. First, the 2021 fishery fleet structure was modified from the 2021 structure where the new assessment separated commercial data into two fleets based on fish landed condition, dead or live, and the recreational data into two fleets, CPFV and PR. The 2021 recreational and commercial data were reprocessed into the new model structure through 2021 and new selectivity parameters were added to the 2021 for the newly split data. The new data available in for this assessment were then added to the model retaining the same model structure where feasible in the following order:

\begin{enumerate}
\def\labelenumi{\arabic{enumi}.}
\tightlist
\item
  Update externally estimated biology parameters for length-at-age, weight-at-age, and maturity.
\item
  Add new catch data for all fishery fleets.
\item
  Add all updated commercial and recreational length and age data.
\item
  Add the new fishery-dependent indices of abundance.
\item
  Add the CDFW ROV survey index of abundance and length data.
\item
  Add the CCFRP index of abundance, length, and age data.
\item
  Add selectivity blocks for the commercial and recreational fishing fleets.
\item
  Adjust the estimation of annual recruitment deviations.
\item
  Add conditional-age-at-length data for the growth fleet and estimate growth parameters for both sexes.
\end{enumerate}

The data bridging are shown in Figures \ref{fig:data-bridge-ssb-1}-\ref{fig:data-bridge-depl-2}. Revising the model structure, updating biology, and removals resulted in small changes to the estimated spawning output and stock status (Figures \ref{fig:data-bridge-ssb-1} and \ref{fig:data-bridge-depl-1}). Updating and adding the fishery lengths, ages, and indices resulted in a less depleted final population at the end of the time-series. Adding and updating survey data, adding selectivity blocks, and estimating annual recruitment deviations and growth resulted in only minimal revisions in the population estimates (Figures \ref{fig:data-bridge-ssb-2} and \ref{fig:data-bridge-depl-2}). Adjusting the annual recruitment deviations (years estimates and bias adjustment) resulted in a small decline in end spawning biomass and stock status. The final bridging step that added the condition-age-at-length data for the growth fleet and allowing the estimation of growth resulte in an increase in spawning output and stock status at the end of the time-series.

To arrive at a final base model additional revisions to the model structure, selectivity blocks, revised selectivity parameterization were done in order to determine the best fit to the data.

\hypertarget{key-assumptions-and-structural-choices}{%
\subsubsection{Key Assumptions and Structural Choices}\label{key-assumptions-and-structural-choices}}

A decision was made by the STAT after discussions with the Pacific Fishery Management Council's Groundfish Management Team and Groundfish Advisory Panel to model the areas north and south of Point Conception independently for a number of reasons. These included a discussion of the evidence of a change in growth with latitude and the fundamental differences in the fisheries north and south of Point Conception. The preliminary exploration of length data also suggested that the size composition of landed fish north and south of Point Conception differed in both the commercial and recreational fleets. The STAT maintained consistency across the two models when the data supported the decisions, i.e., maintaining the same recreational and commercial fleet structures and sharing biological data across models when appropriate.

The specifications of the assessment are listed in Table \ref{tab:model-structure}. The structure of the California models north and south of Point Conception are very similar. Population dynamics in both regions operate on an annual time step and are initialized from an unfished equilibrium condition in 1916. The model is a two-sex, age-structured model with an accumulated age group at 50 years. Growth and natural mortality were assumed time invariant with constant growth estimated and natural mortality fixed at the median of the prior for both sexes. Sex-specific age and length structure is modeled from age 0 (recruitment age) to an accumulator age (plus group) of 50, with 1-cm population length bins ranging from 10-54 cm in the south and 10-58 cm in the north. Length data bins are 2-cm wide, and range from 10-54 cm. Expected recruitment is assumed to follow a Beverton-Holt function of spawning output, with lognormally-distributed recruitment deviations.

Stock Synthesis estimates growth in the age and size plus group. To avoid issues with additional estimated growth in the plus groups, the selection of the maximum age and length bins were selected to ensure that the numbers of fish in the plus group would be low. Growth (male and female) is modeled using the Schnute parameterization of von Bertalanffy growth, with two estimated lengths (ages 2 and 20) and a growth rate coefficient (\emph{k}). The major differences between the two models are the availability of fishery-independent data sources that are region-specific, and the parameterization of male growth and mortality parameters (details below).

The models in both regions are conditioned on catches from the commercial and recreational sectors. The commercial sector is divided into two fleets, one representing fish landed for the live-fish market and the second representing all other landings. The recreational fleets were divided into two groups according to boat mode: CPFV (party/charter) and PR (private/rental/shoreside) and includes both estimated retained and discarded catch.

Copper rockfish is a desirable species and discards are a small component of total fishing mortality in both the commercial and recreational sectors. Estimated discards based on West Coast Groundfish Observer Program were used to estimate discard mortality for both commercial fleets and were added to the landings to estimate total catch. The size distribution of recreational discards from the CDFW and Cal Poly onboard observer programs represented smaller fish than those retained prior to the one fish sub-bag for copper rockfish enacted in January 2023. The estimates of discard mortality available in RecFIN were combined with retained catch to estimate total recreational landings.

The northern California model is fit to two fishery-dependent indices of relative abundance: 1) CDFW CRFS onboard observer survey, and 2) CDFW CRFS private/rentals PR1 dockside survey. Additionally, a MRFSS CPFV dockside index was generated but due to data concerns was not included in the base model. The MRFSS CPFV dockside index is assumed to be proportional to changes in the relative abundance of the recreational party/charter fleet and represents retained fish only. The CDFW onboard observer index represents the same recreational party/charter fleet and includes both retained and discarded fish. The onboard index is specified as a separate ``survey'' fleet in the model because it overlaps in time with the MRFSS dockside time series. Both the MRFSS and onboard indices use the recreational party/charter fleet's selectivity curve to define vulnerable size classes. The CRFS PR1 dockside index is linked to the recreational private/rental boat fleet, and had selectivity curve different from the party/charter fleet.

Recreational length measurements are included as marginal length compositions (proportions at length, sexes combined) by year starting in 1959 for both the CPFV and PR modes. Fishery-dependent length composition data are also included for the commercial fleets starting in 1980 for the dead fishery and 1994 for the commercial live fleet. There were limited historical age data available for copper rockfish. Age data available from the CPFV fleet in 1978, 1981, and 1984 were included as marginal ages in the base model. In recent years, age data was avaialble from the commercial dead, CPFV, and PR fleets and were input as conditional-age-at-length data, except for a subset of ages based on filleted length from the CPFV fleet in 2022 which were used as marginal ages.

Fishery-independent data sources in the southern California model are organized into two fleets: 1) CDFW ROV survey and 2) CCFRP Hook and Line survey. Each of the survey data sources were used to create an index of relative abundance and included marginal length compositions by sex and year. Age data from the CCFRP Hook and Line survey were included as conditional-age-at-length data by sex and year.

Additional available age structures that could not be linked to one of the fleets above or represented a subset of information from a fleet were included in a growth fleet. The NWFSC WCGBT survey was not considered for an index of abundance, but is a source of conditional-age-at-length data and associated marginal length comps, both by sex and year. Additional age data collected by Adams and Pearson Research studies were inlcude in the growth fleet. Selectivity by the growth fleet was age-based and estimated to have full selectivity starting at age-1.

Time blocks on selectivity were explored extensively when setting up the initial model structure. A range of management changes to the commercial fishery were considered when determining periods of when selectivity may have been expected to change. Commercial removals for copper rockfish are relatively low when compared to recreational removals for this species which limited the amount of composition data available to support estimation of changes in selectivity. Given this and the limited evidence in the available length data, time-invariant selectivity was selected for the commercial dead fleet. A single time block was assumed in the commercial live fleet due to a sharp decline in the mean size observed in this fleet starting in 2010. The STAT contacted various participants in the live fish fishery to determine what regulations and/or market changes may have created this shift in targeting but no clear cause was identified.

The same time block structure was assumed both the recreational CPFV and PR fleets based on gear restrictions and depth closures. Selectivity blocked into three periods: 1916-2001, 2002-20016, and 2017-2022. In 2022 the sub-bag limit in the California recreational fishery was reduce to only allow one copper rockfish. However, the amount of length samples in 2022 were not informative about a change in selectivity. Finally, a time block in selectivity was applied to the CCFRP Hook and Line survey starting in 2017 when the survey extended their sampling across California.

The specification of when to estimate recruitment deviations is an assumption that affects the estimate of early model uncertainty around stock scale and status. Recruitment deviations were estimated from 1900 - 2019 to appropriately quantify uncertainty in the early model years. The earliest length-composition data occur in 1959 and limited age-composition data available in between 1975-1984. However, age data were not consistently on a yearly basis until 2002. The most informed years for estimating recruitment deviations were from about the early-1980s to 2018. The period from 1900 - 1969 was fit using an early recruitment deviation series with little or no bias adjustment, the main period of recruitment deviates occurred from 1970 - 2019 with an upward and downward ramping of bias adjustment, and 2020 onward recruitment deviations were assumed to be 0 due to limited information at the end of the time-series. (2011) summarize the reasoning behind varying levels of bias adjustment based on the information available to estimate the deviates. The standard deviation of recruitment variability was assumed to be 0.50 based on the estimated variation in recruitment from the base model. Annual recruitment deviations were not forced to be fully zero centered during the main recruitment period in order to allow the data to fully inform the estimation and to avoid this constraint altering the annual estimates. Early model explorations revealed that forcing annual deviations to be zero-centered and when the selected period of the main recruitment impacted the annual estimates. Allowing annual recruitment deviations to not be fully zero-centered allowed the annual estimates to be informed by data rather than the model structure.

The following distributions were assumed for data fitting where the survey indices were lognormal and the compositional data had a multinomial error structure.

\hypertarget{priors}{%
\subsubsection{Priors}\label{priors}}

Priors were used to determine fixed parameter values for natural mortality and steepness in the base model. The prior distribution for natural mortality was based on the Hamel (2022) meta-analytic approach with an assumed maximum age of 50 years. The prior assumed a log normal distribution for natural mortality. The log normal prior has a median of 0.108 yr\textsuperscript{-1} and a standard error of 0.31.

The prior for steepness assumed a beta distribution with mean of 0.72 and standard error of 0.15. The prior parameters are based on the Thorson-Dorn rockfish prior (commonly used in past West Coast rockfish assessments) conducted by James Thorson (personal communication, NWFSC, NOAA), which was reviewed and endorsed by the Scientific and Statistical Committee (SSC) in 2017. However, this approach was subsequently rejected for future analysis in 2019 when the new meta-analysis resulted in a mean value of approximately 0.95. In the absence of a new method for generating a prior for steepness the default approach reverts to the previously endorsed method, the 2017 value.

\hypertarget{data-weighting}{%
\subsubsection{Data Weighting}\label{data-weighting}}

Length composition data for the commercial fishery started with a sample size determined from the equation listed in Sections \ref{commercial-fishery}. The input sample size for the length composition data from recreational fishery was determined based on the number of estimated trips (described in Sections \ref{recreational-fishery}), the CCFRP Hook and Line survey was set equal to the number of positive drops by year, and the CDFW ROV were set equal to the number of positive transects. The majority of age-composition data were input as conditional-age-at-length with input sample size equal to the number of ages. The age-composition data from the historical CPFV samples were used as marginal age-composition with the input sample size set equal to the number of ages.

The base model was weighted using the ``Francis method'', which was based on equation TA1.8 in Francis (2011) which was selected based on model stability and consistency in the model to identify the minimum likelihood estimate (MLE). This formulation looks at the mean length or age and the variance of the mean to determine if across years, the variability is explained by the model. If the variability around the mean does not encompass the model predictions, then that data source should be down-weighted. This method accounts for correlation in the data (i.e., the multinomial distribution). Since Francis data weighting is determined by the mean and variance in observations across years, the age data from the PR fleet had only one year of composition data were weighted based on recommended values from the McAllister-Ianelli Harmonic Mean Weight (1997).

Sensitivities were performed examining the difference in the model fits and results due to in weighting using McAllister-Ianelli Harmonic Mean Weighting (1997) and the Dirichlet Multinomial Weighting (2017).

\hypertarget{model-parameters}{%
\subsubsection{Model Parameters}\label{model-parameters}}

There were 154 estimated parameters in the base model. These included one parameter for \(R_0\), 8 parameters for growth, 5 parameters for extra variability for the fishery and survey indices, 20 parameters for length-based selectivity and time blocking of the fleets and the surveys, and 120 recruitment deviations including 16 pre-model estimates (Table \ref{tab:params}).

Fixed parameters in the model were as follows. Steepness was fixed at 0.72, the mean of the prior. A sensitivity analysis and a likelihood profile were performed for steepness. Natural mortality was fixed at 0.108 yr\textsuperscript{-1} for females and males, the median of the prior. Estimation of natural mortality was explored during model development. The estimate of natural mortality uncertain and poorly informed by the data. When estimated natural mortality was low (around 0.090 yr\textsuperscript{-1}) relative to the median of the prior which was based on maximum age; however, well within the prior distribution (SE = 0.31). The observations of fish greater than 50 years of age across the West Coast, not only this assessment area, is rare, so a natural mortality that would be associated with a maximum around 60 years of age did not seem well supported. Natural mortality was fixed in the base model but estimation of this parameter for both sexes was explored via sensitivities.

The standard deviation of recruitment deviates was fixed at 0.50 (\(\sigma_R\)). Maturity-at-length was fixed as described above in Section \ref{maturation-and-fecundity}. A single California sex-specific length-weight relationship were fixed at externally derived estimates using the NWFSC Hook and Line survey and NWFSC WCGBT survey length-weight observations (Figure \ref{fig:est-len-wght}). Finally, the length at age 2 for both sexes in the growth function was fixed based on the model estimates due to large uncertainty around these values.

\hypertarget{base-model-results}{%
\subsection{Base Model Results}\label{base-model-results}}

The base model described here is only for the portion of the stock copper rockfish in California from Point Conception, $34^\circ 27^\prime$ N. lat. to the California/Oregon border, $42^\circ 00^\prime$ N. lat. Descriptions of the summed biomass and stock status for the California stock of copper rockfish are described in later sections.

The base model parameter estimates along with approximate asymptotic standard errors are shown in Table \ref{tab:north-params} and the likelihood components are shown in Table \ref{tab:likes}. Estimates of derived reference points and approximate 95 percent asymptotic confidence intervals are shown in Table \ref{tab:referenceES}. Estimates of stock size and status over time are shown in Table \ref{tab:timeseries}.

The full r4ss plotting output is available in the supplementary material on the Council's website.

\hypertarget{parameter-estimates}{%
\subsubsection{Parameter Estimates}\label{parameter-estimates}}

Estimated parameter values are provided in Table \ref{tab:params}. The log(\(R_0\)) was estimated at 6.34.

The northern California base model estimated reasonable growth parameters for \(k\) and lengths at age 2 and age 20 for males and females. The estimates differed from those estimated externally, which was not unexpected given the lack of consistent age data across fleets and years. The direct estimation of male \(L_{age=2}=12.6\) cm was reasonable compared to female \$L\_\{age=20\}=14.6. While \(k\) was estimated larger for males (0.20) than females (0.15), female \(L_{age=30}\) of 48.3 cm was larger than males at 46.4 cm. These results are consistent with other studies that have looked at sex-specific growth in copper rockfish and similar to estimates from the southern California pre-STAR base model.

Length-based selectivity curves were estimated for the fishery and survey fleets, and age-based selectivity of 1.0 starting at age 1 for the growth fleet. Model explorations included parameterizing the fleets with double normal selectivity. Selectivity of the commercial dead fleet and the CDFW ROV survey were continually estimated as asypmtotic through base model development and were simplified to two parameter logistic selectivity in the base model. Peak selectivity for the commercial dead fleet was estimated at 34 cm and 32 cm for the ROV survey. Plots of the estimated selectivities are shown in Figure \ref{fig:est-selex}.

The commercial live fishery selectivity was estimated in two blocks of time; 1916 - 2010 and 2011 - 2022. The block in selectivity was included to capture a shift from asymptotic selectivity prior to 2011 to the selection of plate-sized (approx. 2 pounds) fish preferred in the live-fish fishery (Figure \ref{fig:com-len-data}). Both recreational fleets were fit to the same three time blocks. From 1916-2001, peak selectivity was estimated around 36 cm with selectivity decreasing for larger fish; dome-shaped selectivity was estimated from 2002-2016 representing the years the fishery was restricted to shallower depths, and asymptotic selectivity starting in 2017 when the fishery gained access to deeper depths. The two estimated PR fleet selecitivities were both dome-shape with the wider peak selecitivity estimted in 2017-2022 representing the change in depth regulations.

The CCFRP survey estimated peak selectivity at 33 cm in both time blocks with the first time blocks estimating decreased selectivity of larger fish. The survey expanded to northern California in 2017 where larger copper rockfish were observed and estimated asymptotic selecitivity for fish larger than 33 cm.

The catchability for each of the surveys was analytically solved comparing observed to expected vulnerable biomass across all years. The analytical values for catchability were very small given the survey methodologies and are reported in Table \ref{tab:params} in log-space. Additional fishery and survey index variability, process error added directly to each year's input standard deviation for the were estimated within the model. The model estimated the largest added variance of for the recreational PR fishery index. In contrast the model estimated only limited additional variability in order to fit the recreational CRFS CPFV fishery index (0.095). The model fit the trend in the CCFRP survey with added variance estimated to fit the time series of 0.22, while the model added and still did not fit the trend in the index. The model fit the CDFW ROV survey index well and estimated a small added variance of 0.089.

The estimated annual recruitment and recruitment deviations are shown in Figures \ref{fig:recruits} and \ref{fig:rec-devs}. The bias adjustment applied to the annual recruimtent deviations across time is shown in Figure \ref{fig:bias-adj}. Strong recruitments are estimated to have occurred in 1966-1967, 2007 and 2017 and the years of lowest estimated recuitment being 1979 and 1980. The uncertainy in recruitment deviations is highest for the first two years 1970 and 1971 and relatively consistent for the remainder of the time series. There is limited information in the data on recruitment variability from the available data. During model explorations the recruitment deviations were most sensitive to the removal of the available age and fishery index data.

Recruitment is estimated based on the spawner-recruit curve in 2021 and 2022 (Figure \ref{fig:bh-curve}). The recruitment bias adjustment applied within the model across years is shown in Figure \ref{fig:bias-adjust}.

\hypertarget{fits-to-the-data}{%
\subsubsection{Fits to the Data}\label{fits-to-the-data}}

\hypertarget{fits-to-length-and-age-composition}{%
\paragraph{Fits to length and age composition}\label{fits-to-length-and-age-composition}}

Fits to the length data are shown based on the Pearson residuals-at-length, the annual mean lengths, and aggregated length composition data for the commercial and recreational fleets. Annual length composition fits are shown in the Appendix, Section \ref{append-fit}. Aggregate fits by fleet are shown in Figure \ref{fig:agg-len-fit}.

The aggregated lengths for the commercial dead fleet reflected a wide selection across sizes, with the model under-predicting the selection for both small males and females. The majority of the length data for the commercial dead fleet consisted of unsexed fish with sex-specific lengths available from 1980, 1984, 1999, and 2019-2022. The aggregate length compositionfit well with the asymptotic selectivity curve for the commercial dead fleet. Multiple sensitivities were conducted to explore alternative parameterization of commercial dead fleet selectivity. The Pearson residuals for the commercial dead fishery length data area shown in Figure \ref{fig:com-dead-pearson}. The mean length observed in the commercial lengths of unsexed fish were generally stable between 1990 - 2019 and decreased to smaller sizes from 2019 - 2022, with high undertainty in the mean lengths of unsexed fish in 2022 (Figure \ref{fig:com-dead-mean-len-fit}). The observations of larger fish, greater than 40 cm, are minimally greater than the model expectations after 2010. A limited number of ages from the commercial dead fleet were available from 2019-2022. The model estimated mean age was within the bounds of uncertainty, but not well fit (Figure \ref{fig:agg-marg-age-fit})

Starting in 2010, the commercial live fleet length data shifts to smaller fish with observations greater than model expectations for fish between 25 - 30 cm. All available lengths for the commercial live fleet were from unsexed fish and the aggregated length data were fit relativley well given the change in selectivity in 2011. There were no ages available from the commercial live fleet and the The Pearson residuals for the commercial dead fishery length data area shown in Figure \ref{fig:com-live-pearson}. The mean length observed in the commercial lengths of unsexed fish were not stable prior to 2011 (Figure \ref{fig:com-dead-mean-len-fit}). From 2011-2022 the mean length of fish in the live fishery are relatively stable, with a notable decreased in 2016.

The lenth composition fits to the recreational CPFV fleet were relatively well fit throughout the time series, except for a efw years where the a number of fish in a single size class were observed that the model did not expect given the selectivity. The Pearson residuals do not show an indication of any strong year classes from the avaialble lengths (Figure \ref{fig:rec-cpfv-pearson}). The mean length observed, unsexed fish from the CPFV fleet was fit relatively well, indicating a slight increase in mean size around 2000, a decrease from 2007-2011 and a slight increase again from 2013-2018 (Figure \textbackslash ref\{fig:rec-cpfv-mean-len-fit). The number of sexed fish available from the CPFV fleet is small andthe last year of data was not well fit and estimated with high uncertainty. Only one year of age data were available, 2022, from a combination of NMFS Coopeartive Reserach collections and the CDFW groundfish group. A small fraction of these fish were unsexed, and the Pearson residuls indicate these data were generally well-fit (Figure \ref{fig:rec-cpfv-age-pearson}).

The Pearson residuals for the recreational PR length data are variable by year (Figure \ref{fig:rec-pr-pearson}). Pearson residuals were positive, observations greater than expected, for small fish prior to 1997 and are generally variable showing no clear misfit in the model in recent years. The aggregate length composition data from the PR fleet had a slightly higher peak around 29 cm with fewer observations. The length composition across yeras is fit well from 2004-2022 when CDFW implemented the CRFS sampling program. A wide range of sizes were observed from 1959-1987 with poorer fits in years with less data such as 1989 and 1996-2002. The mean length by year for the recreational PR fleet was highly variable across years (Figure \ref{fig:rec-pr-mean-len-fit}). The implementation of the MPA network may have impacted the shift to smaller mean size in those years. The CDFW collected ages from the recreational PR fleet in 2022. The peak of the age distribution was underestimated by the model (Figure \ref{fig:agg-marg-age-fit}).

The aggregated length composition to both fishery-independent surveys CCFRP and the CDFW ROV, were fit reasonably well with an underestimation of fish at around 40-45 cm in the ROV survey. Both of these surveys were conducted in California state waters and represent samples from inside and outside the MPAs. The annual fits to the CCFRP length data were not as well fit as other data sources in any given year, but the observation of larger fish when the survey expanded north in 2017 is pronounced. The Pearson residuals are presented in Figure \ref{fig:ccfrp-len-pearson} and exhibit no clear pattern. The model estiamted mean length was increaseing from 2014-2016 priot to the survey's expansion (Figure \ref{fig:ccfrp-mean-len-fit}). The model did not fit the decreased observed mean length in 2019. However, the model estimated mean length given the observed data from this survey. Age data were available from 2018, 2019 and 2022 from the CCFRP survey and input as condition age-at-length data. The data had a slightly higher proportion of older fish given estimated growth (Figure \ref{fig:ccfrp-age-pearson}). Of note is that all of these ages represent the time period after the survey expanded and selectivity was estimated to be asymptotic.

The length composition data for the ROV survey were available at a finer scale than the super years available for the index of abundance. Not surprising, the year with the most available length observations, 2021, had the best fit to the length data, although was an underestimate for fish in the 35-45 cm range. No trend was observed from the Pearson residuals (Figure \ref{fig:ccfrp-len-pearson}). The survey covered a wide range of depths and the same increasing trend in mean size as the CCFRP data show, was observed in the ROV survey from 2019-2021 (Figure \ref{fig:rov-mean-len-fit}).

\hypertarget{fits-to-indices-of-abundance}{%
\paragraph{Fits to Indices of Abundance}\label{fits-to-indices-of-abundance}}

Fits to the indices vary in quality. The Deb Wilson-Vandenberg onboard survey from 1988-1998 indicated a decline from 1992-1998 that was not fit well by the model. However, this is the highest quality data source for the time period and with the added variance, the model fit was fairly flat and uninformative, a (Figure @ref(fig:dwv\_cpfv-index-fit)). The CDFW and Cal Poly onboard index was flat from 2004-2015 and the increase in relative CPUE in the ending years 2017-2019 represent time periods when the fishery had access to deeper water, but the increase in relative CPUE in 2016 was not due to changes in regulations (Figure @ref(fig:rec\_cpfv-index-fit)). The model fit the ending years or data to the upper bound of the added variability.

The recreaitonal dockside PR index showed a similar trends to the CPFV onboard index. Tndex was well fit during the first part of the time series when it was relatively flat (2004-2015), but the increase in relative CPUE in the ending years 2017-2019 was not well captured by the model. Even with selectivity time blocks for these periods, the index was not fit in 2017.

The CDFW ROW survey contained data grouped into two super years and the model estimated a relatively flat line with the added variance (Figure \ref{fig:crfs-pr-index-fit}). The CCFRP index reflects the same increase in relative cpue in 2016 as the CPFV and PR indicies, prior to the survey expansion and releae of recreational depth restrictions. This index was weighted by the area within the MPAs, which exhibits an increasing trends compared to sites outside the survey at the end of the time series (Figure \ref{fig:ccfrp-index-fit}). The fit to the early part of the time series was reasonable given the available data. Similar to the 2019 gopher/black-and-yellow rockfish complex CCFRP survey, the lowest estimate year in the CCFRP year was 2013, which was also not fit in the 2019 gopher stock assessment. No explanation for the decrease in relative CPUE was identified.

\hypertarget{population-trajectory-in-the-modeled-area}{%
\subsubsection{Population Trajectory in the Modeled Area}\label{population-trajectory-in-the-modeled-area}}

The predicted spawning output (in millions of eggs) is given in Table \ref{tab:timeseries} and shown in Figure \ref{fig:north-ssb}. The estimated spawning output decreases sharply in the late-1970s and continues to decline until reaching low levels in the late-1990s. The spawning output slowly increases between 2000 - 2010 with the rate of population growth increasing after 2011 as fish from recent years of above average recruitment begin to mature. The estimate of total biomass follows the same trend over time is shown in Figure \ref{fig:tot-bio}. The estimated spawning output relative to unfished equilibrium spawning output for the sub-area north of Point Conception reached a minimum of 0.17 in 1994 and then increased over the recent time period, with an ending year estimate of 0.52 in 2022 (Figure \ref{fig:depl}).

\hypertarget{population-trajectory-for-the-stock}{%
\subsubsection{Population Trajectory for the Stock}\label{population-trajectory-for-the-stock}}

The predicted spawning output for the California stock or copper rockfish is given in Table \ref{tab:ca-status} and shown in Figure \ref{fig:sb-all}. The predicted trajectory of spawning output for the stock is generally similar to the trend observed for each area north and south of Point Conception with spawning output declining starting late 1970s when catches across California peaked. The spawning output of the stock declined the the lowest level in the mid-1990s and then began to steadily increase through the end of the time series. The spawning output relative unfished spawning output declined to the stock's lowest point in 1994 of 0.15 of spawning output (Figure \ref{fig:depl-all}). After hitting a low in 1994, the relative spawning output of the stock has steadily increased with an estimated final stock status of 40 percent in 2023.

\hypertarget{reference-points-1}\) reference harvest rate. The spawning output equivalent to 40\% of the unfished level (\(SB_{40\%}\)) was 194 million eggs.

The 2022 spawning biomass relative to unfished equilibrium spawning biomass is just below the target of 40\% of unfished levels (Figure \ref{fig:depl}). The relative fishing intensity, \((1-SPR)/(1-SPR_{50\%})\), was near the management target in 2020, and has fluctuated around the target level for the past decade (Figure \ref{fig:1-spr} and \ref{fig:phase}).

Table \ref{tab:north-referenceES} shows the full suite of estimated reference points for the base model and Figures \ref{fig:north-yield2} and \ref{fig:north-yield3} show the equilibrium yield curve and net production based on a steepness value fixed at 0.72.

\hypertarget{model-diagnostics}{%
\subsection{Model Diagnostics}\label{model-diagnostics}}

\hypertarget{convergence}{%
\subsubsection{Convergence}\label{convergence}}

Proper convergence was determined by starting the minimization process from dispersed values of the maximum likelihood estimates to determine if the model found a better minimum. Starting parameters were jittered using the jitter function built into Stock Synthesis, using jitter input of 0.10. This was repeated 100 times with 89 out of 100 runs returning to the base model likelihood. A better, lower negative log-likelihood, model fit was not found. Through the jittering done as explained and likelihood profiles, we are confident that the base model as presented represents the best fit to the data given the assumptions made. There were no difficulties in inverting the Hessian to obtain estimates of variability, although much of the early model investigation was done without attempting to estimate a Hessian.

\hypertarget{sensitivity-analyses}{%
\subsubsection{Sensitivity Analyses}\label{sensitivity-analyses}}

Sensitivity analyses were conducted to examine the relative influence of specific changes to data inputs and model structural assumptions to further address uncertainty associated with the base model estimates and derived management quantities. The majority of the sensitivity models are the result of a single change relative to base model (i.e., they are not the result of cumulative changes such as the modeling approach used with the bridging analysis). Comparisons of likelihood values and estimates of key parameters from the sensitivity analysis are shown in Tables \ref{tab:sensitivities-1}-\ref{tab:sensitivities-3}. Comparison of all sensitivities relative to the base model is shown in Figure \ref{fig:sens-all} with Figures \ref{fig:sens-ssb-1}-\ref{fig:sens-depl-3} showing the change in spawning output and the relative spawning output trajectories. Many additional sensitivity runs were explored during development and testing of the base model. This section focuses on the main data and structural sensitivity model runs and includes the following:

Structural Sensitivities

\begin{enumerate}
   
  \item  Estimate natural mortality ($M$) for each sex.
  
  \item Estimate steepness ($h$).
  
  \item Estimate $M$ for each sex and $h$.
  
  \item Fix growth at external Schnute growth estimates.
  
  \item Fix recruitment to be equal to the stock-recruitment curve (no recruitment deviations).
  
  \item Apply Dirichlet data weights.
  
  \item Apply McAllister-Ianelli data weights.

\end{enumerate}

Data Sensitivities

\begin{enumerate}
   
  \item Reduce the CPFV and PR catch between 1970-1982 to half of each fleet's average catch for that period of time.
  
  \item Move the historical CPFV ages into the growth fleet.
  
  \item Retain only length data within the model (no indices or ages) and fix growth at the estimates.
  
  \item Remove the CPFV cooperative collection ages in 2022.
  
  \item Remove all age data and fix growth at the estimates.
  
  \item Remove the CCFRP Hook and Line survey data (index, lengths, and ages).
  
  \item Remove the CDFW ROV survey data (index and lengths).
  
  \item Remove both the CCFRP Hook and Line and CDFW ROV survey data (indices, lengths, and ages).
  
  \item Remove the CRFS CPFV index of abundance.
  
  \item Remove the Deb-Wilson Vandenberg CPFV index of abundance.
  
  \item Remove the CRFS PR index of abundance.
  
\end{enumerate}

Across all the sensitivities conducted only a subset resulted in distinct changes in the model estimates and the discussion here will focus on the sensitivities that were identified in Figure \ref{fig:sens-all}. The sensitivity without recruitment deviaations and recruitment was predicted directly from the stock-recruitment curve, the estimate of initial unfished reccruitment (\(R_0\)) increased and the contribution of the length contributed to the increased model likelihood (Table \textbackslash ref\{\{tab:sensitivities-1\}, Figures \ref{fig:sens-ssb-1} and \ref{fig:sens-depl-1}). In addition, the model estimated a 63\% increase in the 2023 spawning biomass.

The sensitivities estimating naturaly moratlity and/or steepness all resulted in a decreased 2023 spawning biomass. In both senstivities where natural mortality was estimated, the estimates were considered too low for a rockfish with a max age of less than 55. The model estimating only steepness also resulted in an estimate of less than half the prior (Table \textbackslash ref\{\{tab:sensitivities-1\}).

The leave one out sensitivities provide insight into the contribution of each datastream to the base model. While the Deb Wilson-Vandenberg index was not fit well by the model, the remove of the index results in the same directional increase in estimated 2023 spawning biomass, as the removal of the more recent onboard observer CPFV survey (Figure \ref{fig:sens-all}). Remove all survey fleets (2 fishery-dependent and 2 fishery-independent) xxxxx.

\hypertarget{retrospective-analysis}{%
\subsubsection{Retrospective Analysis}\label{retrospective-analysis}}

A ten-year retrospective analysis was conducted by successively removing years of data ranging from 2013 - 2022 (i.e., ``Data -1 Years'' corresponds to data through 2021).

Figure \ref{fig:retro-ssb} Figure \ref{fig:retro-depl}

\hypertarget{likelihood-profiles}{%
\subsubsection{Likelihood Profiles}\label{likelihood-profiles}}

Likelihood profiles were conducted for \(R_0\), steepness, and sex-specific natural mortality values separately. These likelihood profiles were conducted by fixing the parameter at specific values and estimated the remaining parameters based on the fixed parameter value.

Figure \ref{fig:r0-profile} Figure \ref{fig:r0-ssb} Figure \ref{fig:r0-depl}

Figure \ref{fig:h-profile} Figure \ref{fig:h-ssb} Figure \ref{fig:h-depl}

Figure \ref{fig:m-profile} Figure \ref{fig:m-ssb} Figure \ref{fig:m-depl}

\hypertarget{historical-analysis}{%
\subsubsection{Historical Analysis}\label{historical-analysis}}

Figure \ref{fig:comp-assess-sb}

Figure \ref{fig:comp-assess-depl}

\hypertarget{management}{%
\section{Management}\label{management}}

\hypertarget{reference-points-2}\) reference harvest rate. The spawning output equivalent to 40 percent of unfished spawning output (\(\text{SB}_{40\%}\)) was 275.11 million eggs.

The 2022 spawning output for the California stock relative to unfished equilibrium spawning output, 42 percent, is above the management target of 40 percent of unfished spawning output (Figure \ref{fig:depl-all}). The fishing intensity, \(1-\text{SPR}\), has been above the harvest rate limit (\(\text{SPR}_{50\%}\)) in recent years, except 2020 when overall removals declined due to impacts of COVID-19 which reduced recreational fishing effort (Table \ref{tab:ca-status} and Figure \ref{fig:1-spr}). The stock is estimated to be below the management target with fishing intensity exceeding the target across recent years (Figure \ref{fig:phase}). Table \ref{tab:south-reference} shows the full suite of estimated reference points for the base model and Figures \ref{fig:yield-north} and \ref{fig:yield-south} shows the equilibrium curve based on a steepness value fixed at 0.72.

\hypertarget{unresolved-problems-and-major-uncertainties-1}{%
\subsection{Unresolved Problems and Major Uncertainties}\label{unresolved-problems-and-major-uncertainties-1}}

shared text

\hypertarget{harvest-projections-and-decision-tables}{%
\subsection{Harvest Projections and Decision Tables}\label{harvest-projections-and-decision-tables}}

A ten year projection of both base models south and north of Point Conception in California with catches equal to the estimated Acceptable Biological Catch (ABC) based on the category 1 time-varying \(\sigma\) with \(P^*\) = 0.45 for years 2025-2034 (Table \ref{tab:ca-proj}). Since the stock is estimated to be above the management target of 40 percent the buffer value in Table \ref{tab:ca-proj} reflects the ABC values with the time-varying scientific uncertainty buffer.

The removals in 2023 and 2024 were set equal to the portion of copper rockfish species-specific ACL for California determined by summing the adopted ACLs South of 40\(^\circ\) 10' Lat. N. and the portion of the North of 40\(^\circ\) 10' Lat. N. allocated to California (25 percent - PFMC Groundfish Management Team pers. comm.). The portion of ACL to allocate to each sub-area was determined based on the proportion of the total removals by area in 2022 (71 percent north and 29 percent south).

Describe the decision table when determined

\hypertarget{evaluation-of-scientific-uncertainty}{%
\subsection{Evaluation of Scientific Uncertainty}\label{evaluation-of-scientific-uncertainty}}

The model estimated uncertainty around the 2023 spawning output was for the sub-area south of Point Conception is \(\sigma\) = 0.29 and north of Point Conception is \(\sigma\) = 0.26. The uncertainty around the OFL south and north of Point Conception was \(\sigma\) = 0.27 and 0.25, respectively. These is likely an underestimates of overall uncertainty because of the necessity to fix several population dynamic parameters (e.g., steepness, recruitment variance, natural mortality) and no explicit incorporation of model structural uncertainty (although see the decision table for alternative states of nature).

\hypertarget{research-and-data-needs-1}{%
\subsection{Research and Data Needs}\label{research-and-data-needs-1}}

There were some major sources of uncertainty within the assessments for copper rockfish. To improve our understanding of the copper rockfish stock in California waters the following research and data collection should be prioritized:

\begin{enumerate}

  \item Continue to investigate historical catch reconstructions and all other historical data sources.

  \item Continue to investigate the most appropriate model structure for the NWFSC Hook and Line survey index. The NWFSC Hook and Line survey is the only long-term fishery-independent survey in rocky (untrawlable) habitat in the Southern California Bight. We also recommend evaluating how to structure the NWFSC Hook and Line survey index, given its expansion into the CCAs and increase of site now within designated MPAs, and independent analysis of information content in NWFSC Hook and Line survey across observed species. Finally, increased spatiotemporal sampling around Point Conception would aid in identifying stock boundaries.


    \item The assessment area appears to have a mixture of observations from areas experiencing variable fishing mortality. In the region there are likely a mixture of areas: open access rocky reefs that are close to port that are heavily fished, open access rocky reefs that are inaccessible via day-trips that are fished but likely lower levels, and rocky reefs that fall within marine protect areas.  A spatially-explicit assessment model may be able to capture this complexity but will require data (indices of abundance and composition data) from each of the regions. 
    
    \item Future nearshore assessments would greatly benefit from additional CDFW ROV surveys which could increase the power of these data to inform assessments.

    \item There are very limited age data for copper rockfish across California arising from fishery-dependent sources.  Collecting otoliths from the recreational fishery, a large source of mortality, would support future assessments  and would improve the understanding of the population structure and life history of copper rockfish. 


\end{enumerate}

\hypertarget{acknowledgments}{%
\section{Acknowledgments}\label{acknowledgments}}

Many people were instrumental in the successful completion of this assessment and their contribution is greatly appreciated. We are very grateful to all the agers at the CAP lab for their hard work reading numerous otoliths and availability to answer questions when needed. Kayleigh Sommers and Kate Richardson assisted with data from the WCGOP and entertained our many questions. We would like to acknowledge our survey team and their dedication to improving the assessments we do. Peter Frey and John Harms were incredibly helpful in helping the STAT team to understand the data and as to why and when each of our assessments either encounter or do not copper rockfish along the coast. Melissa Head provided an area-specific maturity estimate for copper rockfish and provided insight in the complex biological processes that govern maturity processes. We thank all of the CCFRP program partners for conducting and providing the available data. Thank you to CDFW and MARE for providing the ROV data and helping us interpret and model these data. This assessment was greatly benefited by data collection efforts between the SWFSC and volunteer CPFV vessels (F/V Amigo, F/V Coral Sea, F/V Legacy, F/V Mirage, F/V Salty Lady, F/V Sea Wolf, F/V Stardust) funded by the Sportfishing Association of California that collected critical length and age data used in this assessment.

This assessment were greatly benefited by the numerous individuals who took the time to participate in the pre-assessment data webinar and/or the pre-assessment industry meetings. Gerry Richter, Merit McCrea, Louis Zimm, Jamie Diamond, Mike Thompson, Ken Franke, Harison Ibach, Jon Law, and Daniel Platt provided insight to the data and the complexities of the commercial and recreational fisheries off the coast of California which were essential in the production of all of the copper rockfish assessments conducted this year.

The assessment was greatly improved through the streamlining of data processing tools (Kelli Johnson and Ian Taylor, NWFSC) and the many discussions within the Population Ecology team in the FRAM division at the NWFSC and the Habitat and Groundfish Ecology Team at the SWFSC.

\clearpage

\hypertarget{references}{%
\section{References}\label{references}}

\hypertarget{refs}{}
\begin{CSLReferences}{1}{0}
\leavevmode\vadjust pre{\hypertarget{ref-albin_effort_1993}{}}%
Albin, D.P., Karpov, K.A., and Van Buskirk, W.H. 1993. Effort and catch estimates for {Northern} and {Central} {California} marine recreational fisheries, 1981-1986. State of California The Resources Agency Department of Fish; Game.

\leavevmode\vadjust pre{\hypertarget{ref-anderson_sdmtmb_2022}{}}%
Anderson, S.C., Ward, E.J., English, P.A., and Barnett, L.A.K. 2022. {sdmTMB}: An {R} package for fast, flexible, and user-friendly generalized linear mixed effects models with spatial and spatiotemporal random fields. preprint, Ecology. doi:\href{https://doi.org/10.1101/2022.03.24.485545}{10.1101/2022.03.24.485545}.

\leavevmode\vadjust pre{\hypertarget{ref-anderson_identification_1983}{}}%
Anderson, T.W. 1983. Identification and development of nearshore juvenile rockfishes (genus genus{\textbackslash{}}emph\{{Sebastes}\}) in central {California} kelp forests. PhD thesis, California State University, Fresno.

\leavevmode\vadjust pre{\hypertarget{ref-baetscher_dispersal_2019}{}}%
Baetscher, D.S., Anderson, E.C., Horvath, E.A.G., Malone, D.P., Saarman, E.T., Carr, M.H., and Garza, J.C. 2019. Dispersal of a nearshore marine fish connects marine reserves and adjacent fished areas along an open coast. Molecular Ecology \textbf{28}: 1611--1623. doi:\href{https://doi.org/10.1111/mec.15044}{10.1111/mec.15044}.

\leavevmode\vadjust pre{\hypertarget{ref-bizzarro_diet_2017}{}}%
Bizzarro, J.J., Yoklavich, M.M., and Wakefield, W.W. 2017. Diet composition and foraging ecology of {U}.{S}. {Pacific} {Coast} groundfishes with applications for fisheries management. Environmental Biology of Fishes \textbf{100}(4): 375--393. doi:\href{https://doi.org/10.1007/s10641-016-0529-2}{10.1007/s10641-016-0529-2}.

\leavevmode\vadjust pre{\hypertarget{ref-bradburn_2003_2011}{}}%
Bradburn, M.J., Keller, A.A., and Horness, B.H. 2011. The 2003 to 2008 {US} {West} {Coast} bottom trawl surveys of groundfish resources off {Washington}, {Oregon}, and {California}: Estimates of distribution, abundance, length, and age composition. US Department of Commerce, National Oceanic; Atmospheric Administration, National Marine Fisheries Service.

\leavevmode\vadjust pre{\hypertarget{ref-buonaccorsi_population_2002}{}}%
Buonaccorsi, V.P., Kimbrell, C.A., Lynn, E.A., and Vetter, R.D. 2002. Population structure of copper rockfish (\emph{{Sebastes} caurinus}) reflects postglacial colonization and contemporary patterns of larval dispersal. Canadian Journal of Fisheries and Aquatic Sciences \textbf{59}(8): 1374--1384. doi:\href{https://doi.org/10.1139/f02-101}{10.1139/f02-101}.

\leavevmode\vadjust pre{\hypertarget{ref-cope_data-moderate_2013}{}}%
Cope, J., Dick, E.J., MacCall, A., Monk, M., Soper, B., and Wetzel, C. 2013. Data-moderate stock assessments for brown, {China}, copper, sharpchin, stripetail, and yellowtail rockfishes and {English} and rex soles in 2013. Pacific Fishery Management Council, 7700 Ambassador Place NE, Suite 200, Portland, OR. Available from \url{http://www.academia.edu/download/44999856/CopeetalDataModerate2013.pdf} {[}accessed 24 June 2016{]}.

\leavevmode\vadjust pre{\hypertarget{ref-cope_approach_2011}{}}%
Cope, J.M., DeVore, J., Dick, E.J., Ames, K., Budrick, J., Erickson, D.L., Grebel, J., Hanshew, G., Jones, R., Mattes, L., Niles, C., and Williams, S. 2011. An {Approach} to {Defining} {Stock} {Complexes} for {U}.{S}. {West} {Coast} {Groundfishes} {Using} {Vulnerabilities} and {Ecological} {Distributions}. North American Journal of Fisheries Management \textbf{31}(4): 589--604. doi:\href{https://doi.org/10.1080/02755947.2011.591264}{10.1080/02755947.2011.591264}.

\leavevmode\vadjust pre{\hypertarget{ref-dick_meta-analysis_2017}{}}%
Dick, E.J., Beyer, S., Mangel, M., and Ralston, S. 2017. A meta-analysis of fecundity in rockfishes (genus \emph{sebastes}). Fisheries Research \textbf{187}: 73--85. doi:\href{https://doi.org/10.1016/j.fishres.2016.11.009}{10.1016/j.fishres.2016.11.009}.

\leavevmode\vadjust pre{\hypertarget{ref-dick_status_2007}{}}%
Dick, E.J., Ralston, S., and Pearson, D.E. 2007. Status of cowcod, \emph{{Sebastes} levis}, in the {Southern} {California} {Bight}. Pacific Fishery Management Council, 7700 Ambassador Place NE, Suite 200, Portland, OR 97220.

\leavevmode\vadjust pre{\hypertarget{ref-francis_data_2011}{}}%
Francis, R.I.C.C., and Hilborn, R. 2011. Data weighting in statistical fisheries stock assessment models. Canadian Journal of Fisheries and Aquatic Sciences \textbf{68}(6): 1124--1138. doi:\href{https://doi.org/10.1139/f2011-025}{10.1139/f2011-025}.

\leavevmode\vadjust pre{\hypertarget{ref-hamel_method_2015}{}}%
Hamel, O.S. 2015. A method for calculating a meta-analytical prior for the natural mortality rate using multiple life history correlates. ICES Journal of Marine Science \textbf{72}(1): 62--69. doi:\href{https://doi.org/doi:10.1093/icesjms/fsu131}{doi:10.1093/icesjms/fsu131}.

\leavevmode\vadjust pre{\hypertarget{ref-hamel_development_2022}{}}%
Hamel, O.S., and Cope, J.M. 2022. Development and considerations for application of a longevity-based prior for the natural mortality rate. Fisheries Research \textbf{256}: 106477. doi:\href{https://doi.org/10.1016/j.fishres.2022.106477}{10.1016/j.fishres.2022.106477}.

\leavevmode\vadjust pre{\hypertarget{ref-hannah_length_2014}{}}%
Hannah, R.W. 2014. Length and age at maturity of female copper rockfish (\emph{{Sebastes} caurinus}) from {Oregon} waters based on histological evaluation of ovaries. Information \{Reports\}, Oregon Department of Fish; Wildlife.

\leavevmode\vadjust pre{\hypertarget{ref-johansson_influence_2008}{}}%
Johansson, M.L., Banks, M.A., Glunt, K.D., Hassel-Finnegan, H.M., and Buonaccorsi, V.P. 2008. Influence of habitat discontinuity, geographical distance, and oceanography on fine-scale population genetic structure of copper rockfish ( \emph{{Sebastes} caurinus} ). Molecular Ecology \textbf{17}(13): 3051--3061. doi:\href{https://doi.org/10.1111/j.1365-294X.2008.03814.x}{10.1111/j.1365-294X.2008.03814.x}.

\leavevmode\vadjust pre{\hypertarget{ref-lea_biological_1999}{}}%
Lea, R.N., McAllister, R.D., and VenTresca, D.A. 1999. Biological sspects of nearshore rockfishes of the genus sebastes from {Central} {California} with notes on ecologically related sport fishes. State of California The Resources Agency Department of Fish; Game.

\leavevmode\vadjust pre{\hypertarget{ref-love_probably_1996}{}}%
Love, M. 1996. Probably more than you want to know about the fishes of the {Pacific} {Coast}. Really Big Press, Santa Barbara, California.

\leavevmode\vadjust pre{\hypertarget{ref-love_rockfishes_2002}{}}%
Love, M.S., Yoklavich, M.M., and Thorsteinson, L. 2002. Rockfishes of the {Northeast} {Pacific}. University of California Press, Berkeley, CA.

\leavevmode\vadjust pre{\hypertarget{ref-mcallister_bayesian_1997}{}}%
McAllister, M.K., and Ianelli, J.N. 1997. Bayesian stock assessment using catch-age data and the sampling - importance resampling algorithm. Canadian Journal of Fisheries and Aquatic Sciences \textbf{54}: 284--300.

\leavevmode\vadjust pre{\hypertarget{ref-methot_adjusting_2011a}{}}%
Methot, R.D., and Taylor, I.G. 2011. Adjusting for bias due to variability of estimated recruitments in fishery assessment models. Canadian Journal of Fisheries and Aquatic Sciences \textbf{68}(10): 1744--1760. doi:\href{https://doi.org/10.1139/f2011-092}{10.1139/f2011-092}.

\leavevmode\vadjust pre{\hypertarget{ref-methot_stock_2013}{}}%
Methot, R.D., and Wetzel, C.R. 2013. Stock synthesis: A biological and statistical framework for fish stock assessment and fishery management. Fisheries Research \textbf{142}: 86--99. doi:\href{https://doi.org/10.1016/j.fishres.2012.10.012}{10.1016/j.fishres.2012.10.012}.

\leavevmode\vadjust pre{\hypertarget{ref-miller_ocean_1965}{}}%
Miller, D.J., and Gotshall, D. 1965. Ocean {Sportfish} {Catch} and {Effort} {From} {Oregon} to {Point} {Arguello}, {California} {July} 1, 1957--{June} 30, 196. California Department of Fish; Game.

\leavevmode\vadjust pre{\hypertarget{ref-miller_guide_1972}{}}%
Miller, D.J., and Lea, R.N. 1972. Guide to coastal {Marine} {Fishes} of {California}. State of California Department of Fish; Game Bureau of Marine Fisheries.

\leavevmode\vadjust pre{\hypertarget{ref-miller_spatially_2014}{}}%
Miller, R.R., Field, J.C., Santora, J.A., Schroeder, I.D., Huff, D.D., Key, M., Pearson, D.E., and MacCall, A.D. 2014. A {Spatially} {Distinct} {History} of the {Development} of {California} {Groundfish} {Fisheries}. PLoS ONE \textbf{9}(6): e99758. doi:\href{https://doi.org/10.1371/journal.pone.0099758}{10.1371/journal.pone.0099758}.

\leavevmode\vadjust pre{\hypertarget{ref-monk_documentation_2014}{}}%
Monk, M.H., Dick, E.J., and Pearson, D. 2014. Documentation of a relational database for the {California} recreational fisheries survey onboard observer sampling program, 1999-2011. NOAA-TM-NMFS-SWFSC-529.

\leavevmode\vadjust pre{\hypertarget{ref-pearson_reliability_2008}{}}%
Pearson, D., Erwin, B., and Key, M. 2008. Reliability of {California}'s groundfish landing estimates from 1969-2006. \{NOAA\} \{Technical\} \{Memorandum\}, US Department of Commerce, National Oceanic; Atmospheric Administration, National Marine Fisheries Service.

\leavevmode\vadjust pre{\hypertarget{ref-prince_food_1972}{}}%
Prince, E.D. 1972. The food and behavior of the copper rockfish, {Sebastes} caurinus {Richardson}, associated with an artificial reef in {South} {Humboldt} {Bay}, {California}. \{PhD\} \{Thesis\}, California State University.

\leavevmode\vadjust pre{\hypertarget{ref-punt_quantifying_2008}{}}%
Punt, A.E., Smith, D.C., KrusicGolub, K., and Robertson, S. 2008. Quantifying age-reading error for use in fisheries stock assessments, with application to species in {Australia}'s southern and eastern scalefish and shark fishery. Canadian Journal of Fisheries and Aquatic Sciences \textbf{65}(9): 1991--2005. doi:\href{https://doi.org/10.1139/F08-111}{10.1139/F08-111}.

\leavevmode\vadjust pre{\hypertarget{ref-ralston_documentation_2010}{}}%
Ralston, S., Pearson, D.E., Field, J.C., and Key, M. 2010. Documentation of the {California} catch reconstruction project. US Department of Commerce, National Oceanic; Atmospheric Adminstration, National Marine.

\leavevmode\vadjust pre{\hypertarget{ref-reilly_onboard_1998}{}}%
Reilly, P.N., Wilson-Vandenberg, D., Wilson, C.E., and Mayer, K. 1998. Onboard sampling of the rockfish and lingcod commercial passenger fishing vessel industry in northern and central {California}, {January} through {December} 1995. Marine region, Admin. Rep. \textbf{98-1}: 1--110.

\leavevmode\vadjust pre{\hypertarget{ref-reynolds_application_2010}{}}%
Reynolds, B.F., Powers, S.P., and Bishop, M.A. 2010. Application of {Acoustic} {Telemetry} to {Assess} {Residency} and {Movements} of {Rockfish} and {Lingcod} at {Created} and {Natural} {Habitats} in {Prince} {William} {Sound}. PLoS ONE \textbf{5}(8): e12130. doi:\href{https://doi.org/10.1371/journal.pone.0012130}{10.1371/journal.pone.0012130}.

\leavevmode\vadjust pre{\hypertarget{ref-sivasundar_life_2010}{}}%
Sivasundar, A., and Palumbi, S.R. 2010. Life history, ecology and the biogeography of strong genetic breaks among 15 species of {Pacific} rockfish, {Sebastes}. Marine Biology \textbf{157}(7): 1433--1452. doi:\href{https://doi.org/10.1007/s00227-010-1419-3}{10.1007/s00227-010-1419-3}.

\leavevmode\vadjust pre{\hypertarget{ref-starr_variation_2015a}{}}%
Starr, R.M., Wendt, D.E., Barnes, C.L., Marks, C.I., Malone, D., Waltz, G., Schmidt, K.T., Chiu, J., Launer, A.L., and Hall, N.C. 2015a. Variation in responses of fishes across multiple reserves within a network of marine protected areas in temperate waters. PLoS ONE \textbf{10}(3): 1--24. doi:\href{https://doi.org/10.5061/dryad.6hk4h.Funding}{10.5061/dryad.6hk4h.Funding}.

\leavevmode\vadjust pre{\hypertarget{ref-starr_variation_2015}{}}%
Starr, R.M., Wendt, D.E., Barnes, C.L., Marks, C.I., Malone, D., Waltz, G., Schmidt, K.T., Chiu, J., Launer, A.L., Hall, N.C., and Yochum, N. 2015b. Variation in responses of fishes across multiple reserves within a network of marine protected areas in temperate waters. PLoS One2 \textbf{10}(3): p.e0118502.

\leavevmode\vadjust pre{\hypertarget{ref-pfmc_final_2008}{}}%
Status of the pacific coast groundfish fishery: Stock assessment and fishery evaluation. 2008. Pacific Fishery Management Council.

\leavevmode\vadjust pre{\hypertarget{ref-then_evaluating_2015}{}}%
Then, A.Y., Hoenig, J.M., Hall, N.G., and Hewitt, D.A. 2015. Evaluating the predictive performance of empirical estimators of natural mortality rate using information on over 200 fish species. ICES Journal of Marine Science \textbf{72}(1): 82--92. doi:\href{https://doi.org/10.1093/icesjms/fsu136}{10.1093/icesjms/fsu136}.

\leavevmode\vadjust pre{\hypertarget{ref-thompson_larval_2017}{}}%
Thompson, A.R., Chen, D.C., Guo, L.W., Hyde, J.R., and Watson, W. 2017. Larval abundances of rockfishes that were historically targeted by fishing increased over 16 years in association with a large marine protected area. Royal Society Open Science \textbf{4}(9). doi:\href{https://doi.org/10.1098/rsos.170639}{10.1098/rsos.170639}.

\leavevmode\vadjust pre{\hypertarget{ref-thorson_model-based_2017}{}}%
Thorson, J.T., Johnson, K.F., Methot, R.D., and Taylor, I.G. 2017. Model-based estimates of effective sample size in stock assessment models using the {Dirichlet}-multinomial distribution. Fisheries Research \textbf{192}: 84--93. doi:\href{https://doi.org/10.1016/j.fishres.2016.06.005}{10.1016/j.fishres.2016.06.005}.

\leavevmode\vadjust pre{\hypertarget{ref-thorson_nwfscageingerror_2012}{}}%
Thorson, J.T., Stewart, I.J., and Punt, A.E. 2012. {nwfscAgeingError}: A user interface in {R} for the {Punt} {\textbackslash{}}emphet al. (2008) method for calculating ageing error and imprecision. Available from: http://github.com/pfmc-assessments/nwfscAgeingError/.

\leavevmode\vadjust pre{\hypertarget{ref-wendt_collaborative_2009}{}}%
Wendt, D.E., and Starr, R.M. 2009. Collaborative research: An effective way to collect data for stock assessments and evaluate marine protected areas in {California}. Marine and Coastal Fisheries \textbf{1}(1): 315--324. doi:\href{https://doi.org/10.1577/c08-054.1}{10.1577/c08-054.1}.

\leavevmode\vadjust pre{\hypertarget{ref-wetzel_status_2021}{}}%
Wetzel, C.R., Langseth, B.J., Cope, J.M., and Budrick, J. 2021a. The status of copper rockfish (\emph{{Sebastes} caurinus}) in {U}.{S}. Waters off the coast of {California} south of {Point} {Conception} in 2021 using catch and length data. Pacific Fishery Management Council, Portland, Oregon.

\leavevmode\vadjust pre{\hypertarget{ref-wetzel_status_2021-1}{}}%
Wetzel, C.R., Langseth, B.J., Cope, J.M., and Budrick, J.E. 2021b. The status of copper rockfish (\emph{{Sebastes} caurinus}) in {U}.{S}. Waters off the coast of {California} north of {Point} {Conception} in 2021 using catch and length data. Pacific Fishery Management Council, 7700 Ambassador Place NE, Suite 101, Portland, OR 97220.

\leavevmode\vadjust pre{\hypertarget{ref-wilson-vandenberg_implementing_2014}{}}%
Wilson-Vandenberg, D., Larinto, T., and Key, M. 2014. Implementing {California}'s {Nearshore} {Fishery} {Management} {Plan} --- twelve years later. California Department of Fish and Game \textbf{100}(2): 32.

\end{CSLReferences}

\clearpage

\hypertarget{tables}{%
\section{Tables}\label{tables}}

\begingroup\fontsize{10}{12}\selectfont
\begingroup\fontsize{10}{12}\selectfont

\begin{longtable}[t]{l>{\raggedright\arraybackslash}p{1.83cm}>{\raggedright\arraybackslash}p{1.83cm}>{\raggedright\arraybackslash}p{1.83cm}>{\raggedright\arraybackslash}p{1.83cm}>{\raggedright\arraybackslash}p{1.83cm}}
\caption{\label{tab:allcatches}Removals (mt) by fleet and the summed total landings (mt).}\\
\toprule
Year & Commercial (Dead) & Commercial (Live) & Rec. CPFV & Rec. PR & Total Landings\\
\midrule
\endfirsthead
\caption[]{\label{tab:allcatches}Removals (mt) by fleet and the summed total landings (mt). \textit{(continued)}}\\
\toprule
Year & Commercial (Dead) & Commercial (Live) & Rec. CPFV & Rec. PR & Total Landings\\
\midrule
\endhead

\endfoot
\bottomrule
\endlastfoot
1916 & 4.0 & 0.0 & 0.0 & 0.0 & 4.0\\
1917 & 6.2 & 0.0 & 0.0 & 0.0 & 6.2\\
1918 & 7.5 & 0.0 & 0.0 & 0.0 & 7.5\\
1919 & 4.9 & 0.0 & 0.0 & 0.0 & 4.9\\
1920 & 5.1 & 0.0 & 0.0 & 0.0 & 5.1\\
1921 & 4.3 & 0.0 & 0.0 & 0.0 & 4.3\\
1922 & 3.7 & 0.0 & 0.0 & 0.0 & 3.7\\
1923 & 3.9 & 0.0 & 0.0 & 0.0 & 3.9\\
1924 & 2.6 & 0.0 & 0.0 & 0.0 & 2.6\\
1925 & 3.8 & 0.0 & 0.0 & 0.0 & 3.8\\
1926 & 4.9 & 0.0 & 0.0 & 0.0 & 4.9\\
1927 & 3.6 & 0.0 & 0.0 & 0.0 & 3.6\\
1928 & 3.6 & 0.0 & 1.0 & 0.6 & 5.2\\
1929 & 3.0 & 0.0 & 1.9 & 1.2 & 6.2\\
1930 & 5.3 & 0.0 & 2.2 & 1.4 & 9.0\\
1931 & 6.3 & 0.0 & 3.0 & 1.9 & 11.1\\
1932 & 5.7 & 0.0 & 3.7 & 2.4 & 11.7\\
1933 & 4.9 & 0.0 & 4.4 & 2.8 & 12.1\\
1934 & 3.6 & 0.0 & 5.2 & 3.3 & 12.0\\
1935 & 5.7 & 0.0 & 5.9 & 3.8 & 15.3\\
1936 & 5.2 & 0.0 & 6.6 & 4.2 & 16.1\\
1937 & 5.9 & 0.0 & 7.9 & 5.0 & 18.8\\
1938 & 5.2 & 0.0 & 7.7 & 5.0 & 17.9\\
1939 & 5.0 & 0.0 & 6.8 & 4.3 & 16.1\\
1940 & 4.8 & 0.0 & 9.7 & 6.2 & 20.8\\
1941 & 5.2 & 0.0 & 9.0 & 5.8 & 20.0\\
1942 & 1.8 & 0.0 & 4.8 & 3.1 & 9.6\\
1943 & 2.9 & 0.0 & 4.6 & 2.9 & 10.4\\
1944 & 8.7 & 0.0 & 3.8 & 2.4 & 14.8\\
1945 & 21.4 & 0.0 & 5.0 & 3.2 & 29.6\\
1946 & 23.9 & 0.0 & 8.6 & 5.5 & 38.0\\
1947 & 7.2 & 0.0 & 6.8 & 4.4 & 18.3\\
1948 & 9.6 & 0.0 & 13.6 & 8.7 & 31.9\\
1949 & 5.2 & 0.0 & 17.6 & 11.3 & 34.1\\
1950 & 4.1 & 0.0 & 21.5 & 13.8 & 39.3\\
1951 & 8.9 & 0.0 & 24.5 & 20.5 & 53.9\\
1952 & 5.9 & 0.0 & 21.3 & 17.8 & 45.1\\
1953 & 2.9 & 0.0 & 18.2 & 15.2 & 36.3\\
1954 & 5.5 & 0.0 & 22.6 & 18.9 & 46.9\\
1955 & 2.9 & 0.0 & 26.9 & 22.5 & 52.4\\
1956 & 4.9 & 0.0 & 30.1 & 25.1 & 60.1\\
1957 & 5.6 & 0.0 & 28.1 & 24.5 & 58.3\\
1958 & 6.5 & 0.0 & 52.4 & 40.3 & 99.2\\
1959 & 7.4 & 0.0 & 39.2 & 33.7 & 80.3\\
1960 & 10.0 & 0.0 & 32.3 & 26.1 & 68.3\\
1961 & 7.3 & 0.0 & 24.1 & 19.7 & 51.1\\
1962 & 5.2 & 0.0 & 27.1 & 31.3 & 63.6\\
1963 & 6.2 & 0.0 & 32.3 & 40.8 & 79.3\\
1964 & 4.2 & 0.0 & 22.5 & 44.0 & 70.7\\
1965 & 4.5 & 0.0 & 37.1 & 63.3 & 104.9\\
1966 & 5.5 & 0.0 & 40.8 & 74.8 & 121.0\\
1967 & 6.2 & 0.0 & 38.3 & 83.8 & 128.4\\
1968 & 3.3 & 0.0 & 37.6 & 95.1 & 136.0\\
1969 & 2.4 & 0.0 & 36.8 & 106.6 & 145.8\\
1970 & 2.5 & 0.0 & 53.7 & 125.0 & 181.2\\
1971 & 4.4 & 0.0 & 39.8 & 125.0 & 169.2\\
1972 & 6.9 & 0.0 & 60.9 & 147.5 & 215.2\\
1973 & 6.7 & 0.0 & 69.3 & 170.4 & 246.3\\
1974 & 15.7 & 0.0 & 70.4 & 184.3 & 270.4\\
1975 & 8.4 & 0.0 & 67.3 & 192.2 & 268.0\\
1976 & 15.9 & 0.0 & 69.5 & 211.1 & 296.5\\
1977 & 13.9 & 0.0 & 78.6 & 213.7 & 306.1\\
1978 & 2.5 & 0.0 & 62.3 & 216.7 & 281.5\\
1979 & 2.8 & 0.0 & 56.4 & 233.6 & 292.8\\
1980 & 39.6 & 0.0 & 55.1 & 210.4 & 305.2\\
1981 & 9.6 & 0.0 & 106.9 & 171.2 & 287.8\\
1982 & 12.9 & 0.0 & 106.7 & 164.4 & 284.0\\
1983 & 69.0 & 0.0 & 64.4 & 76.3 & 209.8\\
1984 & 43.2 & 0.0 & 49.0 & 92.9 & 185.1\\
1985 & 25.4 & 0.0 & 42.6 & 138.4 & 206.5\\
1986 & 10.4 & 0.0 & 47.6 & 106.9 & 165.0\\
1987 & 13.8 & 0.0 & 17.6 & 68.8 & 100.2\\
1988 & 17.9 & 0.0 & 25.5 & 69.2 & 112.7\\
1989 & 33.8 & 0.0 & 42.3 & 46.3 & 122.4\\
1990 & 43.3 & 0.0 & 28.5 & 61.4 & 133.2\\
1991 & 52.4 & 0.0 & 25.7 & 53.7 & 131.8\\
1992 & 71.3 & 0.0 & 24.7 & 46.0 & 142.0\\
1993 & 68.6 & 0.2 & 22.8 & 71.2 & 162.7\\
1994 & 25.4 & 6.0 & 17.1 & 44.9 & 93.5\\
1995 & 34.3 & 8.5 & 11.3 & 21.9 & 76.1\\
1996 & 36.5 & 17.3 & 10.3 & 19.9 & 84.0\\
1997 & 38.6 & 7.1 & 18.5 & 15.8 & 80.0\\
1998 & 23.2 & 5.3 & 5.2 & 11.1 & 44.9\\
1999 & 8.0 & 7.8 & 11.8 & 9.4 & 37.0\\
2000 & 2.9 & 4.8 & 19.8 & 4.2 & 31.6\\
2001 & 4.3 & 7.4 & 12.3 & 4.9 & 28.9\\
2002 & 3.2 & 6.2 & 10.3 & 2.1 & 21.8\\
2003 & 1.0 & 1.6 & 3.8 & 17.4 & 23.8\\
2004 & 1.3 & 2.0 & 6.5 & 9.1 & 18.9\\
2005 & 0.9 & 2.8 & 18.2 & 13.0 & 34.9\\
2006 & 0.8 & 2.2 & 16.8 & 16.5 & 36.2\\
2007 & 1.1 & 4.7 & 17.4 & 18.8 & 42.0\\
2008 & 1.0 & 4.0 & 9.8 & 17.0 & 31.8\\
2009 & 0.8 & 1.7 & 14.7 & 22.0 & 39.2\\
2010 & 0.6 & 1.1 & 14.3 & 11.5 & 27.5\\
2011 & 0.6 & 1.9 & 8.8 & 14.6 & 25.9\\
2012 & 0.9 & 2.3 & 12.2 & 19.5 & 34.9\\
2013 & 0.7 & 2.1 & 8.8 & 14.0 & 25.6\\
2014 & 0.7 & 2.5 & 16.1 & 17.6 & 36.9\\
2015 & 0.8 & 2.7 & 24.2 & 37.8 & 65.5\\
2016 & 0.8 & 2.6 & 28.7 & 34.2 & 66.3\\
2017 & 1.4 & 4.6 & 56.5 & 76.1 & 138.6\\
2018 & 3.0 & 6.4 & 44.0 & 49.0 & 102.4\\
2019 & 2.5 & 6.9 & 39.2 & 53.4 & 101.9\\
2020 & 3.9 & 7.5 & 59.6 & 85.1 & 156.2\\
2021 & 3.1 & 7.5 & 54.9 & 41.4 & 107.0\\
2022 & 1.2 & 1.9 & 11.5 & 32.5 & 47.1\\*
\end{longtable}
\endgroup{}
\endgroup{}

\newpage

\begingroup\fontsize{10}{12}\selectfont
\begingroup\fontsize{10}{12}\selectfont

\begin{longtable}[t]{c>{\centering\arraybackslash}p{2cm}>{\centering\arraybackslash}p{2cm}>{\centering\arraybackslash}p{2cm}}
\caption{\label{tab:ca-management}The portion of the Overfishing Limit (OFL) and Annual Catch Limit (ACL) and estimated catch in California waters.}\\
\toprule
Year & OFL (mt) & ACL (mt) & Catch (mt)\\
\midrule
\endfirsthead
\caption[]{\label{tab:ca-management}The portion of the Overfishing Limit (OFL) and Annual Catch Limit (ACL) and estimated catch in California waters. \textit{(continued)}}\\
\toprule
Year & OFL (mt) & ACL (mt) & Catch (mt)\\
\midrule
\endhead

\endfoot
\bottomrule
\endlastfoot
2012 & 163.15 & 136.17 & 85.95\\
2013 & 148.00 & 123.42 & 105.18\\
2014 & 148.00 & 123.42 & 98.65\\
2015 & 303.75 & 277.32 & 147.64\\
2016 & 286.88 & 261.95 & 165.27\\
2017 & 313.70 & 286.38 & 225.48\\
2018 & 319.60 & 291.85 & 203.69\\
2019 & 325.08 & 296.83 & 182.59\\
2020 & 330.35 & 301.60 & 242.73\\
2021 & 249.85 & 206.43 & 164.20\\
2022 & 249.48 & 204.02 & 66.67\\*
\end{longtable}
\endgroup{}
\endgroup{}

\begingroup\fontsize{10}{12}\selectfont
\begingroup\fontsize{10}{12}\selectfont

\begin{longtable}[t]{r>{\centering\arraybackslash}p{2cm}>{\centering\arraybackslash}p{2cm}}
\caption{\label{tab:com-ratio}Ratio estimates of total rockfish landings north and south of Point Conception. "Ratio years" are the range of years over which ratio estimates were calculated. Sources include the NMFS SWFSC ERD Live Access Server and several volumes of the CDFG Fish Bulletin series.}\\
\toprule
Year & Ratio & Ratio Years\\
\midrule
\endfirsthead
\caption[]{Ratio estimates of total rockfish landings north and south of Point Conception. "Ratio years" are the range of years over which ratio estimates were calculated. Sources include the NMFS SWFSC ERD Live Access Server and several volumes of the CDFG Fish Bulletin series. \textit{(continued)}}\\
\toprule
Year & Ratio & Ratio Years\\
\midrule
\endhead

\endfoot
\bottomrule
\endlastfoot
1916 & 0.33 & 1928-33\\
1917 & 0.33 & 1928-33\\
1918 & 0.33 & 1928-33\\
1919 & 0.33 & 1928-33\\
1920 & 0.33 & 1928-33\\
1921 & 0.33 & 1928-33\\
1922 & 0.33 & 1928-33\\
1923 & 0.33 & 1928-33\\
1924 & 0.33 & 1928-33\\
1925 & 0.33 & 1928-33\\
1926 & 0.33 & 1928-33\\
1927 & 0.33 & 1928-33\\
1928 & 0.33 & 1949-51\\
1929 & 0.33 & 1949-51\\
1930 & 0.33 & 1949-51\\
1931 & 0.33 & 1949-51\\
1932 & 0.33 & 1949-51\\
1933 & 0.33 & 1949-51\\
1934 & 0.33 & 1949-51\\
1935 & 0.33 & 1949-51\\
1936 & 0.33 & 1949-51\\
1937 & 0.33 & 1949-51\\
1938 & 0.33 & 1949-51\\
1939 & 0.33 & 1949-51\\
1940 & 0.33 & 1949-51\\
1941 & 0.33 & 1949-51\\
1942 & 0.33 & 1949-51\\
1943 & 0.33 & 1949-51\\
1944 & 0.33 & 1949-51\\
1945 & 0.33 & 1949-51\\
1946 & 0.33 & 1949-51\\
1947 & 0.33 & 1949-51\\
1948 & 0.33 & 1949-51\\
1949 & 0.30 & data\\
1950 & 0.19 & data\\
1951 & 0.44 & data\\
1952 & 0.46 & 1949-51\\
1953 & 0.31 & 1954-57\\
1954 & 0.14 & data\\
1955 & 0.01 & data\\
1956 & 0.06 & data\\
1957 & 0.10 & data\\
1958 & 0.14 & 1954-57\\
1959 & 0.24 & 1954-57\\
1960 & 0.23 & 1954-57\\
1961 & 0.44 & 1954-57\\
1962 & 0.28 & data\\
1963 & 0.25 & data\\
1964 & 0.19 & data\\
1965 & 0.37 & data\\
1966 & 0.27 & data\\
1967 & 0.38 & data\\
1968 & 0.46 & data\\*
\end{longtable}
\endgroup{}
\endgroup{}


\newpage

\begingroup\fontsize{10}{12}\selectfont
\begingroup\fontsize{10}{12}\selectfont

\begin{table}[t]{r>{\centering\arraybackslash}p{2cm}>{\centering\arraybackslash}p{2cm}}
\caption{\label{tab:dead-com-len}Summary of the number of trips and length samples for fish landed dead by commercial fisheries.}\\
\toprule
Year & Trips & Lengths\\
\midrule
\endfirsthead
\caption[]{Summary of the number of trips and length samples for fish landed dead by commercial fisheries. \textit{(continued)}}\\
\toprule
Year & Trips & Lengths\\
\midrule
\endhead

\endfoot
\bottomrule
\endlastfoot
1978 & 1 & 2\\
1979 & 3 & 26\\
1980 & 4 & 34\\
1981 & 2 & 4\\
1982 & 3 & 6\\
1983 & 5 & 13\\
1984 & 2 & 25\\
1985 & 1 & 1\\
1986 & 1 & 2\\
1987 & 2 & 2\\
1988 & 3 & 4\\
1990 & 2 & 2\\
1991 & 6 & 126\\
1992 & 106 & 662\\
1993 & 169 & 808\\
1994 & 85 & 334\\
1995 & 66 & 255\\
1996 & 87 & 348\\
1997 & 28 & 116\\
1998 & 16 & 32\\
1999 & 58 & 336\\
2000 & 6 & 36\\
2001 & 5 & 10\\
2002 & 2 & 8\\
2003 & 3 & 21\\
2004 & 3 & 14\\
2005 & 1 & 13\\
2007 & 1 & 5\\
2008 & 2 & 5\\
2009 & 3 & 7\\
2010 & 1 & 1\\
2011 & 5 & 7\\
2012 & 7 & 11\\
2013 & 3 & 3\\
2014 & 4 & 4\\
2015 & 3 & 4\\
2016 & 11 & 22\\
2017 & 9 & 14\\
2018 & 7 & 26\\
2019 & 8 & 53\\
2020 & 14 & 56\\
2021 & 19 & 59\\
2022 & 17 & 79\\
\end{table}
\endgroup{}
\endgroup{}


\begingroup\fontsize{10}{12}\selectfont
\begingroup\fontsize{10}{12}\selectfont

\begin{longtable}[t]{r>{\centering\arraybackslash}p{2cm}>{\centering\arraybackslash}p{2cm}}
\caption{\label{tab:live-com-len}Summary of the number of trips and length samples for fish landed live by commercial fisheries.}\\
\toprule
Year & Trips & Lengths\\
\midrule
\endfirsthead
\caption[]{Summary of the number of trips and length samples for fish landed live by commercial fisheries. \textit{(continued)}}\\
\toprule
Year & Trips & Lengths\\
\midrule
\endhead

\endfoot
\bottomrule
\endlastfoot
1994 & 3 & 8\\
1995 & 4 & 8\\
1996 & 7 & 15\\
1998 & 5 & 5\\
1999 & 23 & 123\\
2000 & 26 & 34\\
2001 & 39 & 223\\
2002 & 6 & 21\\
2003 & 3 & 6\\
2004 & 5 & 22\\
2005 & 1 & 6\\
2006 & 2 & 4\\
2007 & 18 & 69\\
2008 & 15 & 67\\
2009 & 11 & 20\\
2010 & 19 & 31\\
2011 & 21 & 44\\
2012 & 18 & 51\\
2013 & 11 & 25\\
2014 & 6 & 20\\
2015 & 12 & 18\\
2016 & 18 & 25\\
2017 & 16 & 56\\
2018 & 21 & 76\\
2019 & 18 & 31\\
2020 & 15 & 29\\
2021 & 9 & 62\\
2022 & 9 & 40\\*
\end{longtable}
\endgroup{}
\endgroup{}


\begingroup\fontsize{10}{12}\selectfont
\begingroup\fontsize{10}{12}\selectfont

\begin{longtable}[t]{r>{\centering\arraybackslash}p{1.57cm}>{\centering\arraybackslash}p{1.57cm}>{\centering\arraybackslash}p{1.57cm}>{\centering\arraybackslash}p{1.57cm}>{\centering\arraybackslash}p{1.57cm}>{\centering\arraybackslash}p{1.57cm}}
\caption{\label{tab:rec-len-samps}Summary of the recreational length samples by source for the CPFV and PR fleets.}\\
\toprule
Area & Year & Source & CPFV Trips & CPFV Samples & PR Trips & PR Samples\\
\midrule
\endfirsthead
\caption[]{Summary of the recreational length samples by source for the CPFV and PR fleets. \textit{(continued)}}\\
\toprule
Area & Year & Source & CPFV Trips & CPFV Samples & PR Trips & PR Samples\\
\midrule
\endhead

\endfoot
\bottomrule
\endlastfoot
North & 1959 & MILLER & 1 & 202 & 4 & 337\\
North & 1960 & MILLER & 4 & 715 & - & -\\
North & 1961 & MILLER & 2 & 8 & - & -\\
North & 1966 & MILLER & 2 & 20 & - & -\\
North & 1978 & DON PEARSON & 98 & 343 & - & -\\
North & 1979 & DON PEARSON & 75 & 233 & - & -\\
North & 1980 & DON PEARSON & 115 & 199 & - & -\\
North & 1980 & MRFSS & 53 & 92 & 125 & 286\\
North & 1981 & DON PEARSON & 53 & 92 & - & -\\
North & 1981 & MRFSS & 61 & 172 & 91 & 188\\
North & 1982 & DON PEARSON & 78 & 148 & - & -\\
North & 1982 & MRFSS & 41 & 59 & 118 & 310\\
North & 1983 & DON PEARSON & 55 & 98 & - & -\\
North & 1983 & MRFSS & 50 & 82 & 109 & 209\\
North & 1984 & DON PEARSON & 40 & 102 & - & -\\
North & 1984 & MRFSS & 79 & 193 & 122 & 216\\
North & 1985 & MRFSS & 110 & 175 & 148 & 314\\
North & 1986 & MRFSS & 138 & 248 & 152 & 257\\
North & 1987 & DEB WILSON-VANDENBERG & 15 & 26 & - & -\\
North & 1987 & MRFSS & 23 & 67 & 56 & 134\\
North & 1988 & DEB WILSON-VANDENBERG & 92 & 551 & - & -\\
North & 1988 & MRFSS & 39 & 57 & 41 & 94\\
North & 1989 & DEB WILSON-VANDENBERG & 130 & 824 & - & -\\
North & 1989 & MRFSS & 89 & 187 & 39 & 68\\
North & 1990 & DEB WILSON-VANDENBERG & 44 & 378 & - & -\\
North & 1991 & DEB WILSON-VANDENBERG & 49 & 272 & - & -\\
North & 1992 & DEB WILSON-VANDENBERG & 126 & 735 & - & -\\
North & 1993 & DEB WILSON-VANDENBERG & 136 & 977 & - & -\\
North & 1993 & MRFSS & 27 & 37 & 234 & 428\\
North & 1994 & DEB WILSON-VANDENBERG & 130 & 530 & - & -\\
North & 1994 & MRFSS & 22 & 29 & 140 & 270\\
North & 1995 & DEB WILSON-VANDENBERG & 148 & 725 & - & -\\
North & 1995 & MRFSS & 32 & 59 & 62 & 92\\
North & 1996 & DEB WILSON-VANDENBERG & 120 & 457 & - & -\\
North & 1996 & MRFSS & 134 & 194 & 56 & 76\\
North & 1997 & DEB WILSON-VANDENBERG & 142 & 554 & - & -\\
North & 1998 & DEB WILSON-VANDENBERG & 84 & 252 & - & -\\
North & 1999 & MRFSS & 140 & 191 & 35 & 53\\
North & 2000 & MRFSS & 53 & 85 & 14 & 19\\
North & 2001 & MRFSS & 72 & 94 & 9 & 18\\
North & 2002 & MRFSS & 82 & 107 & 18 & 20\\
North & 2003 & MRFSS & 87 & 107 & 45 & 60\\
North & 2004 & CRFS & 65 & 179 & 130 & 396\\
North & 2005 & CRFS & 60 & 348 & 251 & 880\\
North & 2006 & CRFS & 77 & 414 & 322 & 1354\\
North & 2007 & CRFS & 149 & 677 & 303 & 1284\\
North & 2008 & CRFS & 90 & 411 & 279 & 1125\\
North & 2009 & CRFS & 96 & 487 & 268 & 994\\
North & 2010 & CRFS & 99 & 532 & 232 & 826\\
North & 2011 & CRFS & 126 & 416 & 266 & 912\\
North & 2012 & CRFS & 134 & 558 & 277 & 884\\
North & 2013 & CRFS & 144 & 533 & 319 & 1245\\
North & 2014 & CRFS & 136 & 583 & 350 & 1327\\
North & 2015 & CRFS & 148 & 528 & 453 & 2397\\
North & 2016 & CRFS & 134 & 641 & 415 & 2184\\
North & 2017 & CRFS & 157 & 1088 & 478 & 2904\\
North & 2018 & CRFS & 128 & 806 & 450 & 2226\\
North & 2019 & CRFS & 142 & 716 & 458 & 2099\\
North & 2021 & CRFS & 79 & 246 & 254 & 1014\\
North & 2022 & CRFS & 98 & 247 & 420 & 1278\\*
\end{longtable}
\endgroup{}
\endgroup{}


\begingroup\fontsize{10}{12}\selectfont
\begingroup\fontsize{10}{12}\selectfont

\begin{longtable}[t]{r>{\centering\arraybackslash}p{2cm}>{\centering\arraybackslash}p{2cm}>{\centering\arraybackslash}p{2cm}}
\caption{\label{tab:ccfrp-samps}The total number of drifts, length, and age samples collected by year from the CCFRP survey north of Point Conception.}\\
\toprule
Year & Drifts & Lengths & Ages\\
\midrule
\endfirsthead
\caption[]{The total number of drifts, length, and age samples collected by year from the CCFRP survey north of Point Conception. \textit{(continued)}}\\
\toprule
Year & Drifts & Lengths & Ages\\
\midrule
\endhead

\endfoot
\bottomrule
\endlastfoot
2007 & 60 & 92 & 0\\
2008 & 70 & 88 & 0\\
2009 & 67 & 92 & 0\\
2010 & 52 & 73 & 0\\
2011 & 60 & 78 & 0\\
2012 & 76 & 108 & 0\\
2013 & 53 & 70 & 0\\
2014 & 109 & 163 & 0\\
2015 & 30 & 43 & 0\\
2016 & 114 & 214 & 0\\
2017 & 117 & 230 & 7\\
2018 & 185 & 335 & 20\\
2019 & 201 & 403 & 27\\
2020 & 182 & 340 & 11\\
2021 & 193 & 355 & 4\\
2022 & 181 & 393 & 45\\*
\end{longtable}
\endgroup{}
\endgroup{}


\begingroup\fontsize{10}{12}\selectfont
\begingroup\fontsize{10}{12}\selectfont

\begin{longtable}[t]{r>{\centering\arraybackslash}p{2.2cm}>{\centering\arraybackslash}p{2.2cm}>{\centering\arraybackslash}p{2.2cm}>{\centering\arraybackslash}p{2.2cm}}
\caption{\label{tab:rov-obs}Number of transects and number of observations of copper rockfish for each group and survey year.}\\
\toprule
Super Year & Area & Designation & Transects & Observations\\
\midrule
\endfirsthead
\caption[]{Number of transects and number of observations of copper rockfish for each group and survey year. \textit{(continued)}}\\
\toprule
Super Year & Area & Designation & Transects & Observations\\
\midrule
\endhead

\endfoot
\bottomrule
\endlastfoot
2015 & Ano Nuevo & MPA & 4 & 0\\
2020 & Ano Nuevo & MPA & 10 & 7\\
2015 & Big Creek & MPA & 3 & 3\\
2020 & Big Creek & MPA & 4 & 4\\
2015 & Bodega Bay & MPA & 28 & 11\\
2020 & Bodega Bay & MPA & 45 & 84\\
2015 & Montara & MPA & 11 & 4\\
2020 & Montara & MPA & 19 & 8\\
2015 & Piedras Blancas & MPA & 8 & 6\\
2020 & Piedras Blancas & MPA & 8 & 11\\
2015 & Pillar Point & MPA & 4 & 1\\
2020 & Pillar Point & MPA & 8 & 7\\
2015 & Point Arena & MPA & 7 & 7\\
2020 & Point Arena & MPA & 12 & 41\\
2015 & Point Buchon & MPA & 7 & 4\\
2020 & Point Buchon & MPA & 14 & 17\\
2015 & Point Lobos & MPA & 15 & 11\\
2020 & Point Lobos & MPA & 31 & 110\\
2015 & Point St. George & MPA & 21 & 27\\
2020 & Point St. George & MPA & 17 & 17\\
2015 & Point Sur & MPA & 14 & 20\\
2020 & Point Sur & MPA & 22 & 74\\
2015 & Portuguese Ledge & MPA & 6 & 30\\
2020 & Portuguese Ledge & MPA & 11 & 24\\
2015 & Reading Rock & MPA & 14 & 4\\
2020 & Reading Rock & MPA & 17 & 17\\
2015 & SE Farallon Islands & MPA & 12 & 18\\
2020 & SE Farallon Islands & MPA & 22 & 58\\
2015 & Sea Lion Gulch & MPA & 12 & 0\\
2020 & Sea Lion Gulch & MPA & 21 & 16\\
2015 & Ten Mile & MPA & 20 & 30\\
2020 & Ten Mile & MPA & 17 & 51\\
2015 & Ano Nuevo & Reference & 5 & 0\\
2020 & Ano Nuevo & Reference & 9 & 3\\
2015 & Big Creek & Reference & 20 & 54\\
2020 & Big Creek & Reference & 8 & 35\\
2015 & Bodega Bay & Reference & 16 & 3\\
2020 & Bodega Bay & Reference & 32 & 48\\
2015 & Montara/Pillar Point & Reference & 8 & 0\\
2020 & Montara/Pillar Point & Reference & 20 & 3\\
2015 & Point Arena & Reference & 8 & 8\\
2020 & Point Arena & Reference & 12 & 7\\
2015 & Point Buchon & Reference & 8 & 4\\
2020 & Point Buchon & Reference & 12 & 8\\
2015 & Point Lobos & Reference & 8 & 2\\
2020 & Point Lobos & Reference & 22 & 13\\
2015 & Point St. George & Reference & 14 & 3\\
2020 & Point St. George & Reference & 13 & 3\\
2015 & Point Sur & Reference & 8 & 3\\
2020 & Point Sur & Reference & 17 & 8\\
2015 & Portuguese Ledge & Reference & 6 & 9\\
2020 & Portuguese Ledge & Reference & 8 & 11\\
2015 & Reading Rock & Reference & 19 & 21\\
2020 & Reading Rock & Reference & 17 & 26\\
2015 & SE Farallon Islands & Reference & 13 & 1\\
2020 & SE Farallon Islands & Reference & 16 & 8\\
2015 & Sea Lion Gulch & Reference & 9 & 5\\
2020 & Sea Lion Gulch & Reference & 16 & 18\\
2015 & Ten Mile & Reference & 18 & 28\\
2020 & Ten Mile & Reference & 19 & 16\\*
\end{longtable}
\endgroup{}
\endgroup{}


\begingroup\fontsize{10}{12}\selectfont
\begingroup\fontsize{10}{12}\selectfont

\begin{table}[t]{r>{\centering\arraybackslash}p{2cm}>{\centering\arraybackslash}p{2cm}}
\caption{\label{tab:growth-age-samps}Number of ages by year and source used as conditional-age-at-length data to inform estimation of growth.}\\
\toprule
Year & Source & Ages\\
\midrule
\endfirsthead
\caption[]{Number of ages by year and source used as conditional-age-at-length data to inform estimation of growth. \textit{(continued)}}\\
\toprule
Year & Source & Ages\\
\midrule
\endhead

\endfoot
\bottomrule
\endlastfoot
2001 & Pearson Research & 3\\
2002 & Pearson Research & 68\\
2003 & Pearson Research & 260\\
2004 & NWFSC WCGBT & 49\\
2004 & Pearson Research & 82\\
2005 & NWFSC WCGBT & 9\\
2005 & Pearson Research & 13\\
2006 & NWFSC WCGBT & 7\\
2007 & NWFSC WCGBT & 1\\
2008 & NWFSC WCGBT & 25\\
2009 & NWFSC WCGBT & 6\\
2010 & Abrams & 27\\
2010 & NWFSC WCGBT & 10\\
2011 & Abrams & 47\\
2012 & NWFSC WCGBT & 4\\
2013 & NWFSC WCGBT & 8\\
2014 & NWFSC WCGBT & 16\\
2015 & NWFSC WCGBT & 10\\
2016 & NWFSC WCGBT & 2\\
2017 & NWFSC WCGBT & 11\\
2018 & CDFW & 3\\
2018 & NWFSC WCGBT & 12\\
2019 & CDFW & 27\\
2019 & NWFSC WCGBT & 10\\
2021 & CDFW & 15\\
2021 & NWFSC WCGBT & 14\\
2022 & NWFSC WCGBT & 13\\
\end{table}
\endgroup{}
\endgroup{}


\begingroup\fontsize{10}{12}\selectfont
\begingroup\fontsize{10}{12}\selectfont

\begin{longtable}[t]{l>{\raggedright\arraybackslash}p{2.2cm}>{\raggedright\arraybackslash}p{2.2cm}>{\raggedright\arraybackslash}p{2.2cm}>{\raggedright\arraybackslash}p{2.2cm}}
\caption{\label{tab:pisco-data}All and filtered observations by year and sampling institution for PISCO.}\\
\toprule
Year & UCSC Raw Count & UCSC Filtered Count & UCSB Raw Count & UCSB Filtered Count\\
\midrule
\endfirsthead
\caption[]{\label{tab:pisco-data}All and filtered observations by year and sampling institution for PISCO. \textit{(continued)}}\\
\toprule
Year & UCSC Raw Count & UCSC Filtered Count & UCSB Raw Count & UCSB Filtered Count\\
\midrule
\endhead

\endfoot
\bottomrule
\endlastfoot
1999 & 2 & NA & 7 & NA\\
2000 & 1 & NA & 11 & NA\\
2001 & 6 & 4 & 4 & NA\\
2002 & 25 & 21 & 8 & NA\\
2003 & 34 & 25 & 73 & NA\\
2004 & 30 & 9 & 65 & 19\\
2005 & 40 & 6 & 45 & 18\\
2006 & 27 & 12 & 51 & 25\\
2007 & 17 & 4 & 58 & 19\\
2008 & 21 & 5 & 44 & 22\\
2009 & 20 & 7 & 60 & 29\\
2010 & 34 & 10 & 85 & 32\\
2011 & 36 & 1 & 44 & 20\\
2012 & 9 & 4 & 77 & 39\\
2013 & 40 & 17 & 59 & 23\\
2014 & 50 & 28 & 50 & 39\\
2015 & 51 & 16 & 18 & 15\\
2016 & 29 & 17 & 51 & 45\\
2017 & 30 & 11 & 28 & 22\\
2018 & 37 & 15 & 42 & 28\\
2019 & 26 & 15 & 41 & 37\\
2020 & 58 & 26 & 35 & 29\\
2021 & 23 & 12 & 37 & 27\\*
\end{longtable}
\endgroup{}
\endgroup{}

\begingroup\fontsize{10}{12}\selectfont
\begingroup\fontsize{10}{12}\selectfont

\begin{longtable}[t]{r>{\centering\arraybackslash}p{1.83cm}>{\centering\arraybackslash}p{1.83cm}>{\centering\arraybackslash}p{1.83cm}>{\centering\arraybackslash}p{1.83cm}>{\centering\arraybackslash}p{1.83cm}}
\caption{\label{tab:wcgbt-pos-tows}The total number of tows between 55-183 m,  the number of positive tows, the total number of copper rockfish observed, and the number of lengths and agec collected north of Point Conception in California by the NWFSC WCGBT survey.}\\
\toprule
Year & Tows & Positive Tows & Numbers & Lengths & Ages\\
\midrule
\endfirsthead
\caption[]{The total number of tows between 55-183 m,  the number of positive tows, the total number of copper rockfish observed, and the number of lengths and agec collected north of Point Conception in California by the NWFSC WCGBT survey. \textit{(continued)}}\\
\toprule
Year & Tows & Positive Tows & Numbers & Lengths & Ages\\
\midrule
\endhead

\endfoot
\bottomrule
\endlastfoot
2003 & 73 & 4 & 12 & 12 & 0\\
2004 & 75 & 4 & 49 & 49 & 49\\
2005 & 97 & 2 & 9 & 9 & 9\\
2006 & 79 & 2 & 7 & 7 & 7\\
2007 & 80 & 1 & 1 & 1 & 1\\
2008 & 93 & 5 & 25 & 25 & 25\\
2009 & 100 & 5 & 6 & 6 & 6\\
2010 & 103 & 5 & 10 & 10 & 10\\
2011 & 102 & 0 & 0 & 0 & 0\\
2012 & 106 & 3 & 4 & 4 & 4\\
2013 & 74 & 3 & 8 & 8 & 8\\
2014 & 91 & 1 & 23 & 23 & 16\\
2015 & 98 & 4 & 10 & 10 & 10\\
2016 & 91 & 1 & 2 & 2 & 2\\
2017 & 93 & 2 & 11 & 11 & 11\\
2018 & 93 & 5 & 12 & 12 & 12\\
2019 & 48 & 3 & 10 & 10 & 10\\
2021 & 101 & 7 & 14 & 14 & 14\\
2022 & 90 & 5 & 13 & 13 & 13\\*
\end{longtable}
\endgroup{}
\endgroup{}


\begingroup\fontsize{10}{12}\selectfont
\begingroup\fontsize{10}{12}\selectfont

\begin{longtable}[t]{r>{\centering\arraybackslash}p{6cm}}
\caption{\label{tab:model-structure}Specifications and structure of the base model.}\\
\toprule
Model Setup & Base Model\\
\midrule
\endfirsthead
\caption[]{Specifications and structure of the base model \textit{(continued)}}\\
\toprule
\textbackslash{}underline\{Model Setup\} & Base Model\\
\midrule
\endhead

\endfoot
\bottomrule
\endlastfoot
Starting year & 1916\\
 \vphantom{3} \vphantom{2} \vphantom{1} & \\
\underline{Population characteristics} & \\
Maximum age & 50\\
Gender & 2\\
Population lengths & 4-58 cm by 1 cm bins\\
Summary biomass (mt) & Age 3+\\
 & \\
\underline{Data characteristics} & \\
Data lengths & 10-54 cm by 2 cm bins\\
Data ages & 0-50 ages\\
Minimum age for growth calculations & 2\\
Maximum age for growth calculations & 20\\
First mature age & 0\\
Starting year of estimated recruitment in main period & 1970\\
 & \\
\underline{Fishery characteristics} & \\
Fishing mortality method & Hybrid F\\
Maximum F & 3.5\\
Catchability & Analytical estimate\\
Commercial Dead Selectivity & Length-Based Double Normal\\
Commercial Live Selectivity & Length-Based Double Normal\\
Recreational CPFV Selectivity & Length-Based Double Normal\\
Recreational PR Selectivity & Length-Based Double Normal\\
CCFRP Selectivity & Length-Based Double Normal\\
CDFW ROV Selectivity & Length-Based Double Normal\\
Growth Selectivity & Age-Based Double Normal\\
 & \\
\underline{Fishery time blocks} & \\
Commercial Live & 1916-2010, 2011-2022\\
Recreational CPFV & 1916-2001, 2002-2016, 2017-2022\\
Recreational PR & 1916-1999, 2000-2022\\*
\end{longtable}
\endgroup{}
\endgroup{}


\begingroup\fontsize{9}{11}\selectfont

\begin{landscape}\begingroup\fontsize{9}{11}\selectfont

>>>>>>> Stashed changes
\begin{longtable}[t]{>{\raggedright\arraybackslash}p{7cm}lllll>{\raggedright\arraybackslash}p{4cm}}
\caption{\label{tab:north-params}List of parameters used in the base model, including estimated values and standard deviations (SD), bounds (minimum and maximum), estimation phase (negative values not estimated), status (indicates if parameters are near bounds), and prior type information (mean and SD).}\\
\toprule
Parameter & Value & Phase & Bounds & Status & SD & Prior (Exp.Val, SD)\\
\midrule
\endfirsthead
\caption[]{\label{tab:north-params}List of parameters used in the base model, including estimated values and standard deviations (SD), bounds (minimum and maximum), estimation phase (negative values not estimated), status (indicates if parameters are near bounds), and prior type information (mean and SD). \textit{(continued)}}\\
\toprule
Parameter & Value & Phase & Bounds & Status & SD & Prior (Exp.Val, SD)\\
\midrule
\endhead

\endfoot
\bottomrule
\endlastfoot
NatM uniform Fem GP 1 & 0.108 & -2 & (0.05, 0.4) & NA & NA & Log Norm (-2.2256, 0.31)\\
L at Amin Fem GP 1 & 14.583 & -2 & (6, 25) & NA & NA & None\\
L at Amax Fem GP 1 & 48.307 & 2 & (35, 54) & OK & 0.4116760 & None\\
VonBert K Fem GP 1 & 0.154 & 2 & (0.03, 0.35) & OK & 0.0081489 & None\\
CV young Fem GP 1 & 0.157 & 2 & (0.01, 0.3) & OK & 0.0209688 & None\\
CV old Fem GP 1 & 0.074 & 2 & (0.01, 0.3) & OK & 0.0063267 & None\\
Wtlen 1 Fem GP 1 & 0.000 & -9 & (0, 0.1) & NA & NA & None\\
Wtlen 2 Fem GP 1 & 3.190 & -9 & (2, 4) & NA & NA & None\\
Mat50\% Fem GP 1 & 34.040 & -9 & (10, 50) & NA & NA & None\\
Mat slope Fem GP 1 & -0.410 & -9 & (-1, 0) & NA & NA & None\\
Eggs scalar Fem GP 1 & 0.000 & -9 & (-3, 3) & NA & NA & None\\
Eggs exp len Fem GP 1 & 3.679 & -9 & (-3, 4) & NA & NA & None\\
NatM uniform Mal GP 1 & 0.108 & -2 & (0.05, 0.4) & NA & NA & Log Norm (-2.2256, 0.31)\\
L at Amin Mal GP 1 & 12.637 & -2 & (6, 25) & NA & NA & None\\
L at Amax Mal GP 1 & 46.488 & 2 & (35, 54) & OK & 0.3839610 & None\\
VonBert K Mal GP 1 & 0.195 & 2 & (0.03, 0.3) & OK & 0.0088325 & None\\
CV young Mal GP 1 & 0.157 & 2 & (0.01, 0.3) & OK & 0.0253907 & None\\
CV old Mal GP 1 & 0.072 & 2 & (0.01, 0.3) & OK & 0.0075141 & None\\
Wtlen 1 Mal GP 1 & 0.000 & -9 & (0, 0.1) & NA & NA & None\\
Wtlen 2 Mal GP 1 & 3.150 & -9 & (2, 4) & NA & NA & None\\
CohortGrowDev & 1.000 & -9 & (0, 1) & NA & NA & None\\
FracFemale GP 1 & 0.500 & -9 & (0.01, 0.99) & NA & NA & None\\
SR LN(R0) & 6.342 & 1 & (2, 20) & OK & 0.1036040 & None\\
SR BH steep & 0.720 & -7 & (0.22, 1) & NA & NA & Normal (0.72, 0.16)\\
SR sigmaR & 0.500 & -99 & (0.15, 0.9) & NA & NA & None\\
SR regime & 0.000 & -99 & (-2, 2) & NA & NA & None\\
SR autocorr & 0.000 & -99 & (0, 0) & NA & NA & None\\
Early InitAge 16 & 0.002 & 5 & (-5, 5) & act & 0.5005760 & dev (NA, NA)\\
Early InitAge 15 & 0.003 & 5 & (-5, 5) & act & 0.5006360 & dev (NA, NA)\\
Early InitAge 14 & 0.003 & 5 & (-5, 5) & act & 0.5007020 & dev (NA, NA)\\
Early InitAge 13 & 0.003 & 5 & (-5, 5) & act & 0.5007730 & dev (NA, NA)\\
Early InitAge 12 & 0.004 & 5 & (-5, 5) & act & 0.5008500 & dev (NA, NA)\\
Early InitAge 11 & 0.004 & 5 & (-5, 5) & act & 0.5009340 & dev (NA, NA)\\
Early InitAge 10 & 0.004 & 5 & (-5, 5) & act & 0.5010240 & dev (NA, NA)\\
Early InitAge 9 & 0.005 & 5 & (-5, 5) & act & 0.5011200 & dev (NA, NA)\\
Early InitAge 8 & 0.005 & 5 & (-5, 5) & act & 0.5012230 & dev (NA, NA)\\
Early InitAge 7 & 0.006 & 5 & (-5, 5) & act & 0.5013330 & dev (NA, NA)\\
Early InitAge 6 & 0.006 & 5 & (-5, 5) & act & 0.5014500 & dev (NA, NA)\\
Early InitAge 5 & 0.007 & 5 & (-5, 5) & act & 0.5015740 & dev (NA, NA)\\
Early InitAge 4 & 0.007 & 5 & (-5, 5) & act & 0.5017090 & dev (NA, NA)\\
Early InitAge 3 & 0.008 & 5 & (-5, 5) & act & 0.5018550 & dev (NA, NA)\\
Early InitAge 2 & 0.008 & 5 & (-5, 5) & act & 0.5020130 & dev (NA, NA)\\
Early InitAge 1 & 0.009 & 5 & (-5, 5) & act & 0.5021850 & dev (NA, NA)\\
Early RecrDev 1916 & 0.010 & 5 & (-5, 5) & act & 0.5023720 & dev (NA, NA)\\
Early RecrDev 1917 & 0.011 & 5 & (-5, 5) & act & 0.5025740 & dev (NA, NA)\\
Early RecrDev 1918 & 0.012 & 5 & (-5, 5) & act & 0.5027920 & dev (NA, NA)\\
Early RecrDev 1919 & 0.013 & 5 & (-5, 5) & act & 0.5030280 & dev (NA, NA)\\
Early RecrDev 1920 & 0.014 & 5 & (-5, 5) & act & 0.5032840 & dev (NA, NA)\\
Early RecrDev 1921 & 0.015 & 5 & (-5, 5) & act & 0.5035600 & dev (NA, NA)\\
Early RecrDev 1922 & 0.016 & 5 & (-5, 5) & act & 0.5038590 & dev (NA, NA)\\
Early RecrDev 1923 & 0.017 & 5 & (-5, 5) & act & 0.5041810 & dev (NA, NA)\\
Early RecrDev 1924 & 0.019 & 5 & (-5, 5) & act & 0.5045290 & dev (NA, NA)\\
Early RecrDev 1925 & 0.020 & 5 & (-5, 5) & act & 0.5049050 & dev (NA, NA)\\
Early RecrDev 1926 & 0.022 & 5 & (-5, 5) & act & 0.5053110 & dev (NA, NA)\\
Early RecrDev 1927 & 0.024 & 5 & (-5, 5) & act & 0.5057490 & dev (NA, NA)\\
Early RecrDev 1928 & 0.026 & 5 & (-5, 5) & act & 0.5062220 & dev (NA, NA)\\
Early RecrDev 1929 & 0.028 & 5 & (-5, 5) & act & 0.5068210 & dev (NA, NA)\\
Early RecrDev 1930 & 0.031 & 5 & (-5, 5) & act & 0.5074760 & dev (NA, NA)\\
Early RecrDev 1931 & 0.034 & 5 & (-5, 5) & act & 0.5081870 & dev (NA, NA)\\
Early RecrDev 1932 & 0.037 & 5 & (-5, 5) & act & 0.5089660 & dev (NA, NA)\\
Early RecrDev 1933 & 0.040 & 5 & (-5, 5) & act & 0.5098070 & dev (NA, NA)\\
Early RecrDev 1934 & 0.044 & 5 & (-5, 5) & act & 0.5107070 & dev (NA, NA)\\
Early RecrDev 1935 & 0.048 & 5 & (-5, 5) & act & 0.5117070 & dev (NA, NA)\\
Early RecrDev 1936 & 0.052 & 5 & (-5, 5) & act & 0.5127780 & dev (NA, NA)\\
Early RecrDev 1937 & 0.057 & 5 & (-5, 5) & act & 0.5139390 & dev (NA, NA)\\
Early RecrDev 1938 & 0.062 & 5 & (-5, 5) & act & 0.5152090 & dev (NA, NA)\\
Early RecrDev 1939 & 0.067 & 5 & (-5, 5) & act & 0.5166160 & dev (NA, NA)\\
Early RecrDev 1940 & 0.073 & 5 & (-5, 5) & act & 0.5182010 & dev (NA, NA)\\
Early RecrDev 1941 & 0.080 & 5 & (-5, 5) & act & 0.5200060 & dev (NA, NA)\\
Early RecrDev 1942 & 0.088 & 5 & (-5, 5) & act & 0.5220630 & dev (NA, NA)\\
Early RecrDev 1943 & 0.097 & 5 & (-5, 5) & act & 0.5243820 & dev (NA, NA)\\
Early RecrDev 1944 & 0.107 & 5 & (-5, 5) & act & 0.5269290 & dev (NA, NA)\\
Early RecrDev 1945 & 0.117 & 5 & (-5, 5) & act & 0.5296400 & dev (NA, NA)\\
Early RecrDev 1946 & 0.128 & 5 & (-5, 5) & act & 0.5324030 & dev (NA, NA)\\
Early RecrDev 1947 & 0.138 & 5 & (-5, 5) & act & 0.5350710 & dev (NA, NA)\\
Early RecrDev 1948 & 0.147 & 5 & (-5, 5) & act & 0.5374910 & dev (NA, NA)\\
Early RecrDev 1949 & 0.154 & 5 & (-5, 5) & act & 0.5394350 & dev (NA, NA)\\
Early RecrDev 1950 & 0.160 & 5 & (-5, 5) & act & 0.5407560 & dev (NA, NA)\\
Early RecrDev 1951 & 0.164 & 5 & (-5, 5) & act & 0.5414200 & dev (NA, NA)\\
Early RecrDev 1952 & 0.166 & 5 & (-5, 5) & act & 0.5415230 & dev (NA, NA)\\
Early RecrDev 1953 & 0.170 & 5 & (-5, 5) & act & 0.5413700 & dev (NA, NA)\\
Early RecrDev 1954 & 0.175 & 5 & (-5, 5) & act & 0.5412900 & dev (NA, NA)\\
Early RecrDev 1955 & 0.182 & 5 & (-5, 5) & act & 0.5416720 & dev (NA, NA)\\
Early RecrDev 1956 & 0.191 & 5 & (-5, 5) & act & 0.5438740 & dev (NA, NA)\\
Early RecrDev 1957 & 0.188 & 5 & (-5, 5) & act & 0.5467260 & dev (NA, NA)\\
Early RecrDev 1958 & 0.186 & 5 & (-5, 5) & act & 0.5470060 & dev (NA, NA)\\
Early RecrDev 1959 & 0.190 & 5 & (-5, 5) & act & 0.5481800 & dev (NA, NA)\\
Early RecrDev 1960 & 0.212 & 5 & (-5, 5) & act & 0.5537480 & dev (NA, NA)\\
Early RecrDev 1961 & 0.258 & 5 & (-5, 5) & act & 0.5659620 & dev (NA, NA)\\
Early RecrDev 1962 & 0.318 & 5 & (-5, 5) & act & 0.5842860 & dev (NA, NA)\\
Early RecrDev 1963 & 0.372 & 5 & (-5, 5) & act & 0.6057910 & dev (NA, NA)\\
Early RecrDev 1964 & 0.451 & 5 & (-5, 5) & act & 0.6383690 & dev (NA, NA)\\
Early RecrDev 1965 & 0.566 & 5 & (-5, 5) & act & 0.6895160 & dev (NA, NA)\\
Early RecrDev 1966 & 0.652 & 5 & (-5, 5) & act & 0.7331070 & dev (NA, NA)\\
Early RecrDev 1967 & 0.642 & 5 & (-5, 5) & act & 0.7014840 & dev (NA, NA)\\
Early RecrDev 1968 & 0.517 & 5 & (-5, 5) & act & 0.6158590 & dev (NA, NA)\\
Early RecrDev 1969 & 0.304 & 5 & (-5, 5) & act & 0.5303580 & dev (NA, NA)\\
Main RecrDev 1970 & 0.110 & 2 & (-5, 5) & act & 0.4613340 & dev (NA, NA)\\
Main RecrDev 1971 & -0.175 & 2 & (-5, 5) & act & 0.4162390 & dev (NA, NA)\\
Main RecrDev 1972 & -0.412 & 2 & (-5, 5) & act & 0.3901040 & dev (NA, NA)\\
Main RecrDev 1973 & -0.366 & 2 & (-5, 5) & act & 0.3627370 & dev (NA, NA)\\
Main RecrDev 1974 & -0.423 & 2 & (-5, 5) & act & 0.3604100 & dev (NA, NA)\\
Main RecrDev 1975 & -0.374 & 2 & (-5, 5) & act & 0.3458820 & dev (NA, NA)\\
Main RecrDev 1976 & -0.462 & 2 & (-5, 5) & act & 0.3376970 & dev (NA, NA)\\
Main RecrDev 1977 & -0.534 & 2 & (-5, 5) & act & 0.3093600 & dev (NA, NA)\\
Main RecrDev 1978 & -0.767 & 2 & (-5, 5) & act & 0.3086280 & dev (NA, NA)\\
Main RecrDev 1979 & -0.968 & 2 & (-5, 5) & act & 0.3113840 & dev (NA, NA)\\
Main RecrDev 1980 & -0.811 & 2 & (-5, 5) & act & 0.3128190 & dev (NA, NA)\\
Main RecrDev 1981 & -0.355 & 2 & (-5, 5) & act & 0.2671490 & dev (NA, NA)\\
Main RecrDev 1982 & -0.683 & 2 & (-5, 5) & act & 0.3180890 & dev (NA, NA)\\
Main RecrDev 1983 & -0.656 & 2 & (-5, 5) & act & 0.3399370 & dev (NA, NA)\\
Main RecrDev 1984 & 0.013 & 2 & (-5, 5) & act & 0.3263580 & dev (NA, NA)\\
Main RecrDev 1985 & 0.219 & 2 & (-5, 5) & act & 0.3167090 & dev (NA, NA)\\
Main RecrDev 1986 & -0.304 & 2 & (-5, 5) & act & 0.3732110 & dev (NA, NA)\\
Main RecrDev 1987 & -0.405 & 2 & (-5, 5) & act & 0.3624990 & dev (NA, NA)\\
Main RecrDev 1988 & -0.296 & 2 & (-5, 5) & act & 0.3624310 & dev (NA, NA)\\
Main RecrDev 1989 & 0.117 & 2 & (-5, 5) & act & 0.3175460 & dev (NA, NA)\\
Main RecrDev 1990 & 0.298 & 2 & (-5, 5) & act & 0.2844420 & dev (NA, NA)\\
Main RecrDev 1991 & -0.044 & 2 & (-5, 5) & act & 0.3326930 & dev (NA, NA)\\
Main RecrDev 1992 & -0.163 & 2 & (-5, 5) & act & 0.3706750 & dev (NA, NA)\\
Main RecrDev 1993 & 0.236 & 2 & (-5, 5) & act & 0.3200240 & dev (NA, NA)\\
Main RecrDev 1994 & 0.016 & 2 & (-5, 5) & act & 0.3474600 & dev (NA, NA)\\
Main RecrDev 1995 & -0.370 & 2 & (-5, 5) & act & 0.3577860 & dev (NA, NA)\\
Main RecrDev 1996 & -0.292 & 2 & (-5, 5) & act & 0.3269950 & dev (NA, NA)\\
Main RecrDev 1997 & -0.162 & 2 & (-5, 5) & act & 0.3181880 & dev (NA, NA)\\
Main RecrDev 1998 & -0.108 & 2 & (-5, 5) & act & 0.3128890 & dev (NA, NA)\\
Main RecrDev 1999 & -0.253 & 2 & (-5, 5) & act & 0.3290230 & dev (NA, NA)\\
Main RecrDev 2000 & -0.467 & 2 & (-5, 5) & act & 0.3274040 & dev (NA, NA)\\
Main RecrDev 2001 & -0.402 & 2 & (-5, 5) & act & 0.2959770 & dev (NA, NA)\\
Main RecrDev 2002 & -0.287 & 2 & (-5, 5) & act & 0.2864020 & dev (NA, NA)\\
Main RecrDev 2003 & -0.460 & 2 & (-5, 5) & act & 0.3080300 & dev (NA, NA)\\
Main RecrDev 2004 & -0.640 & 2 & (-5, 5) & act & 0.3070820 & dev (NA, NA)\\
Main RecrDev 2005 & -0.800 & 2 & (-5, 5) & act & 0.3248390 & dev (NA, NA)\\
Main RecrDev 2006 & -0.239 & 2 & (-5, 5) & act & 0.3438300 & dev (NA, NA)\\
Main RecrDev 2007 & 0.697 & 2 & (-5, 5) & act & 0.2538510 & dev (NA, NA)\\
Main RecrDev 2008 & 0.431 & 2 & (-5, 5) & act & 0.3391750 & dev (NA, NA)\\
Main RecrDev 2009 & 0.608 & 2 & (-5, 5) & act & 0.2872840 & dev (NA, NA)\\
Main RecrDev 2010 & -0.172 & 2 & (-5, 5) & act & 0.3613040 & dev (NA, NA)\\
Main RecrDev 2011 & -0.083 & 2 & (-5, 5) & act & 0.3528960 & dev (NA, NA)\\
Main RecrDev 2012 & 0.471 & 2 & (-5, 5) & act & 0.2792570 & dev (NA, NA)\\
Main RecrDev 2013 & 0.282 & 2 & (-5, 5) & act & 0.3130770 & dev (NA, NA)\\
Main RecrDev 2014 & 0.097 & 2 & (-5, 5) & act & 0.3227050 & dev (NA, NA)\\
Main RecrDev 2015 & 0.320 & 2 & (-5, 5) & act & 0.2687810 & dev (NA, NA)\\
Main RecrDev 2016 & -0.443 & 2 & (-5, 5) & act & 0.3637980 & dev (NA, NA)\\
Main RecrDev 2017 & 0.668 & 2 & (-5, 5) & act & 0.2497840 & dev (NA, NA)\\
Main RecrDev 2018 & 0.268 & 2 & (-5, 5) & act & 0.2998240 & dev (NA, NA)\\
Main RecrDev 2019 & -0.365 & 2 & (-5, 5) & act & 0.3784280 & dev (NA, NA)\\
Late RecrDev 2020 & 0.000 & NA & (NA, NA) & NA & NA & dev (NA, NA)\\
Late RecrDev 2021 & 0.000 & NA & (NA, NA) & NA & NA & dev (NA, NA)\\
Late RecrDev 2022 & 0.000 & NA & (NA, NA) & NA & NA & dev (NA, NA)\\
ForeRecr 2023 & 0.000 & NA & (NA, NA) & NA & NA & dev (NA, NA)\\
ForeRecr 2024 & 0.000 & NA & (NA, NA) & NA & NA & dev (NA, NA)\\
ForeRecr 2025 & 0.000 & NA & (NA, NA) & NA & NA & dev (NA, NA)\\
ForeRecr 2026 & 0.000 & NA & (NA, NA) & NA & NA & dev (NA, NA)\\
ForeRecr 2027 & 0.000 & NA & (NA, NA) & NA & NA & dev (NA, NA)\\
ForeRecr 2028 & 0.000 & NA & (NA, NA) & NA & NA & dev (NA, NA)\\
ForeRecr 2029 & 0.000 & NA & (NA, NA) & NA & NA & dev (NA, NA)\\
ForeRecr 2030 & 0.000 & NA & (NA, NA) & NA & NA & dev (NA, NA)\\
ForeRecr 2031 & 0.000 & NA & (NA, NA) & NA & NA & dev (NA, NA)\\
ForeRecr 2032 & 0.000 & NA & (NA, NA) & NA & NA & dev (NA, NA)\\
ForeRecr 2033 & 0.000 & NA & (NA, NA) & NA & NA & dev (NA, NA)\\
ForeRecr 2034 & 0.000 & NA & (NA, NA) & NA & NA & dev (NA, NA)\\
LnQ base Rec CPFV(3) & -9.898 & -1 & (-15, 15) & NA & NA & None\\
LnQ base Rec PR(4) & -4.634 & -1 & (-15, 15) & NA & NA & None\\
Q extraSD Rec PR(4) & 0.310 & 1 & (0, 0.5) & OK & 0.0716220 & None\\
LnQ base CCFRP(5) & -10.046 & -1 & (-15, 15) & NA & NA & None\\
Q extraSD CCFRP(5) & 0.220 & 1 & (0, 0.5) & OK & 0.0664570 & None\\
LnQ base CDFW ROV(6) & -10.981 & -1 & (-15, 15) & NA & NA & None\\
Q extraSD CDFW ROV(6) & 0.089 & 1 & (0, 0.5) & OK & 0.0970709 & None\\
LnQ base DWV CPFV(7) & -9.848 & -1 & (-15, 15) & NA & NA & None\\
Q extraSD DWV CPFV(7) & 0.248 & 1 & (0, 0.5) & OK & 0.0944555 & None\\
LnQ base CRFS CPFV(8) & -12.051 & -1 & (-15, 15) & NA & NA & None\\
Q extraSD CRFS CPFV(8) & 0.095 & 1 & (0, 0.5) & OK & 0.0431358 & None\\
Size inflection Commercial Dead(1) & 33.927 & 4 & (20, 53) & OK & 1.6891100 & None\\
Size 95\%width Commercial Dead(1) & 10.946 & 4 & (0.001, 50) & OK & 1.6774100 & None\\
Size DblN peak Commercial Live(2) & 27.724 & 1 & (15, 53) & OK & 0.4255870 & None\\
Size DblN top logit Commercial Live(2) & -6.838 & -3 & (-7, 7) & NA & NA & None\\
Size DblN ascend se Commercial Live(2) & 1.716 & -3 & (-10, 10) & NA & NA & None\\
Size DblN descend se Commercial Live(2) & 3.802 & 4 & (-10, 10) & OK & 0.3813730 & None\\
Size DblN start logit Commercial Live(2) & -20.000 & -5 & (-20, 30) & NA & NA & None\\
Size DblN end logit Commercial Live(2) & -3.891 & 4 & (-10, 10) & OK & 1.4258500 & None\\
Size DblN peak Rec CPFV(3) & 37.303 & 2 & (15, 53) & OK & 2.1069200 & None\\
Size DblN top logit Rec CPFV(3) & -1.047 & -3 & (-7, 7) & NA & NA & None\\
Size DblN ascend se Rec CPFV(3) & 4.319 & 3 & (-10, 10) & OK & 0.3801860 & None\\
Size DblN descend se Rec CPFV(3) & 0.738 & -4 & (-10, 10) & NA & NA & None\\
Size DblN start logit Rec CPFV(3) & -20.000 & -9 & (-20, 30) & NA & NA & None\\
Size DblN end logit Rec CPFV(3) & 10.000 & -4 & (-10, 10) & NA & NA & None\\
Size DblN peak Rec PR(4) & 32.369 & 2 & (15, 53) & OK & 0.8489510 & None\\
Size DblN top logit Rec PR(4) & -1.047 & -3 & (-7, 7) & NA & NA & None\\
Size DblN ascend se Rec PR(4) & 3.626 & 3 & (-10, 10) & OK & 0.2146820 & None\\
Size DblN descend se Rec PR(4) & 9.413 & -4 & (-10, 10) & NA & NA & None\\
Size DblN start logit Rec PR(4) & -20.000 & -9 & (-20, 30) & NA & NA & None\\
Size DblN end logit Rec PR(4) & -7.885 & -4 & (-10, 10) & NA & NA & None\\
Size DblN peak CCFRP(5) & 33.183 & 1 & (15, 53) & OK & 1.0263300 & None\\
Size DblN top logit CCFRP(5) & -3.965 & -3 & (-7, 7) & NA & NA & None\\
Size DblN ascend se CCFRP(5) & 3.899 & 3 & (-10, 10) & OK & 0.2554170 & None\\
Size DblN descend se CCFRP(5) & 6.447 & -4 & (-10, 10) & NA & NA & None\\
Size DblN start logit CCFRP(5) & -20.000 & -5 & (-20, 30) & NA & NA & None\\
Size DblN end logit CCFRP(5) & 8.784 & -4 & (-10, 10) & NA & NA & None\\
Size inflection CDFW ROV(6) & 31.698 & 4 & (10, 53) & OK & 4.9517600 & None\\
Size 95\%width CDFW ROV(6) & 13.370 & 4 & (0.001, 50) & OK & 5.8310300 & None\\
Size DblN descend se Commercial Live(2) BLK1repl 1916 & 10.000 & -3 & (-10, 10) & NA & NA & None\\
Size DblN end logit Commercial Live(2) BLK1repl 1916 & 10.000 & -3 & (-10, 10) & NA & NA & None\\
Size DblN peak Rec CPFV(3) BLK2repl 1916 & 36.138 & 3 & (15, 53) & OK & 0.8349190 & None\\
Size DblN peak Rec CPFV(3) BLK2repl 2002 & 34.107 & 3 & (15, 53) & OK & 0.7225480 & None\\
Size DblN ascend se Rec CPFV(3) BLK2repl 1916 & 4.174 & 3 & (-10, 10) & OK & 0.1619150 & None\\
Size DblN ascend se Rec CPFV(3) BLK2repl 2002 & 4.023 & 3 & (-10, 10) & OK & 0.1716760 & None\\
Size DblN descend se Rec CPFV(3) BLK2repl 1916 & 4.443 & -6 & (-10, 10) & NA & NA & None\\
Size DblN descend se Rec CPFV(3) BLK2repl 2002 & 3.003 & -6 & (-10, 10) & NA & NA & None\\
Size DblN end logit Rec CPFV(3) BLK2repl 1916 & -2.190 & -5 & (-10, 10) & NA & NA & None\\
Size DblN end logit Rec CPFV(3) BLK2repl 2002 & -0.916 & -5 & (-10, 10) & NA & NA & None\\
Size DblN peak Rec PR(4) BLK2repl 1916 & 27.796 & -3 & (15, 53) & NA & NA & Normal (28, 3)\\
Size DblN peak Rec PR(4) BLK2repl 2002 & 30.476 & 3 & (15, 53) & OK & 0.4567600 & None\\
Size DblN ascend se Rec PR(4) BLK2repl 1916 & 3.659 & -3 & (-10, 10) & NA & NA & None\\
Size DblN ascend se Rec PR(4) BLK2repl 2002 & 3.331 & 3 & (-10, 10) & OK & 0.1376350 & None\\
Size DblN descend se Rec PR(4) BLK2repl 1916 & -5.198 & -6 & (-10, 10) & NA & NA & None\\
Size DblN descend se Rec PR(4) BLK2repl 2002 & 3.643 & -6 & (-10, 10) & NA & NA & None\\
Size DblN end logit Rec PR(4) BLK2repl 1916 & 0.689 & -5 & (-10, 10) & NA & NA & None\\
Size DblN end logit Rec PR(4) BLK2repl 2002 & -0.868 & 5 & (-10, 10) & OK & 0.2282210 & None\\
Size DblN top logit CCFRP(5) BLK3repl 1916 & -6.798 & -4 & (-7, 7) & NA & NA & None\\
Size DblN descend se CCFRP(5) BLK3repl 1916 & 4.846 & -4 & (-10, 10) & NA & NA & None\\
Size DblN end logit CCFRP(5) BLK3repl 1916 & -1.835 & -5 & (-10, 10) & NA & NA & None\\*
\end{longtable}
\endgroup{}
\end{landscape}
\endgroup{}
<<<<<<< Updated upstream
=======

\begingroup\fontsize{10}{12}\selectfont
\begingroup\fontsize{10}{12}\selectfont

\begin{longtable}[t]{r>{\centering\arraybackslash}p{2cm}}
\caption{\label{tab:north-removalsES}Likelihood components by source.}\\
\toprule
Label & Total\\
\midrule
\endfirsthead
\caption[]{Likelihood components by source. \textit{(continued)}}\\
\toprule
Label & Total\\
\midrule
\endhead

\endfoot
\bottomrule
\endlastfoot
TOTAL & 973.82\\
Catch & 0.00\\
Equil catch & 0.00\\
Survey & -45.27\\
Length comp & 416.49\\
Age comp & 596.55\\
Recruitment & 6.05\\
InitEQ Regime & 0.00\\
Forecast Recruitment & 0.00\\
Parm priors & 0.00\\
Parm softbounds & 0.00\\
Parm devs & 0.00\\
Crash Pen & 0.00\\*
\end{longtable}
\endgroup{}
\endgroup{}


\begingroup\fontsize{10}{12}\selectfont
\begingroup\fontsize{10}{12}\selectfont

\begin{longtable}[t]{r>{\centering\arraybackslash}p{2cm}>{\centering\arraybackslash}p{2cm}>{\centering\arraybackslash}p{2cm}}
	\caption{\label{tab:north-referenceES}Summary of reference points and management quantities, including estimates of the 95 percent intervals for the sub-area model north of Point Conception.}\\
	\toprule
	& Estimate & Lower Interval & Upper Interval\\
	\midrule
	\endfirsthead
	\caption[]{Summary of reference points and management quantities, including estimates of the 95 percent intervals for the sub-area model north of Point Conception. \textit{(continued)}}\\
	\toprule
	& Estimate & Lower Interval & Upper Interval\\
	\midrule
	\endhead
	
	\endfoot
	\bottomrule
	\endlastfoot
	Unfished Spawning Output & 486.15 & 387.43 & 584.87\\
	Unfished Age 3+ Biomass (mt) & 4719.91 & 3777.92 & 5661.90\\
	Unfished Recruitment (R0) & 567.77 & 452.48 & 683.06\\
	Spawning Output (2023) & 262.10 & 124.28 & 399.92\\
	Fraction Unfished (2023) & 0.54 & 0.32 & 0.76\\
	Reference Points Based SB40\% &  &  & \\
	Proxy Spawning Output SB40\% & 194.46 & 154.97 & 233.95\\
	SPR Resulting in SB40\% & 0.46 & 0.46 & 0.46\\
	Exploitation Rate Resulting in SB40\% & 0.06 & 0.06 & 0.06\\
	Yield with SPR Based On SB40\% (mt) & 129.86 & 104.05 & 155.67\\
	Reference Points Based on SPR Proxy for MSY & NA & NA & NA\\
	Proxy Spawning Output (SPR50) & 216.90 & 172.85 & 260.94\\
	SPR50 & 0.50 &  & \\
	Exploitation Rate Corresponding to SPR50 & 0.05 & 0.05 & 0.05\\
	Yield with SPR50 at SB SPR (mt) & 124.05 & 99.39 & 148.71\\
	Reference Points Based on Estimated MSY Values & NA & NA & NA\\
	Spawning Output at MSY (SB MSY) & 134.17 & 106.84 & 161.51\\
	SPR MSY & 0.35 & 0.34 & 0.35\\
	Exploitation Rate Corresponding to SPR MSY & 0.09 & 0.08 & 0.09\\
	MSY (mt) & 137.59 & 110.25 & 164.92\\*
\end{longtable}
\endgroup{}
\endgroup{}


\newpage

\begingroup\fontsize{10}{12}\selectfont
\begingroup\fontsize{10}{12}\selectfont

\begin{longtable}[t]{r>{\centering\arraybackslash}p{1.22cm}>{\centering\arraybackslash}p{1.22cm}>{\centering\arraybackslash}p{1.22cm}>{\centering\arraybackslash}p{1.22cm}>{\centering\arraybackslash}p{1.22cm}>{\centering\arraybackslash}p{1.22cm}>{\centering\arraybackslash}p{1.22cm}>{\centering\arraybackslash}p{1.22cm}}
\caption{\label{tab:north-timeseries}Time series of population estimates from the base model for the sub-area north of Point Conception.}\\
\toprule
Year & Total Biomass (mt) & Spawning Output & Total Biomass 3+ (mt) & Fraction Unfished & Age-0 Recruits & Total Mortality (mt) & 1-SPR & Exploitation Rate\\
\midrule
\endfirsthead
\caption[]{Time series of population estimates from the base model for the sub-area north of Point Conception. \textit{(continued)}}\\
\toprule
Year & Total Biomass (mt) & Spawning Output & Total Biomass 3+ (mt) & Fraction Unfished & Age-0 Recruits & Total Mortality (mt) & 1-SPR & Exploitation Rate\\
\midrule
\endhead

\endfoot
\bottomrule
\endlastfoot
1916 & 4758.75 & 487.03 & 4731.64 & 1.00 & 573.48 & 3.97 & 0.01 & 0.00\\
1917 & 4756.53 & 486.73 & 4729.41 & 1.00 & 573.92 & 6.23 & 0.02 & 0.00\\
1918 & 4752.47 & 486.22 & 4725.31 & 1.00 & 574.38 & 7.50 & 0.02 & 0.00\\
1919 & 4747.63 & 485.61 & 4720.46 & 1.00 & 574.87 & 4.91 & 0.01 & 0.00\\
1920 & 4745.78 & 485.32 & 4718.58 & 1.00 & 575.44 & 5.06 & 0.01 & 0.00\\
1921 & 4744.18 & 485.05 & 4716.96 & 1.00 & 576.07 & 4.31 & 0.01 & 0.00\\
1922 & 4743.72 & 484.91 & 4716.47 & 1.00 & 576.76 & 3.70 & 0.01 & 0.00\\
1923 & 4744.21 & 484.87 & 4716.93 & 1.00 & 577.52 & 3.89 & 0.01 & 0.00\\
1924 & 4744.88 & 484.85 & 4717.56 & 1.00 & 578.34 & 2.57 & 0.01 & 0.00\\
1925 & 4747.14 & 485.02 & 4719.79 & 1.00 & 579.26 & 3.84 & 0.01 & 0.00\\
1926 & 4748.48 & 485.07 & 4721.09 & 1.00 & 580.23 & 4.89 & 0.01 & 0.00\\
1927 & 4749.17 & 485.05 & 4721.73 & 1.00 & 581.27 & 3.64 & 0.01 & 0.00\\
1928 & 4751.45 & 485.20 & 4723.97 & 1.00 & 582.41 & 5.21 & 0.02 & 0.00\\
1929 & 4752.50 & 485.22 & 4724.97 & 1.00 & 583.84 & 6.18 & 0.02 & 0.00\\
1930 & 4752.95 & 485.18 & 4725.36 & 1.00 & 585.39 & 8.96 & 0.03 & 0.00\\
1931 & 4751.19 & 484.88 & 4723.53 & 1.00 & 587.04 & 11.15 & 0.03 & 0.00\\
1932 & 4747.92 & 484.39 & 4720.18 & 1.00 & 588.84 & 11.73 & 0.04 & 0.00\\
1933 & 4744.80 & 483.91 & 4716.98 & 1.00 & 590.78 & 12.11 & 0.04 & 0.00\\
1934 & 4742.11 & 483.45 & 4714.21 & 0.99 & 592.87 & 12.01 & 0.04 & 0.00\\
1935 & 4740.36 & 483.08 & 4712.37 & 0.99 & 595.19 & 15.35 & 0.05 & 0.00\\
1936 & 4736.34 & 482.44 & 4708.24 & 0.99 & 597.65 & 16.09 & 0.05 & 0.00\\
1937 & 4732.71 & 481.81 & 4704.51 & 0.99 & 600.32 & 18.78 & 0.06 & 0.00\\
1938 & 4727.67 & 481.01 & 4699.34 & 0.99 & 603.21 & 17.93 & 0.06 & 0.00\\
1939 & 4724.86 & 480.41 & 4696.41 & 0.99 & 606.46 & 16.05 & 0.05 & 0.00\\
1940 & 4725.39 & 480.14 & 4696.79 & 0.99 & 610.18 & 20.78 & 0.06 & 0.00\\
1941 & 4722.63 & 479.52 & 4693.88 & 0.99 & 614.38 & 19.96 & 0.06 & 0.00\\
1942 & 4722.37 & 479.12 & 4693.43 & 0.99 & 619.23 & 9.64 & 0.03 & 0.00\\
1943 & 4734.07 & 479.95 & 4704.93 & 0.99 & 624.88 & 10.40 & 0.03 & 0.00\\
1944 & 4746.61 & 480.85 & 4717.22 & 0.99 & 631.13 & 14.81 & 0.04 & 0.00\\
1945 & 4756.67 & 481.43 & 4727.01 & 0.99 & 637.77 & 29.60 & 0.08 & 0.01\\
1946 & 4754.55 & 480.61 & 4724.59 & 0.99 & 644.39 & 38.03 & 0.11 & 0.01\\
1947 & 4746.83 & 479.14 & 4716.56 & 0.99 & 650.70 & 18.34 & 0.06 & 0.00\\
1948 & 4761.18 & 480.03 & 4730.59 & 0.99 & 656.79 & 31.93 & 0.10 & 0.01\\
1949 & 4764.49 & 479.75 & 4733.61 & 0.99 & 661.72 & 34.12 & 0.10 & 0.01\\
1950 & 4768.23 & 479.52 & 4737.08 & 0.99 & 665.42 & 39.31 & 0.12 & 0.01\\
1951 & 4769.54 & 479.03 & 4738.17 & 0.99 & 667.86 & 53.91 & 0.16 & 0.01\\
1952 & 4759.34 & 477.30 & 4727.81 & 0.98 & 669.38 & 45.06 & 0.13 & 0.01\\
1953 & 4761.01 & 476.77 & 4729.37 & 0.98 & 671.53 & 36.30 & 0.11 & 0.01\\
1954 & 4774.05 & 477.47 & 4742.33 & 0.98 & 675.16 & 46.94 & 0.14 & 0.01\\
1955 & 4778.65 & 477.39 & 4746.82 & 0.98 & 679.63 & 52.40 & 0.16 & 0.01\\
1956 & 4779.65 & 477.06 & 4747.63 & 0.98 & 686.18 & 60.11 & 0.17 & 0.01\\
1957 & 4774.87 & 476.17 & 4742.63 & 0.98 & 683.59 & 58.27 & 0.17 & 0.01\\
1958 & 4774.03 & 475.66 & 4741.58 & 0.98 & 682.24 & 99.22 & 0.27 & 0.02\\
1959 & 4733.91 & 471.20 & 4701.56 & 0.97 & 684.64 & 80.28 & 0.23 & 0.02\\
1960 & 4715.50 & 468.80 & 4683.16 & 0.96 & 699.37 & 68.30 & 0.20 & 0.01\\
1961 & 4711.74 & 467.80 & 4679.12 & 0.96 & 732.19 & 51.06 & 0.15 & 0.01\\
1962 & 4727.91 & 468.84 & 4694.37 & 0.96 & 777.94 & 63.56 & 0.19 & 0.01\\
1963 & 4734.77 & 468.93 & 4699.55 & 0.96 & 817.20 & 79.29 & 0.23 & 0.02\\
1964 & 4731.24 & 467.64 & 4693.89 & 0.96 & 881.00 & 70.73 & 0.21 & 0.02\\
1965 & 4744.87 & 467.41 & 4705.34 & 0.96 & 983.81 & 104.92 & 0.29 & 0.02\\
1966 & 4736.12 & 464.09 & 4693.15 & 0.95 & 1067.20 & 121.03 & 0.32 & 0.03\\
1967 & 4728.19 & 459.74 & 4680.69 & 0.95 & 1051.05 & 128.37 & 0.34 & 0.03\\
1968 & 4734.70 & 455.64 & 4684.53 & 0.94 & 923.06 & 135.98 & 0.35 & 0.03\\
1969 & 4756.61 & 452.27 & 4708.54 & 0.93 & 741.98 & 145.84 & 0.37 & 0.03\\
1970 & 4786.46 & 449.99 & 4744.98 & 0.93 & 608.42 & 181.20 & 0.43 & 0.04\\
1971 & 4786.28 & 446.80 & 4752.87 & 0.92 & 455.26 & 169.15 & 0.41 & 0.04\\
1972 & 4788.22 & 447.24 & 4761.28 & 0.92 & 357.88 & 215.21 & 0.48 & 0.05\\
1973 & 4718.98 & 444.91 & 4698.51 & 0.92 & 373.18 & 246.30 & 0.52 & 0.05\\
1974 & 4582.63 & 439.32 & 4565.55 & 0.90 & 350.26 & 270.39 & 0.56 & 0.06\\
1975 & 4383.13 & 428.95 & 4365.71 & 0.88 & 365.48 & 267.98 & 0.58 & 0.06\\
1976 & 4150.72 & 414.85 & 4134.02 & 0.85 & 332.35 & 296.45 & 0.63 & 0.07\\
1977 & 3862.72 & 393.09 & 3845.83 & 0.81 & 306.02 & 306.11 & 0.67 & 0.08\\
1978 & 3547.11 & 365.81 & 3531.76 & 0.75 & 239.16 & 281.47 & 0.68 & 0.08\\
1979 & 3245.65 & 337.69 & 3231.98 & 0.69 & 192.93 & 292.80 & 0.72 & 0.09\\
1980 & 2926.86 & 306.26 & 2916.03 & 0.63 & 222.56 & 305.18 & 0.75 & 0.10\\
1981 & 2594.11 & 272.27 & 2584.49 & 0.56 & 344.88 & 287.77 & 0.77 & 0.11\\
1982 & 2277.63 & 239.84 & 2265.84 & 0.49 & 243.16 & 284.03 & 0.80 & 0.13\\
1983 & 1971.16 & 207.66 & 1955.99 & 0.43 & 243.06 & 209.77 & 0.73 & 0.11\\
1984 & 1754.30 & 183.00 & 1742.51 & 0.38 & 462.12 & 185.11 & 0.74 & 0.11\\
1985 & 1576.13 & 161.71 & 1562.04 & 0.33 & 551.70 & 206.53 & 0.81 & 0.13\\
1986 & 1392.78 & 139.51 & 1370.20 & 0.29 & 314.58 & 164.96 & 0.79 & 0.12\\
1987 & 1271.95 & 122.97 & 1248.57 & 0.25 & 274.54 & 100.25 & 0.66 & 0.08\\
1988 & 1238.68 & 114.26 & 1224.21 & 0.24 & 299.29 & 112.68 & 0.70 & 0.09\\
1989 & 1207.29 & 106.33 & 1193.84 & 0.22 & 441.70 & 122.44 & 0.72 & 0.10\\
1990 & 1174.70 & 99.83 & 1158.82 & 0.21 & 518.43 & 133.18 & 0.75 & 0.11\\
1991 & 1138.25 & 94.81 & 1116.68 & 0.20 & 361.75 & 131.79 & 0.75 & 0.12\\
1992 & 1112.37 & 90.96 & 1089.66 & 0.19 & 316.48 & 142.04 & 0.77 & 0.13\\
1993 & 1087.78 & 86.39 & 1070.99 & 0.18 & 462.51 & 162.74 & 0.82 & 0.15\\
1994 & 1048.63 & 80.67 & 1032.13 & 0.17 & 361.77 & 93.49 & 0.68 & 0.09\\
1995 & 1079.12 & 82.08 & 1058.52 & 0.17 & 247.39 & 76.06 & 0.60 & 0.07\\
1996 & 1130.71 & 86.01 & 1114.85 & 0.18 & 272.30 & 83.98 & 0.61 & 0.08\\
1997 & 1174.03 & 90.16 & 1161.98 & 0.19 & 315.62 & 80.03 & 0.58 & 0.07\\
1998 & 1215.54 & 95.08 & 1202.13 & 0.20 & 339.49 & 44.90 & 0.40 & 0.04\\
1999 & 1284.31 & 103.40 & 1269.15 & 0.21 & 302.54 & 37.01 & 0.33 & 0.03\\
2000 & 1354.22 & 112.46 & 1338.64 & 0.23 & 251.04 & 31.63 & 0.27 & 0.02\\
2001 & 1423.65 & 121.31 & 1409.88 & 0.25 & 274.09 & 28.95 & 0.25 & 0.02\\
2002 & 1489.36 & 129.50 & 1477.16 & 0.27 & 313.55 & 21.83 & 0.18 & 0.01\\
2003 & 1554.22 & 137.65 & 1540.85 & 0.28 & 268.18 & 23.78 & 0.19 & 0.02\\
2004 & 1609.02 & 145.10 & 1594.74 & 0.30 & 227.39 & 18.92 & 0.15 & 0.01\\
2005 & 1661.84 & 152.32 & 1649.64 & 0.31 & 196.19 & 34.85 & 0.26 & 0.02\\
2006 & 1690.21 & 157.19 & 1679.61 & 0.32 & 346.50 & 36.21 & 0.27 & 0.02\\
2007 & 1709.37 & 161.04 & 1697.69 & 0.33 & 888.98 & 42.04 & 0.30 & 0.02\\
2008 & 1720.32 & 163.55 & 1698.04 & 0.34 & 683.78 & 31.80 & 0.24 & 0.02\\
2009 & 1752.87 & 166.22 & 1712.94 & 0.34 & 819.71 & 39.21 & 0.29 & 0.02\\
2010 & 1806.79 & 167.32 & 1773.45 & 0.34 & 376.08 & 27.47 & 0.21 & 0.02\\
2011 & 1904.77 & 169.37 & 1870.94 & 0.35 & 412.57 & 25.88 & 0.19 & 0.01\\
2012 & 2031.54 & 173.50 & 2012.93 & 0.36 & 721.98 & 34.91 & 0.23 & 0.02\\
2013 & 2160.53 & 180.75 & 2137.65 & 0.37 & 602.74 & 25.64 & 0.16 & 0.01\\
2014 & 2302.33 & 193.08 & 2269.62 & 0.40 & 508.66 & 36.94 & 0.20 & 0.02\\
2015 & 2435.83 & 207.33 & 2408.19 & 0.43 & 644.80 & 65.46 & 0.31 & 0.03\\
2016 & 2538.95 & 219.54 & 2513.77 & 0.45 & 304.21 & 66.32 & 0.31 & 0.03\\
2017 & 2635.51 & 230.78 & 2608.04 & 0.47 & 932.00 & 138.62 & 0.50 & 0.05\\
2018 & 2657.53 & 234.32 & 2636.38 & 0.48 & 649.02 & 102.38 & 0.41 & 0.04\\
2019 & 2709.53 & 240.87 & 2668.99 & 0.50 & 358.62 & 101.89 & 0.40 & 0.04\\
2020 & 2765.07 & 247.00 & 2737.43 & 0.51 & 518.96 & 156.21 & 0.53 & 0.06\\
2021 & 2765.72 & 247.07 & 2746.93 & 0.51 & 518.99 & 106.99 & 0.41 & 0.04\\
2022 & 2807.65 & 251.40 & 2783.08 & 0.52 & 520.55 & 47.15 & 0.21 & 0.02\\
2023 & 2899.87 & 262.10 & 2875.27 & 0.54 & 524.24 & 91.50 & 0.35 & 0.03\\
2024 & 2938.51 & 268.97 & 2913.82 & 0.55 & 526.48 & 66.90 & 0.27 & 0.02\\
2025 & 2993.09 & 277.24 & 2968.25 & 0.57 & 529.05 & 105.37 & 0.38 & 0.04\\
2026 & 3002.22 & 280.38 & 2977.27 & 0.58 & 529.99 & 104.04 & 0.38 & 0.03\\
2027 & 3007.39 & 282.32 & 2982.33 & 0.58 & 530.56 & 102.85 & 0.38 & 0.03\\
2028 & 3010.14 & 283.48 & 2985.04 & 0.58 & 530.90 & 101.52 & 0.38 & 0.03\\
2029 & 3011.84 & 284.18 & 2986.72 & 0.58 & 531.10 & 99.77 & 0.37 & 0.03\\
2030 & 3013.81 & 284.70 & 2988.68 & 0.59 & 531.26 & 98.37 & 0.37 & 0.03\\
2031 & 3016.28 & 285.14 & 2991.13 & 0.59 & 531.38 & 97.00 & 0.36 & 0.03\\
2032 & 3019.57 & 285.60 & 2994.42 & 0.59 & 531.52 & 95.34 & 0.36 & 0.03\\
2033 & 3024.21 & 286.17 & 2999.06 & 0.59 & 531.68 & 94.10 & 0.36 & 0.03\\
2034 & 3029.85 & 286.82 & 3004.69 & 0.59 & 531.87 & 102.94 & 0.38 & 0.03\\*
\end{longtable}
\endgroup{}
\endgroup{}


\newpage

\begingroup\fontsize{10}{12}\selectfont
\begingroup\fontsize{10}{12}\selectfont

\begin{longtable}[t]{c>{\centering\arraybackslash}p{1.83cm}>{\centering\arraybackslash}p{1.83cm}>{\centering\arraybackslash}p{1.83cm}>{\centering\arraybackslash}p{1.83cm}>{\centering\arraybackslash}p{1.83cm}}
\caption{\label{tab:ca-status}The estimated total biomass (mt), total biomass age 3+ (mt), age-0 recruits, spawning ouput in number of million eggs across California and fraction unfished by year.}\\
\toprule
Year & Total Biomass (mt) & Total Biomass 3+ (mt) & Age-0 Recruits & Spawning Output & Fraction Unfished\\
\midrule
\endfirsthead
\caption[]{\label{tab:ca-status}The estimated total biomass (mt), total biomass age 3+ (mt), age-0 recruits, spawning ouput in number of million eggs across California and fraction unfished by year. \textit{(continued)}}\\
\toprule
Year & Total Biomass (mt) & Total Biomass 3+ (mt) & Age-0 Recruits & Spawning Output & Fraction Unfished\\
\midrule
\endhead

\endfoot
\bottomrule
\endlastfoot
1916 & 6771.66 & 6726.78 & 483.56 & 688.74 & 1.001\\
1917 & 6769.45 & 6724.56 & 483.64 & 688.45 & 1.001\\
1918 & 6765.34 & 6720.40 & 483.72 & 687.93 & 1.000\\
1919 & 6760.48 & 6715.54 & 483.82 & 687.31 & 0.999\\
1920 & 6758.72 & 6713.74 & 483.92 & 687.02 & 0.999\\
1921 & 6757.21 & 6712.21 & 484.04 & 686.76 & 0.999\\
1922 & 6756.88 & 6711.84 & 484.16 & 686.62 & 0.998\\
1923 & 6757.52 & 6712.45 & 484.30 & 686.60 & 0.998\\
1924 & 6758.33 & 6713.22 & 484.45 & 686.59 & 0.998\\
1925 & 6760.72 & 6715.57 & 484.61 & 686.76 & 0.999\\
1926 & 6762.19 & 6716.99 & 484.79 & 686.83 & 0.999\\
1927 & 6763.00 & 6717.75 & 484.98 & 686.81 & 0.999\\
1928 & 6765.48 & 6720.18 & 485.19 & 686.97 & 0.999\\
1929 & 6766.75 & 6721.40 & 485.42 & 687.01 & 0.999\\
1930 & 6767.44 & 6722.01 & 485.67 & 686.99 & 0.999\\
1931 & 6765.92 & 6720.42 & 485.95 & 686.70 & 0.998\\
1932 & 6762.93 & 6717.34 & 486.24 & 686.24 & 0.998\\
1933 & 6760.05 & 6714.37 & 486.56 & 685.76 & 0.997\\
1934 & 6757.78 & 6712.00 & 486.92 & 685.34 & 0.996\\
1935 & 6756.39 & 6710.51 & 487.30 & 685.00 & 0.996\\
1936 & 6752.51 & 6706.51 & 487.71 & 684.36 & 0.995\\
1937 & 6749.24 & 6703.12 & 488.15 & 683.76 & 0.994\\
1938 & 6743.91 & 6697.65 & 488.62 & 682.90 & 0.993\\
1939 & 6741.38 & 6694.98 & 489.14 & 682.32 & 0.992\\
1940 & 6742.49 & 6695.93 & 489.72 & 682.08 & 0.992\\
1941 & 6740.38 & 6693.65 & 490.33 & 681.51 & 0.991\\
1942 & 6740.81 & 6693.87 & 491.00 & 681.16 & 0.990\\
1943 & 6753.76 & 6706.59 & 491.72 & 682.09 & 0.992\\
1944 & 6767.59 & 6720.14 & 492.49 & 683.10 & 0.993\\
1945 & 6779.15 & 6731.41 & 493.30 & 683.81 & 0.994\\
1946 & 6778.55 & 6730.48 & 494.16 & 683.12 & 0.993\\
1947 & 6772.42 & 6724.01 & 495.04 & 681.79 & 0.991\\
1948 & 6787.89 & 6739.13 & 495.92 & 682.78 & 0.993\\
1949 & 6791.34 & 6742.26 & 496.75 & 682.50 & 0.992\\
1950 & 6794.80 & 6745.41 & 497.53 & 682.23 & 0.992\\
1951 & 6795.11 & 6745.48 & 498.22 & 681.61 & 0.991\\
1952 & 6781.60 & 6731.78 & 498.76 & 679.48 & 0.988\\
1953 & 6781.57 & 6731.62 & 499.17 & 678.73 & 0.987\\
1954 & 6793.47 & 6743.42 & 499.31 & 679.27 & 0.988\\
1955 & 6792.50 & 6742.32 & 498.98 & 678.64 & 0.987\\
1956 & 6779.53 & 6729.16 & 498.24 & 676.97 & 0.984\\
1957 & 6759.35 & 6708.78 & 497.29 & 674.55 & 0.981\\
1958 & 6751.38 & 6700.63 & 496.75 & 673.22 & 0.979\\
1959 & 6704.70 & 6654.08 & 496.83 & 667.98 & 0.971\\
1960 & 6685.45 & 6634.85 & 498.53 & 665.34 & 0.967\\
1961 & 6680.50 & 6629.60 & 501.47 & 664.12 & 0.966\\
1962 & 6693.08 & 6641.20 & 505.68 & 664.72 & 0.966\\
1963 & 6700.02 & 6646.33 & 512.94 & 664.72 & 0.966\\
1964 & 6696.71 & 6640.70 & 527.17 & 663.35 & 0.964\\
1965 & 6706.31 & 6647.79 & 557.75 & 662.63 & 0.963\\
1966 & 6688.99 & 6626.36 & 613.84 & 658.30 & 0.957\\
1967 & 6647.19 & 6578.69 & 692.63 & 650.44 & 0.946\\
1968 & 6616.98 & 6543.59 & 741.31 & 642.11 & 0.934\\
1969 & 6600.25 & 6526.41 & 632.10 & 633.79 & 0.922\\
1970 & 6613.12 & 6545.34 & 524.22 & 628.00 & 0.913\\
1971 & 6580.04 & 6524.32 & 456.46 & 619.92 & 0.901\\
1972 & 6554.52 & 6508.80 & 537.03 & 616.60 & 0.897\\
1973 & 6429.40 & 6391.48 & 585.75 & 609.13 & 0.886\\
1974 & 6212.80 & 6175.67 & 472.20 & 596.90 & 0.868\\
1975 & 5902.16 & 5864.17 & 413.01 & 576.83 & 0.839\\
1976 & 5553.05 & 5519.48 & 408.18 & 551.71 & 0.802\\
1977 & 5173.57 & 5141.51 & 460.98 & 520.85 & 0.757\\
1978 & 4773.59 & 4742.85 & 403.81 & 485.31 & 0.706\\
1979 & 4388.22 & 4358.13 & 320.16 & 449.06 & 0.653\\
1980 & 3940.02 & 3915.05 & 278.61 & 405.26 & 0.589\\
1981 & 3480.60 & 3459.55 & 293.65 & 358.87 & 0.522\\
1982 & 3100.46 & 3078.34 & 269.30 & 319.71 & 0.465\\
1983 & 2655.53 & 2629.80 & 243.16 & 274.64 & 0.399\\
1984 & 2375.06 & 2353.56 & 291.89 & 243.45 & 0.354\\
1985 & 2124.47 & 2100.97 & 423.38 & 214.84 & 0.312\\
1986 & 1845.66 & 1811.35 & 375.11 & 183.11 & 0.266\\
1987 & 1649.05 & 1610.57 & 318.75 & 158.20 & 0.230\\
1988 & 1571.30 & 1543.51 & 319.57 & 143.47 & 0.209\\
1989 & 1529.20 & 1504.06 & 302.07 & 132.51 & 0.193\\
1990 & 1489.34 & 1461.83 & 360.71 & 124.09 & 0.180\\
1991 & 1440.61 & 1407.47 & 390.95 & 117.50 & 0.171\\
1992 & 1411.63 & 1375.51 & 274.30 & 112.78 & 0.164\\
1993 & 1405.72 & 1375.57 & 182.29 & 109.06 & 0.159\\
1994 & 1402.16 & 1376.33 & 178.25 & 105.70 & 0.154\\
1995 & 1423.35 & 1396.07 & 206.33 & 106.98 & 0.156\\
1996 & 1471.73 & 1449.13 & 160.47 & 111.22 & 0.162\\
1997 & 1458.28 & 1439.05 & 129.43 & 111.93 & 0.163\\
1998 & 1492.25 & 1473.17 & 160.71 & 116.54 & 0.169\\
1999 & 1537.55 & 1517.25 & 357.39 & 123.32 & 0.179\\
2000 & 1584.44 & 1561.51 & 181.13 & 130.99 & 0.190\\
2001 & 1653.94 & 1628.49 & 123.13 & 139.74 & 0.203\\
2002 & 1730.34 & 1711.89 & 208.19 & 148.22 & 0.216\\
2003 & 1814.31 & 1795.64 & 351.55 & 157.39 & 0.229\\
2004 & 1886.01 & 1863.06 & 161.07 & 166.29 & 0.242\\
2005 & 1958.15 & 1934.51 & 236.37 & 175.25 & 0.255\\
2006 & 1994.02 & 1977.00 & 87.93 & 180.82 & 0.263\\
2007 & 2036.68 & 2017.34 & 330.52 & 186.63 & 0.271\\
2008 & 2050.75 & 2023.30 & 317.18 & 189.90 & 0.276\\
2009 & 2090.87 & 2038.55 & 851.87 & 193.68 & 0.282\\
2010 & 2160.63 & 2111.31 & 913.18 & 195.80 & 0.285\\
2011 & 2291.84 & 2226.56 & 408.77 & 198.79 & 0.289\\
2012 & 2452.86 & 2404.67 & 552.88 & 202.49 & 0.294\\
2013 & 2627.91 & 2588.61 & 928.75 & 210.14 & 0.306\\
2014 & 2795.04 & 2739.48 & 244.85 & 223.28 & 0.325\\
2015 & 2973.06 & 2916.90 & 102.22 & 240.91 & 0.350\\
2016 & 3097.56 & 3064.44 & 245.60 & 255.75 & 0.372\\
2017 & 3182.40 & 3150.11 & 122.12 & 268.46 & 0.390\\
2018 & 3182.40 & 3153.24 & 96.39 & 273.80 & 0.398\\
2019 & 3176.81 & 3131.98 & 112.02 & 279.23 & 0.406\\
2020 & 3179.52 & 3148.18 & 163.48 & 283.32 & 0.412\\
2021 & 3110.47 & 3087.05 & 317.57 & 278.75 & 0.405\\
2022 & 3108.97 & 3077.18 & 302.69 & 279.64 & 0.407\\
2023 & 3199.84 & 3163.70 & 299.87 & 289.74 & 0.421\\*
\end{longtable}
\endgroup{}
\endgroup{}

\newpage

\begingroup\fontsize{9}{11}\selectfont

\begin{landscape}\begingroup\fontsize{9}{11}\selectfont

\begin{longtable}[t]{l>{\centering\arraybackslash}p{0.52cm}>{\centering\arraybackslash}p{0.52cm}>{\centering\arraybackslash}p{0.52cm}>{\centering\arraybackslash}p{0.52cm}>{\centering\arraybackslash}p{0.52cm}>{\centering\arraybackslash}p{0.52cm}>{\centering\arraybackslash}p{0.52cm}>{\centering\arraybackslash}p{0.52cm}>{\centering\arraybackslash}p{0.52cm}>{\centering\arraybackslash}p{0.52cm}>{\centering\arraybackslash}p{0.52cm}>{\centering\arraybackslash}p{0.52cm}>{\centering\arraybackslash}p{0.52cm}>{\centering\arraybackslash}p{0.52cm}>{\centering\arraybackslash}p{0.52cm}>{\centering\arraybackslash}p{0.52cm}>{\centering\arraybackslash}p{0.52cm}>{\centering\arraybackslash}p{0.52cm}>{\centering\arraybackslash}p{0.52cm}>{\centering\arraybackslash}p{0.52cm}c}
\caption{\label{tab:sensitivities-1}Sensitivities relative to the base model.}\\
\toprule
  & Base Model & Estimate M & Estimate h & Estimate M \& h & No Rec. Devs. & CPFV Selectivity Asym. & Dirichlet DW & McAllister-Ianelli DW & Reduce PR Catch 1970-82 & Hist. CPFV Ages Lambda = 1 & Rm. All Ages & Rm. Coop. Ages & Rm. WCGBT Ages & Rm. CPFV \& PR Indices & Rm. CCFRP & Rm. CDFW ROV & Rm. NWFSC HKL All & Rm. NWFSC HKL Ages & Rm. NWFSC HKL Lens. \& Ages & Rm. NWFSC HKL Index & Rm. All Surveys\\
\midrule
\endfirsthead
\caption[]{Sensitivities relative to the base model. \textit{(continued)}}\\
\toprule
  & Base Model & Estimate M & Estimate h & Estimate M \& h & No Rec. Devs. & CPFV Selectivity Asym. & Dirichlet DW & McAllister-Ianelli DW & Reduce PR Catch 1970-82 & Hist. CPFV Ages Lambda = 1 & Rm. All Ages & Rm. Coop. Ages & Rm. WCGBT Ages & Rm. CPFV \& PR Indices & Rm. CCFRP & Rm. CDFW ROV & Rm. NWFSC HKL All & Rm. NWFSC HKL Ages & Rm. NWFSC HKL Lens. \& Ages & Rm. NWFSC HKL Index & Rm. All Surveys\\
\midrule
\endhead

\endfoot
\bottomrule
\endlastfoot
Total Likelihood & 2874.130 & 2872.980 & 2874.030 & 2872.630 & 3193.270 & 2913.360 & 8423.880 & 2357.390 & 2872.950 & 2881.460 & 524.917 & 2780.760 & 1973.940 & 2904.170 & 2806.200 & 2854.980 & 1458.470 & 1570.030 & 1416.450 & 2877.970 & 1337.270\\
Survey Likelihood & -35.823 & -35.975 & -35.860 & -35.928 & -19.256 & -33.891 & -34.281 & -34.251 & -35.899 & -35.884 & -33.379 & -35.783 & -34.409 & -5.531 & -32.358 & -33.977 & -31.200 & -35.049 & -34.787 & -31.817 & -25.995\\
Length Likelihood & 576.193 & 576.341 & 576.382 & 575.535 & 696.910 & 587.355 & 4693.580 & 890.263 & 577.070 & 576.796 & 560.119 & 578.846 & 559.550 & 575.230 & 566.861 & 555.932 & 429.658 & 568.526 & 418.326 & 575.864 & 388.022\\
Age Likelihood & 2328.270 & 2326.550 & 2327.860 & 2327.140 & 2515.280 & 2355.450 & 3715.870 & 1496.880 & 2326.660 & 2332.900 & 0.000 & 2233.940 & 1446.140 & 2329.900 & 2267.050 & 2327.050 & 1058.350 & 1034.180 & 1031.170 & 2329.280 & 974.333\\
Recruitment Likelihood & 4.846 & 5.373 & 4.952 & 5.035 & 0.000 & 3.856 & 10.881 & 3.588 & 4.480 & 7.009 & -2.384 & 3.190 & 1.852 & 3.900 & 4.639 & 5.338 & 0.946 & 1.727 & 1.139 & 3.968 & 0.906\\
Forecast Recruitment Likelihood & 0.000 & 0.000 & 0.000 & 0.000 & 0.000 & 0.000 & 0.000 & 0.000 & 0.000 & 0.000 & 0.000 & 0.000 & 0.000 & 0.000 & 0.000 & 0.000 & 0.000 & 0.000 & 0.000 & 0.000 & 0.000\\
Parameter Priors Likelihood & 0.635 & 0.685 & 0.686 & 0.841 & 0.323 & 0.591 & 37.774 & 0.906 & 0.640 & 0.631 & 0.557 & 0.561 & 0.805 & 0.668 & 0.000 & 0.635 & 0.712 & 0.649 & 0.603 & 0.672 & 0.000\\
log(R0) & 5.485 & 5.349 & 5.514 & 5.236 & 5.544 & 5.422 & 5.503 & 5.507 & 5.305 & 5.439 & 5.515 & 5.480 & 5.521 & 5.497 & 5.479 & 5.482 & 5.576 & 5.539 & 5.539 & 5.498 & 5.545\\
SB Virgin & 201.347 & 219.421 & 207.254 & 219.858 & 212.294 & 183.818 & 195.866 & 205.963 & 168.291 & 192.578 & 207.648 & 204.542 & 204.771 & 203.863 & 201.723 & 200.964 & 213.229 & 203.080 & 207.071 & 204.174 & 211.980\\
SB 2023 & 27.584 & 23.620 & 26.848 & 23.797 & 44.151 & 27.792 & 44.565 & 38.217 & 26.684 & 27.380 & 26.787 & 25.960 & 37.522 & 34.602 & 24.759 & 27.629 & 59.084 & 36.196 & 39.686 & 35.324 & 46.951\\
Fraction Unfished 2023 & 0.137 & 0.108 & 0.130 & 0.108 & 0.208 & 0.151 & 0.228 & 0.186 & 0.159 & 0.142 & 0.129 & 0.127 & 0.183 & 0.170 & 0.123 & 0.137 & 0.277 & 0.178 & 0.192 & 0.173 & 0.221\\
Total Yield - SPR 50 & 47.903 & 46.426 & 48.300 & 44.639 & 52.335 & 48.355 & 46.719 & 48.758 & 39.961 & 45.632 & 51.467 & 48.425 & 49.295 & 48.476 & 47.883 & 47.808 & 50.962 & 49.079 & 49.602 & 48.475 & 50.086\\
Steepness & 0.720 & 0.720 & 0.691 & 0.783 & 0.720 & 0.720 & 0.720 & 0.720 & 0.720 & 0.720 & 0.720 & 0.720 & 0.720 & 0.720 & 0.720 & 0.720 & 0.720 & 0.720 & 0.720 & 0.720 & 0.720\\
Natural Mortality - Female & 0.108 & 0.096 & 0.108 & 0.090 & 0.108 & 0.108 & 0.108 & 0.108 & 0.108 & 0.108 & 0.108 & 0.108 & 0.108 & 0.108 & 0.108 & 0.108 & 0.108 & 0.108 & 0.108 & 0.108 & 0.108\\
Length at Amin - Female & 15.449 & 15.451 & 15.447 & 15.497 & 14.745 & 16.184 & 15.888 & 15.934 & 15.427 & 15.444 & 15.449 & 15.390 & 17.336 & 15.461 & 15.552 & 15.435 & 14.046 & 15.903 & 14.123 & 15.447 & 14.414\\
Length at Amax - Female & 46.829 & 46.806 & 46.827 & 46.792 & 46.431 & 46.602 & 46.109 & 46.778 & 46.827 & 46.856 & 46.829 & 47.123 & 46.373 & 46.816 & 47.004 & 46.828 & 46.409 & 46.471 & 46.570 & 46.814 & 46.989\\
Von Bert. k - Female & 0.191 & 0.191 & 0.191 & 0.191 & 0.209 & 0.183 & 0.192 & 0.190 & 0.192 & 0.191 & 0.191 & 0.191 & 0.189 & 0.192 & 0.187 & 0.192 & 0.198 & 0.177 & 0.195 & 0.192 & 0.184\\
CV young - Female & 0.196 & 0.196 & 0.196 & 0.195 & 0.199 & 0.188 & 0.183 & 0.186 & 0.196 & 0.196 & 0.196 & 0.196 & 0.149 & 0.196 & 0.197 & 0.196 & 0.190 & 0.222 & 0.192 & 0.196 & 0.197\\
CV old - Female & 0.060 & 0.060 & 0.060 & 0.060 & 0.062 & 0.060 & 0.078 & 0.065 & 0.060 & 0.060 & 0.060 & 0.055 & 0.068 & 0.060 & 0.058 & 0.059 & 0.079 & 0.067 & 0.076 & 0.060 & 0.074\\
Natural Mortality - Male & 0.108 & 0.000 & 0.108 & 0.093 & 0.108 & 0.108 & 0.108 & 0.108 & 0.108 & 0.108 & 0.108 & 0.108 & 0.108 & 0.108 & 0.108 & 0.108 & 0.108 & 0.108 & 0.108 & 0.108 & 0.108\\
Length at Amin - Male & 15.771 & 15.793 & 15.774 & 15.749 & 15.760 & 16.725 & 16.065 & 15.478 & 15.792 & 15.785 & 15.771 & 15.621 & 15.966 & 15.779 & 15.641 & 15.782 & 16.379 & 14.925 & 16.369 & 15.761 & 16.173\\
Length at Amax - Male & 45.410 & 45.384 & 45.405 & 45.386 & 45.433 & 45.138 & 45.283 & 45.167 & 45.415 & 45.452 & 45.410 & 45.625 & 44.660 & 45.391 & 45.556 & 45.423 & 45.199 & 44.799 & 45.386 & 45.388 & 45.821\\
Von Bert. k - Male & 0.215 & 0.215 & 0.215 & 0.215 & 0.218 & 0.203 & 0.210 & 0.224 & 0.215 & 0.214 & 0.215 & 0.216 & 0.235 & 0.216 & 0.216 & 0.215 & 0.197 & 0.222 & 0.194 & 0.216 & 0.195\\
CV young - Male & 0.203 & 0.202 & 0.203 & 0.203 & 0.216 & 0.193 & 0.186 & 0.194 & 0.203 & 0.203 & 0.203 & 0.206 & 0.171 & 0.203 & 0.205 & 0.203 & 0.219 & 0.201 & 0.220 & 0.203 & 0.223\\
CV old - Male & 0.060 & 0.060 & 0.060 & 0.060 & 0.061 & 0.060 & 0.080 & 0.066 & 0.060 & 0.060 & 0.060 & 0.055 & 0.071 & 0.060 & 0.055 & 0.060 & 0.068 & 0.068 & 0.065 & 0.060 & 0.056\\*
\end{longtable}
\endgroup{}
\end{landscape}
\endgroup{}


\newpage

\begingroup\fontsize{10}{12}\selectfont
\begingroup\fontsize{10}{12}\selectfont

\begin{longtable}[t]{l>{\raggedright\arraybackslash}p{2.2cm}>{\raggedright\arraybackslash}p{2.2cm}>{\raggedright\arraybackslash}p{2.2cm}>{\raggedright\arraybackslash}p{2.2cm}}
\caption{\label{tab:dw}Suggested data weights for each data weighting methodology by fleet and data type.}\\
\toprule
Fleet & Data Type & Francis & MI & Dirichlet\\
\midrule
\endfirsthead
\caption[]{\label{tab:dw}Suggested data weights for each data weighting methodology by fleet and data type. \textit{(continued)}}\\
\toprule
Fleet & Data Type & Francis & MI & Dirichlet\\
\midrule
\endhead

\endfoot
\bottomrule
\endlastfoot
Commercial Dead & Lengths & 0.20 & 0.23 & 0.97\\
Commercial Live & Lengths & 0.29 & 0.58 & 0.99\\
CPFV & Lengths & 0.33 & 0.24 & 0.99\\
PR & Lengths & 0.45 & 0.42 & 0.99\\
CCFRP & Lengths & 0.49 & 1.28 & 0.99\\
CDFW ROV & Lengths & 0.24 & 1.49 & 0.96\\
Commercial Dead & Ages & 0.17 & 0.60 & 0.99\\
CPFV & Ages & 0.21 & 0.21 & 0.99\\
PR & Ages & 0.56 & 0.56 & 0.99\\
CCFRP & Ages & 0.53 & 0.76 & 0.99\\
Growth & Ages & 0.40 & 0.47 & 0.99\\*
\end{longtable}
\endgroup{}
\endgroup{}

\begingroup\fontsize{9}{11}\selectfont

\begin{landscape}\begingroup\fontsize{9}{11}\selectfont

\begin{longtable}[t]{l>{\centering\arraybackslash}p{1.83cm}>{\centering\arraybackslash}p{1.83cm}>{\centering\arraybackslash}p{1.83cm}>{\centering\arraybackslash}p{1.83cm}>{\centering\arraybackslash}p{1.83cm}c}
\caption{\label{tab:sensitivities-2}Sensitivities relative to the base model.}\\
\toprule
  & Base Model & Reduce Rec. Catch 1970-82 & Add Hist. CPFV Ages to Growth & Lengths Only & Rm. Coop. Ages & Rm. All Ages\\
\midrule
\endfirsthead
\caption[]{Sensitivities relative to the base model. \textit{(continued)}}\\
\toprule
  & Base Model & Reduce Rec. Catch 1970-82 & Add Hist. CPFV Ages to Growth & Lengths Only & Rm. Coop. Ages & Rm. All Ages\\
\midrule
\endhead

\endfoot
\bottomrule
\endlastfoot
Total Likelihood & 1016.640 & 1011.040 & 1173.970 & 370.637 & 986.507 & 369.756\\
Survey Likelihood & -40.986 & -42.563 & -40.608 & 0.000 & -40.908 & -41.635\\
Length Likelihood & 413.455 & 413.094 & 415.038 & 365.665 & 413.595 & 406.900\\
Age Likelihood & 638.670 & 639.129 & 794.919 & 0.000 & 608.022 & 0.000\\
Recruitment Likelihood & 5.496 & 1.371 & 4.617 & 4.968 & 5.793 & 4.486\\
Forecast Recruitment Likelihood & 0.000 & 0.000 & 0.000 & 0.000 & 0.000 & 0.000\\
Parameter Priors Likelihood & 0.002 & 0.002 & 0.002 & 0.002 & 0.002 & 0.002\\
log(R0) & 6.342 & 6.061 & 6.297 & 6.328 & 6.340 & 6.292\\
SB Virgin & 486.149 & 366.975 & 476.660 & 479.608 & 485.185 & 462.345\\
SB 2023 & 262.100 & 162.701 & 252.657 & 242.204 & 260.602 & 200.255\\
Fraction Unfished 2023 & 0.539 & 0.443 & 0.530 & 0.505 & 0.537 & 0.433\\
Total Yield - SPR 50 & 124.049 & 93.956 & 120.774 & 121.859 & 124.030 & 118.046\\
Steepness & 0.720 & 0.720 & 0.720 & 0.720 & 0.720 & 0.720\\
Natural Mortality - Female & 0.108 & 0.108 & 0.108 & 0.108 & 0.108 & 0.108\\
Length at Amin - Female & 14.583 & 14.583 & 14.583 & 14.583 & 14.583 & 14.583\\
Length at Amax - Female & 48.307 & 48.296 & 48.496 & 48.307 & 48.286 & 48.307\\
Von Bert. k - Female & 0.154 & 0.154 & 0.157 & 0.154 & 0.154 & 0.154\\
CV young - Female & 0.157 & 0.157 & 0.162 & 0.157 & 0.159 & 0.157\\
CV old - Female & 0.074 & 0.074 & 0.075 & 0.074 & 0.074 & 0.074\\
Natural Mortality - Male & 0.108 & 0.108 & 0.108 & 0.108 & 0.108 & 0.108\\
Length at Amin - Male & 12.637 & 12.637 & 12.637 & 12.637 & 12.637 & 12.637\\
Length at Amax - Male & 46.488 & 46.514 & 46.700 & 46.488 & 46.436 & 46.488\\
Von Bert. k - Male & 0.195 & 0.195 & 0.194 & 0.195 & 0.197 & 0.195\\
CV young - Male & 0.157 & 0.156 & 0.182 & 0.157 & 0.162 & 0.157\\
CV old - Male & 0.072 & 0.073 & 0.070 & 0.072 & 0.073 & 0.072\\*
\end{longtable}
\endgroup{}
\end{landscape}
\endgroup{}


\newpage

\begingroup\fontsize{9}{11}\selectfont

\begin{landscape}\begingroup\fontsize{9}{11}\selectfont

\begin{longtable}[t]{l>{\centering\arraybackslash}p{1.22cm}>{\centering\arraybackslash}p{1.22cm}>{\centering\arraybackslash}p{1.22cm}>{\centering\arraybackslash}p{1.22cm}>{\centering\arraybackslash}p{1.22cm}>{\centering\arraybackslash}p{1.22cm}>{\centering\arraybackslash}p{1.22cm}>{\centering\arraybackslash}p{1.22cm}>{\centering\arraybackslash}p{1.22cm}}
\caption{\label{tab:sensitivities-3}Sensitivities relative to the base model.}\\
\toprule
  & Base Model & Rm. CCFRP & Add CDFW ROV Survey & Rm. NWFSC HKL All & Rm. NWFSC HKL Ages & Rm. NWFSC HKL Lens. \& Ages & Rm. NWFSC HKL Index & Move NWFSC HKL Data Before 2014 & Rm. All Surveys\\
\midrule
\endfirsthead
\caption[]{Sensitivities relative to the base model. \textit{(continued)}}\\
\toprule
  & Base Model & Rm. CCFRP & Add CDFW ROV Survey & Rm. NWFSC HKL All & Rm. NWFSC HKL Ages & Rm. NWFSC HKL Lens. \& Ages & Rm. NWFSC HKL Index & Move NWFSC HKL Data Before 2014 & Rm. All Surveys\\
\midrule
\endhead

\endfoot
\bottomrule
\endlastfoot
Total Likelihood & 2832.680 & 2763.640 & 2830.550 & 1379.710 & 1529.870 & 1377.270 & 2836.050 & 2718.710 & 1315.760\\
Survey Likelihood & -33.416 & -29.661 & -35.632 & -29.493 & -32.653 & -32.401 & -29.806 & -33.786 & -25.732\\
Length Likelihood & 540.049 & 530.746 & 540.192 & 383.346 & 532.298 & 383.718 & 539.694 & 721.904 & 373.522\\
Age Likelihood & 2319.930 & 2257.490 & 2319.670 & 1023.930 & 1027.120 & 1023.450 & 2321.010 & 2025.620 & 966.632\\
Recruitment Likelihood & 5.415 & 5.058 & 5.617 & 1.190 & 2.367 & 1.812 & 4.404 & 4.366 & 1.331\\
Forecast Recruitment Likelihood & 0.000 & 0.000 & 0.000 & 0.000 & 0.000 & 0.000 & 0.000 & 0.000 & 0.000\\
Parameter Priors Likelihood & 0.699 & 0.000 & 0.700 & 0.733 & 0.741 & 0.692 & 0.743 & 0.601 & 0.000\\
log(R0) & 5.486 & 5.481 & 5.483 & 5.576 & 5.547 & 5.548 & 5.503 & 5.494 & 5.555\\
SO Virgin & 201.064 & 201.628 & 200.611 & 216.348 & 205.104 & 210.422 & 204.603 & 195.923 & 214.888\\
SO 2023 & 32.063 & 29.328 & 30.791 & 66.186 & 44.561 & 49.663 & 42.180 & 25.415 & 57.169\\
Fraction Unfished 2023 & 0.159 & 0.145 & 0.153 & 0.306 & 0.217 & 0.236 & 0.206 & 0.130 & 0.266\\
Total Yield - SPR 50 & 47.777 & 47.784 & 47.682 & 51.185 & 49.253 & 49.874 & 48.504 & 48.104 & 50.412\\
Steepness & 0.720 & 0.720 & 0.720 & 0.720 & 0.720 & 0.720 & 0.720 & 0.720 & 0.720\\
Natural Mortality - Female & 0.108 & 0.108 & 0.108 & 0.108 & 0.108 & 0.108 & 0.108 & 0.108 & 0.108\\
Length at Amin - Female & 15.349 & 15.459 & 15.347 & 13.716 & 15.793 & 13.776 & 15.347 & 15.839 & 14.088\\
Length at Amax - Female & 46.829 & 47.009 & 46.833 & 46.569 & 46.486 & 46.622 & 46.810 & 46.429 & 46.986\\
Von Bert. k - Female & 0.194 & 0.190 & 0.194 & 0.204 & 0.182 & 0.202 & 0.195 & 0.191 & 0.192\\
CV young - Female & 0.197 & 0.198 & 0.197 & 0.191 & 0.224 & 0.191 & 0.197 & 0.197 & 0.195\\
CV old - Female & 0.059 & 0.058 & 0.059 & 0.076 & 0.066 & 0.076 & 0.059 & 0.061 & 0.074\\
Natural Mortality - Male & 0.108 & 0.108 & 0.108 & 0.108 & 0.108 & 0.108 & 0.108 & 0.108 & 0.108\\
Length at Amin - Male & 15.722 & 15.580 & 15.722 & 16.280 & 14.724 & 16.285 & 15.709 & 15.438 & 16.109\\
Length at Amax - Male & 45.415 & 45.567 & 45.421 & 45.340 & 44.751 & 45.393 & 45.388 & 45.167 & 45.824\\
Von Bert. k - Male & 0.218 & 0.219 & 0.217 & 0.200 & 0.229 & 0.198 & 0.219 & 0.220 & 0.200\\
CV young - Male & 0.203 & 0.206 & 0.203 & 0.222 & 0.200 & 0.222 & 0.203 & 0.203 & 0.225\\
CV old - Male & 0.060 & 0.055 & 0.060 & 0.065 & 0.068 & 0.065 & 0.060 & 0.059 & 0.055\\*
\end{longtable}
\endgroup{}
\end{landscape}
\endgroup{}


\newpage

\begingroup\fontsize{10}{12}\selectfont
\begingroup\fontsize{10}{12}\selectfont

\begin{longtable}[t]{r>{\centering\arraybackslash}p{2cm}>{\centering\arraybackslash}p{2cm}>{\centering\arraybackslash}p{2cm}}
	\caption{\label{tab:north-referenceES}Summary of reference points and management quantities, including estimates of the 95 percent intervals for the sub-area model north of Point Conception.}\\
	\toprule
	& Estimate & Lower Interval & Upper Interval\\
	\midrule
	\endfirsthead
	\caption[]{Summary of reference points and management quantities, including estimates of the 95 percent intervals for the sub-area model north of Point Conception. \textit{(continued)}}\\
	\toprule
	& Estimate & Lower Interval & Upper Interval\\
	\midrule
	\endhead
	
	\endfoot
	\bottomrule
	\endlastfoot
	Unfished Spawning Output & 486.15 & 387.43 & 584.87\\
	Unfished Age 3+ Biomass (mt) & 4719.91 & 3777.92 & 5661.90\\
	Unfished Recruitment (R0) & 567.77 & 452.48 & 683.06\\
	Spawning Output (2023) & 262.10 & 124.28 & 399.92\\
	Fraction Unfished (2023) & 0.54 & 0.32 & 0.76\\
	Reference Points Based SB40\% &  &  & \\
	Proxy Spawning Output SB40\% & 194.46 & 154.97 & 233.95\\
	SPR Resulting in SB40\% & 0.46 & 0.46 & 0.46\\
	Exploitation Rate Resulting in SB40\% & 0.06 & 0.06 & 0.06\\
	Yield with SPR Based On SB40\% (mt) & 129.86 & 104.05 & 155.67\\
	Reference Points Based on SPR Proxy for MSY & NA & NA & NA\\
	Proxy Spawning Output (SPR50) & 216.90 & 172.85 & 260.94\\
	SPR50 & 0.50 &  & \\
	Exploitation Rate Corresponding to SPR50 & 0.05 & 0.05 & 0.05\\
	Yield with SPR50 at SB SPR (mt) & 124.05 & 99.39 & 148.71\\
	Reference Points Based on Estimated MSY Values & NA & NA & NA\\
	Spawning Output at MSY (SB MSY) & 134.17 & 106.84 & 161.51\\
	SPR MSY & 0.35 & 0.34 & 0.35\\
	Exploitation Rate Corresponding to SPR MSY & 0.09 & 0.08 & 0.09\\
	MSY (mt) & 137.59 & 110.25 & 164.92\\*
\end{longtable}
\endgroup{}
\endgroup{}


\newpage

\begingroup\fontsize{10}{12}\selectfont

\begin{landscape}\begingroup\fontsize{10}{12}\selectfont

\begin{longtable}[t]{c>{\centering\arraybackslash}p{1.38cm}>{\centering\arraybackslash}p{1.38cm}>{\centering\arraybackslash}p{1.38cm}>{\centering\arraybackslash}p{1.38cm}>{\centering\arraybackslash}p{1.38cm}>{\centering\arraybackslash}p{1.38cm}>{\centering\arraybackslash}p{1.38cm}}
\caption{\label{tab:ca-proj}The estimated spawning output in number of million eggs across California and fraction unfished by year.}\\
\toprule
Year & Adopted OFL (mt) & Adopted ABC (mt) & Assumed Catch (mt) & OFL (mt) & ABC (mt) & Spawning Biomass & Fraction Unfished\\
\midrule
\endfirsthead
\caption[]{\label{tab:ca-proj}The estimated spawning output in number of million eggs across California and fraction unfished by year. \textit{(continued)}}\\
\toprule
Year & Adopted OFL (mt) & Adopted ABC (mt) & Assumed Catch (mt) & OFL (mt) & ABC (mt) & Spawning Biomass & Fraction Unfished\\
\midrule
\endhead

\endfoot
\bottomrule
\endlastfoot
2023 & 116.4 & 91.53 & 70 & - & - & 289.74 & 0.421\\
2024 & 121.32 & 94.69 & 70 & - & - & 295.18 & 0.429\\
2025 & - & - & - & 171.29 & 148.79 & 302.30 & 0.440\\
2026 & - & - & - & 171.74 & 147.12 & 303.82 & 0.442\\
2027 & - & - & - & 172.05 & 145.66 & 304.90 & 0.443\\
2028 & - & - & - & 172.2 & 144 & 305.69 & 0.444\\
2029 & - & - & - & 172.26 & 141.82 & 306.33 & 0.445\\
2030 & - & - & - & 172.31 & 140.09 & 306.97 & 0.446\\
2031 & - & - & - & 172.39 & 138.4 & 307.60 & 0.447\\
2032 & - & - & - & 172.54 & 136.34 & 308.29 & 0.448\\
2033 & - & - & - & 172.8 & 134.85 & 309.09 & 0.449\\
2034 & - & - & - & 173.15 & 143.44 & 309.99 & 0.451\\*
\end{longtable}
\endgroup{}
\end{landscape}
\endgroup{}

\begingroup\fontsize{10}{12}\selectfont
\begingroup\fontsize{10}{12}\selectfont

\begin{longtable}[t]{r>{\centering\arraybackslash}p{1.22cm}>{\centering\arraybackslash}p{1.22cm}>{\centering\arraybackslash}p{1.22cm}>{\centering\arraybackslash}p{1.22cm}>{\centering\arraybackslash}p{1.22cm}>{\centering\arraybackslash}p{1.22cm}>{\centering\arraybackslash}p{1.22cm}>{\centering\arraybackslash}p{1.22cm}}
\caption{\label{tab:north-timeseries}Time series of population estimates from the base model for the sub-area north of Point Conception.}\\
\toprule
Year & Total Biomass (mt) & Spawning Output & Total Biomass 3+ (mt) & Fraction Unfished & Age-0 Recruits & Total Mortality (mt) & 1-SPR & Exploitation Rate\\
\midrule
\endfirsthead
\caption[]{Time series of population estimates from the base model for the sub-area north of Point Conception. \textit{(continued)}}\\
\toprule
Year & Total Biomass (mt) & Spawning Output & Total Biomass 3+ (mt) & Fraction Unfished & Age-0 Recruits & Total Mortality (mt) & 1-SPR & Exploitation Rate\\
\midrule
\endhead

\endfoot
\bottomrule
\endlastfoot
1916 & 4758.75 & 487.03 & 4731.64 & 1.00 & 573.48 & 3.97 & 0.01 & 0.00\\
1917 & 4756.53 & 486.73 & 4729.41 & 1.00 & 573.92 & 6.23 & 0.02 & 0.00\\
1918 & 4752.47 & 486.22 & 4725.31 & 1.00 & 574.38 & 7.50 & 0.02 & 0.00\\
1919 & 4747.63 & 485.61 & 4720.46 & 1.00 & 574.87 & 4.91 & 0.01 & 0.00\\
1920 & 4745.78 & 485.32 & 4718.58 & 1.00 & 575.44 & 5.06 & 0.01 & 0.00\\
1921 & 4744.18 & 485.05 & 4716.96 & 1.00 & 576.07 & 4.31 & 0.01 & 0.00\\
1922 & 4743.72 & 484.91 & 4716.47 & 1.00 & 576.76 & 3.70 & 0.01 & 0.00\\
1923 & 4744.21 & 484.87 & 4716.93 & 1.00 & 577.52 & 3.89 & 0.01 & 0.00\\
1924 & 4744.88 & 484.85 & 4717.56 & 1.00 & 578.34 & 2.57 & 0.01 & 0.00\\
1925 & 4747.14 & 485.02 & 4719.79 & 1.00 & 579.26 & 3.84 & 0.01 & 0.00\\
1926 & 4748.48 & 485.07 & 4721.09 & 1.00 & 580.23 & 4.89 & 0.01 & 0.00\\
1927 & 4749.17 & 485.05 & 4721.73 & 1.00 & 581.27 & 3.64 & 0.01 & 0.00\\
1928 & 4751.45 & 485.20 & 4723.97 & 1.00 & 582.41 & 5.21 & 0.02 & 0.00\\
1929 & 4752.50 & 485.22 & 4724.97 & 1.00 & 583.84 & 6.18 & 0.02 & 0.00\\
1930 & 4752.95 & 485.18 & 4725.36 & 1.00 & 585.39 & 8.96 & 0.03 & 0.00\\
1931 & 4751.19 & 484.88 & 4723.53 & 1.00 & 587.04 & 11.15 & 0.03 & 0.00\\
1932 & 4747.92 & 484.39 & 4720.18 & 1.00 & 588.84 & 11.73 & 0.04 & 0.00\\
1933 & 4744.80 & 483.91 & 4716.98 & 1.00 & 590.78 & 12.11 & 0.04 & 0.00\\
1934 & 4742.11 & 483.45 & 4714.21 & 0.99 & 592.87 & 12.01 & 0.04 & 0.00\\
1935 & 4740.36 & 483.08 & 4712.37 & 0.99 & 595.19 & 15.35 & 0.05 & 0.00\\
1936 & 4736.34 & 482.44 & 4708.24 & 0.99 & 597.65 & 16.09 & 0.05 & 0.00\\
1937 & 4732.71 & 481.81 & 4704.51 & 0.99 & 600.32 & 18.78 & 0.06 & 0.00\\
1938 & 4727.67 & 481.01 & 4699.34 & 0.99 & 603.21 & 17.93 & 0.06 & 0.00\\
1939 & 4724.86 & 480.41 & 4696.41 & 0.99 & 606.46 & 16.05 & 0.05 & 0.00\\
1940 & 4725.39 & 480.14 & 4696.79 & 0.99 & 610.18 & 20.78 & 0.06 & 0.00\\
1941 & 4722.63 & 479.52 & 4693.88 & 0.99 & 614.38 & 19.96 & 0.06 & 0.00\\
1942 & 4722.37 & 479.12 & 4693.43 & 0.99 & 619.23 & 9.64 & 0.03 & 0.00\\
1943 & 4734.07 & 479.95 & 4704.93 & 0.99 & 624.88 & 10.40 & 0.03 & 0.00\\
1944 & 4746.61 & 480.85 & 4717.22 & 0.99 & 631.13 & 14.81 & 0.04 & 0.00\\
1945 & 4756.67 & 481.43 & 4727.01 & 0.99 & 637.77 & 29.60 & 0.08 & 0.01\\
1946 & 4754.55 & 480.61 & 4724.59 & 0.99 & 644.39 & 38.03 & 0.11 & 0.01\\
1947 & 4746.83 & 479.14 & 4716.56 & 0.99 & 650.70 & 18.34 & 0.06 & 0.00\\
1948 & 4761.18 & 480.03 & 4730.59 & 0.99 & 656.79 & 31.93 & 0.10 & 0.01\\
1949 & 4764.49 & 479.75 & 4733.61 & 0.99 & 661.72 & 34.12 & 0.10 & 0.01\\
1950 & 4768.23 & 479.52 & 4737.08 & 0.99 & 665.42 & 39.31 & 0.12 & 0.01\\
1951 & 4769.54 & 479.03 & 4738.17 & 0.99 & 667.86 & 53.91 & 0.16 & 0.01\\
1952 & 4759.34 & 477.30 & 4727.81 & 0.98 & 669.38 & 45.06 & 0.13 & 0.01\\
1953 & 4761.01 & 476.77 & 4729.37 & 0.98 & 671.53 & 36.30 & 0.11 & 0.01\\
1954 & 4774.05 & 477.47 & 4742.33 & 0.98 & 675.16 & 46.94 & 0.14 & 0.01\\
1955 & 4778.65 & 477.39 & 4746.82 & 0.98 & 679.63 & 52.40 & 0.16 & 0.01\\
1956 & 4779.65 & 477.06 & 4747.63 & 0.98 & 686.18 & 60.11 & 0.17 & 0.01\\
1957 & 4774.87 & 476.17 & 4742.63 & 0.98 & 683.59 & 58.27 & 0.17 & 0.01\\
1958 & 4774.03 & 475.66 & 4741.58 & 0.98 & 682.24 & 99.22 & 0.27 & 0.02\\
1959 & 4733.91 & 471.20 & 4701.56 & 0.97 & 684.64 & 80.28 & 0.23 & 0.02\\
1960 & 4715.50 & 468.80 & 4683.16 & 0.96 & 699.37 & 68.30 & 0.20 & 0.01\\
1961 & 4711.74 & 467.80 & 4679.12 & 0.96 & 732.19 & 51.06 & 0.15 & 0.01\\
1962 & 4727.91 & 468.84 & 4694.37 & 0.96 & 777.94 & 63.56 & 0.19 & 0.01\\
1963 & 4734.77 & 468.93 & 4699.55 & 0.96 & 817.20 & 79.29 & 0.23 & 0.02\\
1964 & 4731.24 & 467.64 & 4693.89 & 0.96 & 881.00 & 70.73 & 0.21 & 0.02\\
1965 & 4744.87 & 467.41 & 4705.34 & 0.96 & 983.81 & 104.92 & 0.29 & 0.02\\
1966 & 4736.12 & 464.09 & 4693.15 & 0.95 & 1067.20 & 121.03 & 0.32 & 0.03\\
1967 & 4728.19 & 459.74 & 4680.69 & 0.95 & 1051.05 & 128.37 & 0.34 & 0.03\\
1968 & 4734.70 & 455.64 & 4684.53 & 0.94 & 923.06 & 135.98 & 0.35 & 0.03\\
1969 & 4756.61 & 452.27 & 4708.54 & 0.93 & 741.98 & 145.84 & 0.37 & 0.03\\
1970 & 4786.46 & 449.99 & 4744.98 & 0.93 & 608.42 & 181.20 & 0.43 & 0.04\\
1971 & 4786.28 & 446.80 & 4752.87 & 0.92 & 455.26 & 169.15 & 0.41 & 0.04\\
1972 & 4788.22 & 447.24 & 4761.28 & 0.92 & 357.88 & 215.21 & 0.48 & 0.05\\
1973 & 4718.98 & 444.91 & 4698.51 & 0.92 & 373.18 & 246.30 & 0.52 & 0.05\\
1974 & 4582.63 & 439.32 & 4565.55 & 0.90 & 350.26 & 270.39 & 0.56 & 0.06\\
1975 & 4383.13 & 428.95 & 4365.71 & 0.88 & 365.48 & 267.98 & 0.58 & 0.06\\
1976 & 4150.72 & 414.85 & 4134.02 & 0.85 & 332.35 & 296.45 & 0.63 & 0.07\\
1977 & 3862.72 & 393.09 & 3845.83 & 0.81 & 306.02 & 306.11 & 0.67 & 0.08\\
1978 & 3547.11 & 365.81 & 3531.76 & 0.75 & 239.16 & 281.47 & 0.68 & 0.08\\
1979 & 3245.65 & 337.69 & 3231.98 & 0.69 & 192.93 & 292.80 & 0.72 & 0.09\\
1980 & 2926.86 & 306.26 & 2916.03 & 0.63 & 222.56 & 305.18 & 0.75 & 0.10\\
1981 & 2594.11 & 272.27 & 2584.49 & 0.56 & 344.88 & 287.77 & 0.77 & 0.11\\
1982 & 2277.63 & 239.84 & 2265.84 & 0.49 & 243.16 & 284.03 & 0.80 & 0.13\\
1983 & 1971.16 & 207.66 & 1955.99 & 0.43 & 243.06 & 209.77 & 0.73 & 0.11\\
1984 & 1754.30 & 183.00 & 1742.51 & 0.38 & 462.12 & 185.11 & 0.74 & 0.11\\
1985 & 1576.13 & 161.71 & 1562.04 & 0.33 & 551.70 & 206.53 & 0.81 & 0.13\\
1986 & 1392.78 & 139.51 & 1370.20 & 0.29 & 314.58 & 164.96 & 0.79 & 0.12\\
1987 & 1271.95 & 122.97 & 1248.57 & 0.25 & 274.54 & 100.25 & 0.66 & 0.08\\
1988 & 1238.68 & 114.26 & 1224.21 & 0.24 & 299.29 & 112.68 & 0.70 & 0.09\\
1989 & 1207.29 & 106.33 & 1193.84 & 0.22 & 441.70 & 122.44 & 0.72 & 0.10\\
1990 & 1174.70 & 99.83 & 1158.82 & 0.21 & 518.43 & 133.18 & 0.75 & 0.11\\
1991 & 1138.25 & 94.81 & 1116.68 & 0.20 & 361.75 & 131.79 & 0.75 & 0.12\\
1992 & 1112.37 & 90.96 & 1089.66 & 0.19 & 316.48 & 142.04 & 0.77 & 0.13\\
1993 & 1087.78 & 86.39 & 1070.99 & 0.18 & 462.51 & 162.74 & 0.82 & 0.15\\
1994 & 1048.63 & 80.67 & 1032.13 & 0.17 & 361.77 & 93.49 & 0.68 & 0.09\\
1995 & 1079.12 & 82.08 & 1058.52 & 0.17 & 247.39 & 76.06 & 0.60 & 0.07\\
1996 & 1130.71 & 86.01 & 1114.85 & 0.18 & 272.30 & 83.98 & 0.61 & 0.08\\
1997 & 1174.03 & 90.16 & 1161.98 & 0.19 & 315.62 & 80.03 & 0.58 & 0.07\\
1998 & 1215.54 & 95.08 & 1202.13 & 0.20 & 339.49 & 44.90 & 0.40 & 0.04\\
1999 & 1284.31 & 103.40 & 1269.15 & 0.21 & 302.54 & 37.01 & 0.33 & 0.03\\
2000 & 1354.22 & 112.46 & 1338.64 & 0.23 & 251.04 & 31.63 & 0.27 & 0.02\\
2001 & 1423.65 & 121.31 & 1409.88 & 0.25 & 274.09 & 28.95 & 0.25 & 0.02\\
2002 & 1489.36 & 129.50 & 1477.16 & 0.27 & 313.55 & 21.83 & 0.18 & 0.01\\
2003 & 1554.22 & 137.65 & 1540.85 & 0.28 & 268.18 & 23.78 & 0.19 & 0.02\\
2004 & 1609.02 & 145.10 & 1594.74 & 0.30 & 227.39 & 18.92 & 0.15 & 0.01\\
2005 & 1661.84 & 152.32 & 1649.64 & 0.31 & 196.19 & 34.85 & 0.26 & 0.02\\
2006 & 1690.21 & 157.19 & 1679.61 & 0.32 & 346.50 & 36.21 & 0.27 & 0.02\\
2007 & 1709.37 & 161.04 & 1697.69 & 0.33 & 888.98 & 42.04 & 0.30 & 0.02\\
2008 & 1720.32 & 163.55 & 1698.04 & 0.34 & 683.78 & 31.80 & 0.24 & 0.02\\
2009 & 1752.87 & 166.22 & 1712.94 & 0.34 & 819.71 & 39.21 & 0.29 & 0.02\\
2010 & 1806.79 & 167.32 & 1773.45 & 0.34 & 376.08 & 27.47 & 0.21 & 0.02\\
2011 & 1904.77 & 169.37 & 1870.94 & 0.35 & 412.57 & 25.88 & 0.19 & 0.01\\
2012 & 2031.54 & 173.50 & 2012.93 & 0.36 & 721.98 & 34.91 & 0.23 & 0.02\\
2013 & 2160.53 & 180.75 & 2137.65 & 0.37 & 602.74 & 25.64 & 0.16 & 0.01\\
2014 & 2302.33 & 193.08 & 2269.62 & 0.40 & 508.66 & 36.94 & 0.20 & 0.02\\
2015 & 2435.83 & 207.33 & 2408.19 & 0.43 & 644.80 & 65.46 & 0.31 & 0.03\\
2016 & 2538.95 & 219.54 & 2513.77 & 0.45 & 304.21 & 66.32 & 0.31 & 0.03\\
2017 & 2635.51 & 230.78 & 2608.04 & 0.47 & 932.00 & 138.62 & 0.50 & 0.05\\
2018 & 2657.53 & 234.32 & 2636.38 & 0.48 & 649.02 & 102.38 & 0.41 & 0.04\\
2019 & 2709.53 & 240.87 & 2668.99 & 0.50 & 358.62 & 101.89 & 0.40 & 0.04\\
2020 & 2765.07 & 247.00 & 2737.43 & 0.51 & 518.96 & 156.21 & 0.53 & 0.06\\
2021 & 2765.72 & 247.07 & 2746.93 & 0.51 & 518.99 & 106.99 & 0.41 & 0.04\\
2022 & 2807.65 & 251.40 & 2783.08 & 0.52 & 520.55 & 47.15 & 0.21 & 0.02\\
2023 & 2899.87 & 262.10 & 2875.27 & 0.54 & 524.24 & 91.50 & 0.35 & 0.03\\
2024 & 2938.51 & 268.97 & 2913.82 & 0.55 & 526.48 & 66.90 & 0.27 & 0.02\\
2025 & 2993.09 & 277.24 & 2968.25 & 0.57 & 529.05 & 105.37 & 0.38 & 0.04\\
2026 & 3002.22 & 280.38 & 2977.27 & 0.58 & 529.99 & 104.04 & 0.38 & 0.03\\
2027 & 3007.39 & 282.32 & 2982.33 & 0.58 & 530.56 & 102.85 & 0.38 & 0.03\\
2028 & 3010.14 & 283.48 & 2985.04 & 0.58 & 530.90 & 101.52 & 0.38 & 0.03\\
2029 & 3011.84 & 284.18 & 2986.72 & 0.58 & 531.10 & 99.77 & 0.37 & 0.03\\
2030 & 3013.81 & 284.70 & 2988.68 & 0.59 & 531.26 & 98.37 & 0.37 & 0.03\\
2031 & 3016.28 & 285.14 & 2991.13 & 0.59 & 531.38 & 97.00 & 0.36 & 0.03\\
2032 & 3019.57 & 285.60 & 2994.42 & 0.59 & 531.52 & 95.34 & 0.36 & 0.03\\
2033 & 3024.21 & 286.17 & 2999.06 & 0.59 & 531.68 & 94.10 & 0.36 & 0.03\\
2034 & 3029.85 & 286.82 & 3004.69 & 0.59 & 531.87 & 102.94 & 0.38 & 0.03\\*
\end{longtable}
\endgroup{}
\endgroup{}


\newpage

\begingroup\fontsize{10}{12}\selectfont
\begingroup\fontsize{10}{12}\selectfont

\begin{longtable}[t]{c>{\centering\arraybackslash}p{1.83cm}>{\centering\arraybackslash}p{1.83cm}>{\centering\arraybackslash}p{1.83cm}>{\centering\arraybackslash}p{1.83cm}>{\centering\arraybackslash}p{1.83cm}}
\caption{\label{tab:ca-status}The estimated total biomass (mt), total biomass age 3+ (mt), age-0 recruits, spawning ouput in number of million eggs across California and fraction unfished by year.}\\
\toprule
Year & Total Biomass (mt) & Total Biomass 3+ (mt) & Age-0 Recruits & Spawning Output & Fraction Unfished\\
\midrule
\endfirsthead
\caption[]{\label{tab:ca-status}The estimated total biomass (mt), total biomass age 3+ (mt), age-0 recruits, spawning ouput in number of million eggs across California and fraction unfished by year. \textit{(continued)}}\\
\toprule
Year & Total Biomass (mt) & Total Biomass 3+ (mt) & Age-0 Recruits & Spawning Output & Fraction Unfished\\
\midrule
\endhead

\endfoot
\bottomrule
\endlastfoot
1916 & 6771.66 & 6726.78 & 483.56 & 688.74 & 1.001\\
1917 & 6769.45 & 6724.56 & 483.64 & 688.45 & 1.001\\
1918 & 6765.34 & 6720.40 & 483.72 & 687.93 & 1.000\\
1919 & 6760.48 & 6715.54 & 483.82 & 687.31 & 0.999\\
1920 & 6758.72 & 6713.74 & 483.92 & 687.02 & 0.999\\
1921 & 6757.21 & 6712.21 & 484.04 & 686.76 & 0.999\\
1922 & 6756.88 & 6711.84 & 484.16 & 686.62 & 0.998\\
1923 & 6757.52 & 6712.45 & 484.30 & 686.60 & 0.998\\
1924 & 6758.33 & 6713.22 & 484.45 & 686.59 & 0.998\\
1925 & 6760.72 & 6715.57 & 484.61 & 686.76 & 0.999\\
1926 & 6762.19 & 6716.99 & 484.79 & 686.83 & 0.999\\
1927 & 6763.00 & 6717.75 & 484.98 & 686.81 & 0.999\\
1928 & 6765.48 & 6720.18 & 485.19 & 686.97 & 0.999\\
1929 & 6766.75 & 6721.40 & 485.42 & 687.01 & 0.999\\
1930 & 6767.44 & 6722.01 & 485.67 & 686.99 & 0.999\\
1931 & 6765.92 & 6720.42 & 485.95 & 686.70 & 0.998\\
1932 & 6762.93 & 6717.34 & 486.24 & 686.24 & 0.998\\
1933 & 6760.05 & 6714.37 & 486.56 & 685.76 & 0.997\\
1934 & 6757.78 & 6712.00 & 486.92 & 685.34 & 0.996\\
1935 & 6756.39 & 6710.51 & 487.30 & 685.00 & 0.996\\
1936 & 6752.51 & 6706.51 & 487.71 & 684.36 & 0.995\\
1937 & 6749.24 & 6703.12 & 488.15 & 683.76 & 0.994\\
1938 & 6743.91 & 6697.65 & 488.62 & 682.90 & 0.993\\
1939 & 6741.38 & 6694.98 & 489.14 & 682.32 & 0.992\\
1940 & 6742.49 & 6695.93 & 489.72 & 682.08 & 0.992\\
1941 & 6740.38 & 6693.65 & 490.33 & 681.51 & 0.991\\
1942 & 6740.81 & 6693.87 & 491.00 & 681.16 & 0.990\\
1943 & 6753.76 & 6706.59 & 491.72 & 682.09 & 0.992\\
1944 & 6767.59 & 6720.14 & 492.49 & 683.10 & 0.993\\
1945 & 6779.15 & 6731.41 & 493.30 & 683.81 & 0.994\\
1946 & 6778.55 & 6730.48 & 494.16 & 683.12 & 0.993\\
1947 & 6772.42 & 6724.01 & 495.04 & 681.79 & 0.991\\
1948 & 6787.89 & 6739.13 & 495.92 & 682.78 & 0.993\\
1949 & 6791.34 & 6742.26 & 496.75 & 682.50 & 0.992\\
1950 & 6794.80 & 6745.41 & 497.53 & 682.23 & 0.992\\
1951 & 6795.11 & 6745.48 & 498.22 & 681.61 & 0.991\\
1952 & 6781.60 & 6731.78 & 498.76 & 679.48 & 0.988\\
1953 & 6781.57 & 6731.62 & 499.17 & 678.73 & 0.987\\
1954 & 6793.47 & 6743.42 & 499.31 & 679.27 & 0.988\\
1955 & 6792.50 & 6742.32 & 498.98 & 678.64 & 0.987\\
1956 & 6779.53 & 6729.16 & 498.24 & 676.97 & 0.984\\
1957 & 6759.35 & 6708.78 & 497.29 & 674.55 & 0.981\\
1958 & 6751.38 & 6700.63 & 496.75 & 673.22 & 0.979\\
1959 & 6704.70 & 6654.08 & 496.83 & 667.98 & 0.971\\
1960 & 6685.45 & 6634.85 & 498.53 & 665.34 & 0.967\\
1961 & 6680.50 & 6629.60 & 501.47 & 664.12 & 0.966\\
1962 & 6693.08 & 6641.20 & 505.68 & 664.72 & 0.966\\
1963 & 6700.02 & 6646.33 & 512.94 & 664.72 & 0.966\\
1964 & 6696.71 & 6640.70 & 527.17 & 663.35 & 0.964\\
1965 & 6706.31 & 6647.79 & 557.75 & 662.63 & 0.963\\
1966 & 6688.99 & 6626.36 & 613.84 & 658.30 & 0.957\\
1967 & 6647.19 & 6578.69 & 692.63 & 650.44 & 0.946\\
1968 & 6616.98 & 6543.59 & 741.31 & 642.11 & 0.934\\
1969 & 6600.25 & 6526.41 & 632.10 & 633.79 & 0.922\\
1970 & 6613.12 & 6545.34 & 524.22 & 628.00 & 0.913\\
1971 & 6580.04 & 6524.32 & 456.46 & 619.92 & 0.901\\
1972 & 6554.52 & 6508.80 & 537.03 & 616.60 & 0.897\\
1973 & 6429.40 & 6391.48 & 585.75 & 609.13 & 0.886\\
1974 & 6212.80 & 6175.67 & 472.20 & 596.90 & 0.868\\
1975 & 5902.16 & 5864.17 & 413.01 & 576.83 & 0.839\\
1976 & 5553.05 & 5519.48 & 408.18 & 551.71 & 0.802\\
1977 & 5173.57 & 5141.51 & 460.98 & 520.85 & 0.757\\
1978 & 4773.59 & 4742.85 & 403.81 & 485.31 & 0.706\\
1979 & 4388.22 & 4358.13 & 320.16 & 449.06 & 0.653\\
1980 & 3940.02 & 3915.05 & 278.61 & 405.26 & 0.589\\
1981 & 3480.60 & 3459.55 & 293.65 & 358.87 & 0.522\\
1982 & 3100.46 & 3078.34 & 269.30 & 319.71 & 0.465\\
1983 & 2655.53 & 2629.80 & 243.16 & 274.64 & 0.399\\
1984 & 2375.06 & 2353.56 & 291.89 & 243.45 & 0.354\\
1985 & 2124.47 & 2100.97 & 423.38 & 214.84 & 0.312\\
1986 & 1845.66 & 1811.35 & 375.11 & 183.11 & 0.266\\
1987 & 1649.05 & 1610.57 & 318.75 & 158.20 & 0.230\\
1988 & 1571.30 & 1543.51 & 319.57 & 143.47 & 0.209\\
1989 & 1529.20 & 1504.06 & 302.07 & 132.51 & 0.193\\
1990 & 1489.34 & 1461.83 & 360.71 & 124.09 & 0.180\\
1991 & 1440.61 & 1407.47 & 390.95 & 117.50 & 0.171\\
1992 & 1411.63 & 1375.51 & 274.30 & 112.78 & 0.164\\
1993 & 1405.72 & 1375.57 & 182.29 & 109.06 & 0.159\\
1994 & 1402.16 & 1376.33 & 178.25 & 105.70 & 0.154\\
1995 & 1423.35 & 1396.07 & 206.33 & 106.98 & 0.156\\
1996 & 1471.73 & 1449.13 & 160.47 & 111.22 & 0.162\\
1997 & 1458.28 & 1439.05 & 129.43 & 111.93 & 0.163\\
1998 & 1492.25 & 1473.17 & 160.71 & 116.54 & 0.169\\
1999 & 1537.55 & 1517.25 & 357.39 & 123.32 & 0.179\\
2000 & 1584.44 & 1561.51 & 181.13 & 130.99 & 0.190\\
2001 & 1653.94 & 1628.49 & 123.13 & 139.74 & 0.203\\
2002 & 1730.34 & 1711.89 & 208.19 & 148.22 & 0.216\\
2003 & 1814.31 & 1795.64 & 351.55 & 157.39 & 0.229\\
2004 & 1886.01 & 1863.06 & 161.07 & 166.29 & 0.242\\
2005 & 1958.15 & 1934.51 & 236.37 & 175.25 & 0.255\\
2006 & 1994.02 & 1977.00 & 87.93 & 180.82 & 0.263\\
2007 & 2036.68 & 2017.34 & 330.52 & 186.63 & 0.271\\
2008 & 2050.75 & 2023.30 & 317.18 & 189.90 & 0.276\\
2009 & 2090.87 & 2038.55 & 851.87 & 193.68 & 0.282\\
2010 & 2160.63 & 2111.31 & 913.18 & 195.80 & 0.285\\
2011 & 2291.84 & 2226.56 & 408.77 & 198.79 & 0.289\\
2012 & 2452.86 & 2404.67 & 552.88 & 202.49 & 0.294\\
2013 & 2627.91 & 2588.61 & 928.75 & 210.14 & 0.306\\
2014 & 2795.04 & 2739.48 & 244.85 & 223.28 & 0.325\\
2015 & 2973.06 & 2916.90 & 102.22 & 240.91 & 0.350\\
2016 & 3097.56 & 3064.44 & 245.60 & 255.75 & 0.372\\
2017 & 3182.40 & 3150.11 & 122.12 & 268.46 & 0.390\\
2018 & 3182.40 & 3153.24 & 96.39 & 273.80 & 0.398\\
2019 & 3176.81 & 3131.98 & 112.02 & 279.23 & 0.406\\
2020 & 3179.52 & 3148.18 & 163.48 & 283.32 & 0.412\\
2021 & 3110.47 & 3087.05 & 317.57 & 278.75 & 0.405\\
2022 & 3108.97 & 3077.18 & 302.69 & 279.64 & 0.407\\
2023 & 3199.84 & 3163.70 & 299.87 & 289.74 & 0.421\\*
\end{longtable}
\endgroup{}
\endgroup{}

\begingroup\fontsize{10}{12}\selectfont

\begin{landscape}\begingroup\fontsize{10}{12}\selectfont

\begin{longtable}[t]{c>{\centering\arraybackslash}p{1.38cm}>{\centering\arraybackslash}p{1.38cm}>{\centering\arraybackslash}p{1.38cm}>{\centering\arraybackslash}p{1.38cm}>{\centering\arraybackslash}p{1.38cm}>{\centering\arraybackslash}p{1.38cm}>{\centering\arraybackslash}p{1.38cm}}
\caption{\label{tab:ca-proj}The estimated spawning output in number of million eggs across California and fraction unfished by year.}\\
\toprule
Year & Adopted OFL (mt) & Adopted ABC (mt) & Assumed Catch (mt) & OFL (mt) & ABC (mt) & Spawning Biomass & Fraction Unfished\\
\midrule
\endfirsthead
\caption[]{\label{tab:ca-proj}The estimated spawning output in number of million eggs across California and fraction unfished by year. \textit{(continued)}}\\
\toprule
Year & Adopted OFL (mt) & Adopted ABC (mt) & Assumed Catch (mt) & OFL (mt) & ABC (mt) & Spawning Biomass & Fraction Unfished\\
\midrule
\endhead

\endfoot
\bottomrule
\endlastfoot
2023 & 116.4 & 91.53 & 70 & - & - & 289.74 & 0.421\\
2024 & 121.32 & 94.69 & 70 & - & - & 295.18 & 0.429\\
2025 & - & - & - & 171.29 & 148.79 & 302.30 & 0.440\\
2026 & - & - & - & 171.74 & 147.12 & 303.82 & 0.442\\
2027 & - & - & - & 172.05 & 145.66 & 304.90 & 0.443\\
2028 & - & - & - & 172.2 & 144 & 305.69 & 0.444\\
2029 & - & - & - & 172.26 & 141.82 & 306.33 & 0.445\\
2030 & - & - & - & 172.31 & 140.09 & 306.97 & 0.446\\
2031 & - & - & - & 172.39 & 138.4 & 307.60 & 0.447\\
2032 & - & - & - & 172.54 & 136.34 & 308.29 & 0.448\\
2033 & - & - & - & 172.8 & 134.85 & 309.09 & 0.449\\
2034 & - & - & - & 173.15 & 143.44 & 309.99 & 0.451\\*
\end{longtable}
\endgroup{}
\end{landscape}
\endgroup{}

\newpage

\clearpage

\hypertarget{figures}{%
\section{Figures}\label{figures}}

\hypertarget{data-1}{%
\subsection{Data}\label{data-1}}

\begin{figure}
\centering
\includegraphics[width=1\textwidth,height=1\textheight]{C:/Assessments/2023/copper_rockfish_2023/documents/shared_figures/copper_length_by_half_month.png}
\caption{Distribution of YOY copper rockfish lengths from fish genetically identified from D. Baetscher.\label{fig:copper-smurf-length}}
\end{figure}

\pagebreak

\begin{figure}
\centering
\includegraphics[width=1\textwidth,height=1\textheight]{C:/Assessments/2023/copper_rockfish_2023/documents/shared_figures/map.png}
\caption{Map of management area and the 2023 assessments areas for copper rockfish.\label{fig:ca-map}}
\end{figure}

\pagebreak

\begin{figure}
\centering
\includegraphics[width=1\textwidth,height=1\textheight]{N:/Assessments/CurrentAssessments/copper_rockfish_2023/models/nca/9.8_selex_fix_forecast/plots/catch2 landings stacked.png}
\caption{Landings by fleet used in the base model where catches in metric tons by fleet are stacked.\label{fig:catch}}
\end{figure}

\pagebreak

\begin{figure}
\centering
\includegraphics[width=1\textwidth,height=1\textheight]{N:/Assessments/CurrentAssessments/copper_rockfish_2023/models/nca/9.8_selex_fix_forecast/plots/data_plot.png}
\caption{Summary of data sources used in the base model.\label{fig:data-plot}}
\end{figure}

\pagebreak

\begin{figure}
\centering
\includegraphics[width=1\textwidth,height=1\textheight]{N:/Assessments/CurrentAssessments/copper_rockfish_2023/models/nca/9.8_selex_fix_forecast/plots/comp_lendat_bubflt1mkt0.png}
\caption{Length composition data from the commercial dead fleet.\label{fig:com-dead-len-data}}
\end{figure}

\pagebreak

\begin{figure}
\centering
\includegraphics[width=1\textwidth,height=1\textheight]{N:/Assessments/CurrentAssessments/copper_rockfish_2023/models/nca/9.8_selex_fix_forecast/plots/comp_lendat_data_weighting_TA1.8_Commercial_Dead.png}
\caption{Mean length for commercial dead fleet with 95 percent confidence intervals.\label{fig:mean-com-dead-len-data}}
\end{figure}

\pagebreak

\begin{figure}
\centering
\includegraphics[width=1\textwidth,height=1\textheight]{N:/Assessments/CurrentAssessments/copper_rockfish_2023/models/nca/9.8_selex_fix_forecast/plots/comp_condAALdat_bubflt1mkt0.png}
\caption{Conditional age-at-length composition data from the commercial dead fleet.\label{fig:com-dead-age-data}}
\end{figure}

\pagebreak

\begin{figure}
\centering
\includegraphics[width=1\textwidth,height=1\textheight]{N:/Assessments/CurrentAssessments/copper_rockfish_2023/models/nca/9.8_selex_fix_forecast/plots/comp_lendat_bubflt2mkt0.png}
\caption{Length composition data from the commercial live fleet.\label{fig:com-live-len-data}}
\end{figure}

\pagebreak

\begin{figure}
\centering
\includegraphics[width=1\textwidth,height=1\textheight]{N:/Assessments/CurrentAssessments/copper_rockfish_2023/models/nca/9.8_selex_fix_forecast/plots/comp_lendat_data_weighting_TA1.8_Commercial_Live.png}
\caption{Mean length for commercial live fleet with 95 percent confidence intervals.\label{fig:mean-com-live-len-data}}
\end{figure}

\pagebreak

\begin{figure}
\centering
\includegraphics[width=1\textwidth,height=1\textheight]{N:/Assessments/CurrentAssessments/copper_rockfish_2023/data/rec_indices/mrfss_cpfv_dockside/north/forSS/Index.png}
\caption{Estimated annual index of abundances for the CPFV fleet based on MRFSS survey data.\label{fig:mrfss-index-main}}
\end{figure}

\pagebreak

\begin{figure}
\centering
\includegraphics[width=1\textwidth,height=1\textheight]{N:/Assessments/CurrentAssessments/copper_rockfish_2023/data/rec_indices/debwv_cpfv_onboard/deltalogn/Index.png}
\caption{Estimated annual index of abundances for the CPFV fleet based on the Deb Wilson-Vandenberg survey data.\label{fig:dwv-index-main}}
\end{figure}

\pagebreak

\begin{figure}
\centering
\includegraphics[width=1\textwidth,height=1\textheight]{N:/Assessments/CurrentAssessments/copper_rockfish_2023/data/rec_indices/crfs_cpfv_onboard/north/area_weighted/Index.png}
\caption{Estimated annual index of abundances for the CPFV fleet based on CRFS survey data.\label{fig:crfs-index-main}}
\end{figure}

\pagebreak

\begin{figure}
\centering
\includegraphics[width=1\textwidth,height=1\textheight]{N:/Assessments/CurrentAssessments/copper_rockfish_2023/data/rec_indices/crfs_pr_dockside/north/rm_last2yrs_area_weighted/Index.png}
\caption{Estimated annual index of abundances for the CPFV fleet based on CRFS survey data.\label{fig:crfs-pr-index-main}}
\end{figure}

\pagebreak

\begin{figure}
\centering
\includegraphics[width=1\textwidth,height=1\textheight]{N:/Assessments/CurrentAssessments/copper_rockfish_2023/models/nca/9.8_selex_fix_forecast/plots/comp_lendat_bubflt3mkt0_page2.png}
\caption{Length composition data from the recreational CPFV fleet.\label{fig:rec-cpfv-len-data}}
\end{figure}

\pagebreak

\begin{figure}
\centering
\includegraphics[width=1\textwidth,height=1\textheight]{N:/Assessments/CurrentAssessments/copper_rockfish_2023/models/nca/9.8_selex_fix_forecast/plots/comp_lendat_data_weighting_TA1.8_Rec_CPFV.png}
\caption{Mean length for recreational CPFV fleet with 95 percent confidence intervals.\label{fig:mean-rec-cpfv-len-data}}
\end{figure}

\pagebreak

\begin{figure}
\centering
\includegraphics[width=1\textwidth,height=1\textheight]{N:/Assessments/CurrentAssessments/copper_rockfish_2023/models/nca/9.8_selex_fix_forecast/plots/comp_condAALdat_bubflt3mkt0.png}
\caption{Conditional age-at-length composition data from the recreational CPFV fleet.\label{fig:rec-cpfv-caal-data}}
\end{figure}

\pagebreak

\begin{figure}
\centering
\includegraphics[width=1\textwidth,height=1\textheight]{N:/Assessments/CurrentAssessments/copper_rockfish_2023/models/nca/9.8_selex_fix_forecast/plots/comp_agedat_data_weighting_TA1.8_Rec_CPFV.png}
\caption{Mean age for recreational CPFV fleet with 95 percent confidence intervals.\label{fig:mean-rec-cpfv-age-data}}
\end{figure}

\pagebreak

\begin{figure}
\centering
\includegraphics[width=1\textwidth,height=1\textheight]{N:/Assessments/CurrentAssessments/copper_rockfish_2023/data/ages/plots/coop_crfs_length_comparison.png}
\caption{Comparison of all length collected by the CRFS sampling program for the CPFV fleet to the lengths from the fish with ages from the cooperative sampling program. The length distributions in the area north of Point Conception are in general agreement while the distribution of lengths collected by this program does not align with the length samples from CRFS.\label{fig:coop-len-comparison}}
\end{figure}

\pagebreak

\begin{figure}
\centering
\includegraphics[width=1\textwidth,height=1\textheight]{N:/Assessments/CurrentAssessments/copper_rockfish_2023/models/nca/9.8_selex_fix_forecast/plots/comp_condAALdat_bubflt4mkt0.png}
\caption{Conditional age-at-length data for recreational PR collected in 2022.\label{fig:rec-pr-caal-data}}
\end{figure}

\pagebreak

\begin{figure}
\centering
\includegraphics[width=1\textwidth,height=1\textheight]{N:/Assessments/CurrentAssessments/copper_rockfish_2023/models/nca/9.8_selex_fix_forecast/plots/comp_lendat_flt4mkt0_page2.png}
\caption{Length composition data from the recreational PR fleet.\label{fig:rec-pr-len-data}}
\end{figure}

\pagebreak

\begin{figure}
\centering
\includegraphics[width=1\textwidth,height=1\textheight]{N:/Assessments/CurrentAssessments/copper_rockfish_2023/models/nca/9.8_selex_fix_forecast/plots/comp_lendat_data_weighting_TA1.8_Rec_PR.png}
\caption{Mean length for recreational PR fleet with 95 percent confidence intervals.\label{fig:mean-rec-pr-len-data}}
\end{figure}

\pagebreak

\begin{figure}
\centering
\includegraphics[width=1\textwidth,height=1\textheight]{N:/Assessments/CurrentAssessments/copper_rockfish_2023/data/survey_indices/plots/north_survey_locations_designation.png}
\caption{Sample locations by each of the fishery-independent data sources used in the base model with indices of abundance, lengths, and ages if collected.\label{fig:survey-locations}}
\end{figure}

\pagebreak

\begin{figure}
\centering
\includegraphics[width=1\textwidth,height=1\textheight]{N:/Assessments/CurrentAssessments/copper_rockfish_2023/data/survey_indices/plots/north_survey_locations.png}
\caption{Sample locations by area, areas open to fishing (reference) and MPAS, for each of the fishery-independent data sources used in the base model with indices of abundance, lengths, and ages if collected.\label{fig:ref-mpa}}
\end{figure}

\pagebreak

\begin{figure}
\centering
\includegraphics[width=1\textwidth,height=1\textheight]{N:/Assessments/CurrentAssessments/copper_rockfish_2023/data/survey_indices/ccfrp/north/area_weighted/Index.png}
\caption{Estimated index of abundance from the CCFRP survey.\label{fig:ccfrp-index-main}}
\end{figure}

\pagebreak

\begin{figure}
\centering
\includegraphics[width=1\textwidth,height=1\textheight]{N:/Assessments/CurrentAssessments/copper_rockfish_2023/models/nca/9.8_selex_fix_forecast/plots/comp_lendat_bubflt5mkt0.png}
\caption{Length composition data from the CCFRP survey.\label{fig:ccfrp-len-data}}
\end{figure}

\pagebreak

\begin{figure}
\centering
\includegraphics[width=1\textwidth,height=1\textheight]{N:/Assessments/CurrentAssessments/copper_rockfish_2023/models/nca/9.8_selex_fix_forecast/plots/comp_lendat_data_weighting_TA1.8_CCFRP.png}
\caption{Mean length for the CCFRP survey with 95 percent confidence intervals.\label{fig:ccfrp-mean-len-data}}
\end{figure}

\pagebreak

\begin{figure}
\centering
\includegraphics[width=1\textwidth,height=1\textheight]{N:/Assessments/CurrentAssessments/copper_rockfish_2023/models/nca/9.8_selex_fix_forecast/plots/comp_condAALdat_bubflt5mkt0.png}
\caption{Conditional age-at-length data from the CCFRP survey.\label{fig:ccfrp-age-data}}
\end{figure}

\pagebreak

\begin{figure}
\centering
\includegraphics[width=1\textwidth,height=1\textheight]{N:/Assessments/CurrentAssessments/copper_rockfish_2023/data/survey_indices/rov/plots/rov_transect_collapsed_copper_north_protection_count.png}
\caption{The location and size of observations across all years and transects.\label{fig:rov-obs-loc}}
\end{figure}

\pagebreak

\begin{figure}
\centering
\includegraphics[width=1\textwidth,height=1\textheight]{N:/Assessments/CurrentAssessments/copper_rockfish_2023/data/survey_indices/rov/plots/north_raw_cpue_by_mpa_group.png}
\caption{The trend of the calculated CPUE by each MPA and Reference group.\label{fig:rov-raw-cpue}}
\end{figure}

\pagebreak

\begin{figure}
\centering
\includegraphics[width=1\textwidth,height=1\textheight]{N:/Assessments/CurrentAssessments/copper_rockfish_2023/data/survey_indices/rov/glm_negbin_north_designation_depth/Index.png}
\caption{The estimated weighted relative index of abundance.\label{fig:rov-index-main}}
\end{figure}

\pagebreak

\begin{figure}
\centering
\includegraphics[width=1\textwidth,height=1\textheight]{N:/Assessments/CurrentAssessments/copper_rockfish_2023/data/survey_indices/rov/plots/rov_length_by_area_designation.png}
\caption{The distribution of lengths across all years for MPA and Reference area north and south of Point Conception.\label{fig:rov-len}}
\end{figure}

\pagebreak

\begin{figure}
\centering
\includegraphics[width=1\textwidth,height=1\textheight]{N:/Assessments/CurrentAssessments/copper_rockfish_2023/models/nca/9.8_selex_fix_forecast/plots/comp_lendat_flt6mkt0.png}
\caption{Length composition data from the CDFW ROV survey.\label{fig:rov-len-data}}
\end{figure}

\pagebreak

\begin{figure}
\centering
\includegraphics[width=1\textwidth,height=1\textheight]{N:/Assessments/CurrentAssessments/copper_rockfish_2023/models/nca/9.8_selex_fix_forecast/plots/comp_lendat_data_weighting_TA1.8_CDFW_ROV.png}
\caption{Mean length for CDWF ROV survey with 95 percent confidence intervals.\label{fig:mean-rov-len-data}}
\end{figure}

\pagebreak

\begin{figure}
\centering
\includegraphics[width=1\textwidth,height=1\textheight]{N:/Assessments/CurrentAssessments/copper_rockfish_2023/data/ages/plots/south_growth_length_comparison.png}
\caption{Length distribution of fish by collection source that were used as conditional age-at-length data in the growth fleet.\label{fig:growth-len-dist}}
\end{figure}

\pagebreak

\begin{figure}
\centering
\includegraphics[width=1\textwidth,height=1\textheight]{N:/Assessments/CurrentAssessments/copper_rockfish_2023/data/ages/plots/south_growth_age_comparison.png}
\caption{Age distribution of fish by collection source that were used as conditional age-at-length data in the growth fleet.\label{fig:growth-age-dist}}
\end{figure}

\pagebreak

\begin{figure}
\centering
\includegraphics[width=1\textwidth,height=1\textheight]{N:/Assessments/CurrentAssessments/copper_rockfish_2023/data/wcgbt/north/plots/cpue_map.png}
\caption{Location and catch-per-unit-effort by location caught north of Point Conception by the NWFSC WCGBT survey.\label{fig:wcgbt-cpue}}
\end{figure}

\pagebreak

\begin{figure}
\centering
\includegraphics[width=1\textwidth,height=1\textheight]{N:/Assessments/CurrentAssessments/copper_rockfish_2023/data/wcgbt/north/plots/presence-absence_proportion_by_depth.png}
\caption{Number of positive tows across all years by depth in meters.\label{fig:wcgbt-depth}}
\end{figure}

\pagebreak

\begin{figure}
\centering
\includegraphics[width=1\textwidth,height=1\textheight]{N:/Assessments/CurrentAssessments/copper_rockfish_2023/data/wcgbt/plots/wcgbt_north_age_at_length_by_area.png}
\caption{Age and length by sex for copper rockfish caught north of Point Conception by the NWFSC WCGBT survey.\label{fig:wcgbt-len-age}}
\end{figure}

\pagebreak

\hypertarget{biology}{%
\subsection{Biology}\label{biology}}

\begin{figure}
\centering
\includegraphics[width=1\textwidth,height=1\textheight]{N:/Assessments/CurrentAssessments/copper_rockfish_2023/models/nca/9.8_selex_fix_forecast/plots/bio6_maturity.png}
\caption{Maturity as a function of length.\label{fig:maturity}}
\end{figure}

\pagebreak

\begin{figure}
\centering
\includegraphics[width=1\textwidth,height=1\textheight]{N:/Assessments/CurrentAssessments/copper_rockfish_2023/models/nca/9.8_selex_fix_forecast/plots/bio9_fecundity_len.png}
\caption{Fecundity as a function of length.\label{fig:fecundity}}
\end{figure}

\pagebreak

\begin{figure}
\centering
\includegraphics[width=1\textwidth,height=1\textheight]{N:/Assessments/CurrentAssessments/copper_rockfish_2023/data/wcgbt/plots/length_fraction_female.png}
\caption{Fraction of each sex by length by the NWFSC WCGBT survey.\label{fig:frac-sex-len}}
\end{figure}

\pagebreak

\begin{figure}
\centering
\includegraphics[width=1\textwidth,height=1\textheight]{N:/Assessments/CurrentAssessments/copper_rockfish_2023/data/biology/plots/Length_Weight_All.png}
\caption{Estimated weight-at-length.\label{fig:weight-length}}
\end{figure}

\pagebreak

\begin{figure}
\centering
\includegraphics[width=1\textwidth,height=1\textheight]{N:/Assessments/CurrentAssessments/copper_rockfish_2023/data/ages/ageing_error/B0_S3/Reader_1_vs_Reader_2.png}
\caption{Distribution of double reads between age reader 1 and 2.\label{fig:age-error-dist}}
\end{figure}

\pagebreak

\begin{figure}
\centering
\includegraphics[width=1\textwidth,height=1\textheight]{N:/Assessments/CurrentAssessments/copper_rockfish_2023/models/nca/9.8_selex_fix_forecast/plots/numbers5_ageerrorSD.png}
\caption{Ageing imprecision standard deviation of observed age in years.\label{fig:age-error}}
\end{figure}

\pagebreak

\begin{figure}
\centering
\includegraphics[width=1\textwidth,height=1\textheight]{N:/Assessments/CurrentAssessments/copper_rockfish_2023/models/nca/9.8_selex_fix_forecast/plots/numbers10_ageerror_matrix_1.png}
\caption{Distribution of observed age at true age for ageing error type 1.\label{fig:age-error-matrix}}
\end{figure}

\pagebreak

\hypertarget{model-results}{%
\subsection{Model Results}\label{model-results}}

\hypertarget{model-bridging}{%
\subsubsection{Model Bridging}\label{model-bridging}}

\begin{figure}
\centering
\includegraphics[width=1\textwidth,height=1\textheight]{N:/Assessments/CurrentAssessments/copper_rockfish_2023/models/nca/_bridging/_plots/0_model_convert_compare2_spawnbio_uncertainty.png}
\caption{Model version bridge comparison of estimated spawning output.\label{fig:bridge-ssb}}
\end{figure}

\pagebreak

\begin{figure}
\centering
\includegraphics[width=1\textwidth,height=1\textheight]{N:/Assessments/CurrentAssessments/copper_rockfish_2023/models/nca/_bridging/_plots/0_model_convert_compare4_Bratio_uncertainty.png}
\caption{Model version bridge comparison of estimated fraction unfished.\label{fig:bridge-depl}}
\end{figure}

\pagebreak

\begin{figure}
\centering
\includegraphics[width=1\textwidth,height=1\textheight]{N:/Assessments/CurrentAssessments/copper_rockfish_2023/models/nca/_bridging/_plots/full_bridge_1_compare2_spawnbio_uncertainty.png}
\caption{Model structure and data bridging comparison of estimated spawning output.\label{fig:data-bridge-ssb-1}}
\end{figure}

\pagebreak

\begin{figure}
\centering
\includegraphics[width=1\textwidth,height=1\textheight]{N:/Assessments/CurrentAssessments/copper_rockfish_2023/models/nca/_bridging/_plots/full_bridge_1_compare4_Bratio_uncertainty.png}
\caption{Model structure and data bridging comparison of estimated fraction unfished.\label{fig:data-bridge-depl-1}}
\end{figure}

\pagebreak

\begin{figure}
\centering
\includegraphics[width=1\textwidth,height=1\textheight]{N:/Assessments/CurrentAssessments/copper_rockfish_2023/models/nca/_bridging/_plots/full_bridge_2_compare2_spawnbio_uncertainty.png}
\caption{Model structure and data bridging comparison of estimated spawning output.\label{fig:data-bridge-ssb-2}}
\end{figure}

\pagebreak

\begin{figure}
\centering
\includegraphics[width=1\textwidth,height=1\textheight]{N:/Assessments/CurrentAssessments/copper_rockfish_2023/models/nca/_bridging/_plots/full_bridge_2_compare4_Bratio_uncertainty.png}
\caption{Model structure and data bridging comparison of estimated fraction unfished.\label{fig:data-bridge-depl-2}}
\end{figure}

\pagebreak

\pagebreak

\hypertarget{biology-1}{%
\subsubsection{Biology}\label{biology-1}}

\begin{figure}
\centering
\includegraphics[width=1\textwidth,height=1\textheight]{N:/Assessments/CurrentAssessments/copper_rockfish_2023/models/nca/9.8_selex_fix_forecast/plots/bio1_sizeatage.png}
\caption{Model estimated length-at-age in the beginning of the year. Shaded area indicates 95 percent distribution of length-at-age around the estimated growth curve.\label{fig:mod-est-len-age}}
\end{figure}

\pagebreak

\hypertarget{selectivity}{%
\subsubsection{Selectivity}\label{selectivity}}

\pagebreak

\hypertarget{recruitment-1}{%
\subsubsection{Recruitment}\label{recruitment-1}}

\begin{figure}
\centering
\includegraphics[width=1\textwidth,height=1\textheight]{N:/Assessments/CurrentAssessments/copper_rockfish_2023/models/nca/9.8_selex_fix_forecast/plots/ts11_Age-0_recruits_(1000s)_with_95_asymptotic_intervals.png}
\caption{Estimated time series of age-0 recruits (1000s).\label{fig:recruits}}
\end{figure}
>>>>>>> Stashed changes

\begingroup\fontsize{10}{12}\selectfont
\begingroup\fontsize{10}{12}\selectfont

\begin{longtable}[t]{r>{\centering\arraybackslash}p{2cm}}
\caption{\label{tab:north-removalsES}Likelihood components by source.}\\
\toprule
Label & Total\\
\midrule
\endfirsthead
\caption[]{Likelihood components by source. \textit{(continued)}}\\
\toprule
Label & Total\\
\midrule
\endhead

\endfoot
\bottomrule
\endlastfoot
TOTAL & 973.82\\
Catch & 0.00\\
Equil catch & 0.00\\
Survey & -45.27\\
Length comp & 416.49\\
Age comp & 596.55\\
Recruitment & 6.05\\
InitEQ Regime & 0.00\\
Forecast Recruitment & 0.00\\
Parm priors & 0.00\\
Parm softbounds & 0.00\\
Parm devs & 0.00\\
Crash Pen & 0.00\\*
\end{longtable}
\endgroup{}
\endgroup{}


\begingroup\fontsize{10}{12}\selectfont
\begingroup\fontsize{10}{12}\selectfont

\begin{longtable}[t]{r>{\centering\arraybackslash}p{2cm}>{\centering\arraybackslash}p{2cm}>{\centering\arraybackslash}p{2cm}}
	\caption{\label{tab:north-referenceES}Summary of reference points and management quantities, including estimates of the 95 percent intervals for the sub-area model north of Point Conception.}\\
	\toprule
	& Estimate & Lower Interval & Upper Interval\\
	\midrule
	\endfirsthead
	\caption[]{Summary of reference points and management quantities, including estimates of the 95 percent intervals for the sub-area model north of Point Conception. \textit{(continued)}}\\
	\toprule
	& Estimate & Lower Interval & Upper Interval\\
	\midrule
	\endhead
	
	\endfoot
	\bottomrule
	\endlastfoot
	Unfished Spawning Output & 486.15 & 387.43 & 584.87\\
	Unfished Age 3+ Biomass (mt) & 4719.91 & 3777.92 & 5661.90\\
	Unfished Recruitment (R0) & 567.77 & 452.48 & 683.06\\
	Spawning Output (2023) & 262.10 & 124.28 & 399.92\\
	Fraction Unfished (2023) & 0.54 & 0.32 & 0.76\\
	Reference Points Based SB40\% &  &  & \\
	Proxy Spawning Output SB40\% & 194.46 & 154.97 & 233.95\\
	SPR Resulting in SB40\% & 0.46 & 0.46 & 0.46\\
	Exploitation Rate Resulting in SB40\% & 0.06 & 0.06 & 0.06\\
	Yield with SPR Based On SB40\% (mt) & 129.86 & 104.05 & 155.67\\
	Reference Points Based on SPR Proxy for MSY & NA & NA & NA\\
	Proxy Spawning Output (SPR50) & 216.90 & 172.85 & 260.94\\
	SPR50 & 0.50 &  & \\
	Exploitation Rate Corresponding to SPR50 & 0.05 & 0.05 & 0.05\\
	Yield with SPR50 at SB SPR (mt) & 124.05 & 99.39 & 148.71\\
	Reference Points Based on Estimated MSY Values & NA & NA & NA\\
	Spawning Output at MSY (SB MSY) & 134.17 & 106.84 & 161.51\\
	SPR MSY & 0.35 & 0.34 & 0.35\\
	Exploitation Rate Corresponding to SPR MSY & 0.09 & 0.08 & 0.09\\
	MSY (mt) & 137.59 & 110.25 & 164.92\\*
\end{longtable}
\endgroup{}
\endgroup{}


\newpage

\begingroup\fontsize{10}{12}\selectfont
\begingroup\fontsize{10}{12}\selectfont

\begin{longtable}[t]{r>{\centering\arraybackslash}p{1.22cm}>{\centering\arraybackslash}p{1.22cm}>{\centering\arraybackslash}p{1.22cm}>{\centering\arraybackslash}p{1.22cm}>{\centering\arraybackslash}p{1.22cm}>{\centering\arraybackslash}p{1.22cm}>{\centering\arraybackslash}p{1.22cm}>{\centering\arraybackslash}p{1.22cm}}
\caption{\label{tab:north-timeseries}Time series of population estimates from the base model for the sub-area north of Point Conception.}\\
\toprule
Year & Total Biomass (mt) & Spawning Output & Total Biomass 3+ (mt) & Fraction Unfished & Age-0 Recruits & Total Mortality (mt) & 1-SPR & Exploitation Rate\\
\midrule
\endfirsthead
\caption[]{Time series of population estimates from the base model for the sub-area north of Point Conception. \textit{(continued)}}\\
\toprule
Year & Total Biomass (mt) & Spawning Output & Total Biomass 3+ (mt) & Fraction Unfished & Age-0 Recruits & Total Mortality (mt) & 1-SPR & Exploitation Rate\\
\midrule
\endhead

\endfoot
\bottomrule
\endlastfoot
1916 & 4758.75 & 487.03 & 4731.64 & 1.00 & 573.48 & 3.97 & 0.01 & 0.00\\
1917 & 4756.53 & 486.73 & 4729.41 & 1.00 & 573.92 & 6.23 & 0.02 & 0.00\\
1918 & 4752.47 & 486.22 & 4725.31 & 1.00 & 574.38 & 7.50 & 0.02 & 0.00\\
1919 & 4747.63 & 485.61 & 4720.46 & 1.00 & 574.87 & 4.91 & 0.01 & 0.00\\
1920 & 4745.78 & 485.32 & 4718.58 & 1.00 & 575.44 & 5.06 & 0.01 & 0.00\\
1921 & 4744.18 & 485.05 & 4716.96 & 1.00 & 576.07 & 4.31 & 0.01 & 0.00\\
1922 & 4743.72 & 484.91 & 4716.47 & 1.00 & 576.76 & 3.70 & 0.01 & 0.00\\
1923 & 4744.21 & 484.87 & 4716.93 & 1.00 & 577.52 & 3.89 & 0.01 & 0.00\\
1924 & 4744.88 & 484.85 & 4717.56 & 1.00 & 578.34 & 2.57 & 0.01 & 0.00\\
1925 & 4747.14 & 485.02 & 4719.79 & 1.00 & 579.26 & 3.84 & 0.01 & 0.00\\
1926 & 4748.48 & 485.07 & 4721.09 & 1.00 & 580.23 & 4.89 & 0.01 & 0.00\\
1927 & 4749.17 & 485.05 & 4721.73 & 1.00 & 581.27 & 3.64 & 0.01 & 0.00\\
1928 & 4751.45 & 485.20 & 4723.97 & 1.00 & 582.41 & 5.21 & 0.02 & 0.00\\
1929 & 4752.50 & 485.22 & 4724.97 & 1.00 & 583.84 & 6.18 & 0.02 & 0.00\\
1930 & 4752.95 & 485.18 & 4725.36 & 1.00 & 585.39 & 8.96 & 0.03 & 0.00\\
1931 & 4751.19 & 484.88 & 4723.53 & 1.00 & 587.04 & 11.15 & 0.03 & 0.00\\
1932 & 4747.92 & 484.39 & 4720.18 & 1.00 & 588.84 & 11.73 & 0.04 & 0.00\\
1933 & 4744.80 & 483.91 & 4716.98 & 1.00 & 590.78 & 12.11 & 0.04 & 0.00\\
1934 & 4742.11 & 483.45 & 4714.21 & 0.99 & 592.87 & 12.01 & 0.04 & 0.00\\
1935 & 4740.36 & 483.08 & 4712.37 & 0.99 & 595.19 & 15.35 & 0.05 & 0.00\\
1936 & 4736.34 & 482.44 & 4708.24 & 0.99 & 597.65 & 16.09 & 0.05 & 0.00\\
1937 & 4732.71 & 481.81 & 4704.51 & 0.99 & 600.32 & 18.78 & 0.06 & 0.00\\
1938 & 4727.67 & 481.01 & 4699.34 & 0.99 & 603.21 & 17.93 & 0.06 & 0.00\\
1939 & 4724.86 & 480.41 & 4696.41 & 0.99 & 606.46 & 16.05 & 0.05 & 0.00\\
1940 & 4725.39 & 480.14 & 4696.79 & 0.99 & 610.18 & 20.78 & 0.06 & 0.00\\
1941 & 4722.63 & 479.52 & 4693.88 & 0.99 & 614.38 & 19.96 & 0.06 & 0.00\\
1942 & 4722.37 & 479.12 & 4693.43 & 0.99 & 619.23 & 9.64 & 0.03 & 0.00\\
1943 & 4734.07 & 479.95 & 4704.93 & 0.99 & 624.88 & 10.40 & 0.03 & 0.00\\
1944 & 4746.61 & 480.85 & 4717.22 & 0.99 & 631.13 & 14.81 & 0.04 & 0.00\\
1945 & 4756.67 & 481.43 & 4727.01 & 0.99 & 637.77 & 29.60 & 0.08 & 0.01\\
1946 & 4754.55 & 480.61 & 4724.59 & 0.99 & 644.39 & 38.03 & 0.11 & 0.01\\
1947 & 4746.83 & 479.14 & 4716.56 & 0.99 & 650.70 & 18.34 & 0.06 & 0.00\\
1948 & 4761.18 & 480.03 & 4730.59 & 0.99 & 656.79 & 31.93 & 0.10 & 0.01\\
1949 & 4764.49 & 479.75 & 4733.61 & 0.99 & 661.72 & 34.12 & 0.10 & 0.01\\
1950 & 4768.23 & 479.52 & 4737.08 & 0.99 & 665.42 & 39.31 & 0.12 & 0.01\\
1951 & 4769.54 & 479.03 & 4738.17 & 0.99 & 667.86 & 53.91 & 0.16 & 0.01\\
1952 & 4759.34 & 477.30 & 4727.81 & 0.98 & 669.38 & 45.06 & 0.13 & 0.01\\
1953 & 4761.01 & 476.77 & 4729.37 & 0.98 & 671.53 & 36.30 & 0.11 & 0.01\\
1954 & 4774.05 & 477.47 & 4742.33 & 0.98 & 675.16 & 46.94 & 0.14 & 0.01\\
1955 & 4778.65 & 477.39 & 4746.82 & 0.98 & 679.63 & 52.40 & 0.16 & 0.01\\
1956 & 4779.65 & 477.06 & 4747.63 & 0.98 & 686.18 & 60.11 & 0.17 & 0.01\\
1957 & 4774.87 & 476.17 & 4742.63 & 0.98 & 683.59 & 58.27 & 0.17 & 0.01\\
1958 & 4774.03 & 475.66 & 4741.58 & 0.98 & 682.24 & 99.22 & 0.27 & 0.02\\
1959 & 4733.91 & 471.20 & 4701.56 & 0.97 & 684.64 & 80.28 & 0.23 & 0.02\\
1960 & 4715.50 & 468.80 & 4683.16 & 0.96 & 699.37 & 68.30 & 0.20 & 0.01\\
1961 & 4711.74 & 467.80 & 4679.12 & 0.96 & 732.19 & 51.06 & 0.15 & 0.01\\
1962 & 4727.91 & 468.84 & 4694.37 & 0.96 & 777.94 & 63.56 & 0.19 & 0.01\\
1963 & 4734.77 & 468.93 & 4699.55 & 0.96 & 817.20 & 79.29 & 0.23 & 0.02\\
1964 & 4731.24 & 467.64 & 4693.89 & 0.96 & 881.00 & 70.73 & 0.21 & 0.02\\
1965 & 4744.87 & 467.41 & 4705.34 & 0.96 & 983.81 & 104.92 & 0.29 & 0.02\\
1966 & 4736.12 & 464.09 & 4693.15 & 0.95 & 1067.20 & 121.03 & 0.32 & 0.03\\
1967 & 4728.19 & 459.74 & 4680.69 & 0.95 & 1051.05 & 128.37 & 0.34 & 0.03\\
1968 & 4734.70 & 455.64 & 4684.53 & 0.94 & 923.06 & 135.98 & 0.35 & 0.03\\
1969 & 4756.61 & 452.27 & 4708.54 & 0.93 & 741.98 & 145.84 & 0.37 & 0.03\\
1970 & 4786.46 & 449.99 & 4744.98 & 0.93 & 608.42 & 181.20 & 0.43 & 0.04\\
1971 & 4786.28 & 446.80 & 4752.87 & 0.92 & 455.26 & 169.15 & 0.41 & 0.04\\
1972 & 4788.22 & 447.24 & 4761.28 & 0.92 & 357.88 & 215.21 & 0.48 & 0.05\\
1973 & 4718.98 & 444.91 & 4698.51 & 0.92 & 373.18 & 246.30 & 0.52 & 0.05\\
1974 & 4582.63 & 439.32 & 4565.55 & 0.90 & 350.26 & 270.39 & 0.56 & 0.06\\
1975 & 4383.13 & 428.95 & 4365.71 & 0.88 & 365.48 & 267.98 & 0.58 & 0.06\\
1976 & 4150.72 & 414.85 & 4134.02 & 0.85 & 332.35 & 296.45 & 0.63 & 0.07\\
1977 & 3862.72 & 393.09 & 3845.83 & 0.81 & 306.02 & 306.11 & 0.67 & 0.08\\
1978 & 3547.11 & 365.81 & 3531.76 & 0.75 & 239.16 & 281.47 & 0.68 & 0.08\\
1979 & 3245.65 & 337.69 & 3231.98 & 0.69 & 192.93 & 292.80 & 0.72 & 0.09\\
1980 & 2926.86 & 306.26 & 2916.03 & 0.63 & 222.56 & 305.18 & 0.75 & 0.10\\
1981 & 2594.11 & 272.27 & 2584.49 & 0.56 & 344.88 & 287.77 & 0.77 & 0.11\\
1982 & 2277.63 & 239.84 & 2265.84 & 0.49 & 243.16 & 284.03 & 0.80 & 0.13\\
1983 & 1971.16 & 207.66 & 1955.99 & 0.43 & 243.06 & 209.77 & 0.73 & 0.11\\
1984 & 1754.30 & 183.00 & 1742.51 & 0.38 & 462.12 & 185.11 & 0.74 & 0.11\\
1985 & 1576.13 & 161.71 & 1562.04 & 0.33 & 551.70 & 206.53 & 0.81 & 0.13\\
1986 & 1392.78 & 139.51 & 1370.20 & 0.29 & 314.58 & 164.96 & 0.79 & 0.12\\
1987 & 1271.95 & 122.97 & 1248.57 & 0.25 & 274.54 & 100.25 & 0.66 & 0.08\\
1988 & 1238.68 & 114.26 & 1224.21 & 0.24 & 299.29 & 112.68 & 0.70 & 0.09\\
1989 & 1207.29 & 106.33 & 1193.84 & 0.22 & 441.70 & 122.44 & 0.72 & 0.10\\
1990 & 1174.70 & 99.83 & 1158.82 & 0.21 & 518.43 & 133.18 & 0.75 & 0.11\\
1991 & 1138.25 & 94.81 & 1116.68 & 0.20 & 361.75 & 131.79 & 0.75 & 0.12\\
1992 & 1112.37 & 90.96 & 1089.66 & 0.19 & 316.48 & 142.04 & 0.77 & 0.13\\
1993 & 1087.78 & 86.39 & 1070.99 & 0.18 & 462.51 & 162.74 & 0.82 & 0.15\\
1994 & 1048.63 & 80.67 & 1032.13 & 0.17 & 361.77 & 93.49 & 0.68 & 0.09\\
1995 & 1079.12 & 82.08 & 1058.52 & 0.17 & 247.39 & 76.06 & 0.60 & 0.07\\
1996 & 1130.71 & 86.01 & 1114.85 & 0.18 & 272.30 & 83.98 & 0.61 & 0.08\\
1997 & 1174.03 & 90.16 & 1161.98 & 0.19 & 315.62 & 80.03 & 0.58 & 0.07\\
1998 & 1215.54 & 95.08 & 1202.13 & 0.20 & 339.49 & 44.90 & 0.40 & 0.04\\
1999 & 1284.31 & 103.40 & 1269.15 & 0.21 & 302.54 & 37.01 & 0.33 & 0.03\\
2000 & 1354.22 & 112.46 & 1338.64 & 0.23 & 251.04 & 31.63 & 0.27 & 0.02\\
2001 & 1423.65 & 121.31 & 1409.88 & 0.25 & 274.09 & 28.95 & 0.25 & 0.02\\
2002 & 1489.36 & 129.50 & 1477.16 & 0.27 & 313.55 & 21.83 & 0.18 & 0.01\\
2003 & 1554.22 & 137.65 & 1540.85 & 0.28 & 268.18 & 23.78 & 0.19 & 0.02\\
2004 & 1609.02 & 145.10 & 1594.74 & 0.30 & 227.39 & 18.92 & 0.15 & 0.01\\
2005 & 1661.84 & 152.32 & 1649.64 & 0.31 & 196.19 & 34.85 & 0.26 & 0.02\\
2006 & 1690.21 & 157.19 & 1679.61 & 0.32 & 346.50 & 36.21 & 0.27 & 0.02\\
2007 & 1709.37 & 161.04 & 1697.69 & 0.33 & 888.98 & 42.04 & 0.30 & 0.02\\
2008 & 1720.32 & 163.55 & 1698.04 & 0.34 & 683.78 & 31.80 & 0.24 & 0.02\\
2009 & 1752.87 & 166.22 & 1712.94 & 0.34 & 819.71 & 39.21 & 0.29 & 0.02\\
2010 & 1806.79 & 167.32 & 1773.45 & 0.34 & 376.08 & 27.47 & 0.21 & 0.02\\
2011 & 1904.77 & 169.37 & 1870.94 & 0.35 & 412.57 & 25.88 & 0.19 & 0.01\\
2012 & 2031.54 & 173.50 & 2012.93 & 0.36 & 721.98 & 34.91 & 0.23 & 0.02\\
2013 & 2160.53 & 180.75 & 2137.65 & 0.37 & 602.74 & 25.64 & 0.16 & 0.01\\
2014 & 2302.33 & 193.08 & 2269.62 & 0.40 & 508.66 & 36.94 & 0.20 & 0.02\\
2015 & 2435.83 & 207.33 & 2408.19 & 0.43 & 644.80 & 65.46 & 0.31 & 0.03\\
2016 & 2538.95 & 219.54 & 2513.77 & 0.45 & 304.21 & 66.32 & 0.31 & 0.03\\
2017 & 2635.51 & 230.78 & 2608.04 & 0.47 & 932.00 & 138.62 & 0.50 & 0.05\\
2018 & 2657.53 & 234.32 & 2636.38 & 0.48 & 649.02 & 102.38 & 0.41 & 0.04\\
2019 & 2709.53 & 240.87 & 2668.99 & 0.50 & 358.62 & 101.89 & 0.40 & 0.04\\
2020 & 2765.07 & 247.00 & 2737.43 & 0.51 & 518.96 & 156.21 & 0.53 & 0.06\\
2021 & 2765.72 & 247.07 & 2746.93 & 0.51 & 518.99 & 106.99 & 0.41 & 0.04\\
2022 & 2807.65 & 251.40 & 2783.08 & 0.52 & 520.55 & 47.15 & 0.21 & 0.02\\
2023 & 2899.87 & 262.10 & 2875.27 & 0.54 & 524.24 & 91.50 & 0.35 & 0.03\\
2024 & 2938.51 & 268.97 & 2913.82 & 0.55 & 526.48 & 66.90 & 0.27 & 0.02\\
2025 & 2993.09 & 277.24 & 2968.25 & 0.57 & 529.05 & 105.37 & 0.38 & 0.04\\
2026 & 3002.22 & 280.38 & 2977.27 & 0.58 & 529.99 & 104.04 & 0.38 & 0.03\\
2027 & 3007.39 & 282.32 & 2982.33 & 0.58 & 530.56 & 102.85 & 0.38 & 0.03\\
2028 & 3010.14 & 283.48 & 2985.04 & 0.58 & 530.90 & 101.52 & 0.38 & 0.03\\
2029 & 3011.84 & 284.18 & 2986.72 & 0.58 & 531.10 & 99.77 & 0.37 & 0.03\\
2030 & 3013.81 & 284.70 & 2988.68 & 0.59 & 531.26 & 98.37 & 0.37 & 0.03\\
2031 & 3016.28 & 285.14 & 2991.13 & 0.59 & 531.38 & 97.00 & 0.36 & 0.03\\
2032 & 3019.57 & 285.60 & 2994.42 & 0.59 & 531.52 & 95.34 & 0.36 & 0.03\\
2033 & 3024.21 & 286.17 & 2999.06 & 0.59 & 531.68 & 94.10 & 0.36 & 0.03\\
2034 & 3029.85 & 286.82 & 3004.69 & 0.59 & 531.87 & 102.94 & 0.38 & 0.03\\*
\end{longtable}
\endgroup{}
\endgroup{}


\newpage

\begingroup\fontsize{10}{12}\selectfont
\begingroup\fontsize{10}{12}\selectfont

\begin{longtable}[t]{c>{\centering\arraybackslash}p{1.83cm}>{\centering\arraybackslash}p{1.83cm}>{\centering\arraybackslash}p{1.83cm}>{\centering\arraybackslash}p{1.83cm}>{\centering\arraybackslash}p{1.83cm}}
\caption{\label{tab:ca-status}The estimated total biomass (mt), total biomass age 3+ (mt), age-0 recruits, spawning ouput in number of million eggs across California and fraction unfished by year.}\\
\toprule
Year & Total Biomass (mt) & Total Biomass 3+ (mt) & Age-0 Recruits & Spawning Output & Fraction Unfished\\
\midrule
\endfirsthead
\caption[]{\label{tab:ca-status}The estimated total biomass (mt), total biomass age 3+ (mt), age-0 recruits, spawning ouput in number of million eggs across California and fraction unfished by year. \textit{(continued)}}\\
\toprule
Year & Total Biomass (mt) & Total Biomass 3+ (mt) & Age-0 Recruits & Spawning Output & Fraction Unfished\\
\midrule
\endhead

\endfoot
\bottomrule
\endlastfoot
1916 & 6771.66 & 6726.78 & 483.56 & 688.74 & 1.001\\
1917 & 6769.45 & 6724.56 & 483.64 & 688.45 & 1.001\\
1918 & 6765.34 & 6720.40 & 483.72 & 687.93 & 1.000\\
1919 & 6760.48 & 6715.54 & 483.82 & 687.31 & 0.999\\
1920 & 6758.72 & 6713.74 & 483.92 & 687.02 & 0.999\\
1921 & 6757.21 & 6712.21 & 484.04 & 686.76 & 0.999\\
1922 & 6756.88 & 6711.84 & 484.16 & 686.62 & 0.998\\
1923 & 6757.52 & 6712.45 & 484.30 & 686.60 & 0.998\\
1924 & 6758.33 & 6713.22 & 484.45 & 686.59 & 0.998\\
1925 & 6760.72 & 6715.57 & 484.61 & 686.76 & 0.999\\
1926 & 6762.19 & 6716.99 & 484.79 & 686.83 & 0.999\\
1927 & 6763.00 & 6717.75 & 484.98 & 686.81 & 0.999\\
1928 & 6765.48 & 6720.18 & 485.19 & 686.97 & 0.999\\
1929 & 6766.75 & 6721.40 & 485.42 & 687.01 & 0.999\\
1930 & 6767.44 & 6722.01 & 485.67 & 686.99 & 0.999\\
1931 & 6765.92 & 6720.42 & 485.95 & 686.70 & 0.998\\
1932 & 6762.93 & 6717.34 & 486.24 & 686.24 & 0.998\\
1933 & 6760.05 & 6714.37 & 486.56 & 685.76 & 0.997\\
1934 & 6757.78 & 6712.00 & 486.92 & 685.34 & 0.996\\
1935 & 6756.39 & 6710.51 & 487.30 & 685.00 & 0.996\\
1936 & 6752.51 & 6706.51 & 487.71 & 684.36 & 0.995\\
1937 & 6749.24 & 6703.12 & 488.15 & 683.76 & 0.994\\
1938 & 6743.91 & 6697.65 & 488.62 & 682.90 & 0.993\\
1939 & 6741.38 & 6694.98 & 489.14 & 682.32 & 0.992\\
1940 & 6742.49 & 6695.93 & 489.72 & 682.08 & 0.992\\
1941 & 6740.38 & 6693.65 & 490.33 & 681.51 & 0.991\\
1942 & 6740.81 & 6693.87 & 491.00 & 681.16 & 0.990\\
1943 & 6753.76 & 6706.59 & 491.72 & 682.09 & 0.992\\
1944 & 6767.59 & 6720.14 & 492.49 & 683.10 & 0.993\\
1945 & 6779.15 & 6731.41 & 493.30 & 683.81 & 0.994\\
1946 & 6778.55 & 6730.48 & 494.16 & 683.12 & 0.993\\
1947 & 6772.42 & 6724.01 & 495.04 & 681.79 & 0.991\\
1948 & 6787.89 & 6739.13 & 495.92 & 682.78 & 0.993\\
1949 & 6791.34 & 6742.26 & 496.75 & 682.50 & 0.992\\
1950 & 6794.80 & 6745.41 & 497.53 & 682.23 & 0.992\\
1951 & 6795.11 & 6745.48 & 498.22 & 681.61 & 0.991\\
1952 & 6781.60 & 6731.78 & 498.76 & 679.48 & 0.988\\
1953 & 6781.57 & 6731.62 & 499.17 & 678.73 & 0.987\\
1954 & 6793.47 & 6743.42 & 499.31 & 679.27 & 0.988\\
1955 & 6792.50 & 6742.32 & 498.98 & 678.64 & 0.987\\
1956 & 6779.53 & 6729.16 & 498.24 & 676.97 & 0.984\\
1957 & 6759.35 & 6708.78 & 497.29 & 674.55 & 0.981\\
1958 & 6751.38 & 6700.63 & 496.75 & 673.22 & 0.979\\
1959 & 6704.70 & 6654.08 & 496.83 & 667.98 & 0.971\\
1960 & 6685.45 & 6634.85 & 498.53 & 665.34 & 0.967\\
1961 & 6680.50 & 6629.60 & 501.47 & 664.12 & 0.966\\
1962 & 6693.08 & 6641.20 & 505.68 & 664.72 & 0.966\\
1963 & 6700.02 & 6646.33 & 512.94 & 664.72 & 0.966\\
1964 & 6696.71 & 6640.70 & 527.17 & 663.35 & 0.964\\
1965 & 6706.31 & 6647.79 & 557.75 & 662.63 & 0.963\\
1966 & 6688.99 & 6626.36 & 613.84 & 658.30 & 0.957\\
1967 & 6647.19 & 6578.69 & 692.63 & 650.44 & 0.946\\
1968 & 6616.98 & 6543.59 & 741.31 & 642.11 & 0.934\\
1969 & 6600.25 & 6526.41 & 632.10 & 633.79 & 0.922\\
1970 & 6613.12 & 6545.34 & 524.22 & 628.00 & 0.913\\
1971 & 6580.04 & 6524.32 & 456.46 & 619.92 & 0.901\\
1972 & 6554.52 & 6508.80 & 537.03 & 616.60 & 0.897\\
1973 & 6429.40 & 6391.48 & 585.75 & 609.13 & 0.886\\
1974 & 6212.80 & 6175.67 & 472.20 & 596.90 & 0.868\\
1975 & 5902.16 & 5864.17 & 413.01 & 576.83 & 0.839\\
1976 & 5553.05 & 5519.48 & 408.18 & 551.71 & 0.802\\
1977 & 5173.57 & 5141.51 & 460.98 & 520.85 & 0.757\\
1978 & 4773.59 & 4742.85 & 403.81 & 485.31 & 0.706\\
1979 & 4388.22 & 4358.13 & 320.16 & 449.06 & 0.653\\
1980 & 3940.02 & 3915.05 & 278.61 & 405.26 & 0.589\\
1981 & 3480.60 & 3459.55 & 293.65 & 358.87 & 0.522\\
1982 & 3100.46 & 3078.34 & 269.30 & 319.71 & 0.465\\
1983 & 2655.53 & 2629.80 & 243.16 & 274.64 & 0.399\\
1984 & 2375.06 & 2353.56 & 291.89 & 243.45 & 0.354\\
1985 & 2124.47 & 2100.97 & 423.38 & 214.84 & 0.312\\
1986 & 1845.66 & 1811.35 & 375.11 & 183.11 & 0.266\\
1987 & 1649.05 & 1610.57 & 318.75 & 158.20 & 0.230\\
1988 & 1571.30 & 1543.51 & 319.57 & 143.47 & 0.209\\
1989 & 1529.20 & 1504.06 & 302.07 & 132.51 & 0.193\\
1990 & 1489.34 & 1461.83 & 360.71 & 124.09 & 0.180\\
1991 & 1440.61 & 1407.47 & 390.95 & 117.50 & 0.171\\
1992 & 1411.63 & 1375.51 & 274.30 & 112.78 & 0.164\\
1993 & 1405.72 & 1375.57 & 182.29 & 109.06 & 0.159\\
1994 & 1402.16 & 1376.33 & 178.25 & 105.70 & 0.154\\
1995 & 1423.35 & 1396.07 & 206.33 & 106.98 & 0.156\\
1996 & 1471.73 & 1449.13 & 160.47 & 111.22 & 0.162\\
1997 & 1458.28 & 1439.05 & 129.43 & 111.93 & 0.163\\
1998 & 1492.25 & 1473.17 & 160.71 & 116.54 & 0.169\\
1999 & 1537.55 & 1517.25 & 357.39 & 123.32 & 0.179\\
2000 & 1584.44 & 1561.51 & 181.13 & 130.99 & 0.190\\
2001 & 1653.94 & 1628.49 & 123.13 & 139.74 & 0.203\\
2002 & 1730.34 & 1711.89 & 208.19 & 148.22 & 0.216\\
2003 & 1814.31 & 1795.64 & 351.55 & 157.39 & 0.229\\
2004 & 1886.01 & 1863.06 & 161.07 & 166.29 & 0.242\\
2005 & 1958.15 & 1934.51 & 236.37 & 175.25 & 0.255\\
2006 & 1994.02 & 1977.00 & 87.93 & 180.82 & 0.263\\
2007 & 2036.68 & 2017.34 & 330.52 & 186.63 & 0.271\\
2008 & 2050.75 & 2023.30 & 317.18 & 189.90 & 0.276\\
2009 & 2090.87 & 2038.55 & 851.87 & 193.68 & 0.282\\
2010 & 2160.63 & 2111.31 & 913.18 & 195.80 & 0.285\\
2011 & 2291.84 & 2226.56 & 408.77 & 198.79 & 0.289\\
2012 & 2452.86 & 2404.67 & 552.88 & 202.49 & 0.294\\
2013 & 2627.91 & 2588.61 & 928.75 & 210.14 & 0.306\\
2014 & 2795.04 & 2739.48 & 244.85 & 223.28 & 0.325\\
2015 & 2973.06 & 2916.90 & 102.22 & 240.91 & 0.350\\
2016 & 3097.56 & 3064.44 & 245.60 & 255.75 & 0.372\\
2017 & 3182.40 & 3150.11 & 122.12 & 268.46 & 0.390\\
2018 & 3182.40 & 3153.24 & 96.39 & 273.80 & 0.398\\
2019 & 3176.81 & 3131.98 & 112.02 & 279.23 & 0.406\\
2020 & 3179.52 & 3148.18 & 163.48 & 283.32 & 0.412\\
2021 & 3110.47 & 3087.05 & 317.57 & 278.75 & 0.405\\
2022 & 3108.97 & 3077.18 & 302.69 & 279.64 & 0.407\\
2023 & 3199.84 & 3163.70 & 299.87 & 289.74 & 0.421\\*
\end{longtable}
\endgroup{}
\endgroup{}

\newpage

\begingroup\fontsize{9}{11}\selectfont

\begin{landscape}\begingroup\fontsize{9}{11}\selectfont

\begin{longtable}[t]{l>{\centering\arraybackslash}p{0.52cm}>{\centering\arraybackslash}p{0.52cm}>{\centering\arraybackslash}p{0.52cm}>{\centering\arraybackslash}p{0.52cm}>{\centering\arraybackslash}p{0.52cm}>{\centering\arraybackslash}p{0.52cm}>{\centering\arraybackslash}p{0.52cm}>{\centering\arraybackslash}p{0.52cm}>{\centering\arraybackslash}p{0.52cm}>{\centering\arraybackslash}p{0.52cm}>{\centering\arraybackslash}p{0.52cm}>{\centering\arraybackslash}p{0.52cm}>{\centering\arraybackslash}p{0.52cm}>{\centering\arraybackslash}p{0.52cm}>{\centering\arraybackslash}p{0.52cm}>{\centering\arraybackslash}p{0.52cm}>{\centering\arraybackslash}p{0.52cm}>{\centering\arraybackslash}p{0.52cm}>{\centering\arraybackslash}p{0.52cm}>{\centering\arraybackslash}p{0.52cm}c}
\caption{\label{tab:sensitivities-1}Sensitivities relative to the base model.}\\
\toprule
  & Base Model & Estimate M & Estimate h & Estimate M \& h & No Rec. Devs. & CPFV Selectivity Asym. & Dirichlet DW & McAllister-Ianelli DW & Reduce PR Catch 1970-82 & Hist. CPFV Ages Lambda = 1 & Rm. All Ages & Rm. Coop. Ages & Rm. WCGBT Ages & Rm. CPFV \& PR Indices & Rm. CCFRP & Rm. CDFW ROV & Rm. NWFSC HKL All & Rm. NWFSC HKL Ages & Rm. NWFSC HKL Lens. \& Ages & Rm. NWFSC HKL Index & Rm. All Surveys\\
\midrule
\endfirsthead
\caption[]{Sensitivities relative to the base model. \textit{(continued)}}\\
\toprule
  & Base Model & Estimate M & Estimate h & Estimate M \& h & No Rec. Devs. & CPFV Selectivity Asym. & Dirichlet DW & McAllister-Ianelli DW & Reduce PR Catch 1970-82 & Hist. CPFV Ages Lambda = 1 & Rm. All Ages & Rm. Coop. Ages & Rm. WCGBT Ages & Rm. CPFV \& PR Indices & Rm. CCFRP & Rm. CDFW ROV & Rm. NWFSC HKL All & Rm. NWFSC HKL Ages & Rm. NWFSC HKL Lens. \& Ages & Rm. NWFSC HKL Index & Rm. All Surveys\\
\midrule
\endhead

\endfoot
\bottomrule
\endlastfoot
Total Likelihood & 2874.130 & 2872.980 & 2874.030 & 2872.630 & 3193.270 & 2913.360 & 8423.880 & 2357.390 & 2872.950 & 2881.460 & 524.917 & 2780.760 & 1973.940 & 2904.170 & 2806.200 & 2854.980 & 1458.470 & 1570.030 & 1416.450 & 2877.970 & 1337.270\\
Survey Likelihood & -35.823 & -35.975 & -35.860 & -35.928 & -19.256 & -33.891 & -34.281 & -34.251 & -35.899 & -35.884 & -33.379 & -35.783 & -34.409 & -5.531 & -32.358 & -33.977 & -31.200 & -35.049 & -34.787 & -31.817 & -25.995\\
Length Likelihood & 576.193 & 576.341 & 576.382 & 575.535 & 696.910 & 587.355 & 4693.580 & 890.263 & 577.070 & 576.796 & 560.119 & 578.846 & 559.550 & 575.230 & 566.861 & 555.932 & 429.658 & 568.526 & 418.326 & 575.864 & 388.022\\
Age Likelihood & 2328.270 & 2326.550 & 2327.860 & 2327.140 & 2515.280 & 2355.450 & 3715.870 & 1496.880 & 2326.660 & 2332.900 & 0.000 & 2233.940 & 1446.140 & 2329.900 & 2267.050 & 2327.050 & 1058.350 & 1034.180 & 1031.170 & 2329.280 & 974.333\\
Recruitment Likelihood & 4.846 & 5.373 & 4.952 & 5.035 & 0.000 & 3.856 & 10.881 & 3.588 & 4.480 & 7.009 & -2.384 & 3.190 & 1.852 & 3.900 & 4.639 & 5.338 & 0.946 & 1.727 & 1.139 & 3.968 & 0.906\\
Forecast Recruitment Likelihood & 0.000 & 0.000 & 0.000 & 0.000 & 0.000 & 0.000 & 0.000 & 0.000 & 0.000 & 0.000 & 0.000 & 0.000 & 0.000 & 0.000 & 0.000 & 0.000 & 0.000 & 0.000 & 0.000 & 0.000 & 0.000\\
Parameter Priors Likelihood & 0.635 & 0.685 & 0.686 & 0.841 & 0.323 & 0.591 & 37.774 & 0.906 & 0.640 & 0.631 & 0.557 & 0.561 & 0.805 & 0.668 & 0.000 & 0.635 & 0.712 & 0.649 & 0.603 & 0.672 & 0.000\\
log(R0) & 5.485 & 5.349 & 5.514 & 5.236 & 5.544 & 5.422 & 5.503 & 5.507 & 5.305 & 5.439 & 5.515 & 5.480 & 5.521 & 5.497 & 5.479 & 5.482 & 5.576 & 5.539 & 5.539 & 5.498 & 5.545\\
SB Virgin & 201.347 & 219.421 & 207.254 & 219.858 & 212.294 & 183.818 & 195.866 & 205.963 & 168.291 & 192.578 & 207.648 & 204.542 & 204.771 & 203.863 & 201.723 & 200.964 & 213.229 & 203.080 & 207.071 & 204.174 & 211.980\\
SB 2023 & 27.584 & 23.620 & 26.848 & 23.797 & 44.151 & 27.792 & 44.565 & 38.217 & 26.684 & 27.380 & 26.787 & 25.960 & 37.522 & 34.602 & 24.759 & 27.629 & 59.084 & 36.196 & 39.686 & 35.324 & 46.951\\
Fraction Unfished 2023 & 0.137 & 0.108 & 0.130 & 0.108 & 0.208 & 0.151 & 0.228 & 0.186 & 0.159 & 0.142 & 0.129 & 0.127 & 0.183 & 0.170 & 0.123 & 0.137 & 0.277 & 0.178 & 0.192 & 0.173 & 0.221\\
Total Yield - SPR 50 & 47.903 & 46.426 & 48.300 & 44.639 & 52.335 & 48.355 & 46.719 & 48.758 & 39.961 & 45.632 & 51.467 & 48.425 & 49.295 & 48.476 & 47.883 & 47.808 & 50.962 & 49.079 & 49.602 & 48.475 & 50.086\\
Steepness & 0.720 & 0.720 & 0.691 & 0.783 & 0.720 & 0.720 & 0.720 & 0.720 & 0.720 & 0.720 & 0.720 & 0.720 & 0.720 & 0.720 & 0.720 & 0.720 & 0.720 & 0.720 & 0.720 & 0.720 & 0.720\\
Natural Mortality - Female & 0.108 & 0.096 & 0.108 & 0.090 & 0.108 & 0.108 & 0.108 & 0.108 & 0.108 & 0.108 & 0.108 & 0.108 & 0.108 & 0.108 & 0.108 & 0.108 & 0.108 & 0.108 & 0.108 & 0.108 & 0.108\\
Length at Amin - Female & 15.449 & 15.451 & 15.447 & 15.497 & 14.745 & 16.184 & 15.888 & 15.934 & 15.427 & 15.444 & 15.449 & 15.390 & 17.336 & 15.461 & 15.552 & 15.435 & 14.046 & 15.903 & 14.123 & 15.447 & 14.414\\
Length at Amax - Female & 46.829 & 46.806 & 46.827 & 46.792 & 46.431 & 46.602 & 46.109 & 46.778 & 46.827 & 46.856 & 46.829 & 47.123 & 46.373 & 46.816 & 47.004 & 46.828 & 46.409 & 46.471 & 46.570 & 46.814 & 46.989\\
Von Bert. k - Female & 0.191 & 0.191 & 0.191 & 0.191 & 0.209 & 0.183 & 0.192 & 0.190 & 0.192 & 0.191 & 0.191 & 0.191 & 0.189 & 0.192 & 0.187 & 0.192 & 0.198 & 0.177 & 0.195 & 0.192 & 0.184\\
CV young - Female & 0.196 & 0.196 & 0.196 & 0.195 & 0.199 & 0.188 & 0.183 & 0.186 & 0.196 & 0.196 & 0.196 & 0.196 & 0.149 & 0.196 & 0.197 & 0.196 & 0.190 & 0.222 & 0.192 & 0.196 & 0.197\\
CV old - Female & 0.060 & 0.060 & 0.060 & 0.060 & 0.062 & 0.060 & 0.078 & 0.065 & 0.060 & 0.060 & 0.060 & 0.055 & 0.068 & 0.060 & 0.058 & 0.059 & 0.079 & 0.067 & 0.076 & 0.060 & 0.074\\
Natural Mortality - Male & 0.108 & 0.000 & 0.108 & 0.093 & 0.108 & 0.108 & 0.108 & 0.108 & 0.108 & 0.108 & 0.108 & 0.108 & 0.108 & 0.108 & 0.108 & 0.108 & 0.108 & 0.108 & 0.108 & 0.108 & 0.108\\
Length at Amin - Male & 15.771 & 15.793 & 15.774 & 15.749 & 15.760 & 16.725 & 16.065 & 15.478 & 15.792 & 15.785 & 15.771 & 15.621 & 15.966 & 15.779 & 15.641 & 15.782 & 16.379 & 14.925 & 16.369 & 15.761 & 16.173\\
Length at Amax - Male & 45.410 & 45.384 & 45.405 & 45.386 & 45.433 & 45.138 & 45.283 & 45.167 & 45.415 & 45.452 & 45.410 & 45.625 & 44.660 & 45.391 & 45.556 & 45.423 & 45.199 & 44.799 & 45.386 & 45.388 & 45.821\\
Von Bert. k - Male & 0.215 & 0.215 & 0.215 & 0.215 & 0.218 & 0.203 & 0.210 & 0.224 & 0.215 & 0.214 & 0.215 & 0.216 & 0.235 & 0.216 & 0.216 & 0.215 & 0.197 & 0.222 & 0.194 & 0.216 & 0.195\\
CV young - Male & 0.203 & 0.202 & 0.203 & 0.203 & 0.216 & 0.193 & 0.186 & 0.194 & 0.203 & 0.203 & 0.203 & 0.206 & 0.171 & 0.203 & 0.205 & 0.203 & 0.219 & 0.201 & 0.220 & 0.203 & 0.223\\
CV old - Male & 0.060 & 0.060 & 0.060 & 0.060 & 0.061 & 0.060 & 0.080 & 0.066 & 0.060 & 0.060 & 0.060 & 0.055 & 0.071 & 0.060 & 0.055 & 0.060 & 0.068 & 0.068 & 0.065 & 0.060 & 0.056\\*
\end{longtable}
\endgroup{}
\end{landscape}
\endgroup{}


\newpage

\begingroup\fontsize{10}{12}\selectfont
\begingroup\fontsize{10}{12}\selectfont

\begin{longtable}[t]{l>{\raggedright\arraybackslash}p{2.2cm}>{\raggedright\arraybackslash}p{2.2cm}>{\raggedright\arraybackslash}p{2.2cm}>{\raggedright\arraybackslash}p{2.2cm}}
\caption{\label{tab:dw}Suggested data weights for each data weighting methodology by fleet and data type.}\\
\toprule
Fleet & Data Type & Francis & MI & Dirichlet\\
\midrule
\endfirsthead
\caption[]{\label{tab:dw}Suggested data weights for each data weighting methodology by fleet and data type. \textit{(continued)}}\\
\toprule
Fleet & Data Type & Francis & MI & Dirichlet\\
\midrule
\endhead

\endfoot
\bottomrule
\endlastfoot
Commercial Dead & Lengths & 0.20 & 0.23 & 0.97\\
Commercial Live & Lengths & 0.29 & 0.58 & 0.99\\
CPFV & Lengths & 0.33 & 0.24 & 0.99\\
PR & Lengths & 0.45 & 0.42 & 0.99\\
CCFRP & Lengths & 0.49 & 1.28 & 0.99\\
CDFW ROV & Lengths & 0.24 & 1.49 & 0.96\\
Commercial Dead & Ages & 0.17 & 0.60 & 0.99\\
CPFV & Ages & 0.21 & 0.21 & 0.99\\
PR & Ages & 0.56 & 0.56 & 0.99\\
CCFRP & Ages & 0.53 & 0.76 & 0.99\\
Growth & Ages & 0.40 & 0.47 & 0.99\\*
\end{longtable}
\endgroup{}
\endgroup{}

\begingroup\fontsize{9}{11}\selectfont

\begin{landscape}\begingroup\fontsize{9}{11}\selectfont

\begin{longtable}[t]{l>{\centering\arraybackslash}p{1.83cm}>{\centering\arraybackslash}p{1.83cm}>{\centering\arraybackslash}p{1.83cm}>{\centering\arraybackslash}p{1.83cm}>{\centering\arraybackslash}p{1.83cm}c}
\caption{\label{tab:sensitivities-2}Sensitivities relative to the base model.}\\
\toprule
  & Base Model & Reduce Rec. Catch 1970-82 & Add Hist. CPFV Ages to Growth & Lengths Only & Rm. Coop. Ages & Rm. All Ages\\
\midrule
\endfirsthead
\caption[]{Sensitivities relative to the base model. \textit{(continued)}}\\
\toprule
  & Base Model & Reduce Rec. Catch 1970-82 & Add Hist. CPFV Ages to Growth & Lengths Only & Rm. Coop. Ages & Rm. All Ages\\
\midrule
\endhead

\endfoot
\bottomrule
\endlastfoot
Total Likelihood & 1016.640 & 1011.040 & 1173.970 & 370.637 & 986.507 & 369.756\\
Survey Likelihood & -40.986 & -42.563 & -40.608 & 0.000 & -40.908 & -41.635\\
Length Likelihood & 413.455 & 413.094 & 415.038 & 365.665 & 413.595 & 406.900\\
Age Likelihood & 638.670 & 639.129 & 794.919 & 0.000 & 608.022 & 0.000\\
Recruitment Likelihood & 5.496 & 1.371 & 4.617 & 4.968 & 5.793 & 4.486\\
Forecast Recruitment Likelihood & 0.000 & 0.000 & 0.000 & 0.000 & 0.000 & 0.000\\
Parameter Priors Likelihood & 0.002 & 0.002 & 0.002 & 0.002 & 0.002 & 0.002\\
log(R0) & 6.342 & 6.061 & 6.297 & 6.328 & 6.340 & 6.292\\
SB Virgin & 486.149 & 366.975 & 476.660 & 479.608 & 485.185 & 462.345\\
SB 2023 & 262.100 & 162.701 & 252.657 & 242.204 & 260.602 & 200.255\\
Fraction Unfished 2023 & 0.539 & 0.443 & 0.530 & 0.505 & 0.537 & 0.433\\
Total Yield - SPR 50 & 124.049 & 93.956 & 120.774 & 121.859 & 124.030 & 118.046\\
Steepness & 0.720 & 0.720 & 0.720 & 0.720 & 0.720 & 0.720\\
Natural Mortality - Female & 0.108 & 0.108 & 0.108 & 0.108 & 0.108 & 0.108\\
Length at Amin - Female & 14.583 & 14.583 & 14.583 & 14.583 & 14.583 & 14.583\\
Length at Amax - Female & 48.307 & 48.296 & 48.496 & 48.307 & 48.286 & 48.307\\
Von Bert. k - Female & 0.154 & 0.154 & 0.157 & 0.154 & 0.154 & 0.154\\
CV young - Female & 0.157 & 0.157 & 0.162 & 0.157 & 0.159 & 0.157\\
CV old - Female & 0.074 & 0.074 & 0.075 & 0.074 & 0.074 & 0.074\\
Natural Mortality - Male & 0.108 & 0.108 & 0.108 & 0.108 & 0.108 & 0.108\\
Length at Amin - Male & 12.637 & 12.637 & 12.637 & 12.637 & 12.637 & 12.637\\
Length at Amax - Male & 46.488 & 46.514 & 46.700 & 46.488 & 46.436 & 46.488\\
Von Bert. k - Male & 0.195 & 0.195 & 0.194 & 0.195 & 0.197 & 0.195\\
CV young - Male & 0.157 & 0.156 & 0.182 & 0.157 & 0.162 & 0.157\\
CV old - Male & 0.072 & 0.073 & 0.070 & 0.072 & 0.073 & 0.072\\*
\end{longtable}
\endgroup{}
\end{landscape}
\endgroup{}


\newpage

\begingroup\fontsize{9}{11}\selectfont

\begin{landscape}\begingroup\fontsize{9}{11}\selectfont

\begin{longtable}[t]{l>{\centering\arraybackslash}p{1.22cm}>{\centering\arraybackslash}p{1.22cm}>{\centering\arraybackslash}p{1.22cm}>{\centering\arraybackslash}p{1.22cm}>{\centering\arraybackslash}p{1.22cm}>{\centering\arraybackslash}p{1.22cm}>{\centering\arraybackslash}p{1.22cm}>{\centering\arraybackslash}p{1.22cm}>{\centering\arraybackslash}p{1.22cm}}
\caption{\label{tab:sensitivities-3}Sensitivities relative to the base model.}\\
\toprule
  & Base Model & Rm. CCFRP & Add CDFW ROV Survey & Rm. NWFSC HKL All & Rm. NWFSC HKL Ages & Rm. NWFSC HKL Lens. \& Ages & Rm. NWFSC HKL Index & Move NWFSC HKL Data Before 2014 & Rm. All Surveys\\
\midrule
\endfirsthead
\caption[]{Sensitivities relative to the base model. \textit{(continued)}}\\
\toprule
  & Base Model & Rm. CCFRP & Add CDFW ROV Survey & Rm. NWFSC HKL All & Rm. NWFSC HKL Ages & Rm. NWFSC HKL Lens. \& Ages & Rm. NWFSC HKL Index & Move NWFSC HKL Data Before 2014 & Rm. All Surveys\\
\midrule
\endhead

\endfoot
\bottomrule
\endlastfoot
Total Likelihood & 2832.680 & 2763.640 & 2830.550 & 1379.710 & 1529.870 & 1377.270 & 2836.050 & 2718.710 & 1315.760\\
Survey Likelihood & -33.416 & -29.661 & -35.632 & -29.493 & -32.653 & -32.401 & -29.806 & -33.786 & -25.732\\
Length Likelihood & 540.049 & 530.746 & 540.192 & 383.346 & 532.298 & 383.718 & 539.694 & 721.904 & 373.522\\
Age Likelihood & 2319.930 & 2257.490 & 2319.670 & 1023.930 & 1027.120 & 1023.450 & 2321.010 & 2025.620 & 966.632\\
Recruitment Likelihood & 5.415 & 5.058 & 5.617 & 1.190 & 2.367 & 1.812 & 4.404 & 4.366 & 1.331\\
Forecast Recruitment Likelihood & 0.000 & 0.000 & 0.000 & 0.000 & 0.000 & 0.000 & 0.000 & 0.000 & 0.000\\
Parameter Priors Likelihood & 0.699 & 0.000 & 0.700 & 0.733 & 0.741 & 0.692 & 0.743 & 0.601 & 0.000\\
log(R0) & 5.486 & 5.481 & 5.483 & 5.576 & 5.547 & 5.548 & 5.503 & 5.494 & 5.555\\
SO Virgin & 201.064 & 201.628 & 200.611 & 216.348 & 205.104 & 210.422 & 204.603 & 195.923 & 214.888\\
SO 2023 & 32.063 & 29.328 & 30.791 & 66.186 & 44.561 & 49.663 & 42.180 & 25.415 & 57.169\\
Fraction Unfished 2023 & 0.159 & 0.145 & 0.153 & 0.306 & 0.217 & 0.236 & 0.206 & 0.130 & 0.266\\
Total Yield - SPR 50 & 47.777 & 47.784 & 47.682 & 51.185 & 49.253 & 49.874 & 48.504 & 48.104 & 50.412\\
Steepness & 0.720 & 0.720 & 0.720 & 0.720 & 0.720 & 0.720 & 0.720 & 0.720 & 0.720\\
Natural Mortality - Female & 0.108 & 0.108 & 0.108 & 0.108 & 0.108 & 0.108 & 0.108 & 0.108 & 0.108\\
Length at Amin - Female & 15.349 & 15.459 & 15.347 & 13.716 & 15.793 & 13.776 & 15.347 & 15.839 & 14.088\\
Length at Amax - Female & 46.829 & 47.009 & 46.833 & 46.569 & 46.486 & 46.622 & 46.810 & 46.429 & 46.986\\
Von Bert. k - Female & 0.194 & 0.190 & 0.194 & 0.204 & 0.182 & 0.202 & 0.195 & 0.191 & 0.192\\
CV young - Female & 0.197 & 0.198 & 0.197 & 0.191 & 0.224 & 0.191 & 0.197 & 0.197 & 0.195\\
CV old - Female & 0.059 & 0.058 & 0.059 & 0.076 & 0.066 & 0.076 & 0.059 & 0.061 & 0.074\\
Natural Mortality - Male & 0.108 & 0.108 & 0.108 & 0.108 & 0.108 & 0.108 & 0.108 & 0.108 & 0.108\\
Length at Amin - Male & 15.722 & 15.580 & 15.722 & 16.280 & 14.724 & 16.285 & 15.709 & 15.438 & 16.109\\
Length at Amax - Male & 45.415 & 45.567 & 45.421 & 45.340 & 44.751 & 45.393 & 45.388 & 45.167 & 45.824\\
Von Bert. k - Male & 0.218 & 0.219 & 0.217 & 0.200 & 0.229 & 0.198 & 0.219 & 0.220 & 0.200\\
CV young - Male & 0.203 & 0.206 & 0.203 & 0.222 & 0.200 & 0.222 & 0.203 & 0.203 & 0.225\\
CV old - Male & 0.060 & 0.055 & 0.060 & 0.065 & 0.068 & 0.065 & 0.060 & 0.059 & 0.055\\*
\end{longtable}
\endgroup{}
\end{landscape}
\endgroup{}


\newpage

\begingroup\fontsize{10}{12}\selectfont
\begingroup\fontsize{10}{12}\selectfont

\begin{longtable}[t]{r>{\centering\arraybackslash}p{2cm}>{\centering\arraybackslash}p{2cm}>{\centering\arraybackslash}p{2cm}}
	\caption{\label{tab:north-referenceES}Summary of reference points and management quantities, including estimates of the 95 percent intervals for the sub-area model north of Point Conception.}\\
	\toprule
	& Estimate & Lower Interval & Upper Interval\\
	\midrule
	\endfirsthead
	\caption[]{Summary of reference points and management quantities, including estimates of the 95 percent intervals for the sub-area model north of Point Conception. \textit{(continued)}}\\
	\toprule
	& Estimate & Lower Interval & Upper Interval\\
	\midrule
	\endhead
	
	\endfoot
	\bottomrule
	\endlastfoot
	Unfished Spawning Output & 486.15 & 387.43 & 584.87\\
	Unfished Age 3+ Biomass (mt) & 4719.91 & 3777.92 & 5661.90\\
	Unfished Recruitment (R0) & 567.77 & 452.48 & 683.06\\
	Spawning Output (2023) & 262.10 & 124.28 & 399.92\\
	Fraction Unfished (2023) & 0.54 & 0.32 & 0.76\\
	Reference Points Based SB40\% &  &  & \\
	Proxy Spawning Output SB40\% & 194.46 & 154.97 & 233.95\\
	SPR Resulting in SB40\% & 0.46 & 0.46 & 0.46\\
	Exploitation Rate Resulting in SB40\% & 0.06 & 0.06 & 0.06\\
	Yield with SPR Based On SB40\% (mt) & 129.86 & 104.05 & 155.67\\
	Reference Points Based on SPR Proxy for MSY & NA & NA & NA\\
	Proxy Spawning Output (SPR50) & 216.90 & 172.85 & 260.94\\
	SPR50 & 0.50 &  & \\
	Exploitation Rate Corresponding to SPR50 & 0.05 & 0.05 & 0.05\\
	Yield with SPR50 at SB SPR (mt) & 124.05 & 99.39 & 148.71\\
	Reference Points Based on Estimated MSY Values & NA & NA & NA\\
	Spawning Output at MSY (SB MSY) & 134.17 & 106.84 & 161.51\\
	SPR MSY & 0.35 & 0.34 & 0.35\\
	Exploitation Rate Corresponding to SPR MSY & 0.09 & 0.08 & 0.09\\
	MSY (mt) & 137.59 & 110.25 & 164.92\\*
\end{longtable}
\endgroup{}
\endgroup{}


\newpage

\begingroup\fontsize{10}{12}\selectfont

\begin{landscape}\begingroup\fontsize{10}{12}\selectfont

\begin{longtable}[t]{c>{\centering\arraybackslash}p{1.38cm}>{\centering\arraybackslash}p{1.38cm}>{\centering\arraybackslash}p{1.38cm}>{\centering\arraybackslash}p{1.38cm}>{\centering\arraybackslash}p{1.38cm}>{\centering\arraybackslash}p{1.38cm}>{\centering\arraybackslash}p{1.38cm}}
\caption{\label{tab:ca-proj}The estimated spawning output in number of million eggs across California and fraction unfished by year.}\\
\toprule
Year & Adopted OFL (mt) & Adopted ABC (mt) & Assumed Catch (mt) & OFL (mt) & ABC (mt) & Spawning Biomass & Fraction Unfished\\
\midrule
\endfirsthead
\caption[]{\label{tab:ca-proj}The estimated spawning output in number of million eggs across California and fraction unfished by year. \textit{(continued)}}\\
\toprule
Year & Adopted OFL (mt) & Adopted ABC (mt) & Assumed Catch (mt) & OFL (mt) & ABC (mt) & Spawning Biomass & Fraction Unfished\\
\midrule
\endhead

\endfoot
\bottomrule
\endlastfoot
2023 & 116.4 & 91.53 & 70 & - & - & 289.74 & 0.421\\
2024 & 121.32 & 94.69 & 70 & - & - & 295.18 & 0.429\\
2025 & - & - & - & 171.29 & 148.79 & 302.30 & 0.440\\
2026 & - & - & - & 171.74 & 147.12 & 303.82 & 0.442\\
2027 & - & - & - & 172.05 & 145.66 & 304.90 & 0.443\\
2028 & - & - & - & 172.2 & 144 & 305.69 & 0.444\\
2029 & - & - & - & 172.26 & 141.82 & 306.33 & 0.445\\
2030 & - & - & - & 172.31 & 140.09 & 306.97 & 0.446\\
2031 & - & - & - & 172.39 & 138.4 & 307.60 & 0.447\\
2032 & - & - & - & 172.54 & 136.34 & 308.29 & 0.448\\
2033 & - & - & - & 172.8 & 134.85 & 309.09 & 0.449\\
2034 & - & - & - & 173.15 & 143.44 & 309.99 & 0.451\\*
\end{longtable}
\endgroup{}
\end{landscape}
\endgroup{}

\begingroup\fontsize{10}{12}\selectfont
\begingroup\fontsize{10}{12}\selectfont

\begin{longtable}[t]{r>{\centering\arraybackslash}p{1.22cm}>{\centering\arraybackslash}p{1.22cm}>{\centering\arraybackslash}p{1.22cm}>{\centering\arraybackslash}p{1.22cm}>{\centering\arraybackslash}p{1.22cm}>{\centering\arraybackslash}p{1.22cm}>{\centering\arraybackslash}p{1.22cm}>{\centering\arraybackslash}p{1.22cm}}
\caption{\label{tab:north-timeseries}Time series of population estimates from the base model for the sub-area north of Point Conception.}\\
\toprule
Year & Total Biomass (mt) & Spawning Output & Total Biomass 3+ (mt) & Fraction Unfished & Age-0 Recruits & Total Mortality (mt) & 1-SPR & Exploitation Rate\\
\midrule
\endfirsthead
\caption[]{Time series of population estimates from the base model for the sub-area north of Point Conception. \textit{(continued)}}\\
\toprule
Year & Total Biomass (mt) & Spawning Output & Total Biomass 3+ (mt) & Fraction Unfished & Age-0 Recruits & Total Mortality (mt) & 1-SPR & Exploitation Rate\\
\midrule
\endhead

\endfoot
\bottomrule
\endlastfoot
1916 & 4758.75 & 487.03 & 4731.64 & 1.00 & 573.48 & 3.97 & 0.01 & 0.00\\
1917 & 4756.53 & 486.73 & 4729.41 & 1.00 & 573.92 & 6.23 & 0.02 & 0.00\\
1918 & 4752.47 & 486.22 & 4725.31 & 1.00 & 574.38 & 7.50 & 0.02 & 0.00\\
1919 & 4747.63 & 485.61 & 4720.46 & 1.00 & 574.87 & 4.91 & 0.01 & 0.00\\
1920 & 4745.78 & 485.32 & 4718.58 & 1.00 & 575.44 & 5.06 & 0.01 & 0.00\\
1921 & 4744.18 & 485.05 & 4716.96 & 1.00 & 576.07 & 4.31 & 0.01 & 0.00\\
1922 & 4743.72 & 484.91 & 4716.47 & 1.00 & 576.76 & 3.70 & 0.01 & 0.00\\
1923 & 4744.21 & 484.87 & 4716.93 & 1.00 & 577.52 & 3.89 & 0.01 & 0.00\\
1924 & 4744.88 & 484.85 & 4717.56 & 1.00 & 578.34 & 2.57 & 0.01 & 0.00\\
1925 & 4747.14 & 485.02 & 4719.79 & 1.00 & 579.26 & 3.84 & 0.01 & 0.00\\
1926 & 4748.48 & 485.07 & 4721.09 & 1.00 & 580.23 & 4.89 & 0.01 & 0.00\\
1927 & 4749.17 & 485.05 & 4721.73 & 1.00 & 581.27 & 3.64 & 0.01 & 0.00\\
1928 & 4751.45 & 485.20 & 4723.97 & 1.00 & 582.41 & 5.21 & 0.02 & 0.00\\
1929 & 4752.50 & 485.22 & 4724.97 & 1.00 & 583.84 & 6.18 & 0.02 & 0.00\\
1930 & 4752.95 & 485.18 & 4725.36 & 1.00 & 585.39 & 8.96 & 0.03 & 0.00\\
1931 & 4751.19 & 484.88 & 4723.53 & 1.00 & 587.04 & 11.15 & 0.03 & 0.00\\
1932 & 4747.92 & 484.39 & 4720.18 & 1.00 & 588.84 & 11.73 & 0.04 & 0.00\\
1933 & 4744.80 & 483.91 & 4716.98 & 1.00 & 590.78 & 12.11 & 0.04 & 0.00\\
1934 & 4742.11 & 483.45 & 4714.21 & 0.99 & 592.87 & 12.01 & 0.04 & 0.00\\
1935 & 4740.36 & 483.08 & 4712.37 & 0.99 & 595.19 & 15.35 & 0.05 & 0.00\\
1936 & 4736.34 & 482.44 & 4708.24 & 0.99 & 597.65 & 16.09 & 0.05 & 0.00\\
1937 & 4732.71 & 481.81 & 4704.51 & 0.99 & 600.32 & 18.78 & 0.06 & 0.00\\
1938 & 4727.67 & 481.01 & 4699.34 & 0.99 & 603.21 & 17.93 & 0.06 & 0.00\\
1939 & 4724.86 & 480.41 & 4696.41 & 0.99 & 606.46 & 16.05 & 0.05 & 0.00\\
1940 & 4725.39 & 480.14 & 4696.79 & 0.99 & 610.18 & 20.78 & 0.06 & 0.00\\
1941 & 4722.63 & 479.52 & 4693.88 & 0.99 & 614.38 & 19.96 & 0.06 & 0.00\\
1942 & 4722.37 & 479.12 & 4693.43 & 0.99 & 619.23 & 9.64 & 0.03 & 0.00\\
1943 & 4734.07 & 479.95 & 4704.93 & 0.99 & 624.88 & 10.40 & 0.03 & 0.00\\
1944 & 4746.61 & 480.85 & 4717.22 & 0.99 & 631.13 & 14.81 & 0.04 & 0.00\\
1945 & 4756.67 & 481.43 & 4727.01 & 0.99 & 637.77 & 29.60 & 0.08 & 0.01\\
1946 & 4754.55 & 480.61 & 4724.59 & 0.99 & 644.39 & 38.03 & 0.11 & 0.01\\
1947 & 4746.83 & 479.14 & 4716.56 & 0.99 & 650.70 & 18.34 & 0.06 & 0.00\\
1948 & 4761.18 & 480.03 & 4730.59 & 0.99 & 656.79 & 31.93 & 0.10 & 0.01\\
1949 & 4764.49 & 479.75 & 4733.61 & 0.99 & 661.72 & 34.12 & 0.10 & 0.01\\
1950 & 4768.23 & 479.52 & 4737.08 & 0.99 & 665.42 & 39.31 & 0.12 & 0.01\\
1951 & 4769.54 & 479.03 & 4738.17 & 0.99 & 667.86 & 53.91 & 0.16 & 0.01\\
1952 & 4759.34 & 477.30 & 4727.81 & 0.98 & 669.38 & 45.06 & 0.13 & 0.01\\
1953 & 4761.01 & 476.77 & 4729.37 & 0.98 & 671.53 & 36.30 & 0.11 & 0.01\\
1954 & 4774.05 & 477.47 & 4742.33 & 0.98 & 675.16 & 46.94 & 0.14 & 0.01\\
1955 & 4778.65 & 477.39 & 4746.82 & 0.98 & 679.63 & 52.40 & 0.16 & 0.01\\
1956 & 4779.65 & 477.06 & 4747.63 & 0.98 & 686.18 & 60.11 & 0.17 & 0.01\\
1957 & 4774.87 & 476.17 & 4742.63 & 0.98 & 683.59 & 58.27 & 0.17 & 0.01\\
1958 & 4774.03 & 475.66 & 4741.58 & 0.98 & 682.24 & 99.22 & 0.27 & 0.02\\
1959 & 4733.91 & 471.20 & 4701.56 & 0.97 & 684.64 & 80.28 & 0.23 & 0.02\\
1960 & 4715.50 & 468.80 & 4683.16 & 0.96 & 699.37 & 68.30 & 0.20 & 0.01\\
1961 & 4711.74 & 467.80 & 4679.12 & 0.96 & 732.19 & 51.06 & 0.15 & 0.01\\
1962 & 4727.91 & 468.84 & 4694.37 & 0.96 & 777.94 & 63.56 & 0.19 & 0.01\\
1963 & 4734.77 & 468.93 & 4699.55 & 0.96 & 817.20 & 79.29 & 0.23 & 0.02\\
1964 & 4731.24 & 467.64 & 4693.89 & 0.96 & 881.00 & 70.73 & 0.21 & 0.02\\
1965 & 4744.87 & 467.41 & 4705.34 & 0.96 & 983.81 & 104.92 & 0.29 & 0.02\\
1966 & 4736.12 & 464.09 & 4693.15 & 0.95 & 1067.20 & 121.03 & 0.32 & 0.03\\
1967 & 4728.19 & 459.74 & 4680.69 & 0.95 & 1051.05 & 128.37 & 0.34 & 0.03\\
1968 & 4734.70 & 455.64 & 4684.53 & 0.94 & 923.06 & 135.98 & 0.35 & 0.03\\
1969 & 4756.61 & 452.27 & 4708.54 & 0.93 & 741.98 & 145.84 & 0.37 & 0.03\\
1970 & 4786.46 & 449.99 & 4744.98 & 0.93 & 608.42 & 181.20 & 0.43 & 0.04\\
1971 & 4786.28 & 446.80 & 4752.87 & 0.92 & 455.26 & 169.15 & 0.41 & 0.04\\
1972 & 4788.22 & 447.24 & 4761.28 & 0.92 & 357.88 & 215.21 & 0.48 & 0.05\\
1973 & 4718.98 & 444.91 & 4698.51 & 0.92 & 373.18 & 246.30 & 0.52 & 0.05\\
1974 & 4582.63 & 439.32 & 4565.55 & 0.90 & 350.26 & 270.39 & 0.56 & 0.06\\
1975 & 4383.13 & 428.95 & 4365.71 & 0.88 & 365.48 & 267.98 & 0.58 & 0.06\\
1976 & 4150.72 & 414.85 & 4134.02 & 0.85 & 332.35 & 296.45 & 0.63 & 0.07\\
1977 & 3862.72 & 393.09 & 3845.83 & 0.81 & 306.02 & 306.11 & 0.67 & 0.08\\
1978 & 3547.11 & 365.81 & 3531.76 & 0.75 & 239.16 & 281.47 & 0.68 & 0.08\\
1979 & 3245.65 & 337.69 & 3231.98 & 0.69 & 192.93 & 292.80 & 0.72 & 0.09\\
1980 & 2926.86 & 306.26 & 2916.03 & 0.63 & 222.56 & 305.18 & 0.75 & 0.10\\
1981 & 2594.11 & 272.27 & 2584.49 & 0.56 & 344.88 & 287.77 & 0.77 & 0.11\\
1982 & 2277.63 & 239.84 & 2265.84 & 0.49 & 243.16 & 284.03 & 0.80 & 0.13\\
1983 & 1971.16 & 207.66 & 1955.99 & 0.43 & 243.06 & 209.77 & 0.73 & 0.11\\
1984 & 1754.30 & 183.00 & 1742.51 & 0.38 & 462.12 & 185.11 & 0.74 & 0.11\\
1985 & 1576.13 & 161.71 & 1562.04 & 0.33 & 551.70 & 206.53 & 0.81 & 0.13\\
1986 & 1392.78 & 139.51 & 1370.20 & 0.29 & 314.58 & 164.96 & 0.79 & 0.12\\
1987 & 1271.95 & 122.97 & 1248.57 & 0.25 & 274.54 & 100.25 & 0.66 & 0.08\\
1988 & 1238.68 & 114.26 & 1224.21 & 0.24 & 299.29 & 112.68 & 0.70 & 0.09\\
1989 & 1207.29 & 106.33 & 1193.84 & 0.22 & 441.70 & 122.44 & 0.72 & 0.10\\
1990 & 1174.70 & 99.83 & 1158.82 & 0.21 & 518.43 & 133.18 & 0.75 & 0.11\\
1991 & 1138.25 & 94.81 & 1116.68 & 0.20 & 361.75 & 131.79 & 0.75 & 0.12\\
1992 & 1112.37 & 90.96 & 1089.66 & 0.19 & 316.48 & 142.04 & 0.77 & 0.13\\
1993 & 1087.78 & 86.39 & 1070.99 & 0.18 & 462.51 & 162.74 & 0.82 & 0.15\\
1994 & 1048.63 & 80.67 & 1032.13 & 0.17 & 361.77 & 93.49 & 0.68 & 0.09\\
1995 & 1079.12 & 82.08 & 1058.52 & 0.17 & 247.39 & 76.06 & 0.60 & 0.07\\
1996 & 1130.71 & 86.01 & 1114.85 & 0.18 & 272.30 & 83.98 & 0.61 & 0.08\\
1997 & 1174.03 & 90.16 & 1161.98 & 0.19 & 315.62 & 80.03 & 0.58 & 0.07\\
1998 & 1215.54 & 95.08 & 1202.13 & 0.20 & 339.49 & 44.90 & 0.40 & 0.04\\
1999 & 1284.31 & 103.40 & 1269.15 & 0.21 & 302.54 & 37.01 & 0.33 & 0.03\\
2000 & 1354.22 & 112.46 & 1338.64 & 0.23 & 251.04 & 31.63 & 0.27 & 0.02\\
2001 & 1423.65 & 121.31 & 1409.88 & 0.25 & 274.09 & 28.95 & 0.25 & 0.02\\
2002 & 1489.36 & 129.50 & 1477.16 & 0.27 & 313.55 & 21.83 & 0.18 & 0.01\\
2003 & 1554.22 & 137.65 & 1540.85 & 0.28 & 268.18 & 23.78 & 0.19 & 0.02\\
2004 & 1609.02 & 145.10 & 1594.74 & 0.30 & 227.39 & 18.92 & 0.15 & 0.01\\
2005 & 1661.84 & 152.32 & 1649.64 & 0.31 & 196.19 & 34.85 & 0.26 & 0.02\\
2006 & 1690.21 & 157.19 & 1679.61 & 0.32 & 346.50 & 36.21 & 0.27 & 0.02\\
2007 & 1709.37 & 161.04 & 1697.69 & 0.33 & 888.98 & 42.04 & 0.30 & 0.02\\
2008 & 1720.32 & 163.55 & 1698.04 & 0.34 & 683.78 & 31.80 & 0.24 & 0.02\\
2009 & 1752.87 & 166.22 & 1712.94 & 0.34 & 819.71 & 39.21 & 0.29 & 0.02\\
2010 & 1806.79 & 167.32 & 1773.45 & 0.34 & 376.08 & 27.47 & 0.21 & 0.02\\
2011 & 1904.77 & 169.37 & 1870.94 & 0.35 & 412.57 & 25.88 & 0.19 & 0.01\\
2012 & 2031.54 & 173.50 & 2012.93 & 0.36 & 721.98 & 34.91 & 0.23 & 0.02\\
2013 & 2160.53 & 180.75 & 2137.65 & 0.37 & 602.74 & 25.64 & 0.16 & 0.01\\
2014 & 2302.33 & 193.08 & 2269.62 & 0.40 & 508.66 & 36.94 & 0.20 & 0.02\\
2015 & 2435.83 & 207.33 & 2408.19 & 0.43 & 644.80 & 65.46 & 0.31 & 0.03\\
2016 & 2538.95 & 219.54 & 2513.77 & 0.45 & 304.21 & 66.32 & 0.31 & 0.03\\
2017 & 2635.51 & 230.78 & 2608.04 & 0.47 & 932.00 & 138.62 & 0.50 & 0.05\\
2018 & 2657.53 & 234.32 & 2636.38 & 0.48 & 649.02 & 102.38 & 0.41 & 0.04\\
2019 & 2709.53 & 240.87 & 2668.99 & 0.50 & 358.62 & 101.89 & 0.40 & 0.04\\
2020 & 2765.07 & 247.00 & 2737.43 & 0.51 & 518.96 & 156.21 & 0.53 & 0.06\\
2021 & 2765.72 & 247.07 & 2746.93 & 0.51 & 518.99 & 106.99 & 0.41 & 0.04\\
2022 & 2807.65 & 251.40 & 2783.08 & 0.52 & 520.55 & 47.15 & 0.21 & 0.02\\
2023 & 2899.87 & 262.10 & 2875.27 & 0.54 & 524.24 & 91.50 & 0.35 & 0.03\\
2024 & 2938.51 & 268.97 & 2913.82 & 0.55 & 526.48 & 66.90 & 0.27 & 0.02\\
2025 & 2993.09 & 277.24 & 2968.25 & 0.57 & 529.05 & 105.37 & 0.38 & 0.04\\
2026 & 3002.22 & 280.38 & 2977.27 & 0.58 & 529.99 & 104.04 & 0.38 & 0.03\\
2027 & 3007.39 & 282.32 & 2982.33 & 0.58 & 530.56 & 102.85 & 0.38 & 0.03\\
2028 & 3010.14 & 283.48 & 2985.04 & 0.58 & 530.90 & 101.52 & 0.38 & 0.03\\
2029 & 3011.84 & 284.18 & 2986.72 & 0.58 & 531.10 & 99.77 & 0.37 & 0.03\\
2030 & 3013.81 & 284.70 & 2988.68 & 0.59 & 531.26 & 98.37 & 0.37 & 0.03\\
2031 & 3016.28 & 285.14 & 2991.13 & 0.59 & 531.38 & 97.00 & 0.36 & 0.03\\
2032 & 3019.57 & 285.60 & 2994.42 & 0.59 & 531.52 & 95.34 & 0.36 & 0.03\\
2033 & 3024.21 & 286.17 & 2999.06 & 0.59 & 531.68 & 94.10 & 0.36 & 0.03\\
2034 & 3029.85 & 286.82 & 3004.69 & 0.59 & 531.87 & 102.94 & 0.38 & 0.03\\*
\end{longtable}
\endgroup{}
\endgroup{}


\newpage

\begingroup\fontsize{10}{12}\selectfont
\begingroup\fontsize{10}{12}\selectfont

\begin{longtable}[t]{c>{\centering\arraybackslash}p{1.83cm}>{\centering\arraybackslash}p{1.83cm}>{\centering\arraybackslash}p{1.83cm}>{\centering\arraybackslash}p{1.83cm}>{\centering\arraybackslash}p{1.83cm}}
\caption{\label{tab:ca-status}The estimated total biomass (mt), total biomass age 3+ (mt), age-0 recruits, spawning ouput in number of million eggs across California and fraction unfished by year.}\\
\toprule
Year & Total Biomass (mt) & Total Biomass 3+ (mt) & Age-0 Recruits & Spawning Output & Fraction Unfished\\
\midrule
\endfirsthead
\caption[]{\label{tab:ca-status}The estimated total biomass (mt), total biomass age 3+ (mt), age-0 recruits, spawning ouput in number of million eggs across California and fraction unfished by year. \textit{(continued)}}\\
\toprule
Year & Total Biomass (mt) & Total Biomass 3+ (mt) & Age-0 Recruits & Spawning Output & Fraction Unfished\\
\midrule
\endhead

\endfoot
\bottomrule
\endlastfoot
1916 & 6771.66 & 6726.78 & 483.56 & 688.74 & 1.001\\
1917 & 6769.45 & 6724.56 & 483.64 & 688.45 & 1.001\\
1918 & 6765.34 & 6720.40 & 483.72 & 687.93 & 1.000\\
1919 & 6760.48 & 6715.54 & 483.82 & 687.31 & 0.999\\
1920 & 6758.72 & 6713.74 & 483.92 & 687.02 & 0.999\\
1921 & 6757.21 & 6712.21 & 484.04 & 686.76 & 0.999\\
1922 & 6756.88 & 6711.84 & 484.16 & 686.62 & 0.998\\
1923 & 6757.52 & 6712.45 & 484.30 & 686.60 & 0.998\\
1924 & 6758.33 & 6713.22 & 484.45 & 686.59 & 0.998\\
1925 & 6760.72 & 6715.57 & 484.61 & 686.76 & 0.999\\
1926 & 6762.19 & 6716.99 & 484.79 & 686.83 & 0.999\\
1927 & 6763.00 & 6717.75 & 484.98 & 686.81 & 0.999\\
1928 & 6765.48 & 6720.18 & 485.19 & 686.97 & 0.999\\
1929 & 6766.75 & 6721.40 & 485.42 & 687.01 & 0.999\\
1930 & 6767.44 & 6722.01 & 485.67 & 686.99 & 0.999\\
1931 & 6765.92 & 6720.42 & 485.95 & 686.70 & 0.998\\
1932 & 6762.93 & 6717.34 & 486.24 & 686.24 & 0.998\\
1933 & 6760.05 & 6714.37 & 486.56 & 685.76 & 0.997\\
1934 & 6757.78 & 6712.00 & 486.92 & 685.34 & 0.996\\
1935 & 6756.39 & 6710.51 & 487.30 & 685.00 & 0.996\\
1936 & 6752.51 & 6706.51 & 487.71 & 684.36 & 0.995\\
1937 & 6749.24 & 6703.12 & 488.15 & 683.76 & 0.994\\
1938 & 6743.91 & 6697.65 & 488.62 & 682.90 & 0.993\\
1939 & 6741.38 & 6694.98 & 489.14 & 682.32 & 0.992\\
1940 & 6742.49 & 6695.93 & 489.72 & 682.08 & 0.992\\
1941 & 6740.38 & 6693.65 & 490.33 & 681.51 & 0.991\\
1942 & 6740.81 & 6693.87 & 491.00 & 681.16 & 0.990\\
1943 & 6753.76 & 6706.59 & 491.72 & 682.09 & 0.992\\
1944 & 6767.59 & 6720.14 & 492.49 & 683.10 & 0.993\\
1945 & 6779.15 & 6731.41 & 493.30 & 683.81 & 0.994\\
1946 & 6778.55 & 6730.48 & 494.16 & 683.12 & 0.993\\
1947 & 6772.42 & 6724.01 & 495.04 & 681.79 & 0.991\\
1948 & 6787.89 & 6739.13 & 495.92 & 682.78 & 0.993\\
1949 & 6791.34 & 6742.26 & 496.75 & 682.50 & 0.992\\
1950 & 6794.80 & 6745.41 & 497.53 & 682.23 & 0.992\\
1951 & 6795.11 & 6745.48 & 498.22 & 681.61 & 0.991\\
1952 & 6781.60 & 6731.78 & 498.76 & 679.48 & 0.988\\
1953 & 6781.57 & 6731.62 & 499.17 & 678.73 & 0.987\\
1954 & 6793.47 & 6743.42 & 499.31 & 679.27 & 0.988\\
1955 & 6792.50 & 6742.32 & 498.98 & 678.64 & 0.987\\
1956 & 6779.53 & 6729.16 & 498.24 & 676.97 & 0.984\\
1957 & 6759.35 & 6708.78 & 497.29 & 674.55 & 0.981\\
1958 & 6751.38 & 6700.63 & 496.75 & 673.22 & 0.979\\
1959 & 6704.70 & 6654.08 & 496.83 & 667.98 & 0.971\\
1960 & 6685.45 & 6634.85 & 498.53 & 665.34 & 0.967\\
1961 & 6680.50 & 6629.60 & 501.47 & 664.12 & 0.966\\
1962 & 6693.08 & 6641.20 & 505.68 & 664.72 & 0.966\\
1963 & 6700.02 & 6646.33 & 512.94 & 664.72 & 0.966\\
1964 & 6696.71 & 6640.70 & 527.17 & 663.35 & 0.964\\
1965 & 6706.31 & 6647.79 & 557.75 & 662.63 & 0.963\\
1966 & 6688.99 & 6626.36 & 613.84 & 658.30 & 0.957\\
1967 & 6647.19 & 6578.69 & 692.63 & 650.44 & 0.946\\
1968 & 6616.98 & 6543.59 & 741.31 & 642.11 & 0.934\\
1969 & 6600.25 & 6526.41 & 632.10 & 633.79 & 0.922\\
1970 & 6613.12 & 6545.34 & 524.22 & 628.00 & 0.913\\
1971 & 6580.04 & 6524.32 & 456.46 & 619.92 & 0.901\\
1972 & 6554.52 & 6508.80 & 537.03 & 616.60 & 0.897\\
1973 & 6429.40 & 6391.48 & 585.75 & 609.13 & 0.886\\
1974 & 6212.80 & 6175.67 & 472.20 & 596.90 & 0.868\\
1975 & 5902.16 & 5864.17 & 413.01 & 576.83 & 0.839\\
1976 & 5553.05 & 5519.48 & 408.18 & 551.71 & 0.802\\
1977 & 5173.57 & 5141.51 & 460.98 & 520.85 & 0.757\\
1978 & 4773.59 & 4742.85 & 403.81 & 485.31 & 0.706\\
1979 & 4388.22 & 4358.13 & 320.16 & 449.06 & 0.653\\
1980 & 3940.02 & 3915.05 & 278.61 & 405.26 & 0.589\\
1981 & 3480.60 & 3459.55 & 293.65 & 358.87 & 0.522\\
1982 & 3100.46 & 3078.34 & 269.30 & 319.71 & 0.465\\
1983 & 2655.53 & 2629.80 & 243.16 & 274.64 & 0.399\\
1984 & 2375.06 & 2353.56 & 291.89 & 243.45 & 0.354\\
1985 & 2124.47 & 2100.97 & 423.38 & 214.84 & 0.312\\
1986 & 1845.66 & 1811.35 & 375.11 & 183.11 & 0.266\\
1987 & 1649.05 & 1610.57 & 318.75 & 158.20 & 0.230\\
1988 & 1571.30 & 1543.51 & 319.57 & 143.47 & 0.209\\
1989 & 1529.20 & 1504.06 & 302.07 & 132.51 & 0.193\\
1990 & 1489.34 & 1461.83 & 360.71 & 124.09 & 0.180\\
1991 & 1440.61 & 1407.47 & 390.95 & 117.50 & 0.171\\
1992 & 1411.63 & 1375.51 & 274.30 & 112.78 & 0.164\\
1993 & 1405.72 & 1375.57 & 182.29 & 109.06 & 0.159\\
1994 & 1402.16 & 1376.33 & 178.25 & 105.70 & 0.154\\
1995 & 1423.35 & 1396.07 & 206.33 & 106.98 & 0.156\\
1996 & 1471.73 & 1449.13 & 160.47 & 111.22 & 0.162\\
1997 & 1458.28 & 1439.05 & 129.43 & 111.93 & 0.163\\
1998 & 1492.25 & 1473.17 & 160.71 & 116.54 & 0.169\\
1999 & 1537.55 & 1517.25 & 357.39 & 123.32 & 0.179\\
2000 & 1584.44 & 1561.51 & 181.13 & 130.99 & 0.190\\
2001 & 1653.94 & 1628.49 & 123.13 & 139.74 & 0.203\\
2002 & 1730.34 & 1711.89 & 208.19 & 148.22 & 0.216\\
2003 & 1814.31 & 1795.64 & 351.55 & 157.39 & 0.229\\
2004 & 1886.01 & 1863.06 & 161.07 & 166.29 & 0.242\\
2005 & 1958.15 & 1934.51 & 236.37 & 175.25 & 0.255\\
2006 & 1994.02 & 1977.00 & 87.93 & 180.82 & 0.263\\
2007 & 2036.68 & 2017.34 & 330.52 & 186.63 & 0.271\\
2008 & 2050.75 & 2023.30 & 317.18 & 189.90 & 0.276\\
2009 & 2090.87 & 2038.55 & 851.87 & 193.68 & 0.282\\
2010 & 2160.63 & 2111.31 & 913.18 & 195.80 & 0.285\\
2011 & 2291.84 & 2226.56 & 408.77 & 198.79 & 0.289\\
2012 & 2452.86 & 2404.67 & 552.88 & 202.49 & 0.294\\
2013 & 2627.91 & 2588.61 & 928.75 & 210.14 & 0.306\\
2014 & 2795.04 & 2739.48 & 244.85 & 223.28 & 0.325\\
2015 & 2973.06 & 2916.90 & 102.22 & 240.91 & 0.350\\
2016 & 3097.56 & 3064.44 & 245.60 & 255.75 & 0.372\\
2017 & 3182.40 & 3150.11 & 122.12 & 268.46 & 0.390\\
2018 & 3182.40 & 3153.24 & 96.39 & 273.80 & 0.398\\
2019 & 3176.81 & 3131.98 & 112.02 & 279.23 & 0.406\\
2020 & 3179.52 & 3148.18 & 163.48 & 283.32 & 0.412\\
2021 & 3110.47 & 3087.05 & 317.57 & 278.75 & 0.405\\
2022 & 3108.97 & 3077.18 & 302.69 & 279.64 & 0.407\\
2023 & 3199.84 & 3163.70 & 299.87 & 289.74 & 0.421\\*
\end{longtable}
\endgroup{}
\endgroup{}

\begingroup\fontsize{10}{12}\selectfont

\begin{landscape}\begingroup\fontsize{10}{12}\selectfont

\begin{longtable}[t]{c>{\centering\arraybackslash}p{1.38cm}>{\centering\arraybackslash}p{1.38cm}>{\centering\arraybackslash}p{1.38cm}>{\centering\arraybackslash}p{1.38cm}>{\centering\arraybackslash}p{1.38cm}>{\centering\arraybackslash}p{1.38cm}>{\centering\arraybackslash}p{1.38cm}}
\caption{\label{tab:ca-proj}The estimated spawning output in number of million eggs across California and fraction unfished by year.}\\
\toprule
Year & Adopted OFL (mt) & Adopted ABC (mt) & Assumed Catch (mt) & OFL (mt) & ABC (mt) & Spawning Biomass & Fraction Unfished\\
\midrule
\endfirsthead
\caption[]{\label{tab:ca-proj}The estimated spawning output in number of million eggs across California and fraction unfished by year. \textit{(continued)}}\\
\toprule
Year & Adopted OFL (mt) & Adopted ABC (mt) & Assumed Catch (mt) & OFL (mt) & ABC (mt) & Spawning Biomass & Fraction Unfished\\
\midrule
\endhead

\endfoot
\bottomrule
\endlastfoot
2023 & 116.4 & 91.53 & 70 & - & - & 289.74 & 0.421\\
2024 & 121.32 & 94.69 & 70 & - & - & 295.18 & 0.429\\
2025 & - & - & - & 171.29 & 148.79 & 302.30 & 0.440\\
2026 & - & - & - & 171.74 & 147.12 & 303.82 & 0.442\\
2027 & - & - & - & 172.05 & 145.66 & 304.90 & 0.443\\
2028 & - & - & - & 172.2 & 144 & 305.69 & 0.444\\
2029 & - & - & - & 172.26 & 141.82 & 306.33 & 0.445\\
2030 & - & - & - & 172.31 & 140.09 & 306.97 & 0.446\\
2031 & - & - & - & 172.39 & 138.4 & 307.60 & 0.447\\
2032 & - & - & - & 172.54 & 136.34 & 308.29 & 0.448\\
2033 & - & - & - & 172.8 & 134.85 & 309.09 & 0.449\\
2034 & - & - & - & 173.15 & 143.44 & 309.99 & 0.451\\*
\end{longtable}
\endgroup{}
\end{landscape}
\endgroup{}

\newpage

\clearpage

\hypertarget{figures}{%
\section{Figures}\label{figures}}

\hypertarget{data}{%
\subsection{Data}\label{data}}

\begin{figure}
\centering
\includegraphics[width=1\textwidth,height=1\textheight]{C:/Users/melissa.monk/Documents/GitHub/copper_rockfish_2023/documents/shared_figures/copper_length_by_half_month.png}
\caption{Distribution of YOY copper rockfish lengths from fish genetically identified from D. Baetscher.\label{fig:copper-smurf-length}}
\end{figure}

\pagebreak

\begin{figure}
\centering
\includegraphics[width=1\textwidth,height=1\textheight]{C:/Users/melissa.monk/Documents/GitHub/copper_rockfish_2023/documents/shared_figures/map.png}
\caption{Map of management area and the 2023 assessments areas for copper rockfish.\label{fig:ca-map}}
\end{figure}

\pagebreak

\begin{figure}
\centering
\includegraphics[width=1\textwidth,height=1\textheight]{S:/copper_rockfish_2023/models/nca/9.8_selex_fix_forecast/plots/catch2 landings stacked.png}
\caption{Landings by fleet used in the base model where catches in metric tons by fleet are stacked.\label{fig:catch}}
\end{figure}

\pagebreak

\begin{figure}
\centering
\includegraphics[width=1\textwidth,height=1\textheight]{S:/copper_rockfish_2023/models/nca/9.8_selex_fix_forecast/plots/data_plot.png}
\caption{Summary of data sources used in the base model.\label{fig:data-plot}}
\end{figure}

\pagebreak

\begin{figure}
\centering
\includegraphics[width=1\textwidth,height=1\textheight]{S:/copper_rockfish_2023/models/nca/9.8_selex_fix_forecast/plots/comp_lendat_bubflt1mkt0.png}
\caption{Length composition data from the commercial dead fleet.\label{fig:com-dead-len-data}}
\end{figure}

\pagebreak

\begin{figure}
\centering
\includegraphics[width=1\textwidth,height=1\textheight]{S:/copper_rockfish_2023/models/nca/9.8_selex_fix_forecast/plots/comp_lendat_data_weighting_TA1.8_Commercial_Dead.png}
\caption{Mean length for commercial dead fleet with 95 percent confidence intervals.\label{fig:mean-com-dead-len-data}}
\end{figure}

\pagebreak

\begin{figure}
\centering
\includegraphics[width=1\textwidth,height=1\textheight]{S:/copper_rockfish_2023/models/nca/9.8_selex_fix_forecast/plots/comp_condAALdat_bubflt1mkt0.png}
\caption{Conditional age-at-length composition data from the commercial dead fleet.\label{fig:com-dead-age-data}}
\end{figure}

\pagebreak

\begin{figure}
\centering
\includegraphics[width=1\textwidth,height=1\textheight]{S:/copper_rockfish_2023/models/nca/9.8_selex_fix_forecast/plots/comp_lendat_bubflt2mkt0.png}
\caption{Length composition data from the commercial live fleet.\label{fig:com-live-len-data}}
\end{figure}

\pagebreak

\begin{figure}
\centering
\includegraphics[width=1\textwidth,height=1\textheight]{S:/copper_rockfish_2023/models/nca/9.8_selex_fix_forecast/plots/comp_lendat_data_weighting_TA1.8_Commercial_Live.png}
\caption{Mean length for commercial live fleet with 95 percent confidence intervals.\label{fig:mean-com-live-len-data}}
\end{figure}

\pagebreak

\begin{figure}
\centering
\includegraphics[width=1\textwidth,height=1\textheight]{S:/copper_rockfish_2023/data/rec_indices/mrfss_cpfv_dockside/north/forSS/Index.png}
\caption{Estimated annual index of abundances for the CPFV fleet based on MRFSS survey data.\label{fig:mrfss-index-main}}
\end{figure}

\pagebreak

\begin{figure}
\centering
\includegraphics[width=1\textwidth,height=1\textheight]{S:/copper_rockfish_2023/data/rec_indices/debwv_cpfv_onboard/deltalogn/Index.png}
\caption{Estimated annual index of abundances for the CPFV fleet based on the Deb Wilson-Vandenberg survey data.\label{fig:dwv-index-main}}
\end{figure}

\pagebreak

\begin{figure}
\centering
\includegraphics[width=1\textwidth,height=1\textheight]{S:/copper_rockfish_2023/data/rec_indices/crfs_cpfv_onboard/north/area_weighted/Index.png}
\caption{Estimated annual index of abundances for the CPFV fleet based on CRFS survey data.\label{fig:crfs-index-main}}
\end{figure}

\pagebreak

\begin{figure}
\centering
\includegraphics[width=1\textwidth,height=1\textheight]{S:/copper_rockfish_2023/data/rec_indices/crfs_pr_dockside/north/rm_last2yrs_area_weighted/Index.png}
\caption{Estimated annual index of abundances for the CPFV fleet based on CRFS survey data.\label{fig:crfs-pr-index-main}}
\end{figure}

\pagebreak

\begin{figure}
\centering
\includegraphics[width=1\textwidth,height=1\textheight]{S:/copper_rockfish_2023/models/nca/9.8_selex_fix_forecast/plots/comp_lendat_bubflt3mkt0_page2.png}
\caption{Length composition data from the recreational CPFV fleet.\label{fig:rec-cpfv-len-data}}
\end{figure}

\pagebreak

\begin{figure}
\centering
\includegraphics[width=1\textwidth,height=1\textheight]{S:/copper_rockfish_2023/models/nca/9.8_selex_fix_forecast/plots/comp_lendat_data_weighting_TA1.8_Rec_CPFV.png}
\caption{Mean length for recreational CPFV fleet with 95 percent confidence intervals.\label{fig:mean-rec-cpfv-len-data}}
\end{figure}

\pagebreak

\begin{figure}
\centering
\includegraphics[width=1\textwidth,height=1\textheight]{S:/copper_rockfish_2023/models/nca/9.8_selex_fix_forecast/plots/comp_condAALdat_bubflt3mkt0.png}
\caption{Conditional age-at-length composition data from the recreational CPFV fleet.\label{fig:rec-cpfv-caal-data}}
\end{figure}

\pagebreak

\begin{figure}
\centering
\includegraphics[width=1\textwidth,height=1\textheight]{S:/copper_rockfish_2023/models/nca/9.8_selex_fix_forecast/plots/comp_agedat_data_weighting_TA1.8_Rec_CPFV.png}
\caption{Mean age for recreational CPFV fleet with 95 percent confidence intervals.\label{fig:mean-rec-cpfv-age-data}}
\end{figure}

\pagebreak

\begin{figure}
\centering
\includegraphics[width=1\textwidth,height=1\textheight]{S:/copper_rockfish_2023/data/ages/plots/coop_crfs_length_comparison.png}
\caption{Comparison of all length collected by the CRFS sampling program for the CPFV fleet to the lengths from the fish with ages from the cooperative sampling program. The length distributions in the area north of Point Conception are in general agreement while the distribution of lengths collected by this program does not align with the length samples from CRFS.\label{fig:coop-len-comparison}}
\end{figure}

\pagebreak

\begin{figure}
\centering
\includegraphics[width=1\textwidth,height=1\textheight]{S:/copper_rockfish_2023/models/nca/9.8_selex_fix_forecast/plots/comp_condAALdat_bubflt4mkt0.png}
\caption{Conditional age-at-length data for recreational PR collected in 2022.\label{fig:rec-pr-caal-data}}
\end{figure}

\pagebreak

\begin{figure}
\centering
\includegraphics[width=1\textwidth,height=1\textheight]{S:/copper_rockfish_2023/models/nca/9.8_selex_fix_forecast/plots/comp_lendat_flt4mkt0_page2.png}
\caption{Length composition data from the recreational PR fleet.\label{fig:rec-pr-len-data}}
\end{figure}

\pagebreak

\begin{figure}
\centering
\includegraphics[width=1\textwidth,height=1\textheight]{S:/copper_rockfish_2023/models/nca/9.8_selex_fix_forecast/plots/comp_lendat_data_weighting_TA1.8_Rec_PR.png}
\caption{Mean length for recreational PR fleet with 95 percent confidence intervals.\label{fig:mean-rec-pr-len-data}}
\end{figure}

\pagebreak

\begin{figure}
\centering
\includegraphics[width=1\textwidth,height=1\textheight]{S:/copper_rockfish_2023/data/survey_indices/plots/north_survey_locations_designation.png}
\caption{Sample locations by each of the fishery-independent data sources used in the base model with indices of abundance, lengths, and ages if collected.\label{fig:survey-locations}}
\end{figure}

\pagebreak

\begin{figure}
\centering
\includegraphics[width=1\textwidth,height=1\textheight]{S:/copper_rockfish_2023/data/survey_indices/plots/north_survey_locations.png}
\caption{Sample locations by area, areas open to fishing (reference) and MPAS, for each of the fishery-independent data sources used in the base model with indices of abundance, lengths, and ages if collected.\label{fig:ref-mpa}}
\end{figure}

\pagebreak

\begin{figure}
\centering
\includegraphics[width=1\textwidth,height=1\textheight]{S:/copper_rockfish_2023/data/survey_indices/ccfrp/north/area_weighted/Index.png}
\caption{Estimated index of abundance from the CCFRP survey.\label{fig:ccfrp-index-main}}
\end{figure}

\pagebreak

\begin{figure}
\centering
\includegraphics[width=1\textwidth,height=1\textheight]{S:/copper_rockfish_2023/models/nca/9.8_selex_fix_forecast/plots/comp_lendat_bubflt5mkt0.png}
\caption{Length composition data from the CCFRP survey.\label{fig:ccfrp-len-data}}
\end{figure}

\pagebreak

\begin{figure}
\centering
\includegraphics[width=1\textwidth,height=1\textheight]{S:/copper_rockfish_2023/models/nca/9.8_selex_fix_forecast/plots/comp_lendat_data_weighting_TA1.8_CCFRP.png}
\caption{Mean length for the CCFRP survey with 95 percent confidence intervals.\label{fig:ccfrp-mean-len-data}}
\end{figure}

\pagebreak

\begin{figure}
\centering
\includegraphics[width=1\textwidth,height=1\textheight]{S:/copper_rockfish_2023/models/nca/9.8_selex_fix_forecast/plots/comp_condAALdat_bubflt5mkt0.png}
\caption{Conditional age-at-length data from the CCFRP survey.\label{fig:ccfrp-age-data}}
\end{figure}

\pagebreak

\begin{figure}
\centering
\includegraphics[width=1\textwidth,height=1\textheight]{S:/copper_rockfish_2023/data/survey_indices/rov/plots/rov_transect_collapsed_copper_north_protection_count.png}
\caption{The location and size of observations across all years and transects.\label{fig:rov-obs-loc}}
\end{figure}

\pagebreak

\begin{figure}
\centering
\includegraphics[width=1\textwidth,height=1\textheight]{S:/copper_rockfish_2023/data/survey_indices/rov/plots/north_raw_cpue_by_mpa_group.png}
\caption{The trend of the calculated CPUE by each MPA and Reference group.\label{fig:rov-raw-cpue}}
\end{figure}

\pagebreak

\begin{figure}
\centering
\includegraphics[width=1\textwidth,height=1\textheight]{S:/copper_rockfish_2023/data/survey_indices/rov/glm_negbin_north_designation_depth/Index.png}
\caption{The estimated weighted relative index of abundance.\label{fig:rov-index-main}}
\end{figure}

\pagebreak

\begin{figure}
\centering
\includegraphics[width=1\textwidth,height=1\textheight]{S:/copper_rockfish_2023/data/survey_indices/rov/plots/rov_length_by_area_designation.png}
\caption{The distribution of lengths across all years for MPA and Reference area north and south of Point Conception.\label{fig:rov-len}}
\end{figure}

\pagebreak

\begin{figure}
\centering
\includegraphics[width=1\textwidth,height=1\textheight]{S:/copper_rockfish_2023/models/nca/9.8_selex_fix_forecast/plots/comp_lendat_flt6mkt0.png}
\caption{Length composition data from the CDFW ROV survey.\label{fig:rov-len-data}}
\end{figure}

\pagebreak

\begin{figure}
\centering
\includegraphics[width=1\textwidth,height=1\textheight]{S:/copper_rockfish_2023/models/nca/9.8_selex_fix_forecast/plots/comp_lendat_data_weighting_TA1.8_CDFW_ROV.png}
\caption{Mean length for CDWF ROV survey with 95 percent confidence intervals.\label{fig:mean-rov-len-data}}
\end{figure}

\pagebreak

\begin{figure}
\centering
\includegraphics[width=1\textwidth,height=1\textheight]{S:/copper_rockfish_2023/data/ages/plots/south_growth_length_comparison.png}
\caption{Length distribution of fish by collection source that were used as conditional age-at-length data in the growth fleet.\label{fig:growth-len-dist}}
\end{figure}

\pagebreak

\begin{figure}
\centering
\includegraphics[width=1\textwidth,height=1\textheight]{S:/copper_rockfish_2023/data/ages/plots/south_growth_age_comparison.png}
\caption{Age distribution of fish by collection source that were used as conditional age-at-length data in the growth fleet.\label{fig:growth-age-dist}}
\end{figure}

\pagebreak

\begin{figure}
\centering
\includegraphics[width=1\textwidth,height=1\textheight]{S:/copper_rockfish_2023/data/wcgbt/north/plots/cpue_map.png}
\caption{Location and catch-per-unit-effort by location caught north of Point Conception by the NWFSC WCGBT survey.\label{fig:wcgbt-cpue}}
\end{figure}

\pagebreak

\begin{figure}
\centering
\includegraphics[width=1\textwidth,height=1\textheight]{S:/copper_rockfish_2023/data/wcgbt/north/plots/presence-absence_proportion_by_depth.png}
\caption{Number of positive tows across all years by depth in meters.\label{fig:wcgbt-depth}}
\end{figure}

\pagebreak
<<<<<<< Updated upstream
=======

\hypertarget{mrfss-cpfv-index}{%
\section{Appendix B. MRFSS CPFV Dockside Index of Abundance}\label{mrfss-cpfv-index}}

\hypertarget{onboard-cpfv-index}{%
\section{Appendix C. California Onboard CPFV Index of Abundance}\label{onboard-cpfv-index}}

The state of California implemented a statewide onboard observer sampling program in 1999 (Monk et al. 2014). California Polytechnic State University (Cal Poly) has conducted an independent onboard sampling program as of 2003 for boats in Port San Luis and Morro Bay, and follows the protocols established in Reilly et al. (1998). During an onboard observer trip the sampler rides along on the CPFV and records location-specific catch and discard information to the species level for a subset of anglers onboard the vessel. The subset of observed anglers is usually a maximum of 15 people and the observed anglers change during each fishing stop.

The catch cannot be linked to an individual, but rather to a specific fishing location. The sampler also records the starting and ending time, number of anglers observed, starting and ending depth, and measures discarded fish. The fine-scale catch and effort data allow us to better filter the data for indices to fishing stops within suitable habitat for copper rockfish. Cal Poly has modified protocols to reflect sampling changes that CDFW has also adopted, e.g., observing fish as they are encountered instead of at the level of a fisher's bag. Therefore, the Cal Poly data are incorporated in the same index as the CDFW data from 1999-2019. The only difference is that Cal Poly measures the length of both retained and discarded fish.

In the assessment model, the recreational CPFV fleet is modeled as retained plus discarded fish. The proportion of observed discarded copper rockfish is small, averaging 3.5\% over the time series (\ref{tab:onboard-keepdiscard}) and are included in the index.

As described above the CDFW and Cal Poly onboard observer programs are identical in that the same protocols are followed. The only difference is that Cal Poly measures both retained and discarded fish from the observed anglers and CDFW measures only discarded fish from the observed anglers. CDFW measures retained fish as part of the angler interview at the bag and trip level.\\
Therefore, only retained fish were modeled in this index. The data went through a QA/QC process at the SWFSC which included mapping fishing drifts in ArcPro to determine if the recorded latitude and longitude were correct.

We applied a number of data filters to the available data presented in Table \ref{tab:onboard-filter}. The onboard CPFV index restricts the time series to 2004-2019. The onboard observer survey began in 1999, but the sample sizes were small during the first years of the program. The years 1999-2003 also represent years where a number of regulations changed including gear limits, bag limits, and spatial closures. Due to COVID-19, no onboard sampling took place in 2020. In 2021 the onboard sampling resumed in August, but not at full capacity. The 2021 stock assessment had also been released by August 2021 indicating the stock was below the biomass at 40\%. The southern California CPFV began an organized effort to avoid copper rockfish and encourage their clientele to release and descend copper rockfish when encountered. The northern California fleet also participated in this effort to an extent. In 2022, the CDFW implemented the one copper rockfish sub-bag limit and combined with avoidance by the fleet, the data for this year do not represent the available copper rockfish biomass. See the online supplementary material or the history of regulation changes section for details.

The filters also included removal of the number of observed anglers and time fished at the tail ends of the distributions and removal of drifts occurring in depths outside copper rockfish's range (Table \ref{tab:onboard-filter} and Figure \ref{fig:onboard-depths}). We retained 17,458 drifts for index standardization, and 3,303 of those drifts encountered copper rockfish Table \ref{tab:onboard-percentpos}.

We modeled catch per angler minutes fished (CPUE) by fishing drift. Prior to any modeling, the SWFSC QA/QC'd the data to ensure the location information was correct. Each drift was overlaid with the available interpreted substrate layer that characterizes rocky and hard substrate, assigned to a rocky reef, and the distance of the drift start location calculated. In addition, the depth of the start location was interpreted from the 2 m resolution bathymetry as well as 90 m resolution bathymetry layer for comparison. For drifts missing depth location, we assigned depth based on the best available depth based on the bathymetry.

To appropriately weight the onboard observer survey index by the available rocky substrate within a region, each drift was assigned to the closest area of rocky habitat. Hard bottom was extracted from the \href{http://seafloor.otterlabs.org/index.html}{California Seafloor Mapping Project}, along the mainland coast of southern California. These data were collected in state waters at a resolution of two meters, but did not extend into state waters past the mainland coast. Additional interpreted bathymetric data classifying the bottom type as rock or soft bottom were compiled by analysts at the University of California Santa Cruz and are now also available from CDFW's website. We used the available interpreted rocky substrate data to expand the known area of rocky substrate to areas in southern California that lack substrate type. This expansion of the estimated rocky substrate assumes that the proportions of rocky substrate within and outside state waters are similar. Copper rockfish are a nearshore species and the majority of observed encounters were within state waters (Table \ref{tab:onboard-waterarea}). This is, of course, an estimation of the amount of rocky substrate, and represents the best available data. The calculations can be found in the online supplementary material. Starting in 2017, depth restrictions eased in districts north of Point Conception and the recreational fleet targeted these depths (Figure \ref{fig:onboard-depths}). The deeper waters (40-50 fm) are outside of the mapped hard bottom habitat, but could be assigned to the larger areas considered as a factor in the index.

The covariates explored for model selection included year and four categorical region levels (CRFS Districts 3-6), a categorical variable for month, and continuous depth and depth-squared.\\
Trends in the average CPUE by region were similar in the filtered data set (Figure \ref{fig:onboard-regioncpue}). A year and region interaction was included after visualizing the trends in average CPUE over time, but was not significant (Figure @ref(fig:onboard-average\_cpue\_by\_region)). The full model was selected by AICc and included year, depth, depth squared and region (Table @ref(tab:onboard-model\_selection)).

Indices were fit via MLE from the sdmTMB package in R. The QQ plot for the negative binoimal model indicated a poor fit to the data, which as not surprising given the low percent of observed drifts encountering copper rockfish. A delta-Lognormal was selected over a delta-gamma by a delta AIC of 487. The QQ plot indicated a much improved fit compared to the negative binomial model (Table \ref{fig:onboard-qq}). The relative abundance is predominantly flat during the time series, with a visible increase in CPUE in 2017 when deeper waters opened in portions of northern California after a 17 year closure (Table \ref{tab:onboard-index} and Figure \ref{fig:onboard-index}).

\begingroup\fontsize{10}{12}\selectfont
\begingroup\fontsize{10}{12}\selectfont

\begin{longtable}[t]{c>{\centering\arraybackslash}p{2cm}>{\centering\arraybackslash}p{2cm}>{\centering\arraybackslash}p{2cm}}
\caption{\label{tab:onboard-keepdiscard}Number of observed copper rockfish retained and discarded by year.}\\
\toprule
year & kept & discard & Proportion.discarded\\
\midrule
\endfirsthead
\caption[]{\label{tab:onboard-keepdiscard}Number of observed copper rockfish retained and discarded by year. \textit{(continued)}}\\
\toprule
year & kept & discard & Proportion.discarded\\
\midrule
\endhead

\endfoot
\bottomrule
\endlastfoot
1999 & 43 & 0 & 0.0000000\\
2000 & 44 & 0 & 0.0000000\\
2001 & 66 & 2 & 0.0294118\\
2002 & 66 & 3 & 0.0434783\\
2003 & 129 & 8 & 0.0583942\\
2004 & 348 & 29 & 0.0769231\\
2005 & 431 & 29 & 0.0630435\\
2006 & 535 & 38 & 0.0663176\\
2007 & 523 & 17 & 0.0314815\\
2008 & 266 & 4 & 0.0148148\\
2009 & 262 & 9 & 0.0332103\\
2010 & 480 & 19 & 0.0380762\\
2011 & 313 & 16 & 0.0486322\\
2012 & 327 & 19 & 0.0549133\\
2013 & 332 & 11 & 0.0320700\\
2014 & 374 & 11 & 0.0285714\\
2015 & 369 & 8 & 0.0212202\\
2016 & 404 & 12 & 0.0288462\\
2017 & 823 & 5 & 0.0060386\\
2018 & 584 & 7 & 0.0118443\\
2019 & 398 & 7 & 0.0172840\\*
\end{longtable}
\endgroup{}
\endgroup{}

\newpage

\begingroup\fontsize{10}{12}\selectfont
\begingroup\fontsize{10}{12}\selectfont

\begin{longtable}[t]{c>{\centering\arraybackslash}p{1.83cm}>{\centering\arraybackslash}p{1.83cm}>{\centering\arraybackslash}p{1.83cm}>{\centering\arraybackslash}p{1.83cm}>{\centering\arraybackslash}p{1.83cm}}
\caption{\label{tab:onboard-waterarea}Number of observed drifts inside and outside of state waters.}\\
\toprule
X & district & year & N & O & percent\_N\\
\midrule
\endfirsthead
\caption[]{\label{tab:onboard-waterarea}Number of observed drifts inside and outside of state waters. \textit{(continued)}}\\
\toprule
X & district & year & N & O & percent\_N\\
\midrule
\endhead

\endfoot
\bottomrule
\endlastfoot
1 & 3 & 2004 & 86 & NA & NA\\
2 & 3 & 2005 & 133 & 13 & 0.9109589\\
3 & 3 & 2006 & 137 & 22 & 0.8616352\\
4 & 3 & 2007 & 144 & 33 & 0.8135593\\
5 & 3 & 2008 & 61 & 10 & 0.8591549\\
6 & 3 & 2009 & 69 & 4 & 0.9452055\\
7 & 3 & 2010 & 128 & 29 & 0.8152866\\
8 & 3 & 2011 & 119 & 13 & 0.9015152\\
9 & 3 & 2012 & 125 & 20 & 0.8620690\\
10 & 3 & 2013 & 175 & 9 & 0.9510870\\
11 & 3 & 2014 & 136 & 9 & 0.9379310\\
12 & 3 & 2015 & 131 & 23 & 0.8506494\\
13 & 3 & 2016 & 153 & 12 & 0.9272727\\
14 & 3 & 2017 & 136 & 29 & 0.8242424\\
15 & 3 & 2018 & 68 & 68 & 0.5000000\\
16 & 3 & 2019 & 109 & 27 & 0.8014706\\
17 & 4 & 2004 & 10 & NA & NA\\
18 & 4 & 2005 & 8 & NA & NA\\
19 & 4 & 2006 & 21 & 2 & 0.9130435\\
20 & 4 & 2007 & 24 & 4 & 0.8571429\\
21 & 4 & 2008 & 18 & NA & NA\\
22 & 4 & 2009 & 23 & 3 & 0.8846154\\
23 & 4 & 2010 & 14 & NA & NA\\
24 & 4 & 2011 & 25 & 3 & 0.8928571\\
25 & 4 & 2012 & 15 & 1 & 0.9375000\\
26 & 4 & 2013 & 17 & NA & NA\\
27 & 4 & 2014 & 33 & 5 & 0.8684211\\
28 & 4 & 2015 & 32 & 7 & 0.8205128\\
29 & 4 & 2016 & 52 & 1 & 0.9811321\\
30 & 4 & 2017 & 40 & 29 & 0.5797101\\
31 & 4 & 2018 & 17 & 8 & 0.6800000\\
32 & 4 & 2019 & 30 & 10 & 0.7500000\\
33 & 5 & 2004 & 1 & NA & NA\\
34 & 5 & 2008 & 4 & NA & NA\\
35 & 5 & 2009 & 2 & NA & NA\\
36 & 5 & 2010 & 4 & NA & NA\\
37 & 5 & 2011 & 3 & NA & NA\\
38 & 5 & 2012 & 3 & NA & NA\\
39 & 5 & 2014 & 5 & NA & NA\\
40 & 5 & 2015 & 1 & NA & NA\\
41 & 5 & 2017 & 1 & NA & NA\\
42 & 6 & 2008 & 3 & NA & NA\\
43 & 6 & 2011 & 1 & NA & NA\\
44 & 6 & 2012 & 2 & NA & NA\\
45 & 6 & 2014 & 3 & NA & NA\\
46 & 6 & 2015 & 1 & NA & NA\\
47 & 6 & 2018 & 7 & NA & NA\\*
\end{longtable}
\endgroup{}
\endgroup{}

\newpage

\begingroup\fontsize{10}{12}\selectfont
\begingroup\fontsize{10}{12}\selectfont

\begin{longtable}[t]{c>{\centering\arraybackslash}p{2.2cm}>{\centering\arraybackslash}p{2.2cm}>{\centering\arraybackslash}p{2.2cm}>{\centering\arraybackslash}p{2.2cm}}
\caption{\label{tab:onboard-percentpos}Data filtering steps for the onboard CPFV survey.}\\
\toprule
year & tripsWithTarget & tripsWOTarget & totalTrips & percentpos\\
\midrule
\endfirsthead
\caption[]{\label{tab:onboard-percentpos}Data filtering steps for the onboard CPFV survey. \textit{(continued)}}\\
\toprule
year & tripsWithTarget & tripsWOTarget & totalTrips & percentpos\\
\midrule
\endhead

\endfoot
\bottomrule
\endlastfoot
1999 & 14 & 167 & 181 & 0.0773481\\
2000 & 13 & 90 & 103 & 0.1262136\\
2001 & 31 & 168 & 199 & 0.1557789\\
2002 & 19 & 159 & 178 & 0.1067416\\
2003 & 57 & 515 & 572 & 0.0996503\\
2004 & 88 & 831 & 919 & 0.0957563\\
2005 & 150 & 559 & 709 & 0.2115656\\
2006 & 172 & 635 & 807 & 0.2131351\\
2007 & 203 & 669 & 872 & 0.2327982\\
2008 & 95 & 694 & 789 & 0.1204056\\
2009 & 100 & 752 & 852 & 0.1173709\\
2010 & 170 & 857 & 1027 & 0.1655307\\
2011 & 158 & 996 & 1154 & 0.1369151\\
2012 & 163 & 864 & 1027 & 0.1587147\\
2013 & 199 & 960 & 1159 & 0.1716997\\
2014 & 186 & 858 & 1044 & 0.1781609\\
2015 & 198 & 767 & 965 & 0.2051813\\
2016 & 221 & 1017 & 1238 & 0.1785137\\
2017 & 240 & 650 & 890 & 0.2696629\\
2018 & 170 & 547 & 717 & 0.2370990\\
2019 & 178 & 621 & 799 & 0.2227785\\*
\end{longtable}
\endgroup{}
\endgroup{}

\newpage

\newpage

\begingroup\fontsize{10}{12}\selectfont
\begingroup\fontsize{10}{12}\selectfont

\begin{longtable}[t]{c>{\centering\arraybackslash}p{2cm}>{\centering\arraybackslash}p{2cm}}
\caption{\label{tab:onboard-index}Estimated relative index of abundance for the onboard CPFV survey.}\\
\toprule
Year & Estimate & logSE\\
\midrule
\endfirsthead
\caption[]{\label{tab:onboard-index}Estimated relative index of abundance for the onboard CPFV survey. \textit{(continued)}}\\
\toprule
Year & Estimate & logSE\\
\midrule
\endhead

\endfoot
\bottomrule
\endlastfoot
2004 & 0.0071710 & 0.1026699\\
2005 & 0.0082869 & 0.0959658\\
2006 & 0.0077566 & 0.0948186\\
2007 & 0.0075606 & 0.0927893\\
2008 & 0.0073246 & 0.1048729\\
2009 & 0.0088566 & 0.1045802\\
2010 & 0.0076344 & 0.0936492\\
2011 & 0.0073209 & 0.0935917\\
2012 & 0.0075198 & 0.0942729\\
2013 & 0.0079753 & 0.0906166\\
2014 & 0.0081479 & 0.0908896\\
2015 & 0.0080234 & 0.0884672\\
2016 & 0.0113242 & 0.0864655\\
2017 & 0.0108471 & 0.0869631\\
2018 & 0.0101541 & 0.0916336\\
2019 & 0.0099923 & 0.0885035\\*
\end{longtable}
\endgroup{}
\endgroup{}

\begingroup\fontsize{10}{12}\selectfont
\begingroup\fontsize{10}{12}\selectfont

\begin{longtable}[t]{c>{\centering\arraybackslash}p{2cm}>{\centering\arraybackslash}p{2cm}>{\centering\arraybackslash}p{2cm}}
\caption{\label{tab:onboard-filter}Data filtering steps for the onboard CPFV survey.}\\
\toprule
Filter & Description & Number of Samples & Positive Samples\\
\midrule
\endfirsthead
\caption[]{\label{tab:onboard-filter}Data filtering steps for the onboard CPFV survey. \textit{(continued)}}\\
\toprule
Filter & Description & Number of Samples & Positive Samples\\
\midrule
\endhead

\endfoot
\bottomrule
\endlastfoot
All data & All data & 28554 & 4551\\
Years & Start time series in 2004 due to sparse data & 25267 & 4291\\
Errors and Missing Data & Remove drifts with missing data and identified errors & 25107 & 4288\\
Area fished & Remove drifts in bays & 24667 & 4288\\
Months fished & Remove Jan-March; recreational rockfish fishery closed & 23935 & 4196\\
Depth & Remove upper and lower 1\% of depth with observed coppers;
                                           Remaining drifts between 50 and 300 feet & 22444 & 4146\\
Observed anglers & Remove upper and lower 2.5\% of observed anglers;
                                           Remaining drifts with 4-12 observed anglers & 21032 & 3988\\
Time fished & Remove upper and lower 2.5\% time fished and
                                         time fished; Remaining drifts with 5-73 minutes time fished & 19406 & 3797\\
Distance from rocky substrate & After removing observations further
than 0.5km from rocky substrate, keep 95\% of the data; drifts within 10.1 m of rocky substrate & 17458 & 3403\\*
\end{longtable}
\endgroup{}
\endgroup{}

\newpage
>>>>>>> Stashed changes

\begin{figure}
\centering
\includegraphics[width=1\textwidth,height=1\textheight]{S:/copper_rockfish_2023/data/wcgbt/plots/wcgbt_north_age_at_length_by_area.png}
\caption{Age and length by sex for copper rockfish caught north of Point Conception by the NWFSC WCGBT survey.\label{fig:wcgbt-len-age}}
\end{figure}

\pagebreak

\hypertarget{biology}{%
\subsection{Biology}\label{biology}}

\begin{figure}
\centering
\includegraphics[width=1\textwidth,height=1\textheight]{S:/copper_rockfish_2023/models/nca/9.8_selex_fix_forecast/plots/bio6_maturity.png}
\caption{Maturity as a function of length.\label{fig:maturity}}
\end{figure}

\pagebreak

\begin{figure}
\centering
\includegraphics[width=1\textwidth,height=1\textheight]{S:/copper_rockfish_2023/models/nca/9.8_selex_fix_forecast/plots/bio9_fecundity_len.png}
\caption{Fecundity as a function of length.\label{fig:fecundity}}
\end{figure}

\pagebreak

\begin{figure}
\centering
\includegraphics[width=1\textwidth,height=1\textheight]{S:/copper_rockfish_2023/data/wcgbt/plots/length_fraction_female.png}
\caption{Fraction of each sex by length by the NWFSC WCGBT survey.\label{fig:frac-sex-len}}
\end{figure}

\pagebreak

\begin{figure}
\centering
\includegraphics[width=1\textwidth,height=1\textheight]{S:/copper_rockfish_2023/data/biology/plots/Length_Weight_All.png}
\caption{Estimated weight-at-length.\label{fig:weight-length}}
\end{figure}

<<<<<<< Updated upstream
\pagebreak
=======
\newpage

\hypertarget{dwv-cpfv-index}{%
\section{Appendix D. Deb Wilson-Vandenberg Onboard CPFV Index of Abundance}\label{dwv-cpfv-index}}

The Deb Wilson-Vandenberg data set is derived from an onboard observer survey conducted by CDFW in California north of Point Conception from 1987-1998 and referred to as the Deb Wilson-Vandenberg onboard observer survey, (\textbf{reilly-onboard-1998?}). During an observer trip the sampler rode along on the CPFV and recorded location-specific catch and discard information to the species level for a subset of anglers onboard the vessel. The subset of observed anglers is usually a maximum of 15 people and the observed anglers change during each fishing stop. The catch cannot be linked to an individual, but rather to a specific fishing location. The sampler also recorded the starting and ending time, number of anglers observed, starting and ending depth, and measured retained and discarded fish. The fine-scale catch and effort data allow us to better filter the data for indices to fishing stops within suitable habitat for the target species.

A large effort was made by the SWFSC to recover data from the original data sheets for this survey and develop a relational database (\textbf{monk-documentation-2016?}). The specific fishing locations at each fishing stop were recorded at a finer scale than the catch data for this survey. We aggregated the relevant location information (time and number of observed anglers) to match the available catch information. Between April 1987 and July 1992 the number of observed anglers was not recorded for each fishing stop, but the number of anglers aboard the vessel is available. We imputed the number of observed anglers using the number of anglers aboard the vessel and the number of observed anglers at each fishing stop from the August 1992-December 1998 data (see Supplemental materials for details). In 1987, trips were only observed in Monterey, CA and were therefore excluded from the analysis (Table \ref{tab:deb-filter}). Sampling mainly targeted areas of central California. Of the 2,256 trips observed, only 12 of those launched from port in District 6, the most northern district in California.

Each fishing location was assigned to a reef based on the on the bathymetric maps and interpretation of hard bottom was extracted from the \href{http://seafloor.otterlabs.org/index.html}{California Seafloor Mapping Project}. Reefs were aggregated to four regions to produce adequate sample sizes; the California/Oregon border to San Francisco (V1), San Francisco to Moss Landing (V2), Moss Landing to Big Sur (V3), and San Luis Obispo county to Point Conception (V4). The ports in San Luis Obispo county were sampled more frequently than other regions and the arithmetic mean of CPUE by year was also higher in this area (Figure \ref{fig:fig-areacpue-debwv})

The filters also included removal of the number of observed anglers and time fished at the tail ends of the distributions and removal of drifts occurring in depths outside copper rockfish's range (Table \ref{tab:deb-filter} and Figure \ref{fig:deb-depths}). We retained 5,546 drifts for index standardization, with 1,389 fishing locations encountering copper rockfish.\\
Tables of the number of samples and positive observations by factors depth, region and year, can be found in Tables \ref{tab:deb}, \ref{tab:tab-region-debwv}, and \ref{tab:tab-year-debwv}.

We modeled catch per angler hour fished (CPUE) by fishing stop where the angler hours were summed across drifts at a fishing stop. To explore weighting of the onboard observer survey index by the available rocky substrate within a region, each drift was assigned to the closest area of rocky habitat. Hard bottom was extracted from the \href{http://seafloor.otterlabs.org/index.html}{California Seafloor Mapping Project}, along the mainland coast of southern California. These data were collected in state waters at a resolution of two meters, but did not extend into state waters past the mainland coast. Additional interpreted bathymetric data classifying the bottom type as rock or soft bottom were compiled by analysts at the University of California Santa Cruz and are now also available from CDFW's website. We used the available interpreted rocky substrate data to expand the known area of rocky substrate to areas in southern California that lack substrate type. This expansion of the estimated rocky substrate assumes that the proportions of rocky substrate within and outside state waters are similar.

The covariates explored for model selection included year, four categorical region levels, and continuous depth and depth-squared.\\
Trends in the average CPUE by region were similar in the filtered data set (Figure \ref{fig:deb-regioncpue}). A year and region interaction was included after visualizing the trends in average CPUE over time, but was not significant (Figure @ref(fig:deb-average\_cpue\_by\_region)). The full model was selected by AICc and included year, depth, depth squared and region (Table @ref(tab:deb-model\_selection)).
>>>>>>> Stashed changes

\begin{figure}
\centering
\includegraphics[width=1\textwidth,height=1\textheight]{S:/copper_rockfish_2023/data/ages/ageing_error/B0_S3/Reader_1_vs_Reader_2.png}
\caption{Distribution of double reads between age reader 1 and 2.\label{fig:age-error-dist}}
\end{figure}

\pagebreak

\begin{figure}
\centering
\includegraphics[width=1\textwidth,height=1\textheight]{S:/copper_rockfish_2023/models/nca/9.8_selex_fix_forecast/plots/numbers5_ageerrorSD.png}
\caption{Ageing imprecision standard deviation of observed age in years.\label{fig:age-error}}
\end{figure}

\pagebreak

\begin{figure}
\centering
\includegraphics[width=1\textwidth,height=1\textheight]{S:/copper_rockfish_2023/models/nca/9.8_selex_fix_forecast/plots/numbers10_ageerror_matrix_1.png}
\caption{Distribution of observed age at true age for ageing error type 1.\label{fig:age-error-matrix}}
\end{figure}

\pagebreak

\hypertarget{model-results}{%
\subsection{Model Results}\label{model-results}}

\hypertarget{model-bridging}{%
\subsubsection{Model Bridging}\label{model-bridging}}

\begin{figure}
\centering
\includegraphics[width=1\textwidth,height=1\textheight]{S:/copper_rockfish_2023/models/nca/_bridging/_plots/0_model_convert_compare2_spawnbio_uncertainty.png}
\caption{Model version bridge comparison of estimated spawning output.\label{fig:bridge-ssb}}
\end{figure}

\pagebreak

\begin{figure}
\centering
\includegraphics[width=1\textwidth,height=1\textheight]{S:/copper_rockfish_2023/models/nca/_bridging/_plots/0_model_convert_compare4_Bratio_uncertainty.png}
\caption{Model version bridge comparison of estimated fraction unfished.\label{fig:bridge-depl}}
\end{figure}

\pagebreak

\begin{figure}
\centering
\includegraphics[width=1\textwidth,height=1\textheight]{S:/copper_rockfish_2023/models/nca/_bridging/_plots/full_bridge_1_compare2_spawnbio_uncertainty.png}
\caption{Model structure and data bridging comparison of estimated spawning output.\label{fig:data-bridge-ssb-1}}
\end{figure}

\pagebreak

\begin{figure}
\centering
\includegraphics[width=1\textwidth,height=1\textheight]{S:/copper_rockfish_2023/models/nca/_bridging/_plots/full_bridge_1_compare4_Bratio_uncertainty.png}
\caption{Model structure and data bridging comparison of estimated fraction unfished.\label{fig:data-bridge-depl-1}}
\end{figure}

\pagebreak

\begin{figure}
\centering
\includegraphics[width=1\textwidth,height=1\textheight]{S:/copper_rockfish_2023/models/nca/_bridging/_plots/full_bridge_2_compare2_spawnbio_uncertainty.png}
\caption{Model structure and data bridging comparison of estimated spawning output.\label{fig:data-bridge-ssb-2}}
\end{figure}

\pagebreak

\begin{figure}
\centering
\includegraphics[width=1\textwidth,height=1\textheight]{S:/copper_rockfish_2023/models/nca/_bridging/_plots/full_bridge_2_compare4_Bratio_uncertainty.png}
\caption{Model structure and data bridging comparison of estimated fraction unfished.\label{fig:data-bridge-depl-2}}
\end{figure}

\pagebreak

\pagebreak

\hypertarget{biology-1}{%
\subsubsection{Biology}\label{biology-1}}

\begin{figure}
\centering
\includegraphics[width=1\textwidth,height=1\textheight]{S:/copper_rockfish_2023/models/nca/9.8_selex_fix_forecast/plots/bio1_sizeatage.png}
\caption{Model estimated length-at-age in the beginning of the year. Shaded area indicates 95 percent distribution of length-at-age around the estimated growth curve.\label{fig:mod-est-len-age}}
\end{figure}

<<<<<<< Updated upstream
\pagebreak
=======
The survey design for CCFRP consists 500 x 500 m cells both within and adjacent to each MPA. On any given survey day site cells are randomly selected within a stratum (MPA and/or reference cells). CPFVs are chartered for the survey and the fishing captain is allowed to search within the cell for a fishing location. During a sampling event, each cell is fished for a total of 30-45 minutes by volunteer anglers. Each fish encountered is recorded, measured, and released (or descended to depth) and can later be linked back to a particular angler, . CCFRP samples shallower depths to avoid barotrauma-induced mortality.\\
Starting in 2017, a subset of fish have been retained to collect otoliths and fin clips that provide needed biological information for nearshore species. For the index of abundance, CPUE was modeled at the level of the drift, similar to the fishery-dependent onboard observer survey described above.
>>>>>>> Stashed changes

\hypertarget{selectivity}{%
\subsubsection{Selectivity}\label{selectivity}}

<<<<<<< Updated upstream
\pagebreak

\hypertarget{recruitment}{%
\subsubsection{Recruitment}\label{recruitment}}

\begin{figure}
\centering
\includegraphics[width=1\textwidth,height=1\textheight]{S:/copper_rockfish_2023/models/nca/9.8_selex_fix_forecast/plots/ts11_Age-0_recruits_(1000s)_with_95_asymptotic_intervals.png}
\caption{Estimated time series of age-0 recruits (1000s).\label{fig:recruits}}
\end{figure}
=======
The CCFRP index includes all of the MPAs currently sampled from 2017-2020 and the core central California sampling sites from 2007-2016. Trends among all of the MPAs sampled increased along the entire coast from 2017-2020. The final index (Table \ref{tab:index-ccfrp}) represents a similar trend to the arithmetic mean of the annual CPUE (Figure \ref{fig:ccfrp-avg-cpue}).

We modeled retained catch per angler hour (CPUE; number of fish per angler hour) using MLE fr. Indices with a year and area (location along the coast) interaction were not considered in model selection; trends in the average CPUE by region were similar in the filtered data set (Figure \ref{fig:fig-areacpue-ccfrp}). Figure \ref{fig:fig-sitecpue-ccfrp} shows arithmetic mean by inside (MPA) and outside (REF) MPAs over time\\
and illustrates the distinct trends of increasing average CPUE over time.

A negative binomial model was fit to the drift-level data (catch with a log offset for angler hours). Because the average observed CPUE inside MPAs and in the reference sites exhibited differing trends, we explored a YEAR:SITE interaction, which was selected as the best fit model by AIC Table \ref{tab:ccfrp-model-selection}), The final model included yrea, mpa/reference categorization, depth, depth squared, and a year:mpa/reference interaction. The model was fit using the sdmTMB R package (version 0.3.0).
>>>>>>> Stashed changes

\pagebreak

\begin{figure}
\centering
\includegraphics[width=1\textwidth,height=1\textheight]{S:/copper_rockfish_2023/models/nca/9.8_selex_fix_forecast/plots/recdevs2_withbars.png}
\caption{Estimated time series of recruitment deviations.\label{fig:rec-devs}}
\end{figure}

\pagebreak

\begin{figure}
\centering
\includegraphics[width=1\textwidth,height=1\textheight]{S:/copper_rockfish_2023/models/nca/9.8_selex_fix_forecast/plots/SR_curve.png}
\caption{Stock-recruit curve. Point colors indicate year, with warmer colors indicating earlier years and cooler colors in showing later years.\label{fig:bh-curve}}
\end{figure}

\pagebreak

\hypertarget{fits-to-data}{%
\subsubsection{Fits to Data}\label{fits-to-data}}

\begin{figure}
\centering
\includegraphics[width=1\textwidth,height=1\textheight]{S:/copper_rockfish_2023/models/nca/9.8_selex_fix_forecast/plots/index5_logcpuefit_Rec_CPFV.png}
\caption{Fit to log index data on log scale for recreational (MRFSS) CPFV. Lines indicate 95\% uncertainty interval around index values based on the model assumption of lognormal error. Thicker lines (if present) indicate input uncertainty before addition of estimated additional uncertainty parameter.\label{fig:mrfss-cpfv-index-fit}}
\end{figure}

\pagebreak

\begin{figure}
\centering
\includegraphics[width=1\textwidth,height=1\textheight]{S:/copper_rockfish_2023/models/nca/9.8_selex_fix_forecast/plots/index5_logcpuefit_DWV_CPFV.png}
\caption{Fit to log index data on log scale for Deb Wilson-Vandenberg CPFV survey. Lines indicate 95\% uncertainty interval around index values based on the model assumption of lognormal error. Thicker lines (if present) indicate input uncertainty before addition of estimated additional uncertainty parameter.\label{fig:dwv-cpfv-index-fit}}
\end{figure}

\pagebreak

\begin{figure}
\centering
\includegraphics[width=1\textwidth,height=1\textheight]{S:/copper_rockfish_2023/models/nca/9.8_selex_fix_forecast/plots/index5_logcpuefit_CRFS_CPFV.png}
\caption{Fit to log index data on log scale for CRFS CPFV survey. Lines indicate 95\% uncertainty interval around index values based on the model assumption of lognormal error. Thicker lines (if present) indicate input uncertainty before addition of estimated additional uncertainty parameter.\label{fig:crfs-cpfv-index-fit}}
\end{figure}

\pagebreak

\begin{figure}
\centering
\includegraphics[width=1\textwidth,height=1\textheight]{S:/copper_rockfish_2023/models/nca/9.8_selex_fix_forecast/plots/index5_logcpuefit_Rec_PR.png}
\caption{Fit to log index data on log scale for recreational (CRFS) PR. Lines indicate 95\% uncertainty interval around index values based on the model assumption of lognormal error. Thicker lines (if present) indicate input uncertainty before addition of estimated additional uncertainty parameter.\label{fig:crfs-pr-index-fit}}
\end{figure}

\pagebreak

\begin{figure}
\centering
\includegraphics[width=1\textwidth,height=1\textheight]{S:/copper_rockfish_2023/models/nca/9.8_selex_fix_forecast/plots/index5_logcpuefit_CCFRP.png}
\caption{Fit to log index data on log scale for CCFRP survey. Lines indicate 95\% uncertainty interval around index values based on the model assumption of lognormal error. Thicker lines (if present) indicate input uncertainty before addition of estimated additional uncertainty parameter.\label{fig:ccfrp-index-fit}}
\end{figure}

\pagebreak

\begin{figure}
\centering
\includegraphics[width=1\textwidth,height=1\textheight]{S:/copper_rockfish_2023/models/nca/9.8_selex_fix_forecast/plots/index5_logcpuefit_CDFW_ROV.png}
\caption{Fit to log index data on log scale for CDFW ROV survey. Lines indicate 95\% uncertainty interval around index values based on the model assumption of lognormal error. Thicker lines (if present) indicate input uncertainty before addition of estimated additional uncertainty parameter.\label{fig:rov-index-fit}}
\end{figure}

\pagebreak

\begin{figure}
\centering
\includegraphics[width=1\textwidth,height=1\textheight]{S:/copper_rockfish_2023/models/nca/9.8_selex_fix_forecast/plots/index9_standcpueall.png}
\caption{Standardized indices overlaid. Each index is rescaled to have mean observation = 1.0.\label{fig:standardized-indices}}
\end{figure}

\pagebreak

\hypertarget{time-series}{%
\subsubsection{Time-series}\label{time-series}}

\begin{figure}
\centering
\includegraphics[width=1\textwidth,height=1\textheight]{S:/copper_rockfish_2023/models/nca/9.8_selex_fix_forecast/plots/ts7_Spawning_output_with_95_asymptotic_intervals_intervals.png}
\caption{Estimated time series of spawning biomass.\label{fig:ssb}}
\end{figure}

<<<<<<< Updated upstream
\pagebreak

\begin{figure}
\centering
\includegraphics[width=1\textwidth,height=1\textheight]{S:/copper_rockfish_2023/models/nca/9.8_selex_fix_forecast/plots/ts1_Total_biomass_(mt).png}
\caption{Estimated time series of total biomass.\label{fig:tot-bio}}
\end{figure}

\pagebreak

\begin{figure}
\centering
\includegraphics[width=1\textwidth,height=1\textheight]{S:/copper_rockfish_2023/models/nca/9.8_selex_fix_forecast/plots/ts9_Relative_spawning_output_intervals.png}
\caption{Estimated time series of fraction of unfished spawning biomass.\label{fig:depl}}
\end{figure}
=======
\hypertarget{cdfw-rov-index}{%
\section{Appendix G. CDFW ROV Index of Abundance}\label{cdfw-rov-index}}

The California Department of Fish and Wildlife (CDFW) in collaboration with Marine Applied Research and Exploration (MARE) have been conducting remotely operated vehicle (ROV) surveys along the California coast in Marine Protected Areas (MPAs) and reference sites adjacent to them since 2004 for the purposes of long-term monitoring of changes in size, density (fish/square meter) and length of fish and invertebrate species along the California coast. Surveys of the entire coast have now been undertaken twice, each taking three years to complete, during 2014-2016 and again in 2019-2021. The survey conducted multiple 500 meter transects across rocky reef survey sites. Transect locations within a site were selected by first randomly selecting the deepest transect at a given site, then placing additional transects on a constant interval into shallower depths. Transects were designed to be oriented parallel to general depth contours, though they were carried out using a fixed bearing that crossed depths in some cases.

Given that each pass of the California coast took a three year period, the STAT opted to explore using the data either by year or grouping it into super years. The selected super years were 2015 and 2020, the middle year of the time grouped sampling efforts. Based on the life history of copper rockfish and the generally limited movement of adult copper rockfish, the super year approach was considered to generate separate indices for north and south of Point Conception. The two sub-area models for copper rockfish represent disparate proportions of the California coast where the model south of Point Conception has a greatly reduced spatial range compared to the model area north of Point Conception. South of Point Conception, nearly all sampling locations were visited either three or four times within the six year sampling period (only one reference location only visited one year) while sampling locations north of Point Conception were visited between two to four times within the six sampling years. These differences in sampling frequency and the areas being sampled informed the selection of modeling these data differently by area. The data south of Point Conception were modeled using the sample year while the data north of Point Conception were modeled using super years.

Minimal filtering was done to the data. Transects were removed based on four factors: 1) extreme estimates of effort (the estimated area of view below the ROV termed usable area), 2) any locations that were not sampled by both super year periods, 3) transects that crossed from MPA into reference areas, and 4) transects conducted across depths that never observed copper rockfish within the survey (Table \ref{tab:rov-filtered}). Once the data were filtered the average calculated CPUE for each MPA and Reference group were plotted to visualize the data (Table \ref{tab:rov-obs} and Figure \ref{fig:rov-raw-cpue}).

A range of alternative model structures were explored to generate an index of abundances including alternative error structures, covariates, and factors. Based on model selection a model with super year, site designation (MPA or Reference), and super year site designation interaction was selected (Table \ref{tab:rov-model-selection}). A negative-binomial model was selected based on the distribution of the data and diagnostics (Figures \ref{fig:rov-qq} and \ref{fig:rov-prop-zero}) using sdmTMB (Anderson et al. 2022). The model estimates were then area-weighted based on the estimated percent of habitat within MPAs based on habitat seafloor mapping data within state waters, which for the region north of Point Conception were estimated to be 20\% of rocky habitat within MPAs and 80\% open to fishing. The weighted relative index of abundance is shown in Table \ref{tab:rov-index} and Figure \ref{fig:rov-index}.
>>>>>>> Stashed changes

\pagebreak

\begin{figure}
\centering
\includegraphics[width=1\textwidth,height=1\textheight]{C:/Users/melissa.monk/Documents/GitHub/copper_rockfish_2023/documents/shared_figures/spawning_output_combined.png}
\caption{Estimated combined time series of spawning output for copper rockfish in California waters.\label{fig:sb-all}}
\end{figure}

\clearpage

\begin{figure}
\centering
\includegraphics[width=1\textwidth,height=1\textheight]{C:/Users/melissa.monk/Documents/GitHub/copper_rockfish_2023/documents/shared_figures/depletion_combined.png}
\caption{Estimated combined time series of fraction of relative spawning output for copper rockfish in California waters.\label{fig:depl-all}}
\end{figure}

\clearpage

\hypertarget{sensitivity-analyses-and-retrospectives}{%
\subsubsection{Sensitivity Analyses and Retrospectives}\label{sensitivity-analyses-and-retrospectives}}

\begin{figure}
\centering
\includegraphics[width=1\textwidth,height=1\textheight]{S:/copper_rockfish_2023/models/nca/_sensitivities/_plots/Sensi_REplot_all_horizontal.png}
\caption{Comparison of the relative change in estimated management quantities as compared to the base model. The quantities compared are the estimate of unfished spawning biomass (SB0), spawning output in 2023 (SB2023), the relative spawnin output (SB2023/SB0), the yield based on a spawner per recruit harvest rate (YieldSPR=0.50), and the fishing mortality at that harvest rate (FSPR=0.50). The colored boxes indicate the 95 percent confidence interval around the point estimate of the quantity from the base model where each color corresponds with a specific quantity in the legend. A model with matching estimates as the base model would reflect a relative change of 0, a model with estimates less than the base model would have a negative relative change, and a model with estimates greater than the base model would have a positive relative change.\label{fig:sens-all}}
\end{figure}

\newpage

\begin{figure}
\centering
\includegraphics[width=1\textwidth,height=1\textheight]{S:/copper_rockfish_2023/models/nca/_sensitivities/_plots/9.8_selex_fix_forecast_final_1_compare2_spawnbio_uncertainty.png}
\caption{Change in estimated spawning output by sensitivity.\label{fig:sens-ssb-1}}
\end{figure}

\newpage

\begin{figure}
\centering
\includegraphics[width=1\textwidth,height=1\textheight]{S:/copper_rockfish_2023/models/nca/_sensitivities/_plots/9.8_selex_fix_forecast_final_1_compare4_Bratio_uncertainty.png}
\caption{Change in estimated fraction unfished by sensitivity.\label{fig:sens-depl-1}}
\end{figure}

\newpage

\begin{figure}
\centering
\includegraphics[width=1\textwidth,height=1\textheight]{S:/copper_rockfish_2023/models/nca/_sensitivities/_plots/9.8_selex_fix_forecast_final_2_compare2_spawnbio_uncertainty.png}
\caption{Change in estimated spawning output by sensitivity.\label{fig:sens-ssb-2}}
\end{figure}

\newpage

\begin{figure}
\centering
\includegraphics[width=1\textwidth,height=1\textheight]{S:/copper_rockfish_2023/models/nca/_sensitivities/_plots/9.8_selex_fix_forecast_final_2_compare4_Bratio_uncertainty.png}
\caption{Change in estimated fraction unfished by sensitivity.\label{fig:sens-depl-2}}
\end{figure}

\newpage

\begin{figure}
\centering
\includegraphics[width=1\textwidth,height=1\textheight]{S:/copper_rockfish_2023/models/nca/_sensitivities/_plots/9.8_selex_fix_forecast_final_3_compare2_spawnbio_uncertainty.png}
\caption{Change in estimated spawning output by sensitivity.\label{fig:sens-ssb-3}}
\end{figure}

\newpage

\begin{figure}
\centering
\includegraphics[width=1\textwidth,height=1\textheight]{S:/copper_rockfish_2023/models/nca/_sensitivities/_plots/9.8_selex_fix_forecast_final_3_compare4_Bratio_uncertainty.png}
\caption{Change in estimated fraction unfished by sensitivity.\label{fig:sens-depl-3}}
\end{figure}

\newpage

\begin{figure}
\centering
\includegraphics[width=1\textwidth,height=1\textheight]{N:/Assessments/CurrentAssessments/copper_rockfish_2023/models/nca/9.8_selex_fix_forecast_retro/compare2_spawnbio_uncertainty.png}
\caption{Change in the estimate of spawning output when the most recent 5 years of data area removed sequentially.\label{fig:retro-ssb}}
\end{figure}

\pagebreak

\begin{figure}
\centering
\includegraphics[width=1\textwidth,height=1\textheight]{N:/Assessments/CurrentAssessments/copper_rockfish_2023/models/nca/9.8_selex_fix_forecast_retro/compare4_Bratio_uncertainty.png}
\caption{Change in the estimate of fraction unfished when the most recent 5 years of data area removed sequentially.\label{fig:retro-depl}}
\end{figure}

\pagebreak

\hypertarget{likelihood-profiles}{%
\subsubsection{Likelihood Profiles}\label{likelihood-profiles}}

\begin{figure}
\centering
\includegraphics[width=1\textwidth,height=1\textheight]{S:/copper_rockfish_2023/models/nca/9.8_selex_fix_forecast_profile_SR_LN(R0)_prior_like_0/piner_panel_SR_LN(R0).png}
\caption{Change in the negative log-likelihood across a range of log(R\textsubscript{0}) values.\label{fig:r0-profile}}
\end{figure}

\pagebreak

\begin{figure}
\centering
\includegraphics[width=1\textwidth,height=1\textheight]{S:/copper_rockfish_2023/models/nca/9.8_selex_fix_forecast_profile_SR_LN(R0)_prior_like_0/SR_LN(R0)_trajectories_compare1_spawnbio.png}
\caption{Change in the estimate of spawning output across a range of log(R\textsubscript{0}) values.\label{fig:r0-ssb}}
\end{figure}

\pagebreak

\begin{figure}
\centering
\includegraphics[width=1\textwidth,height=1\textheight]{S:/copper_rockfish_2023/models/nca/9.8_selex_fix_forecast_profile_SR_LN(R0)_prior_like_0/SR_LN(R0)_trajectories_compare3_Bratio.png}
\caption{Change in the estimate of fraction unfished across a range of log(R\textsubscript{0}) values.\label{fig:r0-depl}}
\end{figure}

\pagebreak

\begin{figure}
\centering
\includegraphics[width=1\textwidth,height=1\textheight]{S:/copper_rockfish_2023/models/nca/9.8_selex_fix_forecast_profile_SR_BH_steep_prior_like_1/piner_panel_SR_BH_steep.png}
\caption{Change in the negative log-likelihood across a range of steepness values.\label{fig:h-profile}}
\end{figure}

\pagebreak

\begin{figure}
\centering
\includegraphics[width=1\textwidth,height=1\textheight]{S:/copper_rockfish_2023/models/nca/9.8_selex_fix_forecast_profile_SR_BH_steep_prior_like_1/SR_BH_steep_trajectories_compare1_spawnbio.png}
\caption{Change in the estimate of spawning output across a range of steepness values.\label{fig:h-ssb}}
\end{figure}

\pagebreak

\begin{figure}
\centering
\includegraphics[width=1\textwidth,height=1\textheight]{S:/copper_rockfish_2023/models/nca/9.8_selex_fix_forecast_profile_SR_BH_steep_prior_like_1/SR_BH_steep_trajectories_compare3_Bratio.png}
\caption{Change in the estimate of fraction unfished across a range of steepness values.\label{fig:h-depl}}
\end{figure}

\pagebreak

\begin{figure}
\centering
\includegraphics[width=1\textwidth,height=1\textheight]{S:/copper_rockfish_2023/models/nca/9.8_selex_fix_forecast_profile_NatM_uniform_Fem_GP_1_prior_like_1/piner_panel_NatM_uniform_Fem_GP_1.png}
\caption{Change in the negative log-likelihood across a range of female natural mortality values.\label{fig:m-profile}}
\end{figure}

\pagebreak

\begin{figure}
\centering
\includegraphics[width=1\textwidth,height=1\textheight]{S:/copper_rockfish_2023/models/nca/9.8_selex_fix_forecast_profile_NatM_uniform_Fem_GP_1_prior_like_1/NatM_uniform_Fem_GP_1_trajectories_compare1_spawnbio.png}
\caption{Change in the estimate of spawning output across a range of female natural mortality values.\label{fig:m-ssb}}
\end{figure}

\pagebreak

\begin{figure}
\centering
\includegraphics[width=1\textwidth,height=1\textheight]{S:/copper_rockfish_2023/models/nca/9.8_selex_fix_forecast_profile_NatM_uniform_Fem_GP_1_prior_like_1/NatM_uniform_Fem_GP_1_trajectories_compare3_Bratio.png}
\caption{Change in the estimate of fraction unfished across a range of female natural values.\label{fig:m-depl}}
\end{figure}

\begin{figure}
\centering
\includegraphics[width=1\textwidth,height=1\textheight]{C:/Users/melissa.monk/Documents/GitHub/copper_rockfish_2023/documents/shared_figures/north_assess_compare_compare2_spawnbio_uncertainty.png}
\caption{Comparison of the estimated spawning output for the base model to previous assessment in 2021.\label{fig:comp-assess-sb}}
\end{figure}

\newpage

\begin{figure}
\centering
\includegraphics[width=1\textwidth,height=1\textheight]{C:/Users/melissa.monk/Documents/GitHub/copper_rockfish_2023/documents/shared_figures/north_assess_compare_compare4_Bratio_uncertainty.png}
\caption{Comparison of the estimated fraction unfished for the base model to previous assessment in 2021.\label{fig:comp-assess-depl}}
\end{figure}

\newpage

\hypertarget{reference-points-and-forecasts}{%
\subsubsection{Reference Points and Forecasts}\label{reference-points-and-forecasts}}

\begin{figure}
\centering
\includegraphics[width=1\textwidth,height=1\textheight]{C:/Users/melissa.monk/Documents/GitHub/copper_rockfish_2023/documents/shared_figures/compare6_SPRratio_uncertainty.png}
\caption{Estimated 1 - relative spawning ratio (SPR) by year for both sub-area models south and north of Point Conception.\label{fig:1-spr}}
\end{figure}

\clearpage

\begin{figure}
\centering
\includegraphics[width=1\textwidth,height=1\textheight]{C:/Users/melissa.monk/Documents/GitHub/copper_rockfish_2023/documents/shared_figures/compare15_phase_plot.png}
\caption{Phase plot of the relative biomass (also referred to as fraction unfished) versus the SPR ratio where each point represents the biomass ratio at the start of the year and the relative fishing intensity in that same year. Lines through the final point show the 95 percent intervals based on the asymptotic uncertainty for each dimension. The shaded ellipse is a 95 percent region which accounts for the estimated correlations between the biomass ratio and SPR ratio.\label{fig:phase}}
\end{figure}

\pagebreak

\begin{figure}
\centering
\includegraphics[width=1\textwidth,height=1\textheight]{S:/copper_rockfish_2023/models/sca/14.0_base_forecast/plots/yield2_yield_curve_with_refpoints.png}
\caption{Equilibrium yield curve for the base case model south of Point Conception. Values are based on the 2022 fishery selectivities and with steepness fixed at 0.72.\label{fig:yield-south}}
\end{figure}

\pagebreak

\begin{figure}
\centering
\includegraphics[width=1\textwidth,height=1\textheight]{S:/copper_rockfish_2023/models/nca/9.8_selex_fix_forecast/plots/yield2_yield_curve_with_refpoints.png}
\caption{Equilibrium yield curve for the base case model north of Point Conception. Values are based on the 2022 fishery selectivities and with steepness fixed at 0.72.\label{fig:yield-north}}
\end{figure}

\pagebreak
\end{document}
